\section{Introduction to \maestro\ Networks}

\maestro\ models multiple species, described by the mass density of
the fluid, $\rho$, and the mass fraction of the species, $X_k \equiv
\rho_k/\rho$, where $\rho_k$ is the mass density of species $k$.  All
\maestro\ problems, regardless of whether they model reactions, need a
network.  In its most basic form, the network supplies the properties
of the species (atomic mass, atomic number) that are interpreted by
the equation of state to compute.

\section{Notes of Specific Networks}

\subsection{{\tt general\_null}}

This is a 'null' network -- i.e. no burning, just define the
properties of the species for thermodynamics.  The twist is that you
can create an inputs file to define what species you want to carry.
For example, the extern/networks/null/ network defines C12, O16, and
Mg24.  To replicate this in {\tt general\_null}, we have the file
{\tt ignition.net} with contents:

\begin{verbatim}
# name       short name    aion     zion
 carbon-12      C12         12.0     6.0
 oxygen-16      O16         16.0     8.0
 magnesium-24   Mg24        24.0    12.0
\end{verbatim}

To use this set of species in your problem, you would set:

\begin{verbatim}
NETWORK_DIR := extern/networks/general_null
GENERAL_NET_INPUTS := ignition.net
\end{verbatim}

It is assumed that the *.net files live in {\tt extern/networks/general\_null/}

Then at compile time, the network.f90 is created using these species and
compiled.  (For the curious, the rule to build {\tt network.f90} lives in
{\tt extern/networks/general\_null/GPackage.mak})

