\label{ch:make}

\section{Build Process Overview}

The \maestro\ build system uses features of GNU make (version 3.81 or
later), which is typically the default on systems.  The \maestro\
executable is built in the problem's directory (one of the directories
under {\tt SCIENCE/}, {\tt TEST\_PROBLEMS}, or {\tt UNIT\_TESTS}).  This
directory will contain a makefile, {\tt GNUmakefile}, that includes
all the necessary information to build the executable.

The main macros that define the build process are split across several
files.  The 4 main files are:
\begin{itemize}
\item {\tt \$\{AMREX\_HOME\}/Tools/F\_mk/GMakedefs.mak}:

  This setups the basic macros, includes the options for the selected
  compiler, builds the list of object and source files, and defines
  the compile and link command lines.

\item {\tt \$\{AMREX\_HOME\}/Tools/F\_mk/GMakeMPI.mak}:

  This implements any changes to the compiler names and library
  locations necessary to build a parallel executable with MPI.

\item {\tt \$\{AMREX\_HOME\}/Tools/F\_mk/GMakerules.mak}:

  This creates the various build targets and specifies the rules for
  building the object files, the list of dependencies, and some other
  lesser-used targets (tags for editors, documentation, etc.)

\item {\tt MAESTRO/GMaestro.mak}:

  This is a \maestro-specific file that gathers all of the various
  modules that are used to build a typical \maestro\ application
  and integrates with the \amrex\ build system.  Every \maestro\
  problem's {\tt GNUmakefile} will include this file.

\end{itemize}

\maestro\ gets the location of the \amrex\ library through the 
{\tt AMREX\_HOME} variable.  This should be set as an environment
variable in your shell start-up files (e.g.\ {\tt .bashrc} or {\tt
.cshrc}).

The \amrex\ build system separates the compiler-specific information
from the machine-specific information---this allows for reuse of the
compiler information.  The only machine-specific parts of the build system
are for the MPI library locations, contained in {\tt GMakeMPI.mak}.
The compiler flags for the various compilers are listed in the
files in {\tt \$\{AMREX\_HOME\}/Tools/F\_mk/comps/}.  The compiler
is set via the {\tt COMP} variable in the problem's {\tt GNUmakefile}.

There are several options in addition to the compiler that affect the
build process: {\tt MPI}, {\tt OMP}, and {\tt NDEBUG}---these turn on/off
MPI parallelization, OpenMP parallelization, and debugging.  Together,
these choices along with the compiler name are reflected in the name
of the executable.  

When the `{\tt make}' command is run, the object and module files are
written into a directory {\tt t/{\em OS}.{\em COMP}.{\em other}/},
where {\tt \em OS} is the operating system detected by the build
system, {\tt \em COMP} is the compiler used, and {\tt \em other}
reflects any other build options (MPI, OpenMP, debugging, etc.) used.
Separating each build into a separate subdirectory under the problem
directory allows for multiple builds of \maestro\ to exist
side-by-side in the problem directory.

\subsection{Finding Source Files}

The \maestro\ executable is built from source distributed across a
number of directories.  In each of these directories containing source
files, there is a {\tt GPackage.mak} file.  This file has a number of
lines of the form:
\begin{verbatim}
f90sources += file.f90
\end{verbatim}
where {\tt file.f90} is a source file that should be built when this
directory is added to the list of build directories.  For old
fixed-form Fortran files, the files should be added to the {\tt
fsources} variable instead of {\tt f90sources}.

The \amrex\ build system relies on the vpath functionality of {\tt
make}.  In a makefile, the {\tt vpath} variable holds search path used
to locate source files.  When a particular file is needed, the
directories listed in {\tt vpath} are searched until the file is
found.  The first instance of a file is used.  We exploit this feature
by always putting the build directory first in {\tt vpath} (this is
done in {\tt GMakerules.mak}).  This means that if a source file is
placed in the build directory, that copy will override any other
version in the source path.

In \maestro, the {\tt vpath} variable is set using the macros defined
in {\tt GMaestro.mak}.  A user does not need to set this variable
explicitly.  Additional source locations are added in the manner
described below (see \S~\ref{sec:make:otherfiles}).

\subsection{Dependencies}

There is no need to explicitly define the dependencies between the
source files for Fortran modules.  The scripts in {\tt
AMREX\_HOME/Tools/F\_scripts/} are run at the start of the build
process and parse all the source files and make an explicit list of
the dependency pairs.  The execution of these scripts is triggered
by including makefiles of the form {\tt *.depends}.  On a fresh build these
will not exist.  When GNU make cannot find an included makefile it will
first attempt to build it using any relevant targets before issuing an 
error.  The targets for the {\tt *.depends} files contain the recipe for
executing the dependency scripts.  Once these makefiles are built by the
scripts GNU make will then read the dependencies for the current build
from them.  This process is defined in {\tt GMakerules.mak}.

A few files use explicit `{\tt include}' statements to include Fortran
source in other source files.  Any include file locations should be
added to {\tt Fmincludes} variable in the problem's {\tt GNUmakefile}.
This does not occur frequently.  For the case of the {\tt helmholtz}
equation of state, this is done automatically in {\tt GMaestro.mak}.


\subsection{Files Created at Compile-time}

Several files are created at build-time:
\begin{itemize}
\item {\tt probin.f90}:

  This is the module that controls runtime parameters.  This is
  created by the script
  {\tt write\_probin.py} in {\tt \$\{AMREX\_HOME\}/Tools/F\_scripts/}.  The
  makefile logic for creating it is in {\tt GMaestro.mak}.  At compile
  time, the problem, main {\tt MAESTRO/}, and any microphysics
  directories (set from the {\tt EOS\_DIR}, {\tt CONDUCTIVITY\_DIR}, and {\tt NETWORK\_DIR} parameters in the {\tt GNUmakefile}
  are searched for {\tt \_parameter} files.  These files
  are parsed and the {\tt probin.f90} file is output containing the 
  runtime parameters, their default values, and the logic for reading
  and interpreting the inputs file.

\item {\tt build\_info.f90}:

  This is a module that contains basic information about the build
  environment (compiler used, build machine, build directory, compiler
  flags, etc.).  This is created by the script {\tt makebuildinfo.py}
  in {\tt \$\{AMREX\_HOME\}/Tools/F\_scripts/} from {\tt
  GMaestro.mak} by passing in a number of makefile variables.  This is
  rewritten everytime the code is compiled.  The primary use of this
  module is writing out the {\tt job\_info} file in the plotfiles.

\item ({\tt network.f90}):

  This is generated at compile time {\em only} for the {\tt
  general\_null} network.  The {\tt general\_null} network allows the
  user to define a list of non-reacting species builds the {\tt
  network.f90} based on this list.  The makefile logic for building
  the {\tt network.f90} is in the {\tt GPackage.mak} in {\tt
  Microphysics/networks/general\_null}.  The script {\tt write\_network.py}
  in that directory does the actual parsing of the species file and
  outputs the {\tt network.f90}.


\end{itemize}



\section{\maestro\ Problem Options}

\subsection{Problem-specific Files}
\label{sec:make:otherfiles}

If a problem has a unique file that is needed as part of the build,
then that file should be added to a {\tt GPackage.mak} file in the
problem's directory.  Since, by default, problems don't have a {\tt
GPackage.mak}, the build system needs to be told to look in the
problem directory for unique sources.  This is accomplished by adding
the problem's directory to the {\tt EXTRA\_DIR} variable in the
problem's {\tt GNUmakefile}.  

Note that this is not necessary if you place a custom version of 
a source file in the problem's directory.  Since that file is already
listed in the {\tt GPackage.mak} in its original location, the build
system will know that it needs to be built.  Since the {\tt vpath}
variable puts the problem's directory at the start of the search
path, the version of the file in the problem's directory will be 
found first.

\subsection{Defining EOS, Network, and Conductivity Routines}

Each \maestro\ problem needs to define an equation of state, a
reaction network, and a routine to compute the conductivities (for
thermal diffusion).  This is true even if the problem is not doing
reactions of thermal diffusion.  These definitions are specified
in the problem's {\tt GNUmakefile}.

\begin{itemize}
\item {\tt EOS\_DIR}:

  This variable points to the directory (by default, relative to {\tt
  Microphysics/EOS/}) of the equation of state used for the build.
  Choices that work with \maestro\ are:
  \begin{itemize}
  \item {\tt helmholtz}
  \item {\tt gamma\_law\_general}
  \item {\tt multigamma}
  \end{itemize}
  To use an EOS contained in a different location, set the variable {\tt
  EOS\_TOP\_DIR} to point to the directory above the alternate EOS
  directory.

\item {\tt CONDUCTIVITY\_DIR}:

  This variable points to the conductivity routine used for the build
  (by default, relative to {\tt Microphysics/conductivity/}).  Choices
  that work with \maestro\ are: 
  %
  \begin{itemize} 
  \item {\tt constant} 
  \item {\tt timmes\_stellar} 
  \end{itemize} 
  % 
  If diffusion is not being used for the problem, this should be set
  to {\tt constant}.  To use an alternate conductivity
  routine, set the variable {\tt CONDUCTIVITY\_TOP\_DIR} to point
  to the directory above the alternate conductivity directory.

\item {\tt NETWORK\_DIR}:

  This variable points to the reaction network used for the build (by
  default, relative to {\tt Microphysics/networks/}).  Several options
  are present in {\tt Microphysics/networks/}.  A network is required even
  if you are not doing reactions, since the network defines the
  species that are advected and interpreted by the equation of state.

  A special choice, {\tt Microphysics/networks/general\_null} is a general
  network that simply defines the properties of one or more species.
  This requires an inputs file, specified by {\tt
  GENERAL\_NET\_INPUTS}.  This inputs file is read at compile-time and
  used to build the {\tt network.f90} file that is compiled into the
  source.

  To use an alternate reaction network, set the variable {\tt
  NETWORK\_TOP\_DIR} to point to the directory above the alternate
  network.

\end{itemize}


\subsection{Core \maestro\ modules}

Several modules are included in all \maestro\ builds by default.
From \amrex, we alway include:
\begin{itemize}
\item {\tt \$\{AMREX\_HOME\}/Src/F\_BaseLib}
\item {\tt \$\{AMREX\_HOME\}/Src/LinearSolvers/F\_MG}
\end{itemize}

\noindent From {\tt Util}, we always include
\begin{itemize}
\item {\tt Util/model\_parser}
\item {\tt Util/random}
\end{itemize}

The microphysics, as described above is also included.  For the
networks, we include a file called {\tt NETWORK\_REQUIRES} into {\tt
  GMaestro.mak} that tells us whether to also include {\tt Util/VODE}
(if {\tt NEED\_VODE := t}).  It is assumed in this case that we need
BLAS and LINPACK, so these are compiled in from {\tt Util/BLAS}  and
{\tt Util/LINPACK}.  

You can instead link in a system-wide optimized BLAS library by setting
{\tt SYSTEM\_BLAS := t}\otherindex{{\tt GNUmakefile}}{BLAS} in the {\tt GNUmakefile}.  This adds {\tt -lblas}
to the link line, and assumes that the library is in your path.  Note
that for some systems, you should have the static BLAS libraries 
available. 


From {\tt MAESTRO/}, we add
\begin{itemize}
\item {\tt MAESTRO/constants}
\item {\tt MAESTRO/Source}
\end{itemize}
(although see the unit tests section below regarding {\tt MAESTRO/Source}.

\noindent Finally, any extra directories listed in the {\tt EXTRA\_DIR}
variable are included.

For each of these included directories, {\tt GMaestro.mak} adds the
list of source files defined in their {\tt GPackage.mak} to the list
of files to be compiled.  It also adds each of these directories to
the {\tt vpath} as a directory for the build process to search in for
source files.


\subsection{Unit Tests}

Sometimes we only want to use a few of the standard \maestro\
routines, for example in a unit test where we are testing only a small
part of the \maestro\ algorithm indepenedently.  In this case, we
don't want to comple all of the files in {\tt MAESTRO/Source}.  If we
set {\tt UNIT\_TEST := t} in our problem's {\tt GNUmakefile}, then the
{\tt GPackage.mak} in {\tt MAESTRO/Source} is not read, so those files
are not automatically put into the list of files to compile.  Instead,
the problem should create its own {\tt GPackage.mak} listing only the
subset of files that are to be compiled.  {\tt MAESTRO/Source} is put
into the {\tt vpath} search path for sources, so those files will
still be found as needed.


\subsection{\amrex-only Tests}

An even more restrictive setting than {\tt UNIT\_TEST := t} is invoked
by setting {\tt AMREX\_ONLY := t}.  This is like the unit test flag,
but does not include {\tt MAESTRO/Source} in the {\tt vpath} search
path for sources.  So this is intended for cases where we don't want
to use any \maestro\ source files.  Typically, this is used in the
small unit tests that live under the various microphysics solvers.  If
a {\tt probin.f90} is built for these tests, it will not include all
the \maestro-specific parameters, but will include any parameters from
the various microphysics routines.


\section{Special Targets}

\subsection{Debugging}

\subsubsection{({\tt print-*})}

To see the contents of any variable in the build system, you can build
the special target {\tt print-{\em varname}}, where {\tt {\em
varname}} is the name of the variable.  For example, to see what the
Fortran compiler flags are, you would do:
\begin{verbatim}
make print-FFLAGS
\end{verbatim}
This would give (for {\tt gfortran}, for example):
\begin{verbatim}
FFLAGS is -Jt/Linux.gfortran/m -I t/Linux.gfortran/m -O2 -fno-range-check
\end{verbatim}
This functionality is useful for debugging the makefiles.

\subsubsection{{\tt file\_locations}}

Source files are found by searching through the {\tt make} {\tt
  vpath}.  The first instance of the file found in the {\tt vpath}
is used in the build.  To see which files are used and their locations,
do:
\begin{verbatim}
make file_locations
\end{verbatim}

This will also show any files that aren't found.  Some are expected
(e.g., {\tt build\_info.f90} and {\tt probin.f90} are created at
compile time), but other files that are not found could indicate
an incomplete {\tt vpath}.

\subsection{{\tt clean} and {\tt realclean}}

Typing `{\tt make clean}' deleted the object and module files for the
current build (i.e., the current choice of {\tt MPI}, {\tt NDEBUG},
{\tt COMP}, and {\tt OMP}).  This also removes any of the compile-time
generated source files.  Any other builds are left unchanged.

Typing `{\tt make realclean}' deletes the object and module files for
all builds---i.e., the entire {\tt t/} directory is removed.


\section{Special Debugging Modes}

\label{ch:makefiles:special}

\amrex\ has several options that produce executables that can help
track down memory issues, uninitialized variables, NaNs, etc.

\begin{itemize}

\item {\tt NDEBUG} 

  \otherindex{{\tt GNUmakefile}}{NDEBUG} To generate an executable
  with debugging information included in the executable (e.g., to be
  interpreted by the debugger, {\tt gdb}), compile with {\tt NDEBUG
    :=\ }.  This will usually add {\tt -g} to the compile line and
  also lower the optimization.  For {\tt gfortran} it will add several
  options to catch uninitialize variables, bounds errors, etc.


\item {\tt TEST}

  Setting {\tt TEST := t}\otherindex{{\tt GNUmakefile}}{TEST} will
  enable routines in \amrex\ initialize multifabs and arrays 
  allowed via {\tt bl\_allocate} to signalliing NaNs.  This behavior
  is the same as {\tt NDEBUG :=}, but {\tt TEST := t} uses the 
  same compiler optimizations as a normal build.  

  This can be useful with compiler flags that trap floating point
  exceptions (FPEs), but checks on floating point exceptions can also
  be enabled through runtime parameters passed to \amrex's
  backtrace functionlity:
  \begin{itemize}
  \item \runparam{boxlib\_fpe\_invalid}: enabling FPE trapping for
    invalid operations (e.g. {\tt 0 * inf}, {\tt sqrt(-1)})

  \item \runparam{boxlib\_fpe\_zero}: enable FPE trapping for
    divide-by-zero

  \item \runparam{boxlib\_fpe\_overflow}: enable FPE trapping for
    overflow
  \end{itemize}

\item backtracing

  When exception trapping is enabled (either via \amrex\ or the
  compiler), the code will abort, and the backtrace information will
  be output to a file {\tt Backtrace.N}, where {\tt N} is the
  processor number.  \amrex\ will also initialize \multifab s with
  signaliing NaNs to help uncover any floating point issues.

  This is also useful to diagnose deadlocks in parallel regions.
  If the code is hanging, doing ``control-C'' will be intercepted
  and the code will generate a backtrace which will identify
  where in the code there was a deadlock.

  Behind the scenes, \amrex\ implements this capability via the
  Linux/Unix {\tt feenableexcept} function (this is in {\tt
    backtrace\_c.cpp} in \amrex).

\item {\tt FSANITIZER}

  For {\tt gfortran}, {\tt gcc}, {\tt g++}, setting {\tt FSANITIZER :=
    t}\otherindex{{\tt GNUmakefile}}{FSANITIZER} will enable the
  address sanitizer support built into GCC.  This is enabled through
  integration with \url{https://github.com/google/sanitizers} in GCC.

  Note: you will need to have the libraries {\tt libasan} and {\tt
    libubsan} installed on your machine to use this functionality.

\end{itemize}


\section{Extending the Build System}

\subsection{Adding a Compiler}

Properties for different compilers are already defined in {\tt
  \$\{AMREX\_HOME\}/Tools/F\_mk/comps/}.  Each compiler is given its
own file.  The appropriate file is included into {\tt GMakedefs.mak}
by looking at the {\tt COMP} variable and the operating system.  These
compiler files define the compiler flags for both optimized and debug
compiling.  Additionally, the variable {\tt FCOMP\_VERSION} should be
defined there, based on the output from the compiler, to provide the
compiler version for output into the {\tt job\_info} file at runtime.


\subsection{Parallel (MPI) Builds}

When building with MPI, the build system needs to know about the
location of the MPI libraries.  If your local MPI has the {\tt mpif90}
and {\tt mpicc} wrappers installed and working, then \maestro\ will
attempt to use these.  Otherwise, you will need to edit {\tt
  GMakeMPI.mak} and add a section specific to your machine with the
compiler and library location.  It is best to simply copy an existing
similar portion of the makefile and adjust it to your system.  Most
national supercomputing facilities are already supported, and parallel
builds on them should work out of the box.



