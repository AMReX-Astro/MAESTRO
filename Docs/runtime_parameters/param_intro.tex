
\section{Introduction to Runtime Parameters}
\label{sec:runtime_parameters}

The runtime parameters are defined in the {\tt MAESTRO/\_parameters}
files, and any other problem-specific parameter files.  These
parameters are then made available to the code through the {\tt
  probin\_module}.

Any runtime parameters defined by the microphysics (either in the
{\tt MAESTRO/Microphysics/} source or the external {\tt Microphysics/}
source are also parsed at build time and read in via the same
namelist in {\tt probin.f90}.  These microphysics runtime parameters
can be accessed via {\tt extern\_probin\_module}.

Parameter definitions take the form of:
\begin{verbatim}
# comment describing the parameter
name              data-type       default-value      priority
\end{verbatim} 
Here, the {\tt priority} is simply an integer.  When two directories
define the same parameter, but with different defaults, the version of
the parameter with the highest priority takes precedence.  This allows
specific implementations to override the general parameter defaults.

The following tables list all of the general
MAESTRO runtime parameters.
These tables are generated automatically using the {\tt
  rp.py} script in {\tt MAESTRO/docs/runtime\_parameters/} by parsing
the {\tt MAESTRO/\_parameters} file.  The problem-specific parameters
are not shown here.
