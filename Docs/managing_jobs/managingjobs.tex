\label{ch:managingjobs}

\section{General Info}

All plotfile directories have a {\tt job\_info} file which lists as
host of parameters about the simulation, including:
\begin{itemize}
\item A descriptive name for the simulation (the {\tt job\_name} runtime
  parameter
\item The number of MPI tasks and OpenMP threads
\item The total amount of CPU-hours used up to this point
\item The data and time of the plotfile creation and the directory it was written to.
\item The build date, machine, and directory
\item The \maestro, \boxlib, and other relevant git hashes for the source
\item The directories used in the build
\item The compilers used and compilation flags
\item The number of levels and boxes for the grid
\item The properties of the species carried
\item The tolerances for the MG solves
\item Any warnings from the initialization procedure (note: these are not currently stored on restart).
\item The value of all runtime parameters (even those that were not explicitly set in the inputs file), along with an indicator showing if the default value
was overridden.
\end{itemize}
This file makes it easy to understand how to recreate the run that
produced the plotfile.


\section{Linux boxes}

\subsection{gfortran}

gfortran is probably the best-supported compiler for \maestro.  Here are some
version-specific notes:

\begin{itemize}

\item {\em gfortran 4.8.x}: This typically works well, but sometimes we get
an error allocating memory in {\tt cluster\_f.f90}.  This
is a compiler bug (affecting atleast 4.8.2 and 4.8.3):

The code runs without any problem if it is compiled with {\tt -O2
-ftree-vectorize -fno-range-check} (our default) but with {\tt
cluster\_f.f90} compiled with {\tt -O2 -ftree-vectorize
-fno-range-check -fno-tree-pre}.  The ``{\tt fno-tree-pre}'' option
turns off ``{\tt ftree-pre}'' that is turned on by ``{\tt O2}''
 
GCC manual says,

\begin{quote}
{\tt -ftree-pre} \newline
Perform partial redundancy elimination (PRE) on trees. This flag is enabled by default at {\tt -O2} and {\tt -O3}.
\end{quote}

gfortran 4.8.5 appears to work without issues


\item {\em gfortran 5.1.1}: These compilers have no known issues.

\item {\em gfortran 5.3.x}: These compilers have no known issues.

\item {\em gfortran 6.1.1}: These compilers have a bug in producing
  optimized code for \boxlib's {\tt init\_layout\_tilearray}.  At the
  moment, this is worked around in \boxlib\ by changing the routine
  from a function to a subroutine.

\item {\em gfortran 6.2}: These compilers work without any known issues.

  gfortran 6.2.1 is used for nightly regression testing.

\end{itemize}

\subsection{PGI compilers}

The \boxlib\ floating point exception trapping is disabled with PGI
compilers earlier than version 16, due to problems with PGI using the
system header files.  From version 16 onward, things should work.

There are no known issues with PGI 16.5 compilers---these are used
for nightly regression testing.


\section{Working at OLCF (ORNL)}

\subsection{Titan Compilers}

The preferred compilers on Titan are the Cray compilers.  
Cray 8.4.0 works well on titan/OLCF with MPI/OpenMP.


\subsection{Monitoring Allocations}

The {\tt showusage} and {\tt showusage -f} commands give an
overview of the usage.

\subsection{Automatic Restarting and Archiving of Data}

The submission script {\tt titan.run} and shell script
{\tt process.titan} in {\tt Util/job\_scripts/titan/}
are designed to allow you to run \maestro\ with minimal interaction,
while being assured that the data is archived to HPSS on the OLCF
machines.

To use the scripts, first create a directory in HPSS that has the same
name as the directory on lustre you are running in (just the directory
name, not the full path).  E.g.\ if you are running in a directory
call {\tt wdconvect\_run}, then do:
\begin{verbatim}
hsi
mkdir wdconvect_run
\end{verbatim}
(Note: if the hsi command prompts you for your password, you will need
to talk to the OLCF help desk to ask for password-less access to
HPSS).

The script {\tt process.titan} is called from {\tt titan.run} and will
run in the background and continually wait until checkpoint or
plotfiles are created (actually, it always leaves the most recent one
alone, since data may still be written to it, so it waits until there
are more than 1 in the directory).

Then the script will use {\tt htar} to archive the plotfiles and
checkpoints to HPSS.  If the {\tt htar} command was successful, then
the plotfiles are copied into a {\tt plotfile/} subdirectory.  This is
actually important, since you don't want to try archiving the data a
second time and overwriting the stored copy, especially if a purge
took place.

Additionally, if the {\tt ftime} executable is in your path ({\tt
ftime.f90} lives in {\tt BoxLib/Tools/Postprocessing/F\_src/}), then
the script will create a file called {\tt ftime.out} that lists the
name of the plotfile and the corresponding simulation time.

Finally, right when the job is submitted, the script will tar up all
of the diagnostic files created by {\tt diag.f90} and {\tt ftime.out}
and archive them on HPSS.  The {\tt .tar} file is given a name that
contains the date-string to allow multiple archives to co-exist.

The {\tt titan.run} submission script has code in it that will look at
the most recently generated checkpoint files, make sure that they were
written out correctly (it looks to see if there is a Header file,
since that is the last thing written), and automatically set the {\tt
--restart} flag on the run line to restart from the most recent
checkpoint file.  This allows you to job-chain a bunch of submission
and have them wait until the previous job finished and then
automatically queue up:
\begin{verbatim}
qsub -W depend=afterany:<JOB-ID>  <QSUB SCRIPT>
\end{verbatim}
where {\tt <JOB-ID>} is the id number of the job that must complete
before the new submission runs and {\tt QSUB SCRIPT} is the submission
script (e.g.\ {\tt titan.run}).  This way you can queue up a bunch of
runs and literally leave things alone and it will restart from the
right place automatically and store the data as it is generated.

When {\tt process.titan} is running, it creates a lockfile (called
{\tt process.pid}) that ensures that only one instance of the script
is running at any one time.  Sometimes if the machine crashes, the
{\tt process.pid} file will be left behind, in which case, the script
aborts.  Just delete that if you know the script is not running.

The {\tt chainsub.sh} script can be used to automatically launch a
number of jobs depending on a single, currently queued (or running)
job.

\subsection{Profiling and Debugging on GPUs}
To get an idea of how code performs on Titan's GPUs, there are a few tools
available.  We'll overview a few here.

\subsubsection{{\tt Score-P} with {\tt CUBE} and {\tt vampir}}

{\tt Score-P} is a profiling and tracing tool that can be used to instrument
code to generate data for other tools to analyze, such as {\tt CUBE} and 
{\tt vampir}.  These tools have been developed to analyze performance of HPC
codes that run on several nodes, not specifically for analyzing GPU usage.
Still, they do support some GPU analysis.  In the next section, we'll discuss
NVIDA's tools specifically for analyzing GPU usage.
At the time of writing, {\tt Score-P} usage is fairly well documented on OLCF's
website here: \url{https://www.olcf.ornl.gov/kb_articles/software-scorep/}.
We'll review the essentials here, but please see the link for more details.

To instrument a code with {\tt Score-P} you must re-compile.  First, your
desired modules will need to be loaded.  Please note that \emph{order is
important} --- you need to load modules needed for compilation before loading 
{\tt Score-P}.  The {\tt Score-P} module is designed to detect the loaded
environment and will configure itself based on that.  These tools have been
tested with the PGI 16.3.0 compilers, and we will use them in our examples.
One possible set of module loads is

\begin{lstlisting}[language=bash,mathescape=false]
$ module load PrgEnv-pgi
$ module swap pgi/15.7.0 pgi/16.3.0
$ module load cudatoolkit
$ module load scorep/3.0-rc1
\end{lstlisting}

In the above we've loaded version 3.0, release candidate 1, which added some
support for analyzing OpenACC code.  The next step is to compile.  You
essentially preface your normal compile (and link) line with the {\tt Score-P}
executable and options.  As an example using the Fortran wrapper required on
Titan, we have

\begin{lstlisting}[language=bash,mathescape=false]
$ scorep --cuda --openacc -v ftn gpuprogram.f90
\end{lstlisting}

One way to achieve this in \maestro\ is to modify the appropriate make file.  For
PGI, this would be {\tt \$BOXLIB\_HOME/Tools/F\_mk/comps/Linux\_pgi.mak}.  If
this proves useful, it may be worth it to build {\tt Score-P} into the build
infrastructure.

Once compiled, we are ready to generate profiling and tracing data.  Among those
that develop these tools, note that they draw a distinction between profiling
and tracing.  Profiling generates a timing (or perhaps other metric) summary of
the entire program's execution while tracing produces a timeline of the
execution. {\tt Score-P}'s analysis is configured with environment variables.
Some of the key configuration variables used in testing include

\begin{lstlisting}[language=bash,mathescape=false]
export SCOREP_ENABLE_PROFILING=yes
export SCOREP_ENABLE_TRACING=yes
export SCOREP_EXPERIMENT_DIRECTORY=profile-trace
export SCOREP_CUDA_ENABLE=yes,kernel_counter,flushatexit
export SCOREP_CUDA_BUFFER=200M
export SCOREP_TOTAL_MEMORY=1G
export SCOREP_OPENACC_ENABLE=yes
\end{lstlisting}

For a full listing and definition of configuration variables, execute
\begin{lstlisting}[language=bash,mathescape=false]
$ scorep-info config-vars --full 
\end{lstlisting}

Except for very simple codes, you will never want to enable both tracing and
profiling.  The overhead is too high, and the code will likely crash or be
excessively slow.  Typically, it's best to profile first and then trace.  The
profiling data can be used to help configure tracing (as we'll see shortly).

Once the configuration is set, simply run the code as you normally would.
Experience suggests you will need to load the same modules that were loaded for
compilation when executing.  If analysis is being done through a batch script,
note that you cannot do a simple {\tt module load ...} in the script.  First you
need to do {\tt source /opt/modules/default/init/bash} in the script, and then
module loads will work as usual.

After executing, analysis data will be stored in the specified
{\tt SCOREP\_EXPERIMENT\_DIRECTORY}.  With profiling, you
will see a file like {\tt profile.cubex}.  This can be opened with cube 
({\tt module load cube}).  

As mentioned, the profiling data can be used to get recommended settings for
tracing.  Running
\begin{lstlisting}[language=bash,mathescape=false]
$ scorep-score -r profile.cubex
\end{lstlisting}
will yield output showing estimated sizes for e.g.~{\tt SCOREP\_TOTAL\_MEMORY}.
It also list functions that are called many times.  If you don't care about
them and they're slowing {\tt Score-P} down (or making an outrageously large output
file), you can configure {\tt Score-P} to ignore them in its analysis.  To filter a set
of functions, you need to provide a filter file, for example
\begin{lstlisting}[language=bash,mathescape=false]
$ export SCOREP_FILTERING_FILE=scorep.filter
\end{lstlisting}
where
\begin{lstlisting}[language=bash,mathescape=false]
$ cat scorep.filter
SCOREP_REGION_NAMES_BEGIN
 EXCLUDE
   matmul_sub
   matvec_sub
SCOREP_REGION_NAMES_END
\end{lstlisting}
This would tell {\tt Score-P} not to trace the routines {\tt matmul\_sub} and 
{\tt matvec\_sub}.  See the OLCF KnowledgeBase article and/or {\tt Score-P}'s
help for more, but this doesn't seem to be the best-documented aspect of the
program.

Running with tracing enabled will generate a traces.otf2 file that can be
inspected with vampir (module load vampir)

\subsubsection{{\tt nvprof} and {\tt nvvp}}

NVIDIA provides tools for specifically analyzing how your code utilizes their
GPUs.  {\tt Score-P} is a fully-featured profiler with some CUDA and OpenACC
support.  It can be useful for providing context for GPU execution and it allows
you to, for example, see line numbers for OpenACC directives that are executed.
{\tt nvprof} will only analyze GPU execution, but in exchange you get much more
detail than is available with {\tt Score-P}.  {\tt nvvp} is NVIDIA's visual
profiler.  It can be used to read data generated by {\tt nvprof}.  Most useful
is the guided analysis it will perform, which analyzes your code's GPU
performance for bottlenecks and suggests ways to improve performance.  Both are
provided when you load the {\tt cudatoolkit} module.

With {\tt nvprof}, no instrumentation is necessary.  Instead, you compile
normally and then run {\tt nvprof} on the executable.  As before, be sure when
executing to load the modules used at compile-time.  Executing {\tt nvprof} on
Titan's compute nodes requires some unexpected options having to do with how
{\tt aprun} and {\tt nvprof} interact.

To get a basic overview printed to the terminal on Titan's compute node, execute
\begin{lstlisting}[language=bash,mathescape=false]
$ aprun -b nvprof --profile-child-processes ./gpuprogram.exe arg1 arg2... 
\end{lstlisting}

To generate tracing data for {\tt nvvp}, execute
\begin{lstlisting}[language=bash,mathescape=false]
$ aprun -b nvprof --profile-child-processes -o nvprof.timeline.out%p 
  ./gpuprogram.exe arg1 arg2... 
\end{lstlisting}
{\tt nvvp} can then be used to read {\tt nvprof.timeline.out\%p}, where the 
{\tt \%p} will be replaced with the process ID.  You \emph{must} include \%p in
the output file's name or the code will crash, even if you're not running a
multi-process code.

To generate profile-like metric data for {\tt nvvp}, execute
\begin{lstlisting}[language=bash,mathescape=false]
$ aprun -b nvprof --profile-child-processes --analysis-metrics 
  -o nvprof.metrics.out%p ./gpuprogram.exe arg1 arg2... 
\end{lstlisting}
This is the output needed for {\tt nvvp}'s guided analysis.

\subsubsection{Target Metrics}
The output from profilers may be difficult to makes sense of.  The purpose of
this section is to note different metrics and reasonable targets for them.
Note that these may be specific to the k20x hardware in Titan.
\begin{itemize}
   \item Threads per block: 256-512.  Note that if your code requires many
      registers per thread, then this will limit the number of threads per block.
   \item Occupancy: 60\% is a reasonable target.  We have had success with codes
      even achieving only 23\% occupancy.  
\end{itemize}

One very useful tool for determining target metrics and what is limiting your
performance is a spreadsheet developed by NVIDIA to calculate occupancy.  Every
installation of the CUDA Toolkit should have this occupancy calculator in a
tools subdirectory.  At time of writing, the calculator is also available at
this link:
\url{http://developer.download.nvidia.com/compute/cuda/CUDA_Occupancy_calculator.xls}.
The document is actually  more than a simple calculator.  It contains quite a
bit of interesting insight into optimizing a GPU code.  More on occupancy can be
found here:
\url{http://docs.nvidia.com/cuda/cuda-c-best-practices-guide/index.html#occupancy}.

\subsection{Batch Submission of \yt\ Visualization Scripts}

\subsubsection{Rhea---preferred method}

{\color{red} this section needs to be updated.  See the titan section}

The best way to do visualization is to use rhea, the OLCF vis machine.
You need to build \yt\ via the {\tt install\_script.sh} script {\em on
rhea}.  It also must be on a {\em Lustre filesystem}, so it is seen by
the compute node.  It is best to build it in your {\tt \$PROJWORK} directory,
since that has a longer time between purges.  Once the installation is
complete, it will tell you what script to source to define the
necessary paths.

The scripts in {\tt MAESTRO/Util/job\_scripts/rhea/} will handle the
visualization.  On rhea, the job script gives you access to the
compute node, and then you can run serial jobs or a parallel job with
{\tt mpirun}.  The {\tt process-rhea.run} script will request the
resources and run the {\tt parallel-yt-rhea} script.  {\tt
parallel-yt-rhea} will launch the visualization process (defined
via the variables at the top of the script) on all the plotfiles
found to match the prefix defined in the script.  Several serial
jobs are run together, with the script using lock files to keep track
of how many processors are in use.  When processors free, the next
file in the list is processed, and so on, until there are no more
plotfiles left to process.  If a {\tt .png} file matching the
plotfile prefix is found, then that plotfile is skipped.

Note: the line in {\tt parallel-yt-rhea} that {\tt source}s the \yt\
{\tt activate} script may need to be modified to point to the
correct \yt\ installation path.


\subsubsection{Titan}

You can run \yt\ python scripts in the titan batch queues to do your
visualization.  You need to install \yt\ and all its dependencies
manually somewhere on a {\em Lustre filesystem}---this ensures that the
compute nodes can see it.  A good choice is the project space, since
that has a longer purge window.  The following procedure will setup
the development version of \yt\ (from source)
\begin{itemize}
\item create a directory in your {\tt \$PROJWORK} directory named {\tt yt/}

\item in {\tt yt/}, down load the \yt\ install script:
\begin{verbatim}
wget https://bitbucket.org/yt_analysis/yt/raw/yt/doc/install_script.sh
\end{verbatim}

\item edit the script to use Conda to get the necessary dependencies
and to build \yt\ from source.  This is accomplished by setting the 
following variables near the top of the script:
\begin{verbatim}
INST_CONDA=1
INST_YT_SOURCE=1
\end{verbatim}

\item run the script:
\begin{verbatim}
source install_script.sh
\end{verbatim}
\end{itemize}

When the script is done, you will have a new python installation in a sub-directory
called {\tt yt-conda/} and the script will tell you how to modify your path
in your {\tt .bashrc}

{\bfseries Important: make sure that you are not loading any other
  python environments in your {\tt .bashrc}, e.g., via modules.}

To test thing out, start up the python shell, and try doing {\tt
  import yt}.  If there are no errors, then you are good.

The python code {\tt vol.py} and submission script {\tt
  yt-vis-titan.run} in {\tt MAESTRO/Util/job\_scripts/titan/vis/} show
how to do a simple visualization in the batch queue using yt.  Note
that {\tt vol.py} is executable, and that we run it via {\tt aprun} to
ensure that it is launched on the compute node.

The scripts {\tt vis-titan.run} and {\tt parallel-yt-new} in that
same directorywill manage the \yt\ jobs by calling a
python script for each file that matches a pattern.  
Note that the actual visualization command itself is launched by
{\tt parallel-yt-new}, again using the {\tt aprun} command.  But
{\tt aprun} can only launch a single job at a time, so this means
we cannot easily do (trivally) parallel visualization on a node.  For
this reason, running on rhea is preferred.


\subsection{Remote \visit\ Visualization on Lens}

{\em Note: this information may be out-of-date.  It is recommended that
 \yt\ be used instead.}

For large data files, visualization with \visit\ should be done with
a client running on your local machine interfacing with \visit\ running
on the remote machine.  For the {\tt lens} machine at NCCS, the proper setup
is described below.

First, on {\tt lens}, in your {\tt .bashrc}, add:
\begin{verbatim}
export PATH="/sw/analysis-x64/visit/bin/":$PATH
\end{verbatim}
(you would think that you could just add {\tt module load visit} but this
does not seem to work with \visit.

On your local machine, launch \visit.  Note: this procedure seems to
work with \visit~2.4.2, but not \visit~2.5.0 for some reason.
\begin{itemize}
\item First we setup a new host

  \begin{itemize}
  \item From the menu, select {\em options $\rightarrow$ host profiles}
  \item Create a new host by clicking on the {\em New Host} button.
  \item Enter the {\em Host nickname} as {\tt lens}
  \item Enter the {\em Remote host name} as {\tt lens.ccs.ornl.gov}
  \item Enter the {\em Path to Visit installation} as {\tt /sw/analysis-x64/visit} (not sure if this is needed)
  \item Make sure that your {\em username} is correct
  \item Check {\em Tunnel data connections through SSH}
  \end{itemize}

\item Setup the {\em Launch Profiles}
  \begin{itemize}
  \item Click on the {\em Launch Profiles} tab
  \item Click on the {\em New Profile} button
  \item Enter the {\em Profile name} as {\tt parallel}
  \item Click on the {\em Parallel} tab
  \item Check {\em Launch parallel engine}
  \item Select the {\em Parallel launch method} as {\tt qsub/mpirun}
  \item Set the {\em Partition / Pool / Queue} to {\tt computation}
  \item Change the {\em Default number of processors} to 8
  \item Set the {\em Default number of nodes} to 2
  \item Set the {\em Default Bank / Account} to {\tt AST006}
  \item Set the {\em Default Time Limit} to {\tt 00:30:00}
  \end{itemize}

\item Click on {\em Apply} and {\em Post}

\item Save your changes by selecting {\em Options $\rightarrow$ Save Settings}
\end{itemize}

To do remote visualization, select {\em File $\rightarrow$ Open}.
From the drop down list at the top, select {\tt lens}.  You will be
prompted for your password.  After that, you can navigate to the
directory on lens with the data.

To make a movie (output sequence of images):
\begin{itemize}
\item save a view in \visit\ you like as a session file (File $\rightarrow$ Save session).  
\item On lens, create a file called {\tt files.visit} which lists all
  of the files you want to visualize, one per line, with {\tt /Header}
  after the filename.  This can be done simply as:
  \begin{verbatim}
  ls -1 | grep -v processed | awk '{print $1"/Header"}' > files.visit
  \end{verbatim}
  %$
  (note: the {\tt processed} bit is for when you used the script above to 
  automatically archive the data).

\item Edit the session file, searching for the name of the plotfile you
  originally visualized, and change it to read {\tt files.visit}.  Make
  sure that the path is correct.  This may appear multiple times.

\item Relaunch \visit\ locally and restore the session (File $\rightarrow$ Restore session).  It will render the first image.  Then reopen (File $\rightarrow$ ReOpen file).  After this is done, the buttons that allow you to move through the files should become active (black).

\item Resave the session file

\item To generate the frames, you have 2 options:

  \begin{enumerate}
  \item File $\rightarrow$ Save movie.  Pick {\em New simple movie},
    then set the format to {\em PNG} and add this to the output box by
    clicking the right arrow, then in the very last screen, select:
    {Later, tell me the command to run}.

   \visit\ will pop up a box showing the command to run.  Trying to
   get the currently running session of \visit\ to generate the frames
   seems problamatic.  Note: you will probably want to edit out the
   {\tt -v x.x.x} argument in the commandline to not have it force
   to use a specific version.

  \item If the session file successfully does the remote visualization
   as desired, you can run the movie via the commandline with something like:

   \begin{verbatim}
   visit -movie -format png -geometry 1080x1080 -output subchandra_cutoff3_ \
       -start 0 -end 44 -sessionfile subchandra_radvel.session
   \end{verbatim}

  \end{enumerate}

\end{itemize}


\section{Working at NERSC}

\subsection{edison compilers}

The default compilers on edison are the Intel compilers, but
PGI and Cray also work well

\begin{itemize}
\item Intel 15.0.1 works well on edison/NERSC with MPI/OpenMP

\item Intel 16.0.0 and 16.0.1 have issues compiling the linear solvers
  in \boxlib\ and should be avoided.

  Intel 16.0.2 works fine.

\item Cray 8.4.x has worked in the past, but it has not been used
  at NERSC in a while.

\end{itemize}

Note: in order to compile, you will need to ensure that both the {\tt
python} and {\tt python\_base} modules are loaded (via the {\tt
module} command).  \MarginPar{this may have changed with the migration of NERSC to anaconda?}

\subsection{Running Jobs}

edison is configured with 24 cores per node split between two Intel
IvyBridge 12-core processors.  Each processor connects to 1/2 of the
node's memory and is called a NUMA node, so there are 2 NUMA nodes per
edison node.  Best performance is seen when running with 6 or 12 threads.

Note: edison switched to SLURM as the batch system.  Your job is submitted
using the {\tt sbatch} command.  Options to {\tt sbatch} are specified at the
top of your submission script with {\tt \#SBATCH} as the prefix.  These options
can be found on the {\tt sbatch manpage}.  For instance, 
\begin{verbatim}
#SBATCH -N 2
#SBATCH -J myjob
#SBATCH -A repo-name
#SBATCH -p regular
#SBATCH -t 12:00:00
\end{verbatim}
will request 2 nodes ({\tt -N}), under the account {\tt repo-name} ({\tt -J}),
in the regular queue, and for a 12-hour window {\tt -t}.

If you are using OpenMP, then your script should set {\tt OMP\_NUM\_THREADS}, e.g.,
\begin{verbatim}
export OMP_NUM_THREADS=12
\end{verbatim}

By default, SLURM will change directory into the submission directory.  The 
job is launched from your script using {\tt srun}, e.g.:
\begin{verbatim}
srun -n 48 ./main.Linux.Cray.mpi.exe inputs_3d
\end{verbatim}
to run 48 MPI tasks (across the 2 nodes), or 
\begin{verbatim}
export OMP_NUM_THREADS=6
srun -n 8 -c 6 ./main.Linux.Cray.mpi.omp.exe inputs_3d
\end{verbatim}
to use 8 MPI tasks each with 6 threads.

The scripts in {\tt Util/job\_scripts/edison/} provides some examples.

To chain jobs, such that one queues up after the previous job finished,
use the {\tt chainslurm.sh} script in that same directory. You can view
the job dependency using:
\begin{verbatim}
squeue -l -j job-id
\end{verbatim}
where {\tt job-id} is the number of the job.

Jobs are submitted with {\tt sbatch}.  A job can be canceled using
{\tt scancel}, and the status can be checked using {\tt squeue -u {\em
username}}.

\subsection{Automatic Restarting and Archiving of Data}

The same set of submission scripts described for titan are available
for edison at NERSC in {\tt Util/job\_scripts/edison/}.  In particular,
the job submission script will set the restart command line parameters
to restart from the most recent checkpoint file in the output directory.

Note: NERSC does not allow for the {\tt process} script that archives
to HPSS to run in the main job submission script.  Instead, a separate
job needs to be run in the ``{\tt xfer}'' queue.  The script {\tt edison.xfer.slurm} 
in {\tt Util/job\_scripts/edison/} shows how this works.  

Jobs in the {\tt xfer} queue start up quickly.  The best approach is
to start one as you start your main job (or make it dependent on the
main job).  The sample {\tt process.xrb} script will wait for output
and then archive it as it is produced, using the techniques described
for titan above.

To check the status of a job in the {\tt xfer} queue, use:
\begin{verbatim}
squeue -u username -M all
\end{verbatim}


\subsection{Batch visualization using \yt}

\yt\ can be built using the {\tt install\_script.sh}.  It has been
tested using the build of {\tt yt} from source and dependencies via conda,
by setting:
\begin{verbatim}
INST_CONDA=1
INST_YT_SOURCE=1
\end{verbatim}
in the {\tt install\_script.sh}.  Once these are set, run:
\begin{verbatim}
source install_script.sh
\end{verbatim}
Note: installation was done in the home directory.

This way of building \yt\ installs it's own python and support
libraries in a directory, {\tt yt-conda}.  {\bf Important: } You need
to make sure that your start-up files (typically {\tt .bashrc.ext} at
NERSC) don't {\tt module load} python or any python libraries, as this
will interfere with the conda installation.  The install script will
direct you to add the install location to your path.


The scripts {\tt parallel-yt} and {\tt process-edison.slurm} in {\tt
  Util/job\_scripts/edison} show how to invoke \yt\ to loop over a
series of plotfiles and do visualization.  A number of tasks are run
at once on the node, each operating on a separate file.  The {\tt
  parallel-yt} script then calls {\tt vol.py} to do the volume
rendering with \yt.  Note: it is important that {\tt srun} be used to
launch the \yt\ script to ensure that it is run on the compute node.

A simple {\tt test-yt.slurm} script shows how to just call the
\yt\ python script directly, using one node and 24 threads, again
using {\tt srun} to execute on the compute node.

If you want to keep up with the development version of \yt, then you
can update the source in {\tt yt-conda/bin/src/yt-hg}, using:
\begin{verbatim}
hg pull
hg update yt
\end{verbatim}
and then rebuild it via:
\begin{verbatim}
python setup.py develop --user
\end{verbatim}


\subsection{Using the {\tt AmrPostprocesing} python plotting scripts on hopper}

To build the {\tt fsnapshot.so} library, you need to do:
\begin{verbatim}
module load gcc
\end{verbatim}
{\tt f2py} is already in the path, so the library should then build without issue.
%

Then edit your {\tt .bashrc.ext} file to set the {\tt PYTHONPATH} to
the {\tt python\_plotfile} directory, e.g.:
\begin{verbatim}
export PYTHONPATH="/global/homes/z/zingale/AmrPostprocessing/python"
\end{verbatim}
%
and set the {\tt PATH} to that directory,
\begin{verbatim}
export PATH="/global/homes/z/zingale/AmrPostprocessing/python:$PATH"
\end{verbatim}

To run the script, you need to do:
\begin{verbatim}
module load matplotlib
module load python
\end{verbatim}



\subsection{Remote visualization on hopper}

\visit\ is already configured to work with hopper.  If the host does not appear
in your local version of visit, copy the {\tt host\_nersc\_hopper.xml} file
from the {\tt .visit/allhosts/} directory under the system's \visit\ install path
to your {\tt $\mathtt{\sim}$/.visit/hosts/} directory. 



\section{Working at NCSA (Blue Waters)}

\subsection{Overview}

Blue Waters consists of 22,640 Cray XE6 compute nodes and 4,224
Cray XK7 compute nodes.

Each XE node has two AMD Interlagos model 6276 compute units, each of
which has 16 integer cores (thus, a single node has a total of 32 integer
cores).  Two integer cores share a multithreaded, 256-bit wide floating 
point unit (FPU).  If both integer cores have their own thread, each has access 
to 128-bit floating point processing, whereas if only one thread is 
assigned the process can access all 256 bits.  In one major science
application on Blue Waters it was found that having an OpenMP thread for
each integer core gave the best performance, but when starting a new
application it's best to experiment.  One OpenMP thread per FPU may
be better in some cases.

Each compute unit is divided into two NUMA nodes.  Cores in
the same NUMA region share a pool of L3 cache.  For the same science
application as before it was found that the best performance was achieved
by assigning an MPI task to each NUMA node.  Thus, each physical node
has four MPI tasks.

The XK nodes consist of one AMD Interlagos model 6276 compute unit
and an NVIDIA GK110 ``Kepler'' GPU accelerator (Tesla K20X).  The
GPU is configured with 14 streaming multiprocessor units (SMXs), each
of which has 192 single-precision or 64 double-precision CUDA cores.  Thus
there are a total of 2688 SP CUDA cores or 896 DP CUDA cores.

For more details, please see 
\url{https://bluewaters.ncsa.illinois.edu/user-guide}

\subsection{BW Compilers}

The Cray compilers are the default on blue waters, and version
8.3.3 works well with \maestro.

\subsection{Monitoring Allocations}

The {\tt usage} command will list the current user's usage and 
{\tt usage -P {\em project}} will
list the usage for all users in a project allocation named ``project''.
