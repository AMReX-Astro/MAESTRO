\documentclass[11pt]{book}    % add ``openany'' option to remove blank pages

\tolerance=600

% longtable package used to split tables across pages
\usepackage{longtable}

% PDF-aware landscape package, used for rotating tables (including the
% longtable)
\usepackage{pdflscape}


% include figures
\usepackage{epsfig}
%\usepackage{epsf}

% AMS symbols
\usepackage{amsmath,amssymb}

% Palatino font (and math symbols) -- looks nicer than the standard
% LaTeX font
\usepackage{mathpazo}

% chapter title styles
%\usepackage[Glenn]{fncychap}
%\ChNameVar{\bfseries\Large}

\usepackage[Sonny]{fncychap}
\ChNameVar{\LARGE}
\ChTitleVar{\LARGE\sl}

% part page style
% see http://tex.stackexchange.com/questions/6609/problems-with-part-labels-using-titlesec
\usepackage{titlesec}

\titleformat{\part}[display]
   {\Huge\filcenter}
   {{\partname{}} \thepart}
   {0em}
   {\hrule}


% hyperlinks -- load after fncychap
\usepackage{hyperref}


% color package
\usepackage[usenames]{color}

% table coloring
\usepackage{colortbl}
\definecolor{tableShade}{rgb}{0.945,0.961,0.980}


% special colors and definitions for framing source code listings
\usepackage{listings}

\definecolor{AntiqueWhite3}{rgb}{0.804,0.753,0.69}
\definecolor{DarkerAntiqueWhite3}{rgb}{0.659,0.635,0.612}
\definecolor{orange}{rgb}{1.0,0.65,0.0}

\lstset{%
  keywordstyle=\color{blue}\ttfamily,%
  commentstyle=\itshape\color[gray]{0.5},%
  mathescape=true,%
  basicstyle=\small\ttfamily,%
  %frameround=fttt,%
  frameround=ffff,%
  %frame=shadowbox,%
  frame=single,%
  rulesepcolor=\color{DarkerAntiqueWhite3},%
  backgroundcolor=\color{AntiqueWhite3},%
  emph={load,add_slice,save}, emphstyle=\color{blue},%
  emph={[2]In}, emphstyle=[2]\color{yellow},%
  emph={[3]Out}, emphstyle=[3]\color{orange},%
  xleftmargin=1em,
  xrightmargin=1em,
  mathescape=false}


% Margins
\usepackage[top=1.0in,bottom=1.0in,inner=0.85in,outer=0.85in]{geometry}

% macros for standard maestro quantities
\input maestrosymbols

\setlength{\marginparwidth}{0.8in}
\newcommand{\MarginPar}[1]{\marginpar{%
\vskip-\baselineskip %raise the marginpar a bit
\raggedright\tiny\sffamily
\hrule\smallskip{\color{red}#1}\par\smallskip\hrule}}


% custom hrule for title page
\newcommand{\HRule}{\rule{\linewidth}{0.125mm}}




% spacing in the table of contents
\usepackage[titles]{tocloft}

\setlength{\cftbeforechapskip}{2ex}
\setlength{\cftbeforesecskip}{0.25ex}

% short table of contents
\usepackage{shorttoc}


% don't put a header on blank pages, see
% http://www.latex-community.org/forum/viewtopic.php?f=4&p=51559
% change ``plain'' to ``empty'' to eliminate the page number
\makeatletter
\renewcommand*\cleardoublepage{\clearpage\if@twoside
\ifodd\c@page\else
\hbox{}
\thispagestyle{empty}
\newpage
\if@twocolumn\hbox{}\newpage\fi\fi\fi}
\makeatother


% don't make the chapter/section headings uppercase.  See the fancyhdr
% documentation (section 9)
\usepackage{fancyhdr}
\renewcommand{\chaptermark}[1]{%
 \markboth{\chaptername
\ \thechapter.\ #1}{}}

\renewcommand{\sectionmark}[1]{\markright{\thesection---#1}}


% skip a bit of space between paragraphs, to enhance readability
\usepackage{parskip}



% figure paths

\newcommand{\figpath}{./initial_models}
\newcommand{\lodensfigpath}{./lo_density}
\newcommand{\flowfigpath}{./flowchart}
\newcommand{\sdcfigpath}{./SDC}
\newcommand{\radbasefigpath}{./radial_base}
\newcommand{\sphericalfigpath}{./spherical_basestate}
\newcommand{\archfigpath}{./architecture}
\newcommand{\unitfigpath}{./unit_tests}
\newcommand{\rotfigpath}{./rotation}
\newcommand{\visfigpath}{./visualization}
\newcommand{\planeinvsqfigpath}{./planar_invsq_base}
\newcommand{\pertfigpath}{./pert_notes}
\newcommand{\suitefigpath}{./test_suite}
\newcommand{\gsfigpath}{./getting_started}
\newcommand{\mgfigpath}{./mg}
\newcommand{\mesafigpath}{./MESA}


%-----------------------------------------------------------------------------
\begin{document}

\frontmatter


\begin{titlepage}

\begin{center}

\ \\[2 in]
%{\Huge \sf MAESTRO} \\[0.25in]
{\sf \Huge MAESTRO} \\[0.25in]


\begin{minipage}{5.5in}
\HRule\\[2mm]
\centering
{\Large \em An adaptive low-Mach number hydrodynamics code\\ for stratified astrophysical flows}


\HRule
\end{minipage}

\ \\[1 in]
{\sf \huge User's Guide}

\ \\[1 in]
{developed at \\[5mm] Center for Computational Sciences and Engineering / \\ Lawrence Berkeley National Laboratory \\[3mm]
  {\em \&}  \\[3mm]
        Stony Brook University}

\vfill

{\large \today}

\end{center}

\end{titlepage}


\shorttoc{Chapter Listing}{0}

\setcounter{tocdepth}{2}
\tableofcontents

\clearpage

\listoffigures
\addcontentsline{toc}{chapter}{list of figures}

\clearpage

\listoftables
\addcontentsline{toc}{chapter}{list of tables}


\clearpage

\chapter*{Preface}
\chaptermark{Preface}
\addcontentsline{toc}{chapter}{preface}
\input preface/preface

\clearpage

\chapter*{Getting Help}
\chaptermark{Getting Help}
\addcontentsline{toc}{chapter}{getting help}

Support for \maestro\ is available through the {\tt maestro-help}
mailing list.  To subscribe to the list or read through the 
archives, visit \\[3mm] 
\url{http://bender.astro.sunysb.edu/mailman/listinfo/maestro-help} \\[3mm]
%
Questions can be sent to \\[3mm]
\href{mailto:maestro-help@bender.astro.sunysb.edu}{maestro-help@bender.astro.sunysb.edu} 


\clearpage

\mainmatter


%-----------------------------------------------------------------------------
\part{Overview of the \maestro\ Algorithm}

\chapter{Introduction}
\input introduction/introduction

\chapter{Algorithm Flowchart}
\label{ch:flowchart}
\input flowchart/flowchart

%-----------------------------------------------------------------------------
\part{Using \maestro}

\chapter{Getting Started}
\input getting_started/getting_started

\chapter{Unit Tests}
\input unit_tests/unit_tests

\chapter{Architecture}
\input architecture/architecture

\chapter{Runtime Parameters}
\label{ch:runtimeparameters}
\input runtime_parameters/param_intro
\input runtime_parameters/runtime_parameters

\chapter{Initial Models}
\input initial_models/initial_models

\chapter{Visualization}
\input visualization/visualization

\chapter{Analysis Routines}
\input analysis/analysis

\chapter{Managing Jobs}
\input managing_jobs/managingjobs

\chapter{Build System}
\input makefiles/makefiles

\chapter{Compiler Lore}
\input compilers/compilers

\chapter{Test Suite}
\input test_suite/test_suite

\chapter{FAQ}
\input FAQ/faq

%-----------------------------------------------------------------------------
\part{\maestro\ Technical Details}

\chapter{Notes on the Low Density Parameters in \maestro}
\input lo_density/lo_density

\chapter{Notes on the Volume Discrepancy Term}
\input volume_discrepancy/volume_discrepancy

\chapter{EOS and Temperature Notes}
\input eos_notes/eos_notes

\chapter{Networks}
\input networks/networks

\chapter{MESA Microphysics}
\input MESA/instructions

\chapter{Notes on $\etarho$}
\input eta_notes/eta

\chapter{Notes on Prediction Types}
\input pert_notes/pert

\chapter{Notes on SDC}
\label{ch:sdc}
\input SDC/sdc

\chapter{Notes on Advection}
\input Godunov_notes/Godunov

\chapter{Multigrid}
\input mg/mg

\chapter{Rotation}
\input rotation/rotation

\chapter{Radial Base State}
\input radial_base/radial_base

\chapter{Spherical Expansion}
\input spherical_basestate/basestate

\chapter{Planar $1/r^2$ Gravity Basestate}
\input planar_invsq_base/basestate

\chapter{Thermodynamic Relations and Stability}
\input thermo_notes/thermo_notes

\chapter{Notes on $\beta_0$}
\input beta0/beta0

\chapter{Notes on Enthalpy}
\input enthalpy/enthalpy

\backmatter

%-----------------------------------------------------------------------------
\part{References}

\renewcommand\bibname{References}
\addcontentsline{toc}{chapter}{References}
\bibliographystyle{plain}
\bibliography{maestro_doc}

\end{document}
