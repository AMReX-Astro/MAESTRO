\section{The {\tt Postprocessing} Routines}
\label{sec:analysis}

The {\tt BoxLib/Tools/Postprocessing/F\_Src/} directory contains a large
number of Fortran-based analysis routines for BoxLib datasets.  Many
of these can be used with both \maestro\ and the compressible
astrophysics code, \castro.

To compile any of the individual routines, edit the {\tt GNUmakefile}
add uncomment the line beginning with `{\tt programs +=}' containing
the routine you want to build.

\subsection{General Analysis Routines}

The following routines are generally applicable for any BoxLib-based
plotfile.  Typing the executable names without any arguments will
provide usage examples.

\begin{itemize}

\item {\tt faverage.f90}

  Laterally average each of the variables in a plotfile (works for
  both 2-d and 3-d).  This is written with \maestro\
  plane-parallel geometry plotfiles in mind, and the averaging is done
  over the coordinate direction(s) perpendicular to gravity.


\item {\tt fboxinfo.f90}

  Print out some basic information about the number of boxes on each
  refinement level and (optionally) the bounds of each of the boxes.

\item {\tt fcompare.f90}

  Compare two plotfiles, zone-by-zone to machine precision, and report
  the L2-norm of the error (both absolute and relative) for each
  variable.  This assumes that the grids are identical. \\[-3mm]

  This is used by in the regression test suite in {\tt
  Parallel/util/regtests/}.


\item {\tt fextract.f90}

  Extract a 1-d line through a dataset (1-, 2-, or 3-d).  This works
  with both uniformly-gridded or AMR datasets.  For multi-dimensional
  datasets, the coordinate direction to extract along can be specified.
  The line is always taken through the center of the domain.  Either
  a single variable or all variables, along with the coordinate 
  information, are output to a file.
  

\item {\tt fextrema.f90}

  Report the min and max of each variable (or only a single variable)
  in one or more plotfiles.


\item {\tt fsnapshot2d.f90}, {\tt fsnapshot3d.f90}

  Create an image (PPM file) of a single variable in a plotfile.  For
  3-d, the slice plane through the center of the domain is specified.
  Separate routines exist for 2-d and 3-d datasets.
  

\item {\tt ftime.f90}
 
  For each plotfile, simply print the simulation time.


\item {\tt fvarnames.f90}

  Simply print out the list of variables stored in a plotfile.


\end{itemize}

\subsection{Data Processing Example}

The routine {\tt fspeciesmass2d.f90} in {\tt
  F\_src/tutorial} serves as a well-commented example of how
to work with \maestro\ plotfile data.  This routine simply computes
the total mass of a particular species on the entire domain for a 2-d
dataset.  It is written to understand a multilevel (AMR) dataset, and
only considers the finest-available data at any physical location in
the computational domain.

{\tt fspeciesmass2d.f90} should provide a good starting point for
writing a new analysis routine for BoxLib data.

\subsection{Particle routines}

\label{analysis:sec:particles}

The {\tt parseparticles.py} routine in the {\tt python/} subdirectory
can read in \maestro\ particle files containing particle histories
(usually named {\tt timestamp\_??}).  See the discussion in
\S~\ref{arch:sec:particles} for details on initializing particles in
\maestro.  The driver {\tt test\_parseparticles.py} shows shows how to
use this module to plot particle histories.  Additional documentation
is available from the module itself.  In the python environment,
type:
\begin{lstlisting}[language=Python]
import parseparticles
help(parseparticles)
\end{lstlisting}
to get information on the classes and functions provided by the {\tt
  parseparticles} module.

As a concrete example, running {\tt test2} with particles enabled
will seed particles in the initial hotspot.  To plot the results,
first set your {\tt PYTHONPATH} environment variable to point to the
{\tt AmrPostprocessing/python/} directory, for example:
\begin{verbatim}
export PYTHONPATH="/home/username/development/AmrPostprocessing/python"
\end{verbatim}
This will allow python to see the {\tt parseparticles.py} routine.
For the {\tt test2} problem, the {\tt plotparticles.py} routine shows
how to plot the particle histories and make an animation of the
particles colored by the ash mass fraction.  This script is run as:
\begin{verbatim}
./plotparticles.py timestamp_*
\end{verbatim}

Note, these python routines require the NumPy and matplotlib packages.
On a Fedora Linux system, the necessary routines can be installed via:
\begin{verbatim}
yum install python-matplotlib lyx-fonts stix-fonts
\end{verbatim}




