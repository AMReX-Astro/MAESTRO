These are working notes for the Godunov step in {\tt MAESTRO} 
and {\tt VARDEN}.

%-----------------------------------------------------------------------------
% Notation
%-----------------------------------------------------------------------------
\section{{\tt MAESTRO} Notation}
\begin{itemize}
\item For 2D, $\Ub = (u,w)$ and $\Ubt = (\ut,\wt)$.  
  Note that $u = \ut$.  We will use the shorthand $\ib = (x,r)$.
\item For 3D plane parallel, $\Ub = (u,v,w)$
  and $\Ubt = (\ut,\vt,\wt)$.  Note that $u = \ut$ and $v = \vt$.
  We will use the shorthand $\ib = (x,y,r)$.
\item For 3D spherical, $\Ub = (u,v,w)$
  and $\Ubt = (\ut,\vt,\wt)$.  We will use the shorthand 
  $\ib = (x,y,z)$.
\end{itemize}
\subsection{Computing $\Ub$ From $\Ubt$}
For plane-parallel problems, in order to compute $w$ from 
$\wt$, we use simple averaging
\begin{equation}
w_{\ib}^n = \wt_{\ib}^n + \frac{w_{0,r-\half} + w_{0,r+\half}}{2}.
\end{equation}
For spherial problems, in order to compute $\Ub$ from $\Ubt$, 
we first map $w_0$ onto $w_0^{\mac}$ using {\tt put\_w0\_on\_edges,}
 where $w_0^{\mac}$ only contains normal velocities at each face.  
Then we construct $\Ub$ by using
\begin{equation}
u_{\ib} = \ut_{\ib} + \frac{w_{0,\ib+\half\eb_x}^{\mac} + w_{0,\ib-\half\eb_x}^{\mac}}{2},
\end{equation}
\begin{equation}
v_{\ib} = \vt_{\ib} + \frac{w_{0,\ib+\half\eb_y}^{\mac} + w_{0,\ib-\half\eb_y}^{\mac}}{2},
\end{equation}
\begin{equation}
w_{\ib} = \wt_{\ib} + \frac{w_{0,\ib+\half\eb_z}^{\mac} + w_{0,\ib-\half\eb_z}^{\mac}}{2}.
\end{equation}
To compute full edge-state velocities, simply add $w_0$ 
(for plane-parallel) or {\tt w0mac} to the perturbational
velocity directly since only edge-based quantities are involved.
\subsection{Computing $\partial w_0/\partial r$}
For plane-parallel problems, the spatial derivatives of $w_0$ 
are given by the two-point centered difference:
\begin{equation}
\left(\frac{\partial w_0}{\partial r}\right)_{\ib} = \frac{w_{0,r+\half}-w_{0,r-\half}}{h}.
\end{equation}
For spherical problems, we compute the radial bin centered gradient using
\begin{equation}
\left(\frac{\partial w_0}{\partial r}\right)_{r} = \frac{w_{0,r+\half}-w_{0,r-\half}}{\Delta r}.
\end{equation}
Then we put $\partial w_0/\partial r$ onto a Cartesian grid 
using {\tt put\_1d\_array\_on\_cart\_3d\_sphr}.

\newpage

%-----------------------------------------------------------------------------
% utrans
%-----------------------------------------------------------------------------
\section{Computing $\Ubt^{\trans}$ in {\tt MAESTRO}}
In {\tt advance\_premac}, we call {\tt mkutrans}, to compute
$\Ubt^{\trans}$.  We will only compute the normal
component of velocity at each face.
These transverse velocities do not contain $w_0$, so immediately
following the call to {\tt mkutrans}, we call {\tt addw0} to compute
$\Ub^{\trans}$ from $\Ubt^{\trans}$.\\ \\
The evolution equation for the perturbational velocity is:
\begin{equation}
\frac{\partial\Ubt}{\partial t} = -\Ub\cdot\nabla\Ubt \underbrace{- (\Ubt\cdot\eb_r)\frac{\partial w_0}{\partial r}\eb_r - \frac{1}{\rho}\nabla\pi + \frac{1}{\rho_0}\frac{\partial\pi_0}{\partial r}\eb_r - \frac{(\rho-\rho_0)}{\rho}g\eb_r}_{\hbox{forcing terms}}.\label{Perturbational Velocity Equation}
\end{equation}
We extrapolate each velocity component to edge-centered, time-centered locations.  For example,
\begin{eqnarray}
\ut_{R,\ib-\half\eb_x} &=& \ut_{\ib}^n + \frac{h}{2}\frac{\partial\ut_{\ib}^n}{\partial x} + \frac{\dt}{2}\frac{\partial\ut_{\ib}^n}{\partial t} \nonumber \\
&=& \ut_{\ib}^n + \frac{h}{2}\frac{\partial\ut_{\ib}^n}{\partial x} + \frac{\dt}{2}
\left(-\ut_{\ib}^n\frac{\partial\ut_{\ib}^n}{\partial x} - \wt_{\ib}^n\frac{\partial\ut_{\ib}^n}{\partial r} + \text{forcing terms}\right)
\end{eqnarray}
We are going to use a 1D Taylor series extrapolation in space and time.
By 1D, we mean that we omit any spatial derivatives that are not in the 
direction of the extrapolation.  We also omit the underbraced forcing terms.
We also use characteristic tracing.
\begin{equation}
\ut_{R,\ib-\half\eb_x} = \ut_{\ib}^n + \left[\frac{1}{2} - \frac{\dt}{2h}\min(0,\ut_{\ib}^n)\right]\partial\ut_{\ib}^n
\end{equation}


\newpage

\subsection{2D Cartesian Case}
We predict $\ut$ to x-faces using a 1D extrapolation:
\begin{eqnarray}
\ut_{L,\ib-\half\eb_x} &=& \ut_{\ib-\eb_x}^n + \left[\half - \frac{\dt}{2h}\max(0,u_{\ib-\eb_x}^n)\right]\Delta_x \ut_{\ib-\eb_x}^n,\\
\ut_{R,\ib-\half\eb_x} &=& \ut_{\ib}^n - \left[\half + \frac{\dt}{2h}\min(0,u_{\ib}^n)\right]\Delta_x \ut_{\ib}^n.
\end{eqnarray}
We pick the final trans states using a Riemann solver:
\begin{equation}
\ut^{\trans}_{\ib-\half\eb_x} =
\begin{cases}
0, & \left(\ut_{L,\ib-\half\eb_x} \le 0 ~~ {\rm AND} ~~ \ut_{R,\ib-\half\eb_x} \ge 0\right) ~~ {\rm OR} ~~ \left|\ut_{L,\ib-\half\eb_x} + \ut_{R,\ib-\half\eb_x}\right| < \epsilon, \\
\ut_{L,\ib-\half\eb_x}, & \ut_{L,\ib-\half\eb_x} + \ut_{R,\ib-\half\eb_x} > 0, \\
\ut_{R,\ib-\half\eb_x}, & \ut_{L,\ib-\half\eb_x} + \ut_{R,\ib-\half\eb_x} < 0, \\
\end{cases}
\end{equation}
We predict $\wt$ to r-faces using a 1D extrapolation:
\begin{eqnarray}
\wt_{L,\ib-\half\eb_r} &=& \wt_{\ib-\eb_r}^n + \left[\half - \frac{\dt}{2h}\max(0,w_{\ib-\eb_r}^n)\right]\Delta_r \wt_{\ib-\eb_r}^n,\\
\wt_{R,\ib-\half\eb_r} &=& \wt_{\ib}^n - \left[\half + \frac{\dt}{2h}\min(0,w_{\ib}^n)\right]\Delta_r \wt_{\ib}^n.
\end{eqnarray}
We pick the final $\trans$ states using a Riemann solver, noting
that we upwind based on the full velocity.
\begin{equation}
\wt^{\trans}_{\ib-\half\eb_r} =
\begin{cases}
0, & \left(w_{L,\ib-\half\eb_r} \le 0 ~~ {\rm AND} ~~ w_{R,\ib-\half\eb_r} \ge 0\right) ~~ {\rm OR} ~~ \left|w_{L,\ib-\half\eb_r} + w_{R,\ib-\half\eb_r}\right| < \epsilon, \\
\wt_{L,\ib-\half\eb_r}, & w_{L,\ib-\half\eb_r} + w_{R,\ib-\half\eb_r} > 0, \\
\wt_{R,\ib-\half\eb_r}, & w_{L,\ib-\half\eb_r} + w_{R,\ib-\half\eb_r} < 0, \\
\end{cases}
\end{equation}

\newpage

\subsection{3D Cartesian Case}
We use the exact same procedure in 2D and 3D to compute $\ut^{\trans}$ and 
$\wt^{\trans}$. The procedure for computing $\vt^{\trans}$ is analogous to
computing $\ut^{\trans}$.  We predict $\vt$ to y-faces using the
1D extrapolation:
\begin{eqnarray}
\vt_{L,\ib-\half\eb_y} &=& \vt_{\ib-\eb_y}^n + \left[\half - \frac{\dt}{2h}\max(0,v_{\ib-\eb_y}^n)\right]\Delta_y \vt_{\ib-\eb_y}^n, \\
\vt_{R,\ib-\half\eb_y} &=& \vt_{\ib}^n - \left[\half + \frac{\dt}{2h}\min(0,v_{\ib}^n)\right]\Delta_y \vt_{\ib}^n,
\end{eqnarray}
\begin{equation}
\vt^{\trans}_{\ib-\half\eb_y} =
\begin{cases}
0, & \left(v_{L,\ib-\half\eb_y} \le 0 ~~ {\rm AND} ~~ v_{R,\ib-\half\eb_y} \ge 0\right) ~~ {\rm OR} ~~ \left|v_{L,\ib-\half\eb_y} + v_{R,\ib-\half\eb_y}\right| < \epsilon, \\
\vt_{L,\ib-\half\eb_y}, & v_{L,\ib-\half\eb_y} + v_{R,\ib-\half\eb_y} > 0, \\
\vt_{R,\ib-\half\eb_y}, & v_{L,\ib-\half\eb_y} + v_{R,\ib-\half\eb_y} < 0. \\
\end{cases}
\end{equation}

\newpage

\subsection{3D Spherical Case}
We predict the normal components of velocity to the normal faces 
using a 1D extrapolation.  The equations for all three directions 
are identical to those given in the 2D and 3D plane-parallel 
sections.  As in the plane-parallel case, make sure
 that the advection velocities, as well as 
the upwind velocity, is done with the full velocity, not the 
perturbational velocity.

\newpage

%-----------------------------------------------------------------------------
% Umac
%-----------------------------------------------------------------------------
\section{Computing $\Ubt^{\mac,*}$ in {\tt MAESTRO}}
In {\tt advance\_premac}, we call {\tt velpred} to compute
$\Ubt^{\mac,*}$.  We will only compute the normal component of
velocity at each face.\\ \\
For reference, here is the perturbational velocity equation from before:
\begin{equation}
\frac{\partial\Ubt}{\partial t} = -\Ub\cdot\nabla\Ubt \underbrace{- (\Ubt\cdot\eb_r)\frac{\partial w_0}{\partial r}\eb_r \underbrace{- \frac{1}{\rho}\nabla\pi + \frac{1}{\rho_0}\frac{\partial\pi_0}{\partial r}\eb_r - \frac{(\rho-\rho_0)}{\rho}g\eb_r}_{\hbox{terms included in $\fb_{\Ubt}$}}}_{\hbox{forcing terms}}.
\end{equation}
Note that the $\partial w_0/\partial r$ term is treated like a forcing
term, but it is not actually part of $\fb_{\Ubt}$.  We make use of the 1D
extrapolations used to compute $\Ubt^{\trans}$ 
($\ut_{L/R,\ib-\half\eb_x}$, $\vt_{L/R,\ib-\half\eb_y}$, 
and $\wt_{L/R,\ib-\half\eb_r}$), as well as the ``$\trans$'' states
($\ut_{\ib-\half\eb_x}^{\trans}$, $\vt_{\ib-\half\eb_y}^{\trans}$, 
and $\wt_{\ib-\half\eb_r}^{\trans}$)

\newpage

\subsection{2D Cartesian Case}
\begin{enumerate}
\item Predict $\ut$ to r-faces using a 1D extrapolation.
\item Predict $\ut$ to x-faces using a full-dimensional extrapolation.
\item Predict $\wt$ to x-faces using a 1D extrapolation.
\item Predict $\wt$ to r-faces using a full-dimensional extrapolation.
\end{enumerate}
Predict $\ut$ to r-faces using a 1D extrapolation:
\begin{eqnarray}
\ut_{L,\ib-\half\eb_r} &=& \ut_{\ib-\eb_r}^n + \left[\half - \frac{\dt}{2h}\max(0,w_{\ib-\eb_r}^n)\right]\Delta_r \ut_{\ib-\eb_r}^n, \\
\ut_{R,\ib-\half\eb_r} &=& \ut_{\ib} - \left[\half + \frac{\dt}{2h}\min(0,w_{\ib}^n)\right]\Delta_r \ut_{\ib}^n.
\end{eqnarray}
Upwind based on $w^{\trans}$:
\begin{equation}
\ut_{\ib-\half\eb_r} =
\begin{cases}
\half\left(\ut_{L,\ib-\half\eb_r} + \ut_{R,\ib-\half\eb_r}\right), & \left|w^{\trans}_{\ib-\half\eb_r}\right| < \epsilon \\
\ut_{L,\ib-\half\eb_r}, & w^{\trans}_{\ib-\half\eb_r} > 0, \\
\ut_{R,\ib-\half\eb_r}, & w^{\trans}_{\ib-\half\eb_r} < 0. \\
\end{cases}
\end{equation}
Predict $\ut$ to x-faces using a full-dimensional extrapolation, 
\begin{eqnarray}
\ut_{L,\ib-\half\eb_x}^{\mac,*} &=& \ut_{L,\ib-\half\eb_x} - \frac{\dt}{4h}\left(w_{\ib-\eb_x+\half\eb_r}^{\trans}+w_{\ib-\eb_x-\half\eb_r}^{\trans}\right)\left(\ut_{\ib-\eb_x+\half\eb_r} - \ut_{\ib-\eb_x-\half\eb_r}\right) + \frac{\dt}{2}f_{\ut,\ib-\eb_x}, \nonumber \\
&& \\
\ut_{R,\ib-\half\eb_x}^{\mac,*} &=& \ut_{R,\ib-\half\eb_x} - \frac{\dt}{4h}\left(w_{\ib+\half\eb_r}^{\trans}+w_{\ib-\half\eb_r}^{\trans}\right)\left(\ut_{\ib+\half\eb_r} - \ut_{\ib-\half\eb_r}\right) + \frac{\dt}{2}f_{\ut,\ib}.
\end{eqnarray}
Solve a Riemann problem:
\begin{equation}
\ut_{\ib-\half\eb_x}^{\mac,*} =
\begin{cases}
0, & \left(u_{L,\ib-\half\eb_x}^{\mac,*} \le 0 ~~ {\rm AND} ~~ u_{R,\ib-\half\eb_x}^{\mac,*} \ge 0\right) ~~ {\rm OR} ~~ \left|u_{L,\ib-\half\eb_x}^{\mac,*} + u_{R,\ib-\half\eb_x}^{\mac,*}\right| < \epsilon, \\
\ut_{L,\ib-\half\eb_x}^{\mac,*}, & u_{L,\ib-\half\eb_x}^{\mac,*} + u_{R,\ib-\half\eb_x}^{\mac,*} > 0, \\
\ut_{R,\ib-\half\eb_x}^{\mac,*}, & u_{L,\ib-\half\eb_x}^{\mac,*} + u_{R,\ib-\half\eb_x}^{\mac,*} < 0. 
\end{cases}
\end{equation}
Predict $\wt$ to x-faces using a 1D extrapolation:
\begin{eqnarray}
\wt_{L,\ib-\half\eb_x} &=& \wt_{\ib-\eb_x}^n + \left[\half - \frac{\dt}{2h}\max(0,u_{\ib-\eb_x}^n)\right]\Delta_x \wt_{\ib-\eb_x}^n, \\
\wt_{R,\ib-\half\eb_x} &=& \wt_{\ib} - \left[\half + \frac{\dt}{2h}\min(0,u_{\ib}^n)\right]\Delta_x \wt_{\ib}^n.
\end{eqnarray}
Upwind based on $u^{\trans}$:
\begin{equation}
\wt_{\ib-\half\eb_x} =
\begin{cases}
\half\left(\wt_{L,\ib-\half\eb_x} + \wt_{R,\ib-\half\eb_x}\right), & \left|u^{\trans}_{\ib-\half\eb_x}\right| < \epsilon \\
\wt_{L,\ib-\half\eb_x}, & u^{\trans}_{\ib-\half\eb_x} > 0, \\
\wt_{R,\ib-\half\eb_x}, & u^{\trans}_{\ib-\half\eb_x} < 0. \\
\end{cases}
\end{equation}
Predict $\wt$ to r-faces using a full-dimensional extrapolation:
\begin{eqnarray}
\wt_{L,\ib-\half\eb_r}^{\mac,*} = \wt_{L,\ib-\half\eb_r} &-& \frac{\dt}{4h}\left(u_{\ib-\eb_r+\half\eb_x}^{\trans}+u_{\ib-\eb_r-\half\eb_x}^{\trans}\right)\left(\wt_{\ib-\eb_r+\half\eb_x} - \wt_{\ib-\eb_r-\half\eb_x}\right) \nonumber \\
&-& \frac{\dt}{4h}\left(\wt_{\ib-\half\eb_r}^{\trans}+\wt_{\ib-\frac{3}{2}\eb_r}^{\trans}\right)\left(w_{0,\ib-\half\eb_r} - w_{0,\ib-\frac{3}{2}\eb_r}\right) + \frac{\dt}{2}f_{\wt,\ib-\eb_r}, \nonumber \\
&& \\
\wt_{R,\ib-\half\eb_r}^{\mac,*} = \wt_{R,\ib-\half\eb_r} &-& \frac{\dt}{4h}\left(u_{\ib+\half\eb_x}^{\trans}+u_{\ib-\half\eb_x}^{\trans}\right)\left(\wt_{\ib+\half\eb_x} - \wt_{\ib-\half\eb_x}\right) \nonumber \\
&-& \frac{\dt}{4h}\left(\wt_{\ib+\half\eb_r}^{\trans}+\wt_{\ib-\half\eb_r}^{\trans}\right)\left(w_{0,\ib+\half\eb_r} - w_{0,\ib-\half\eb_r}\right) + \frac{\dt}{2}f_{\wt,\ib}.
\end{eqnarray}
Solve a Riemann problem:
\begin{equation}
\wt_{\ib-\half\eb_r}^{\mac,*} =
\begin{cases}
0, & \left(w_L^{\mac,*} \le 0 ~~ {\rm AND} ~~ w_R^{\mac,*} \ge 0\right) ~~ {\rm OR} ~~ \left|w_L^{\mac,*} + w_R^{\mac,*}\right| < \epsilon, \\
\wt_{L,\ib-\half\eb_r}^{\mac,*}, & w_L^{\mac,*} + w_R^{\mac,*} > 0, \\
\wt_{R,\ib-\half\eb_r}^{\mac,*}, & w_L^{\mac,*} + w_R^{\mac,*} < 0. 
\end{cases}
\end{equation}

\newpage

\subsection{3D Cartesian Case}
This algorithm is more complicated than the 2D case since we include
the effects of corner coupling.

\begin{enumerate}
\item Predict $\ut$ to y-faces using a 1D extrapolation.
\item Predict $\ut$ to r-faces using a 1D extrapolation.
\item Predict $\vt$ to x-faces using a 1D extrapolation.
\item Predict $\vt$ to r-faces using a 1D extrapolation.
\item Predict $\wt$ to x-faces using a 1D extrapolation.
\item Predict $\wt$ to y-faces using a 1D extrapolation.
\item Update prediction of $\ut$ to y-faces by accounting for $r$-derivatives.
\item Update prediction of $\ut$ to r-faces by accounting for $y$-derivatives.
\item Update prediction of $\vt$ to x-faces by accounting for $r$-derivatives.
\item Update prediction of $\vt$ to r-faces by accounting for $x$-derivatives.
\item Update prediction of $\wt$ to x-faces by accounting for $y$-derivatives.
\item Update prediction of $\wt$ to y-faces by accounting for $x$-derivatives.
\item Predict $\ut$ to x-faces using a full-dimensional extrapolation.
\item Predict $\vt$ to y-faces using a full-dimensional extrapolation.
\item Predict $\wt$ to r-faces using a full-dimensional extrapolation.
\end{enumerate}
Predict $\ut$ to y-faces using a 1D extrapolation.
\begin{eqnarray}
\ut_{L,\ib-\half\eb_y} &=& \ut_{\ib-\eb_y}^n + \left[\half - \frac{\dt}{2h}\max(0,v_{\ib-\eb_y}^n)\right]\Delta_y \ut_{\ib-\eb_y}^n, \\
\ut_{R,\ib-\half\eb_y} &=& \ut_{\ib} - \left[\half + \frac{\dt}{2h}\min(0,v_{\ib}^n)\right]\Delta_y \ut_{\ib}^n.
\end{eqnarray}
Upwind based on $v^{\trans}$:
\begin{equation}
\ut_{\ib-\half\eb_y} =
\begin{cases}
\half\left(\ut_{L,\ib-\half\eb_y} + \ut_{R,\ib-\half\eb_y}\right), & \left|v^{\trans}_{\ib-\half\eb_y}\right| < \epsilon \\
\ut_{L,\ib-\half\eb_y}, & v^{\trans}_{\ib-\half\eb_y} > 0, \\
\ut_{R,\ib-\half\eb_y}, & v^{\trans}_{\ib-\half\eb_y} < 0. \\
\end{cases}
\end{equation}
Predict $\ut$ to r-faces using a 1D extrapolation.\\ \\
Predict $\vt$ to x-faces using a 1D extrapolation.\\ \\
Predict $\vt$ to r-faces using a 1D extrapolation.\\ \\
Predict $\wt$ to x-faces using a 1D extrapolation.\\ \\
Predict $\wt$ to y-faces using a 1D extrapolation.\\ \\
Update prediction of $\ut$ to y-faces by accounting for $r$-derivatives.
The notation $\ut_{\ib-\half\eb_y}^{y|r}$ means state $\ut_{\ib-\half\eb_y}$ that has been updated to account for transverse derives in the r-direction.
\begin{eqnarray}
\ut_{L,\ib-\half\eb_y}^{y|r} &=& \ut_{L,\ib-\half\eb_y} - \frac{\dt}{6h}\left(w_{\ib-\eb_y+\half\eb_r}^{\trans}+w_{\ib-\eb_y-\half\eb_r}^{\trans}\right)\left(\ut_{\ib-\eb_y+\half\eb_r}-\ut_{\ib-\eb_y-\half\eb_r}\right), \\
\ut_{R,\ib-\half\eb_y}^{y|r} &=& \ut_{R,\ib-\half\eb_y} - \frac{\dt}{6h}\left(w_{\ib+\half\eb_r}^{\trans}+w_{\ib-\half\eb_r}^{\trans}\right)\left(\ut_{\ib+\half\eb_r}-\ut_{\ib-\half\eb_r}\right).
\end{eqnarray}
Upwind based on $v^{\trans}$:
\begin{equation}
\ut_{\ib-\half\eb_y}^{y|r} =
\begin{cases}
\half\left(\ut_{L,\ib-\half\eb_y}^{y|r} + \ut_{R,\ib-\half\eb_y}^{y|r}\right), & \left|v_{\ib-\half\eb_y}^{\trans}\right| < \epsilon \\
\ut_{L,\ib-\half\eb_y}^{y|r}, & v_{\ib-\half\eb_y}^{\trans} > 0, \\
\ut_{R,\ib-\half\eb_y}^{y|r}, & v_{\ib-\half\eb_y}^{\trans} < 0.
\end{cases}
\end{equation}
Update prediction of $\ut$ to r-faces by accounting for $y$-derivatives. \\ \\
Update prediction of $\vt$ to x-faces by accounting for $r$-derivatives. \\ \\
Update prediction of $\vt$ to r-faces by accounting for $x$-derivatives. \\ \\
Update prediction of $\wt$ to x-faces by accounting for $y$-derivatives. \\ \\
Update prediction of $\wt$ to y-faces by accounting for $x$-derivatives. \\ \\
Predict $\ut$ to x-faces using a full-dimensional extrapolation.
\begin{eqnarray}
\ut_{L,\ib-\half\eb_x}^{\mac,*} = \ut_{L,\ib-\half\eb_x} &-& \frac{\dt}{4h}\left(v_{\ib-\eb_x+\half\eb_y}^{\trans}+v_{\ib-\eb_x-\half\eb_y}^{\trans}\right)\left(\ut_{\ib-\eb_x+\half\eb_y}^{y|r}-\ut_{\ib-\eb_x-\half\eb_y}^{y|r}\right) \nonumber \\
&-& \frac{\dt}{4h}\left(w_{\ib-\eb_x+\half\eb_r}^{\trans}+w_{\ib-\eb_x-\half\eb_r}^{\trans}\right)\left(\ut_{\ib-\eb_x+\half\eb_r}^{r|y}-\ut_{\ib-\eb_x-\half\eb_r}^{r|y}\right) + \frac{\dt}{2}f_{u,\ib-\eb_x}, \nonumber \\
&& \\
\ut_{R,\ib-\half\eb_x}^{\mac,*} = \ut_{R,\ib-\half\eb_x} &-& \frac{\dt}{4h}\left(v_{\ib+\half\eb_y}^{\trans}+v_{\ib-\half\eb_y}^{\trans}\right)\left(\ut_{\ib+\half\eb_y}^{y|r}-\ut_{\ib-\half\eb_y}^{y|r}\right) \nonumber \\
&-& \frac{\dt}{4h}\left(w_{\ib+\half\eb_r}^{\trans}+w_{\ib-\half\eb_r}^{\trans}\right)\left(\ut_{\ib+\half\eb_r}^{r|y}-\ut_{\ib-\half\eb_r}^{r|y}\right) + \frac{\dt}{2}f_{u,\ib}.
\end{eqnarray}
Solve a Riemann problem:
\begin{equation}
\ut_{\ib-\half\eb_x}^{\mac,*} =
\begin{cases}
0, & \left(u_{L,\ib-\half\eb_x}^{\mac,*} \le 0 ~~ {\rm AND} ~~ u_{R,\ib-\half\eb_x}^{\mac,*} \ge 0\right) ~~ {\rm OR} ~~ \left|u_{L,\ib-\half\eb_x}^{\mac,*} + u_{R,\ib-\half\eb_x}^{\mac,*}\right| < \epsilon, \\
\ut_{L,\ib-\half\eb_x}^{\mac,*}, & u_{L,\ib-\half\eb_x}^{\mac,*} + u_{R,\ib-\half\eb_x}^{\mac,*} > 0, \\
\ut_{R,\ib-\half\eb_x}^{\mac,*}, & u_{L,\ib-\half\eb_x}^{\mac,*} + u_{R,\ib-\half\eb_x}^{\mac,*} < 0. 
\end{cases}
\end{equation}
Predict $\vt$ to y-faces using a full-dimensional extrapolation.\\ \\
Predict $\wt$ to r-faces using a full-dimensional extrapolation.  
In this step, make sure you account for the $\partial w_0/\partial r$ 
term before solving the Riemann problem:
\begin{eqnarray}
\wt_{L,\ib-\half\eb_r}^{\mac,*} &=& \wt_{L,\ib-\half\eb_r}^{\mac,*} - 
\frac{\dt}{4h}\left(\wt^{\trans}_{\ib+\half\eb_r} + \wt^{\trans}_{\ib-\half\eb_r}\right)\left(w_{0,\ib+\half\eb_r}-w_{0,\ib-\half\eb_r}\right) \\
\wt_{R,\ib-\half\eb_r}^{\mac,*} &=& \wt_{R,\ib-\half\eb_r}^{\mac,*} -
\frac{\dt}{4h}\left(\wt^{\trans}_{\ib-\half\eb_r} + \wt^{\trans}_{\ib-\frac{3}{2}\eb_r}\right)\left(w_{0,\ib-\half\eb_r}-w_{0,\ib-\frac{3}{2}\eb_r}\right)
\end{eqnarray}

\newpage

\subsection{3D Spherical Case}
The spherical case is the same as the plane-parallel 3D Cartesian 
case, except the $\partial w_0/\partial r$ term enters 
in the full dimensional extrapolation for each direction.
As in the plane-parallel case, make sure to upwind using the 
full velocity.

\newpage

%-----------------------------------------------------------------------------
% edge states
%-----------------------------------------------------------------------------
\section{Computing $\rho^{'\edge}, X_k^{\edge},(\rho h)^{'\edge} / h^{\edge} / T^{\edge} / h'^{\edge} / T'^{\edge}$, and $\Ubt^{\edge}$ in {\tt MAESTRO}}\label{Scalar Edge State Prediction in MAESTRO}
We call {\tt make\_edge\_scal} to compute $\rho^{'\edge}, X_k^{\edge}, 
(\rho h)^{'\edge} / h^{\edge} / T^{\edge} / h'^{\edge} / T'^{\edge}$, 
and $\Ubt^{\edge}$ at each edge.
The procedure is the same for each quantitiy, so we shall simply denote 
the scalar as $s$.  We always need to compute $\rho'$ and $X_k$ to faces, 
and the choice of energy prediction is as follows:
\begin{itemize}
\item For {\tt enthalpy\_pred\_type} = 1, we predict $(\rho h)'$ to faces.
\item For {\tt enthalpy\_pred\_type} = 2, we predict $h$ to faces.
\item For {\tt enthalpy\_pred\_type} = 3 and 4, we predict $T$ to faces.
\item For {\tt enthalpy\_pred\_type} = 5, we predict $h'$ to faces.
\item For {\tt enthalpy\_pred\_type} = 6, we predict $T'$ to faces.
\end{itemize}
We are using {\tt enthalpy\_pred\_type} = 1 for now.  The equations
of motion are:
\begin{eqnarray}
\frac{\partial \rho'}{\partial t} &=& -\Ub\cdot\nabla\rho' \underbrace{- \rho'\nabla\cdot\Ub - \nabla\cdot(\rho_0\Ubt)}_{f_{\rho'}}, \\
\frac{\partial(\rho X_k)}{\partial t} &=& -\nabla\cdot(\rho X_k\Ub) ~~~ \text{(no forcing)}, \\
\frac{\partial(\rho h)'}{\partial t} &=& -\Ub\cdot\nabla(\rho h)' + \underbrace{(\Ubt\cdot\eb_r)\frac{\partial p_0}{\partial r} -(\rho h)'\nabla\cdot\Ub^{\mac} - \nabla\cdot\left[(\rho h)_0\Ubt^{\mac}\right] + \nabla\cdot\kth\nabla T}_{f_{(\rho h)'}}, \nonumber \\
&& \\
\frac{\partial\Ubt}{\partial t} &=& -\Ub\cdot\nabla\Ubt \underbrace{- (\Ubt\cdot\eb_r)\frac{\partial w_0}{\partial r}\eb_r \underbrace{- \frac{1}{\rho}\nabla\pi + \frac{1}{\rho_0}\frac{\partial\pi_0}{\partial r}\eb_r - \frac{(\rho-\rho_0)}{\rho}g\eb_r}_{\hbox{terms included in $\fb_{\Ubt}$}}}_{\hbox{forcing terms}}.
\end{eqnarray}

\newpage

\subsection{2D Cartesian Case}
\begin{enumerate}
\item Predict $s$ to r-faces using a 1D extrapolation.
\item Predict $s$ to x-faces using a full-dimensional extrapolation.
\item Predict $s$ to x-faces using a 1D extrapolation.
\item Predict $s$ to r-faces using a full-dimensional extrapolation.
\end{enumerate}
Predict $s$ to r-faces using a 1D extrapolation:
\begin{eqnarray}
s_{L,\ib-\half\eb_r} &=& s_{\ib-\eb_r}^n + \left(\half - \frac{\dt}{2h}w_{\ib-\half\eb_r}^{\mac}\right)\Delta_r s_{\ib-\eb_r}^n, \\
s_{R,\ib-\half\eb_r} &=& s_{\ib} - \left(\half + \frac{\dt}{2h}w_{\ib-\half\eb_r}^{\mac}\right)\Delta_r s_{\ib}^n.
\end{eqnarray}
Upwind based on $w^{\mac}$:
\begin{equation}
s_{\ib-\half\eb_r} =
\begin{cases}
\half\left(s_{L,\ib-\half\eb_r} + s_{R,\ib-\half\eb_r}\right), & \left|w^{\mac}_{\ib-\half\eb_r}\right| < \epsilon \\
s_{L,\ib-\half\eb_r}, & w^{\mac}_{\ib-\half\eb_r} > 0, \\
s_{R,\ib-\half\eb_r}, & w^{\mac}_{\ib-\half\eb_r} < 0. \\
\end{cases}
\end{equation}
Predict $s$ to x-faces using a full-dimensional extrapolation.  First, the normal derivative and forcing terms:
\begin{eqnarray}
s_{L,\ib-\half\eb_x}^{\edge} &=& s_{\ib-\eb_x}^n + \left(\half - \frac{\dt}{2h}u_{\ib-\half\eb_x}^{\mac}\right)\Delta_x s_{\ib-\eb_x}^n + \frac{\dt}{2}f_{\ib-\eb_x}^n \\
s_{R,\ib-\half\eb_x}^{\edge} &=& s_{\ib}^n - \left(\half + \frac{\dt}{2h}u_{\ib-\half\eb_x}^{\mac}\right)\Delta_x s_{\ib}^n + \frac{\dt}{2}f_{\ib}^n 
\end{eqnarray}
Account for the transverse terms:\\ \\
{\bf if} {\tt is\_conservative} {\bf then}
\begin{eqnarray}
s_{L,\ib-\half\eb_x}^{\edge} &=& s_{L,\ib-\half\eb_x}^{\edge} -
\frac{\dt}{2h}\left[\left(w^{\mac}s\right)_{\ib-\eb_x+\half\eb_r} - \left(w^{\mac}s\right)_{\ib-\eb_x-\half\eb_r}\right] - \frac{\dt}{2h}s_{\ib-\eb_x}^{n}\left(u_{\ib-\half\eb_x}^{\mac}-u_{\ib-\frac{3}{2}\eb_x}^{\mac}\right)\nonumber \\
&&\\
s_{R,\ib-\half\eb_x}^{\edge} &=& s_{R,\ib-\half\eb_x}^{\edge} -
\frac{\dt}{2h}\left[\left(w^{\mac}s\right)_{\ib+\half\eb_r} - \left(w^{\mac}s\right)_{\ib-\half\eb_r}\right] - \frac{\dt}{2h}s_{\ib}^{n}\left(u_{\ib+\half\eb_x}^{\mac}-u_{\ib-\half\eb_x}^{\mac}\right)
\end{eqnarray}
{\bf else}
\begin{eqnarray}
s_{L,\ib-\half\eb_x}^{\edge} &=& s_{L,\ib-\half\eb_x}^{\edge} -
\frac{\dt}{4h}\left(w^{\mac}_{\ib-\eb_x+\half\eb_r} + w^{\mac}_{\ib-\eb_x-\half\eb_r}\right)\left(s_{\ib-\eb_x+\half\eb_r} - s_{\ib-\eb_x-\half\eb_r}\right)\\
s_{R,\ib-\half\eb_x}^{\edge} &=& s_{R,\ib-\half\eb_x}^{\edge} -
\frac{\dt}{4h}\left(w^{\mac}_{\ib+\half\eb_r} + w^{\mac}_{\ib-\half\eb_r}\right)\left(s_{\ib+\half\eb_r} - s_{\ib-\half\eb_r}\right)
\end{eqnarray}
{\bf end if}\\ \\
Account for the $\partial w_0/\partial r$ term:\\ \\
{\bf if} {\tt is\_vel} {\bf and} {\tt comp} = 2 {\bf then}
\begin{eqnarray}
s_{L,\ib-\half\eb_x}^{\edge} &=& s_{L,\ib-\half\eb_x}^{\edge} - 
\frac{\dt}{4h}\left(\wt^{\mac}_{\ib-\eb_x+\half\eb_r} + \wt^{\mac}_{\ib-\eb_x-\half\eb_r}\right)\left(w_{0,\ib+\half\eb_r}-w_{0,\ib-\half\eb_r}\right) \\
s_{R,\ib-\half\eb_x}^{\edge} &=& s_{R,\ib-\half\eb_x}^{\edge} -
\frac{\dt}{4h}\left(\wt^{\mac}_{\ib+\half\eb_r} + \wt^{\mac}_{\ib-\half\eb_r}\right)\left(w_{0,\ib+\half\eb_r}-w_{0,\ib-\half\eb_r}\right) \\
\end{eqnarray}
{\bf end if}\\ \\
Upwind based on $u^{\mac}$.
\begin{equation}
s_{\ib-\half\eb_x}^{\edge} =
\begin{cases}
\half\left(s_{L,\ib-\half\eb_x}^{\edge} + s_{R,\ib-\half\eb_x}^{\edge}\right), & \left|u^{\mac}_{\ib-\half\eb_x}\right| < \epsilon \\
s_{L,\ib-\half\eb_x}^{\edge}, & u^{\mac}_{\ib-\half\eb_x} > 0, \\
s_{R,\ib-\half\eb_x}^{\edge}, & u^{\mac}_{\ib-\half\eb_x} < 0.
\end{cases}
\end{equation}
Predict $s$ to x-faces using a 1D extrapolation:
\begin{eqnarray}
s_{L,\ib-\half\eb_x} &=& s_{\ib-\eb_x}^n + \left(\half - \frac{\dt}{2h}u_{\ib-\half\eb_x}^{\mac}\right)\Delta_x s_{\ib-\eb_x}^n, \\
s_{R,\ib-\half\eb_x} &=& s_{\ib} - \left(\half + \frac{\dt}{2h}u_{\ib-\half\eb_x}^{\mac}\right)\Delta_x s_{\ib}^n.
\end{eqnarray}
Upwind based on $u^{\mac}$:
\begin{equation}
s_{\ib-\half\eb_x} =
\begin{cases}
\half\left(s_{L,\ib-\half\eb_x} + s_{R,\ib-\half\eb_x}\right), & \left|u^{\mac}_{\ib-\half\eb_x}\right| < \epsilon \\
s_{L,\ib-\half\eb_x}, & u^{\mac}_{\ib-\half\eb_x} > 0, \\
s_{R,\ib-\half\eb_x}, & u^{\mac}_{\ib-\half\eb_x} < 0. \\
\end{cases}
\end{equation}
Predict $s$ to r-faces using a full-dimensional extrapolation.  First, the normal derivative and forcing terms:
\begin{eqnarray}
s_{L,\ib-\half\eb_r}^{\edge} &=&  s_{\ib-\eb_r}^n + \left(\half - \frac{\dt}{2h}w_{\ib-\half\eb_r}^{\mac}\right)\Delta_r s_{\ib-\eb_r}^n + \frac{\dt}{2}f_{\ib-\eb_r}^n \\
s_{R,\ib-\half\eb_r}^{\edge} &=&  s_{\ib}^n - \left(\half + \frac{\dt}{2h}w_{\ib-\half\eb_r}^{\mac}\right)\Delta_r s_{\ib}^n + \frac{\dt}{2}f_{\ib}^n 
\end{eqnarray}
Account for the transverse terms:\\ \\
{\bf if} {\tt is\_conservative} {\bf then}
\begin{eqnarray}
s_{L,\ib-\half\eb_r}^{\edge} &=& s_{L,\ib-\half\eb_r}^{\edge} -
\frac{\dt}{2h}\left[\left(u^{\mac}s\right)_{\ib-\eb_r+\half\eb_x} - \left(u^{\mac}s\right)_{\ib-\eb_r-\half\eb_x}\right] - \frac{\dt}{2h}s_{\ib-\eb_r}^{n}\left(w_{\ib-\half\eb_r}^{\mac}-w_{\ib-\frac{3}{2}\eb_r}^{\mac}\right)\nonumber\\
&& \\
s_{R,\ib-\half\eb_r}^{\edge} &=& s_{R,\ib-\half\eb_r}^{\edge} -
\frac{\dt}{2h}\left[\left(u^{\mac}s\right)_{\ib+\half\eb_x} - \left(u^{\mac}s\right)_{\ib-\half\eb_x}\right] - \frac{\dt}{2h}s_{\ib}^{n}\left(w_{\ib+\half\eb_r}^{\mac}-w_{\ib-\half\eb_r}^{\mac}\right)
\end{eqnarray}
{\bf else}
\begin{eqnarray}
s_{L,\ib-\half\eb_r}^{\edge} &=& s_{L,\ib-\half\eb_r}^{\edge} -
\frac{\dt}{4h}\left(u^{\mac}_{\ib-\eb_r+\half\eb_x} + u^{\mac}_{\ib-\eb_r-\half\eb_x}\right)\left(s_{\ib-\eb_r+\half\eb_x} - s_{\ib-\eb_r-\half\eb_x}\right)\\
s_{R,\ib-\half\eb_r}^{\edge} &=& s_{R,\ib-\half\eb_r}^{\edge} -
\frac{\dt}{4h}\left(u^{\mac}_{\ib+\half\eb_x} + u^{\mac}_{\ib-\half\eb_x}\right)\left(s_{\ib+\half\eb_x} - s_{\ib-\half\eb_x}\right)
\end{eqnarray}
{\bf end if}\\ \\
Account for the $\partial w_0/\partial r$ term:\\ \\
{\bf if} {\tt is\_vel} {\bf and} {\tt comp} = 2 {\bf then}
\begin{eqnarray}
s_{L,\ib-\half\eb_r}^{\edge} &=& s_{L,\ib-\half\eb_r}^{\edge} - 
\frac{\dt}{4h}\left(\wt^{\mac}_{\ib-\half\eb_r} + \wt^{\mac}_{\ib-\frac{3}{2}\eb_r}\right)\left(w_{0,\ib-\half\eb_r}-w_{0,\ib-\frac{3}{2}\eb_r}\right) \\
s_{R,\ib-\half\eb_r}^{\edge} &=& s_{R,\ib-\half\eb_r}^{\edge} -
\frac{\dt}{4h}\left(\wt^{\mac}_{\ib+\half\eb_r} + \wt^{\mac}_{\ib-\half\eb_r}\right)\left(w_{0,\ib+\half\eb_r}-w_{0,\ib-\half\eb_r}\right) \\
\end{eqnarray}
{\bf end if}\\ \\
Upwind based on $w^{\mac}$:
\begin{equation}
s_{\ib-\half\eb_r} =
\begin{cases}
\half\left(s_{L,\ib-\half\eb_r} + s_{R,\ib-\half\eb_r}\right), & \left|w^{\mac}_{\ib-\half\eb_r}\right| < \epsilon \\
u_{L,\ib-\half\eb_r}, & w^{\mac}_{\ib-\half\eb_r} > 0, \\
u_{R,\ib-\half\eb_r}, & w^{\mac}_{\ib-\half\eb_r} < 0. \\
\end{cases}
\end{equation}

\newpage

\subsection{3D Cartesian Case}
This algorithm is more complicated than the 2D case since we include
the effects of corner coupling.
\begin{enumerate}
\item Predict $s$ to x-faces using a 1D extrapolation.
\item Predict $s$ to y-faces using a 1D extrapolation.
\item Predict $s$ to r-faces using a 1D extrapolation.
\item Update prediction of $s$ to x-faces by accounting for y-derivatives.
\item Update prediction of $s$ to x-faces by accounting for r-derivatives.
\item Update prediction of $s$ to y-faces by accounting for x-derivatives.
\item Update prediction of $s$ to y-faces by accounting for r-derivatives.
\item Update prediction of $s$ to r-faces by accounting for x-derivatives.
\item Update prediction of $s$ to r-faces by accounting for y-derivatives.
\item Predict $s$ to x-faces using a full-dimensional extrapolation.
\item Predict $s$ to y-faces using a full-dimensional extrapolation.
\item Predict $s$ to r-faces using a full-dimensional extrapolation.
\end{enumerate}
Predict $s$ to x-faces using a 1D extrapolation.
\begin{eqnarray}
s_{L,\ib-\half\eb_x} &=& s_{\ib-\eb_x}^n + \left(\half - \frac{\dt}{2h}u_{\ib-\half\eb_x}^{\mac}\right)\Delta_x s_{\ib-\eb_x}^n, \label{3D predict s to left}\\
s_{R,\ib-\half\eb_x} &=& s_{\ib} - \left(\half + \frac{\dt}{2h}u_{\ib-\half\eb_x}^{\mac}\right)\Delta_x s_{\ib}^n.\label{3D predict s to right}
\end{eqnarray}
Upwind based on $u^{\mac}$:
\begin{equation}
s_{\ib-\half\eb_x} =
\begin{cases}
\half\left(s_{L,\ib-\half\eb_x} + s_{R,\ib-\half\eb_x}\right), & \left|u^{\mac}_{\ib-\half\eb_x}\right| < \epsilon \\
s_{L,\ib-\half\eb_x}, & u^{\mac}_{\ib-\half\eb_x} > 0, \\
s_{R,\ib-\half\eb_x}, & u^{\mac}_{\ib-\half\eb_x} < 0. \\
\end{cases}
\end{equation}
Predict $s$ to y-faces using a 1D extrapolation.\\ \\
Predict $s$ to r-faces using a 1D extrapolation.\\ \\
Update prediction of $s$ to x-faces by accounting for y-derivatives.
The notation $s_{\ib-\half\eb_x}^{x|y}$ means ``state $s_{\ib-\half\eb_x}$
that has been updated to account for the transverse derivatives in 
the $y$-direction''.\\ \\
{\bf if} {\tt is\_conservative} {\bf then}
\begin{eqnarray}
s_{L,\ib-\half\eb_x}^{x|y} &=& s_{L,\ib-\half\eb_x} - \frac{\dt}{3h}\left[(sv^{\mac})_{\ib-\eb_x+\half\eb_y}-(sv^{\mac})_{\ib-\eb_x-\half\eb_y}\right], \\
s_{R,\ib-\half\eb_x}^{x|y} &=& s_{R,\ib-\half\eb_x} - \frac{\dt}{3h}\left[(sv^{\mac})_{\ib+\half\eb_y}-(sv^{\mac})_{\ib-\half\eb_y}\right].
\end{eqnarray}
{\bf else}
\begin{eqnarray}
s_{L,\ib-\half\eb_x}^{x|y} &=& s_{L,\ib-\half\eb_x} - \frac{\dt}{6h}\left(v_{\ib-\eb_x+\half\eb_y}^{\mac} + v_{\ib-\eb_x-\half\eb_y}^{\mac}\right)\left(s_{\ib-\eb_x+\half\eb_y} - s_{\ib-\eb_x-\half\eb_y}\right), \\
s_{R,\ib-\half\eb_x}^{x|y} &=& s_{R,\ib-\half\eb_x} - \frac{\dt}{6h}\left(v_{\ib+\half\eb_y}^{\mac} + v_{\ib-\half\eb_y}^{\mac}\right)\left(s_{\ib+\half\eb_y} - s_{\ib-\half\eb_y}\right).
\end{eqnarray}
{\bf end if}\\ \\
Upwind based on $u^{\mac}$:
\begin{equation}
s_{\ib-\half\eb_x}^{x|y} =
\begin{cases}
\half\left(s_{L,\ib-\half\eb_x}^{x|y} + s_{R,\ib-\half\eb_x}^{x|y}\right), & \left|u^{\mac}_{\ib-\half\eb_x}\right| < \epsilon \\
s_{L,\ib-\half\eb_x}^{x|y}, & u^{\mac}_{\ib-\half\eb_x} > 0, \\
s_{R,\ib-\half\eb_x}^{x|y}, & u^{\mac}_{\ib-\half\eb_x} < 0.
\end{cases}
\end{equation}
Update prediction of $s$ to x-faces by accounting for r-derivatives.\\ \\
Update prediction of $s$ to y-faces by accounting for x-derivatives.\\ \\
Update prediction of $s$ to y-faces by accounting for r-derivatives.\\ \\
Update prediction of $s$ to r-faces by accounting for x-derivatives.\\ \\
Update prediction of $s$ to r-faces by accounting for y-derivatives.\\ \\
Predict $s$ to x-faces using a full-dimensional extrapolation.\\ \\
{\bf if} {\tt is\_conservative} {\bf then}
\begin{eqnarray}
s_{L,\ib-\half\eb_x}^{\edge} = s_{L,\ib-\half\eb_x} &-& \frac{\dt}{2h}\left[(s^{y|r}v^{\mac})_{\ib-\eb_x+\half\eb_y}-({s^{y|r}v^{\mac})_{\ib-\eb_x-\half\eb_y}}\right] \nonumber \\
&-& \frac{\dt}{2h}\left[(s^{r|y}w^{\mac})_{\ib-\eb_x+\half\eb_r}-({s^{r|y}w^{\mac})_{\ib-\eb_x-\half\eb_r}}\right] \nonumber \\
&-& \frac{\dt}{2h}s_{\ib-\eb_x}\left(u_{\ib-\half\eb_x}^{\mac}-u_{\ib-\frac{3}{2}\eb_x}^{\mac}\right) + \frac{\dt}{2}f_{\ib-\eb_x}, \\
s_{R,\ib-\half\eb_x}^{\edge} = s_{R,\ib-\half\eb_x} &-& \frac{\dt}{2h}\left[(s^{y|r}v^{\mac})_{\ib+\half\eb_y}-({s^{y|r}v^{\mac})_{\ib-\half\eb_y}}\right] \nonumber \\
&-& \frac{\dt}{2h}\left[(s^{r|y}w^{\mac})_{\ib+\half\eb_r}-({s^{r|y}w^{\mac})_{\ib-\half\eb_r}}\right] \nonumber \\
&-& \frac{\dt}{2h}s_{\ib}\left(u_{\ib+\half\eb_x}^{\mac}-u_{\ib-\half\eb_x}^{\mac}\right) + \frac{\dt}{2}f_{\ib}.
\end{eqnarray}
{\bf else}
\begin{eqnarray}
s_{L,\ib-\half\eb_x}^{\edge} = s_{L,\ib-\half\eb_x} &-& \frac{\dt}{4h}\left(v_{\ib-\eb_x+\half\eb_y}^{\mac}+v_{\ib-\eb_x-\half\eb_y}^{\mac}\right)\left(s_{\ib-\eb_x+\half\eb_y}^{y|r}-s_{\ib-\eb_x-\half\eb_y}^{y|r}\right) \nonumber \\
&-& \frac{\dt}{4h}\left(w_{\ib-\eb_x+\half\eb_r}^{\mac}+w_{\ib-\eb_x-\half\eb_r}^{\mac}\right)\left(s_{\ib-\eb_x+\half\eb_r}^{r|y}-s_{\ib-\eb_x-\half\eb_r}^{r|y}\right) + \frac{\dt}{2}f_{\ib-\eb_x}, \nonumber \\
&& \\
s_{R,\ib-\half\eb_x}^{\edge} = s_{R,\ib-\half\eb_x} &-& \frac{\dt}{4h}\left(v_{\ib+\half\eb_y}^{\mac}+v_{\ib-\half\eb_y}^{\mac}\right)\left(s_{\ib+\half\eb_y}^{y|r}-s_{\ib-\half\eb_y}^{y|r}\right) \nonumber \\
&-& \frac{\dt}{4h}\left(w_{\ib+\half\eb_r}^{\mac}+w_{\ib-\half\eb_r}^{\mac}\right)\left(s_{\ib+\half\eb_r}^{r|y}-s_{\ib-\half\eb_r}^{r|y}\right) + \frac{\dt}{2}f_{\ib}.
\end{eqnarray}
{\bf end if}\\ \\
Account for the $\partial w_0/\partial r$ term:\\ \\
{\bf if} {\tt is\_vel} {\bf and} {\tt comp} = 2 {\bf then}
\begin{eqnarray}
s_{L,\ib-\half\eb_x}^{\edge} &=& s_{L,\ib-\half\eb_x}^{\edge} - 
\frac{\dt}{4h}\left(\wt^{\mac}_{\ib-\eb_x+\half\eb_r} + \wt^{\mac}_{\ib-\eb_x-\half\eb_r}\right)\left(w_{0,\ib+\half\eb_r}-w_{0,\ib-\half\eb_r}\right) \\
s_{R,\ib-\half\eb_x}^{\edge} &=& s_{R,\ib-\half\eb_x}^{\edge} -
\frac{\dt}{4h}\left(\wt^{\mac}_{\ib+\half\eb_r} + \wt^{\mac}_{\ib-\half\eb_r}\right)\left(w_{0,\ib+\half\eb_r}-w_{0,\ib-\half\eb_r}\right) \\
\end{eqnarray}
{\bf end if}\\ \\
Upwind based on $u^{\mac}$:
\begin{equation}
s_{\ib-\half\eb_x}^{\edge} =
\begin{cases}
\half\left(s_{L,\ib-\half\eb_x}^{\edge} + s_{R,\ib-\half\eb_x}^{\edge}\right), & \left|u^{\mac}_{\ib-\half\eb_x}\right| < \epsilon \\
s_{L,\ib-\half\eb_x}^{\edge}, & u^{\mac}_{\ib-\half\eb_x} > 0, \\
s_{R,\ib-\half\eb_x}^{\edge}, & u^{\mac}_{\ib-\half\eb_x} < 0.
\end{cases}
\end{equation}
Predict $s$ to y-faces using a full-dimensional extrapolation.\\ \\
Predict $s$ to r-faces using a full-dimensional extrapolation.

\newpage

\subsection{3D Spherical Case}
The spherical case is the same as the plane-parallel 3D Cartesian 
case, except the $\partial w_0/\partial r$ term enters in the full 
dimensional extrapolation for each direction when predicting velocity 
to faces.  As in the plane-parallel case, make sure upwind based on 
the full velocity.

\newpage

%-----------------------------------------------------------------------------
% Umac in Varden
%-----------------------------------------------------------------------------
\section{Computing $\Ub^{\mac,*}$ in {\tt VARDEN}}

\subsection{2D Cartesian Case}
We do a 1D Taylor series extrapolation to get both components of velocity at the x-face:
\begin{eqnarray}
u_{L,\ib-\half\eb_x}^{1D} &=& u_{\ib-\eb_x} + \left[\half - \frac{\dt}{2h}{\rm max}(0,u_{\ib-\eb_x})\right]\Delta_xu_{\ib-\eb_x}, \label{varden U_L^1D} \\
u_{R,\ib-\half\eb_x}^{1D} &=& u_{\ib} + \left[\half - \frac{\dt}{2h}{\rm min}(0,u_{\ib})\right]\Delta_xu_{\ib}.
\end{eqnarray}
\begin{eqnarray}
v_{L,\ib-\half\eb_x}^{1D} &=& v_{\ib-\eb_x} + \left[\half - \frac{\dt}{2h}{\rm max}(0,v_{\ib-\eb_x})\right]\Delta_xv_{\ib-\eb_x}, \\
v_{R,\ib-\half\eb_x}^{1D} &=& v_{\ib} + \left[\half - \frac{\dt}{2h}{\rm min}(0,v_{\ib})\right]\Delta_xv_{\ib}.
\end{eqnarray}
We obtain the normal velocity using the Riemann problem:
\begin{equation}
u_{\ib-\half\eb_x}^{1D} =
\begin{cases}
0, & \left(u_{L,\ib-\half\eb_x}^{1D} \le 0 ~~ {\rm AND} ~~ u_{R,\ib-\half\eb_x}^{1D} \ge 0\right) ~~ {\rm OR} ~~ \left|u_{L,\ib-\half\eb_x}^{1D} + u_{R,\ib-\half\eb_x}^{1D}\right| < \epsilon, \\
u_{L,\ib-\half\eb_x}^{1D}, & u_{L,\ib-\half\eb_x}^{1D} + u_{R,\ib-\half\eb_x}^{1D} > 0, \\
u_{R,\ib-\half\eb_x}^{1D}, & u_{L,\ib-\half\eb_x}^{1D} + u_{R,\ib-\half\eb_x}^{1D} < 0.
\end{cases}
\end{equation}
We obtain the transverse velocity by upwinding based on
$u_{\ib-\half\eb_x}^{1D}$:
\begin{equation}
v_{\ib-\half\eb_x}^{1D} =
\begin{cases}
\half\left(v_{L,\ib-\half\eb_x}^{1D} + v_{R,\ib-\half\eb_x}^{1D}\right), & \left|u_{\ib-\half\eb_x}^{1D}\right| < \epsilon \\
v_{L,\ib-\half\eb_x}^{1D}, & u_{\ib-\half\eb_x}^{1D} > 0, \\
v_{R,\ib-\half\eb_x}^{1D}, & u_{\ib-\half\eb_x}^{1D} < 0.
\end{cases}\label{Transverse Velocity Riemann Problem}
\end{equation}
We perform analogous operations to compute both components of velocity
at the y-faces, $\Ub_{\ib-\half\eb_y}^{1D}$. \\

Now we do a full-dimensional extrapolation to get the MAC velocity at
the x-faces (note that we only compute the normal components):
\begin{eqnarray}
u_{L,\ib-\half\eb_x}^{\mac,*} &=& u_{L,\ib-\half\eb_x}^{1D} - \frac{\dt}{4h}\left(v_{\ib-\eb_x+\half\eb_y}^{1D}+v_{\ib-\eb_x-\half\eb_y}^{1D}\right)\left(u_{\ib-\eb_x+\half\eb_y}^{1D} - u_{\ib-\eb_x-\half\eb_y}^{1D}\right) + \frac{\dt}{2}f_{u,\ib-\eb_x}, \\
u_{R,\ib-\half\eb_x}^{\mac,*} &=& u_{R,\ib-\half\eb_x}^{1D} - \frac{\dt}{4h}\left(v_{\ib+\half\eb_y}^{1D}+v_{\ib-\half\eb_y}^{1D}\right)\left(u_{\ib+\half\eb_y}^{1D} - u_{\ib-\half\eb_y}^{1D}\right) + \frac{\dt}{2}f_{u,\ib}.
\end{eqnarray}
Then we solve a Riemann problem:
\begin{equation}
u_{\ib-\half\eb_x}^{\mac,*} =
\begin{cases}
0, & \left(u_{L,\ib-\half\eb_x}^{\mac,*} \le 0 ~~ {\rm AND} ~~ u_{R,\ib-\half\eb_x}^{\mac,*} \ge 0\right) ~~ {\rm OR} ~~ \left|u_{L,\ib-\half\eb_x}^{\mac,*} + u_{R,\ib-\half\eb_x}^{\mac,*}\right| < \epsilon, \\
u_{L,\ib-\half\eb_x}^{\mac,*}, & u_{L,\ib-\half\eb_x}^{\mac,*} + u_{R,\ib-\half\eb_x}^{\mac,*} > 0, \\
u_{R,\ib-\half\eb_x}^{\mac,*}, & u_{L,\ib-\half\eb_x}^{\mac,*} + u_{R,\ib-\half\eb_x}^{\mac,*} < 0.
\end{cases}\label{umac Riemann Problem}
\end{equation}
We perform analogous operations to compute the normal velocity at the
y-faces, $v^{\mac,*}_{\ib-\half\eb_y}$.

\subsection{3D Cartesian Case}
This is more complicated than the 2D case because we include corner
coupling.  We compute $\Ub_{\ib-\half\eb_x}^{1D},
\Ub_{\ib-\half\eb_y}^{1D}$, and $\Ub_{\ib-\half\eb_z}^{1D}$ in an
analogous manner as equations (\ref{varden U_L^1D})-(\ref{Transverse
  Velocity Riemann Problem}).  Then we compute an intermediate state,
$u_{\ib-\half\eb_y}^{y|z}$, which is described as ``state
$u_{\ib-\half\eb_y}^{1D}$ that has been updated to account for the
transverse derivatives in the z direction'', using:
\begin{eqnarray}
u_{L,\ib-\half\eb_y}^{y|z} &=& u_{L,\ib-\half\eb_y}^{1D} - \frac{\dt}{6h}\left(w_{\ib-\eb_y+\half\eb_z}^{1D}+w_{\ib-\eb_y-\half\eb_z}^{1D}\right)\left(u_{\ib-\eb_y+\half\eb_z}^{1D}-u_{\ib-\eb_y-\half\eb_z}^{1D}\right), \\
u_{R,\ib-\half\eb_y}^{y|z} &=& u_{R,\ib-\half\eb_y}^{1D} - \frac{\dt}{6h}\left(w_{\ib+\half\eb_z}^{1D}+w_{\ib-\half\eb_z}^{1D}\right)\left(u_{\ib+\half\eb_z}^{1D}-u_{\ib-\half\eb_z}^{1D}\right).
\end{eqnarray}
Then upwind based on $v_{\ib-\half\eb_y}^{1D}$:
\begin{equation}
u_{\ib-\half\eb_y}^{y|z} =
\begin{cases}
\half\left(u_{L,\ib-\half\eb_y}^{y|z} + u_{R,\ib-\half\eb_y}^{y|z}\right), & \left|v_{\ib-\half\eb_y}^{1D}\right| < \epsilon \\
u_{L,\ib-\half\eb_y}^{y|z}, & v_{\ib-\half\eb_y}^{1D} > 0, \\
u_{R,\ib-\half\eb_y}^{y|z}, & v_{\ib-\half\eb_y}^{1D} < 0.
\end{cases}
\end{equation}
We use an analogous procedure to compute five more intemediate states,
$u_{\ib-\half\eb_z}^{z|y}, v_{\ib-\half\eb_x}^{x|z},
v_{\ib-\half\eb_z}^{z|x}, w_{\ib-\half\eb_x}^{x|y}$, and
$w_{\ib-\half\eb_y}^{y|x}$.  Then we do a full-dimensional
extrapolation to get the MAC velocities at normal faces:
\begin{eqnarray}
u_{L,\ib-\half\eb_x}^{\mac,*} = u_{L,\ib-\half\eb_x}^{1D} &-& \frac{\dt}{4h}\left(v_{\ib-\eb_x+\half\eb_y}^{1D}+v_{\ib-\eb_x-\half\eb_y}^{1D}\right)\left(u_{\ib-\eb_x+\half\eb_y}^{y|z}-u_{\ib-\eb_x-\half\eb_y}^{y|z}\right) \nonumber \\
&-& \frac{\dt}{4h}\left(w_{\ib-\eb_x+\half\eb_z}^{1D}+w_{\ib-\eb_x-\half\eb_z}^{1D}\right)\left(u_{\ib-\eb_x+\half\eb_z}^{z|y}-u_{\ib-\eb_x-\half\eb_z}^{z|y}\right) + \frac{\dt}{2}f_{u,\ib-\eb_x}, \\
u_{R,\ib-\half\eb_x}^{\mac,*} = u_{R,\ib-\half\eb_x}^{1D} &-& \frac{\dt}{4h}\left(v_{\ib+\half\eb_y}^{1D}+v_{\ib-\half\eb_y}^{1D}\right)\left(u_{\ib+\half\eb_y}^{y|z}-u_{\ib-\half\eb_y}^{y|z}\right) \nonumber \\
&-& \frac{\dt}{4h}\left(w_{\ib+\half\eb_z}^{1D}+w_{\ib-\half\eb_z}^{1D}\right)\left(u_{\ib+\half\eb_z}^{z|y}-u_{\ib-\half\eb_z}^{z|y}\right) + \frac{\dt}{2}f_{u,\ib}.
\end{eqnarray}
Then we use the Riemann solver given above for the 2D case (equation
[\ref{umac Riemann Problem}]) to compute
$u_{\ib-\half\eb_x}^{\mac,*}$.  We use an analogous procedure to
obtain $v_{\ib-\half\eb_y}^{\mac,*}$ and
$w_{\ib-\half\eb_z}^{\mac,*}$.

\newpage

%-----------------------------------------------------------------------------
% Varden edge states
%-----------------------------------------------------------------------------
\section{Computing $\Ub^{\edge}$ and $\rho^{\edge}$ in {\tt VARDEN}}
To compute $\Ub^{\edge}$, {\tt VARDEN} uses the exact same algorithm
as the $s^{\edge}$ case in {\tt MAESTRO}.  The algorithm for
$\rho^{\edge}$ in {\tt VARDEN} is slightly different than the
$s^{\edge}$ case in {\tt MAESTRO} since it uses a ``conservative''
formulation.  Here, $s$ is used in place of either $\rho, u, v$, or
$w$ (in 3D).

\subsection{2D Cartesian Case}
The 1D extrapolation is:
\begin{eqnarray}
s_{L,\ib-\half\eb_x}^{1D} &=& s_{\ib-\eb_x}^n + \left(\half - \frac{\dt}{2h}u_{\ib-\half\eb_x}^{\mac}\right)\Delta_x s_{\ib-\eb_x}^n, \label{varden s_L^1D}\\
s_{R,\ib-\half\eb_x}^{1D} &=& s_{\ib} - \left(\half + \frac{\dt}{2h}u_{\ib-\half\eb_x}^{\mac}\right)\Delta_x s_{\ib}^n. \label{varden s_R^1D}
\end{eqnarray}
Then we upwind based on $u^{\mac}$:
\begin{equation}
s_{\ib-\half\eb_x}^{1D} =
\begin{cases}
\half\left(s_{L,\ib-\half\eb_x}^{1D} + s_{R,\ib-\half\eb_x}^{1D}\right), & \left|u^{\mac}_{\ib-\half\eb_x}\right| < \epsilon \\
s_{L,\ib-\half\eb_x}^{1D}, & u^{\mac}_{\ib-\half\eb_x} > 0, \\
s_{R,\ib-\half\eb_x}^{1D}, & u^{\mac}_{\ib-\half\eb_x} < 0. \\
\end{cases}
\end{equation}
We use an analogous procedure to obtain $s_{\ib-\half\eb_y}^{1D}$.
Now we do a full-dimensional extrapolation of $s$ to each face.  The
extrapolation of a ``non-conserved'' $s$ to x-faces is:
\begin{eqnarray}
s_{L,\ib-\half\eb_x}^{\edge} &=& s_{L,\ib-\half\eb_x}^{1D} - \frac{\dt}{4h}\left(v_{\ib-\eb_x+\half\eb_y}^{\mac}+v_{\ib-\eb_x-\half\eb_y}^{\mac}\right)\left(s_{\ib-\eb_x+\half\eb_y}^{1D} - s_{\ib-\eb_x-\half\eb_y}^{1D}\right) + \frac{\dt}{2}f_{s,\ib-\eb_x}, \\
s_{R,\ib-\half\eb_x}^{\edge} &=& s_{R,\ib-\half\eb_x}^{1D} - \frac{\dt}{4h}\left(v_{\ib+\half\eb_y}^{\mac}+v_{\ib-\half\eb_y}^{\mac}\right)\left(s_{\ib+\half\eb_y}^{1D} - s_{\ib-\half\eb_y}^{1D}\right) + \frac{\dt}{2}f_{s,\ib}.
\end{eqnarray}
The extrapolation of a ``conserved'' $s$ to x-faces is:
\begin{eqnarray}
s_{L,\ib-\half\eb_x}^{\edge} = s_{L,\ib-\half\eb_x}^{1D} &-& \frac{\dt}{2h}\left[(s^{1D} v^{\mac})_{\ib-\eb_x+\half\eb_y} - (s^{1D} v^{\mac})_{\ib-\eb_x-\half\eb_y}\right] \nonumber \\
&-& \frac{\dt}{2}s_{\ib-\eb_x}(\nabla\cdot\Ub^{\mac})_{\ib-\eb_x} + \frac{\dt}{2h}s_{\ib-\eb_x}\left(v_{\ib-\eb_x+\half\eb_y}^{\mac} - v_{\ib-\eb_x-\half\eb_y}^{\mac}\right) + \frac{\dt}{2}f_{s,\ib-\eb_x}, \\
s_{R,\ib-\half\eb_x}^{\edge} = s_{R,\ib-\half\eb_x}^{1D} &-& \frac{\dt}{2h}\left[(s^{1D} v^{\mac})_{\ib+\half\eb_y} - (s^{1D} v^{\mac})_{\ib-\half\eb_y}\right] \nonumber \\
&-& \frac{\dt}{2}s_{\ib}(\nabla\cdot\Ub^{\mac})_{\ib} + \frac{\dt}{2h}s_{\ib}\left(v_{\ib+\half\eb_y}^{\mac} - v_{\ib-\half\eb_y}^{\mac}\right) + \frac{\dt}{2}f_{s,\ib}.
\end{eqnarray}
Then we upwind based on $u^{\mac}$.
\begin{equation}
s_{\ib-\half\eb_x}^{\edge} =
\begin{cases}
\half\left(s_{L,\ib-\half\eb_x}^{\edge} + s_{R,\ib-\half\eb_x}^{\edge}\right), & \left|u^{\mac}_{\ib-\half\eb_x}\right| < \epsilon \\
s_{L,\ib-\half\eb_x}^{\edge}, & u^{\mac}_{\ib-\half\eb_x} > 0, \\
s_{R,\ib-\half\eb_x}^{\edge}, & u^{\mac}_{\ib-\half\eb_x} < 0.
\end{cases}\label{varden s^edge upwind}
\end{equation}
We use an analogous procedure to compute $s_{\ib-\half\eb_y}^{\edge}$.

\subsection{3D Cartesian Case}
This is more complicated than the 2D case because we include corner
coupling.  We first compute $s_{\ib-\half\eb_x}^{1D}$,
$s_{\ib-\half\eb_y}^{1D}$, and $s_{\ib-\half\eb_z}^{1D}$ in an
analogous manner to equations (\ref{varden s_L^1D}) and 
(\ref{varden s_R^1D}).  Then we compute six intermediate states,
$s_{\ib-\half\eb_x}^{x|y}, s_{\ib-\half\eb_x}^{x|z},
s_{\ib-\half\eb_y}^{y|x}, s_{\ib-\half\eb_y}^{y|z},
s_{\ib-\half\eb_z}^{z|x}$, and $s_{\ib-\half\eb_z}^{z|y}$.  For the
``non-conservative case'', we use, for example:
\begin{eqnarray}
s_{L,\ib-\half\eb_x}^{x|y} &=& s_{L,\ib-\half\eb_x}^{1D} - \frac{\dt}{6h}\left(v_{\ib-\eb_x+\half\eb_y}^{\mac} + v_{\ib-\eb_x-\half\eb_y}^{\mac}\right)\left(s_{\ib-\eb_x+\half\eb_y}^{1D} - s_{\ib-\eb_x-\half\eb_y}^{1D}\right), \\
s_{R,\ib-\half\eb_x}^{x|y} &=& s_{R,\ib-\half\eb_x}^{1D} - \frac{\dt}{6h}\left(v_{\ib+\half\eb_y}^{\mac} + v_{\ib-\half\eb_y}^{\mac}\right)\left(s_{\ib+\half\eb_y}^{1D} - s_{\ib-\half\eb_y}^{1D}\right).
\end{eqnarray}
For the ``conservative'' case, we use, for example:
\begin{eqnarray}
s_{L,\ib-\half\eb_x}^{x|y} &=& s_{L,\ib-\half\eb_x}^{1D} - \frac{\dt}{3h}\left[(sv^{\mac})_{\ib-\eb_x+\half\eb_y}-(sv^{\mac})_{\ib-\eb_x-\half\eb_y}\right], \\
s_{R,\ib-\half\eb_x}^{x|y} &=& s_{R,\ib-\half\eb_x}^{1D} - \frac{\dt}{3h}\left[(sv^{\mac})_{\ib+\half\eb_y}-(sv^{\mac})_{\ib-\half\eb_y}\right].
\end{eqnarray}
Then we upwind based on $u^{\mac}$:
\begin{equation}
s_{\ib-\half\eb_x}^{x|y} =
\begin{cases}
\half\left(s_{L,\ib-\half\eb_x}^{x|y} + s_{R,\ib-\half\eb_x}^{x|y}\right), & \left|u^{\mac}_{\ib-\half\eb_x}\right| < \epsilon \\
s_{L,\ib-\half\eb_x}^{x|y}, & u^{\mac}_{\ib-\half\eb_x} > 0, \\
s_{R,\ib-\half\eb_x}^{x|y}, & u^{\mac}_{\ib-\half\eb_x} < 0.
\end{cases}
\end{equation}
We use an analogous procedure to compute the other five intermediate
states.  Now we do a full-dimensional extrapolation of $s$ to each
face.  The extrapolation of a ``non-conserved'' $s$ to x-faces is:
\begin{eqnarray}
s_{L,\ib-\half\eb_x}^{\edge} = s_{L,\ib-\half\eb_x}^{1D} &-& \frac{\dt}{4h}\left(v_{\ib-\eb_x+\half\eb_y}^{\mac}+v_{\ib-\eb_x-\half\eb_y}^{\mac}\right)\left(s_{\ib-\eb_x+\half\eb_y}^{y|z}-s_{\ib-\eb_x-\half\eb_y}^{y|z}\right) \nonumber \\
&-& \frac{\dt}{4h}\left(w_{\ib-\eb_x+\half\eb_z}^{\mac}+w_{\ib-\eb_x-\half\eb_z}^{\mac}\right)\left(s_{\ib-\eb_x+\half\eb_z}^{z|y}-s_{\ib-\eb_x-\half\eb_z}^{z|y}\right) \nonumber \\
&+& \frac{\dt}{2}f_{s,\ib-\eb_x}, \\
s_{R,\ib-\half\eb_x}^{\edge} = s_{R,\ib-\half\eb_x}^{1D} &-& \frac{\dt}{4h}\left(v_{\ib+\half\eb_y}^{\mac}+v_{\ib-\half\eb_y}^{\mac}\right)\left(s_{\ib+\half\eb_y}^{y|z}-s_{\ib-\half\eb_y}^{y|z}\right) \nonumber \\
&-& \frac{\dt}{4h}\left(w_{\ib+\half\eb_z}^{\mac}+w_{\ib-\half\eb_z}^{\mac}\right)\left(s_{\ib+\half\eb_z}^{z|y}-s_{\ib-\half\eb_z}^{z|y}\right) \nonumber \\
&+& \frac{\dt}{2}f_{s,\ib}.
\end{eqnarray}
The extrapolation of a ``conserved'' $s$ to x-faces is:
\begin{eqnarray}
s_{L,\ib-\half\eb_x}^{\edge} = s_{L,\ib-\half\eb_x}^{1D} &-& \frac{\dt}{2h}\left[(s^{y|z}v^{\mac})_{\ib-\eb_x+\half\eb_y}-({s^{y|z}v^{\mac})_{\ib-\eb_x-\half\eb_y}}\right] \nonumber \\
&-& \frac{\dt}{2h}\left[(s^{z|y}w^{\mac})_{\ib-\eb_x+\half\eb_z}-({s^{z|y}w^{\mac})_{\ib-\eb_x-\half\eb_z}}\right] \nonumber \\
&-& \frac{\dt}{2}s_{\ib-\eb_x}(\nabla\cdot\Ub^{\mac})_{\ib-\eb_x} \nonumber \\
&+& \frac{\dt}{2h}s_{\ib-\eb_x}\left(v_{\ib-\eb_x+\half\eb_y}^{\mac}-v_{\ib-\eb_x-\half\eb_y}^{\mac}+w_{\ib-\eb_x+\half\eb_z}^{\mac}-w_{\ib-\eb_x-\half\eb_z}^{\mac}\right) \nonumber \\
&+& \frac{\dt}{2}f_{s,\ib-\eb_x}, \\
s_{R,\ib-\half\eb_x}^{\edge} = s_{R,\ib-\half\eb_x}^{1D} &-& \frac{\dt}{2h}\left[(s^{y|z}v^{\mac})_{\ib+\half\eb_y}-({s^{y|z}v^{\mac})_{\ib-\half\eb_y}}\right] \nonumber \\
&-& \frac{\dt}{2h}\left[(s^{z|y}w^{\mac})_{\ib+\half\eb_z}-({s^{z|y}w^{\mac})_{\ib-\half\eb_z}}\right] \nonumber \\
&-& \frac{\dt}{2}s_{\ib}(\nabla\cdot\Ub^{\mac})_{\ib} \nonumber \\
&+& \frac{\dt}{2h}s_{\ib}\left(v_{\ib+\half\eb_y}^{\mac}-v_{\ib-\half\eb_y}^{\mac}+w_{\ib+\half\eb_z}^{\mac}-w_{\ib-\half\eb_z}^{\mac}\right) \nonumber \\
&+& \frac{\dt}{2}f_{s,\ib}.
\end{eqnarray}
Then we upwind based on $u^{\mac}$, as in equation 
(\ref{varden s^edge upwind}).  
We use an analogous procedure to compute both
$s_{\ib-\half\eb_y}^{\edge}$ and $s_{\ib-\half\eb_z}$.

\newpage

%-----------------------------------------------------------------------------
% ESTATE_FPU
%-----------------------------------------------------------------------------
\section{ESTATE\_FPU in GODUNOV\_2D/3D.f}
First, the normal predictor.
\begin{eqnarray}
s_L^x &=& s_{\ib-\eb_x} + \left(\half - \frac{\dt}{h_x}\text{UEDGE}_{\ib-\half\eb_x}\right)\Delta^x s_{\ib-\eb_x} + \underbrace{\frac{\dt}{2}\text{TFORCES}_{\ib-\eb_x}}_{\text{IF USE\_MINION}} \\
s_R^x &=& s_{\ib} - \left(\half + \frac{\dt}{h_x}\text{UEDGE}_{\ib-\half\eb_x}\right)\Delta^x s_{\ib} + \underbrace{\frac{\dt}{2}\text{TFORCES}_{\ib}}_{\text{IF USE\_MINION}}
\end{eqnarray}
{\bf If} USE\_MINION {\bf and} ICONSERVE {\bf then:}
\begin{eqnarray}
s_L^x &=& s_L^x - \frac{\dt}{2}s_{\ib-\eb_x}\text{DIVU}_{\ib-\eb_x} \\
s_R^x &=& s_R^x - \frac{\dt}{2}s_{\ib}\text{DIVU}_{\ib}
\end{eqnarray}
Apply boundary conditions on $s_L^x$ and $s_R^x$.  Then,
\begin{equation}
\text{s}_{\ib-\half\eb_x}^x =
\begin{cases}
s_L^x, & \text{UEDGE}_{\ib-\half\eb_x} > 0, \\
s_R^x, & \text{else}. \\
\end{cases}\label{ESTATE_FPU Upwind}
\end{equation}
Then, if $|\text{UEDGE}_{\ib-\half\eb_x}| \le \epsilon$, we set $s_{\ib-\half\eb_x}^x = (s_L^x+s_R^x)/2$.  The procedure to obtain $s_{\ib-\half\eb_y}^y$ is analogous.\\ \\
Now, the transverse terms.\\ \\
{\bf If} ICONSERVE {\bf then:}
\begin{eqnarray}
\text{sedge}_L^x &=& s_{\ib-\eb_x} + \left(\half - \frac{\dt}{h_x}\text{UEDGE}_{\ib-\half\eb_x}\right)\Delta^x s_{\ib-\eb_x} + \frac{\dt}{2}\text{TFORCES}_{\ib-\eb_x} \nonumber\\
&& - \frac{\dt}{2}\left[\frac{\text{VEDGE}_{\ib-\eb_x+\half\eb_y}s_{\ib-\eb_x+\half\eb_y}^y - \text{VEDGE}_{\ib-\eb_x-\half\eb_y}s_{\ib-\eb_x-\half\eb_y}^y}{h_y}\right.\nonumber\\
&& ~~~~~~~~~~ \left. - \frac{s_{\ib-\eb_x}(\text{VEDGE}_{\ib-\eb_x+\half\eb_y}-\text{VEDGE}_{\ib-\eb_x-\half\eb_y})}{h_y}+s_{\ib-\eb_x}\text{DIVU}_{\ib-\eb_x}\right]\\
\text{sedge}_R^x &=& s_{\ib} - \left(\half + \frac{\dt}{h_x}\text{UEDGE}_{\ib-\half\eb_x}\right)\Delta^x s_{\ib} + \frac{\dt}{2}\text{TFORCES}_{\ib} \nonumber\\
&& - \frac{\dt}{2}\left[\frac{\text{VEDGE}_{\ib+\half\eb_y}s_{\ib+\half\eb_y}^y - \text{VEDGE}_{\ib-\half\eb_y}s_{\ib-\half\eb_y}^y}{h_y}\right.\nonumber\\
&& ~~~~~~~~~~ \left. - \frac{s_{\ib}(\text{VEDGE}_{\ib+\half\eb_y}-\text{VEDGE}_{\ib-\half\eb_y})}{h_y}+s_{\ib}\text{DIVU}_{\ib}\right]
\end{eqnarray}
Now, define $\text{VBAR}_{\ib} = (\text{VEDGE}_{\ib+\half\eb_y}+\text{VEDGE}_{\ib-\half\eb_y})/2$.\\ \\
{\bf If} NOT ICONSERVE {\bf and} $\text{VEDGE}_{\ib+\half\eb_y}\cdot\text{VEDGE}_{\ib-\half\eb_y} < 0$ {\bf and} $\text{VBAR}_{\ib} < 0$ {\bf then:}
\begin{eqnarray}
\text{sedge}_L^x = s_{\ib-\eb_x} &+& \left(\half - \frac{\dt}{h_x}\text{UEDGE}_{\ib-\half\eb_x}\right)\Delta^x s_{\ib} + \frac{\dt}{2}\text{TFORCES}_{\ib-\eb_x} \nonumber\\
&& - \frac{\dt}{2}\left[\frac{\text{VBAR}_{\ib-\eb_x}(s_{\ib-\eb_x+\eb_y}-s_{\ib-\eb_x})}{h_y}\right] \label{transverse upwinding 1} \\
\text{sedge}_R^x = s_{\ib} &-& \left(\half + \frac{\dt}{h_x}\text{UEDGE}_{\ib-\half\eb_x}\right)\Delta^x s_{\ib} + \frac{\dt}{2}\text{TFORCES}_{\ib} \nonumber\\
&& - \frac{\dt}{2}\left[\frac{\text{VBAR}_{\ib}(s_{\ib+\eb_y}-s_{\ib})}{h_y}\right]
\end{eqnarray}
{\bf Else If} NOT ICONSERVE {\bf and} $\text{VEDGE}_{\ib+\half\eb_y}\cdot\text{VEDGE}_{\ib-\half\eb_y} < 0$ {\bf and} $\text{VBAR}_{\ib} \ge 0$ {\bf then:}
\begin{eqnarray}
\text{sedge}_L^x = s_{\ib-\eb_x} &+& \left(\half - \frac{\dt}{h_x}\text{UEDGE}_{\ib-\half\eb_x}\right)\Delta^x s_{\ib-\eb_x} + \frac{\dt}{2}\text{TFORCES}_{\ib-\eb_x} \nonumber\\
&& - \frac{\dt}{2}\left[\frac{\text{VBAR}_{\ib-\eb_x}(s_{\ib-\eb_x}-s_{\ib-\eb_x-\eb_y})}{h_y}\right] \\
\text{sedge}_R^x = s_{\ib} &-& \left(\half + \frac{\dt}{h_x}\text{UEDGE}_{\ib-\half\eb_x}\right)\Delta^x s_{\ib} + \frac{\dt}{2}\text{TFORCES}_{\ib} \nonumber\\
&& - \frac{\dt}{2}\left[\frac{\text{VBAR}_{\ib}(s_{\ib}-s_{\ib-\eb_y})}{h_y}\right]
\end{eqnarray}
{\bf Else If} NOT ICONSERVE {\bf and} $\text{VEDGE}_{\ib+\half\eb_y}\cdot\text{VEDGE}_{\ib-\half\eb_y} \ge 0$ {\bf then:}
\begin{eqnarray}
\text{sedge}_L^x &=& s_{\ib-\eb_x} + \left(\half - \frac{\dt}{h_x}\text{UEDGE}_{\ib-\half\eb_x}\right)\Delta^x s_{\ib-\eb_x} + \frac{\dt}{2}\text{TFORCES}_{\ib-\eb_x} \nonumber\\
&& - \frac{\dt}{2}\left[\frac{(\text{VEDGE}_{\ib-\eb_x+\half\eb_y}+\text{VEDGE}_{\ib-\eb_x-\half\eb_y})(s_{\ib-\eb_x+\half\eb_y}-s_{\ib-\eb_x-\half\eb_y})}{2h_y}\right] \\
\text{sedge}_R^x &=& s_{\ib} - \left(\half + \frac{\dt}{h_x}\text{UEDGE}_{\ib-\half\eb_x}\right)\Delta^x s_{\ib} + \frac{\dt}{2}\text{TFORCES}_{\ib} \nonumber\\
&& - \frac{\dt}{2}\left[\frac{(\text{VEDGE}_{\ib+\half\eb_y}+\text{VEDGE}_{\ib-\half\eb_y})(s_{\ib+\half\eb_y}-s_{\ib-\half\eb_y})}{2h_y}\right]\label{transverse upwinding 6}
\end{eqnarray}
Finally, upwind analogous to equation (\ref{ESTATE_FPU Upwind}) to get $\text{sedge}_{\ib-\half\eb_x}$.

\newpage

%-----------------------------------------------------------------------------
% ESTATE
%-----------------------------------------------------------------------------
\section{ESTATE in GODUNOV\_2D/3D.f}
First, the normal predictor.
\begin{eqnarray}
s_L^x &=& s_{\ib-\eb_x} + \left(\half - \frac{\dt}{h_x}u_{\ib-\eb_x}\right)\Delta^x s_{\ib-\eb_x} \\
s_R^x &=& s_{\ib} - \left(\half + \frac{\dt}{h_x}u_{\ib}\right)\Delta^x s_{\ib}
\end{eqnarray}
{\bf If} USE\_MINION {\bf then:}
\begin{eqnarray}
s_L^x &=& s_L^x + \frac{\dt}{2}\text{TFORCES}_{\ib-\eb_x} \\
s_R^x &=& s_R^x + \frac{\dt}{2}\text{TFORCES}_{\ib}
\end{eqnarray}
Apply boundary conditions on $s_L^x$ and $s_R^x$.  Then,
\begin{equation}
\text{s}_{\ib-\half\eb_x}^x =
\begin{cases}
s_L^x, & \text{UAD}_{\ib-\half\eb_x} > 0, \\
s_R^x, & \text{else}. \\
\end{cases}\label{ESTATE Upwind}
\end{equation}
Then, if $|\text{UAD}_{\ib-\half\eb_x}| \le \epsilon$, we set $s_{\ib-\half\eb_x}^x = (s_L^x+s_R^x)/2$.
\begin{eqnarray}
\text{sedge}_L^x = s_{\ib-\eb_x} &+& \left(\half - \frac{\dt}{h_x}u_{\ib-\eb_x}\right)\Delta^x s_{\ib-\eb_x} + \frac{\dt}{2}\text{TFORCES}_{\ib-\eb_x} \nonumber\\
&& - \frac{\dt}{2}\left[\frac{(\text{VAD}_{\ib-\eb_x+\half\eb_y}+\text{VAD}_{\ib-\eb_x-\half\eb_y})(s_{\ib-\eb_x+\half\eb_y}-s_{\ib-\eb_x-\half\eb_y})}{2h_y}\right] \\
\text{sedge}_R^x = s_{\ib} &-& \left(\half + \frac{\dt}{h_x}u_{\ib}\right)\Delta^x s_{\ib} + \frac{\dt}{2}\text{TFORCES}_{\ib} \nonumber\\
&& - \frac{\dt}{2}\left[\frac{(\text{VAD}_{\ib+\half\eb_y}+\text{VAD}_{\ib-\half\eb_y})(s_{\ib+\half\eb_y}-s_{\ib-\half\eb_y})}{2h_y}\right]
\end{eqnarray}
Note that the 2D and 3D algorithms are different - in 3D the transverse 
terms use upwinding analogous to equations 
(\ref{transverse upwinding 1})-(\ref{transverse upwinding 6}), using UAD 
instead of UEDGE.  Finally, upwind analogous to equation (\ref{ESTATE Upwind}) 
to get $\text{sedge}_{\ib-\half\eb_x}$, but use UEDGE instead of UAD.

\newpage

%-----------------------------------------------------------------------------
% PPM
%-----------------------------------------------------------------------------
\section{Piecewise Parabolic Method (PPM)}
Let's consider a scalar, $s$, which we wish to predict to edges.  The PPM method
modifies how we perform the 1D extrapolation to get left and right edge states
(for example, we modify equations (\ref{3D predict s to left}) and 
(\ref{3D predict s to right}) in Section 
\ref{Scalar Edge State Prediction in MAESTRO} to obtain higher-order estimates
of $s_{L/R,\ib-\half}$).  Once these 1D edge states are obtained, we
continue with the full-dimensional extrapolations as described before.\\ \\
The PPM method is described in a series of papers:
\begin{itemize}
\item Colella and Woodward 1984 - describes the basic method
\item Miller and Colella 2002 - describes how to apply PPM to a multidimensional
unsplit Godunov method and generalizes the characteristic tracing for more complicated
systems (thankfully, we only upwind based on the fluid velocity)
\item Colella and Sekora 2008 - describes new fancy quadratic limiters
\end{itemize}
Here are the steps for the $x$-direction.  For simplicity, we replace the vector index notation with a simple scalar notation ($\ib+\eb_x \rightarrow i+1$, etc.):
\begin{itemize}
\item {\bf Step 1:} Spatially interpolate $s$ to edges.\\ \\
Use a 6th-order interpolation in space to obtain edge values:
\begin{equation}
s_{i+\half}^{\text{H.O.}} = \frac{37}{60}\left(s_{i}+s_{i+1}\right) - \frac{2}{15}\left(s_{i-1}+s_{i+2}\right) + \frac{1}{60}\left(s_{i-2}+s_{i+3}\right),
\end{equation}
Next, we must ensure that $s_{i+\half}^{\text{H.O.}}$ lies between the adjacent 
cell-centered values:
\begin{equation}
\min\left(s_{i},s_{i+1}\right) \le s_{i+\half}^{\text{H.O.}} \le \max\left(s_{i},s_{i+1}\right).
\end{equation}
If $s_{i+\half}^{\text{H.O.}}$ does not satisfy this condition, we limit this value
by using a nonlinear combination of approximations to the second derivative.
First, define:
\begin{eqnarray}
(D^2s)_{i+\half} &=& \frac{3}{h^2}\left(s_{i}-2s_{i+\half}^{\text{H.O.}}+s_{i+1}\right) \\
(D^2s)_{i+\half,L} &=& \frac{1}{h^2}\left(s_{i-1}-2s_{i}+s_{i+1}\right) \\
(D^2s)_{i+\half,R} &=& \frac{1}{h^2}\left(s_{i}-2s_{i+1}^{\text{H.O.}}+s_{i+2}\right)
\end{eqnarray}
If the signs of $(D^2s)_{i+s\half}$, $(D^2s)_{i+\half,L/R}$ are all the
same, we define
\begin{equation}
(D^2s)_{i+\half,\text{lim}} = \text{sign}\left[(D^2s)_{i+\half}\right]\min\left[C|(D^2s)_{i+\half,L}|,C|(D^2s)_{i+\half,R}|,|(D^2s)_{i+\half}|\right],
\end{equation}
where $C=1.25$ was used in Colella and Sekora.  Otherwise, 
$(D^2s)_{i+\half,\text{lim}}=0$.  Then,
\begin{equation}
s_{i+\half}^{\text{H.O.}} = \frac{1}{2}\left(s_{i}+s_{i+1}\right) - \frac{h^3}{3}(D^2s)_{i+\half,\text{lim}}.
\end{equation}
In anticipation of further limiting, we set double-valued face-centered values:
\begin{equation}
s_{i,+} = s_{i+1,-} = s_{i+\half}^{\text{H.O.}}.
\end{equation}
\item {\bf Step 1 Alternative:} Spatially interpolate $s$ to edges.  This alternative
requires 3 ghost cells instead of 4 (since the interpolation must be done in the first
ghost cell)\\ \\
Use a 4th-order interpolation in space with van Leer limiting to obtain edge values:
\begin{equation}
s_{i+\half}^{\text{H.O.}} = \frac{1}{2}\left(s_{i} + s_{i+1}\right) - \frac{1}{6}\left(\delta s_{i+1}^{vL} - \delta s_{i}^{vL}\right),
\end{equation}
\begin{equation}
\delta s_i = \frac{1}{2}\left(s_{i+1}-s_{i-1}\right),
\end{equation}
\begin{equation}
\delta s_i^{vL} = \text{sign}(\delta s_i)\min\left(|\delta s_i|, ~ 2|s_{i+1}-s_{i}|, ~ 2|s_i-s_{i-1}|\right).
\end{equation}
Without the limiters, this is the familiar 4th-order spatial interpolation formula:
\begin{equation}
s_{i+\half}^{\text{H.O.}} = \frac{7}{12}\left(s_{i+1}+s_i\right) - \frac{1}{12}\left(s_{i+2}+s_{i-1}\right).
\end{equation}
Next, we must ensure that $s_{i+\half}^{\text{H.O.}}$ lies between the adjacent 
cell-centered values:
\begin{equation}
\min\left(s_{i},s_{i+1}\right) \le s_{i+\half}^{\text{H.O.}} \le \max\left(s_{i},s_{i+1}\right).
\end{equation}
In anticipation of further limiting, we set double-valued face-centered values:
\begin{equation}
s_{i,+} = s_{i+1,-} = s_{i+\half}^{\text{H.O.}}.
\end{equation}
\item {\bf Step 2:} Modify $s_{i,\pm}$ using a discontinuity detector.\\ \\
This is done in Colella and Woodward, and I don't think we need it.
\item {\bf Step 3:} Modify $s_{i,\pm}$ using a quadratic limiter.\\ \\
{\bf First limiter:} There are two quadratic limiters.  
For the first limiter, we test whether 
$s_i$ is a local extreumum with the conditions:
\begin{equation}
\left(s_{i,+}-s_i\right)\left(s_i-s_{i,-}\right) \le 0,
\end{equation}
\begin{equation}
\left(s_{i+1}-s_{i}\right)\left(s_{i}-s_{i-1}\right) \le 0.
\end{equation}
If either of these are true, we constrain $s_{i,\pm}$ in a less restrictive 
way than originally used in Colella and Woodward.  First, we define:
\begin{eqnarray}
(D^2s)_i &=& -\frac{2s_{6,i}}{h^2}; ~~~~~~~~~ s_6 = 6s_{i} - 3\left(s_{i,-}+s_{i,+}\right) \\
(D^2s)_{i,C} &=& \frac{1}{h^2}\left(s_{i-1}-2s_{i}+s_{i+1}\right) \\
(D^2s)_{i,L} &=& \frac{1}{h^2}\left(s_{i-2}-2s_{i-1}+s_{i}\right) \\
(D^2s)_{i,R} &=& \frac{1}{h^2}\left(s_{i}-2s_{i+1}+s_{i+2}\right)
\end{eqnarray}
If the signs of $(D^2s)_i$, $(D^2s)_{i,C/L/R}$ are all the
same, we define
\begin{equation}
(D^2s)_{i,\text{lim}} = \text{sign}\left[(D^2s)_{i}\right]\min\left[C|(D^2s)_{i,C}|,C|(D^2s)_{i,L}|,C|(D^2s)_{i,R}|,|(D^2s)_{i}|\right],
\end{equation}
where $C=1.25$ was used in Colella and Sekora.  Otherwise, 
$(D^2s)_{i+\half,\text{lim}}=0$.  Then,
\begin{equation}
s_{i,\pm} = s_i + \left(s_{i,\pm}-s_{i}\right)\frac{(D^2s)_{i,\text{lim}}}{(D^2s)_i}.
\end{equation}
If $(D^2s)_i=0$, then we set $s_{i,\pm} = s_i$.\\ \\
{\bf Second limiter:} If we do not use the first limiter, we then apply a 
second test to determine 
whether $s_i$ is sufficiently close to $s_{i,\pm}$ so that a quadratic 
interpolate would contain a local extremum.  We define 
$\alpha_{i,\pm} = s_{i,\pm} - s_i$.  If one of $|\alpha_{i,\pm}| \ge 2|\alpha_{i,\mp}|$
holds, then for that choice of $\pm = +,-$ we compute:
\begin{equation}
\delta\mathcal{I}_{\text{ext}} = \frac{-\alpha_{i,\pm}^2}{4\left(\alpha_{j,+}+\alpha_{j,-}\right)},
\end{equation}
\begin{equation}
\delta s = s_{i\pm 1} - s_i,
\end{equation}
\begin{equation}
s = \text{sign}\left(s_{i+1}-s_{i-1}\right).
\end{equation}
If $s ~ \delta\mathcal{I}_{\text{ext}} \ge s ~ \delta s$, then we set
\begin{equation}
s_{i,\pm} = s_i - \left\{2\delta s + 2s\left[(\delta s)^2 - \delta s ~  \alpha_{i,\mp}\right]^{\myhalf}\right\}.
\end{equation}
\item {\bf Step 3 Alternative:} Modify $s_{i,\pm}$ using a quadratic limiter.
Here are the simpler quadratic limiters.\\ \\
{\bf First limiter:} There are two quadratic limiters.  
For the first limiter, we test whether 
$s_i$ is a local extreumum with the condition:
\begin{equation}
\left(s_{i,+}-s_i\right)\left(s_i-s_{i,-}\right) \le 0,
\end{equation}
If this condition is true, we constrain $s_{i,\pm}$ by setting 
$s_{i,+} = s_{i,-} = s_i$\\ \\
{\bf Second limiter:} If we do not use the first limiter, we then apply a 
second test to determine 
whether $s_i$ is sufficiently close to $s_{i,\pm}$ so that a quadratic 
interpolate would contain a local extremum.  We define 
$\alpha_{i,\pm} = s_{i,\pm} - s_i$.  If one of $|\alpha_{i,\pm}| \ge 2|\alpha_{i,\mp}|$
holds, then for that choice of $\pm = +,-$ we set:
\begin{equation}
s_{i,\pm} = 3s_i - 2s_{i,\mp}.
\end{equation}
\item {\bf Step 4:} Use flattening.\\ \\
This is done in Miller and Colella, and I don't think we need it.
\item {\bf Step 5:} Construct quadratic profiles.\\ \\
\begin{equation}
s_i^I(x) = s_{i,-} + \xi\left[s_{i,+} - s_{i,-} + s_{6,i}(1-\xi)\right], ~~~ s_6 = 6s_{i} - 3\left(s_{i,-}+s_{i,+}\right),\label{Quadratic Interp}
\end{equation}
\begin{equation}
\xi = \frac{x - ih}{h}, ~ 0 \le \xi \le 1.
\end{equation}
\item {\bf Step 6:} Integrate quadratic profiles.\\ \\
Define the following integrals, where $\sigma = |u|\Delta t/h$:
\begin{eqnarray}
\mathcal{I}_{i,+}(\sigma) &=& \frac{1}{\sigma h}\int_{(i+\half)h-\sigma h}^{(i+\half)h}s_i^I(x)dx \\
\mathcal{I}_{i,-}(\sigma) &=& \frac{1}{\sigma h}\int_{(i-\half)h}^{(i-\half)h+\sigma h}s_i^I(x)dx
\end{eqnarray}
Plugging in (\ref{Quadratic Interp}) gives:
\begin{eqnarray}
\mathcal{I}_{i,+}(\sigma) &=& s_{j,+} - \frac{\sigma}{2}\left[s_{j,+}-s_{j,-}-\left(1-\frac{2}{3}\sigma\right)s_{6,i}\right], \\
\mathcal{I}_{i,-}(\sigma) &=& s_{j,-} + \frac{\sigma}{2}\left[s_{j,+}-s_{j,-}+\left(1-\frac{2}{3}\sigma\right)s_{6,i}\right].
\end{eqnarray}
\item {\bf Step 7:} Obtain 1D edge states.\\ \\
Perform a 1D extrapolation, without source terms, to get 
left and right edge states.  Add the source terms later if desired/necessary.\\ \\
\begin{eqnarray}
s_{L,\ib-\half\eb_x} &=&
\begin{cases}
s_{\ib-\eb_x} + \mathcal{I}_{\ib-\eb_x,+}(\sigma), & u_{\ib-\eb_x} ~ \text{or} ~ u_{\ib-\half\eb_x}^{\mac} > 0 \\
s_{\ib-\eb_x}, & \text{else}.
\end{cases}\\
s_{R,\ib-\half\eb_x} &=& 
\begin{cases}
s_{\ib} + \mathcal{I}_{\ib,-}(\sigma), & u_{\ib} ~ \text{or} ~ u_{\ib-\half\eb_x}^{\mac} < 0 \\
s_{\ib}, & \text{else}.
\end{cases}
\end{eqnarray}
\end{itemize}
