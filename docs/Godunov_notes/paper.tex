\documentclass[11pt]{article} 

\tolerance=600

\usepackage{amsmath}

% Margins
\usepackage[lmargin=1.0in,rmargin=1.0in,tmargin=0.75in,bmargin=0.75in]{geometry}

\def\half  {\frac{1}{2}}
\def\dt    {\Delta t}

\def\edge  {\rm EDGE}
\def\mac   {\rm MAC}
\def\trans {\rm TRANS}

\def\eb    {{\bf e}}
\def\ib    {{\bf i}}
\def\Ub    {{\bf U}}
\def\er    {{\bf e}_r}

\def\Ubt   {\widetilde{\bf U}}
\def\ut    {\tilde{u}}
\def\vt    {\tilde{v}}
\def\wt    {\tilde{w}}

\title{Notes on the Godunov Step in {\tt MAESTRO}}

\begin{document}
\maketitle

These are working notes for the Godunov step in {\tt MAESTRO}.

\section{Computing $\Ub^{\trans}$}
In {\tt advance\_premac.f90}, we call the function {\tt mkutrans}, which computes the edge-centered transverse velocities, $\Ubt^{\trans}$.  These transverse velocities do not contain $w_0$, so immediately following the call to {\tt mkutrans}, we call {\tt addw0} to compute $\Ub^{\trans}$.  We will only compute the normal component of velocity at each face.\\

The evolution equation for the perturbational velocity is:
\begin{equation}
\frac{\partial\Ubt}{\partial t} = -\Ubt\cdot\nabla\Ubt - w_0\frac{\partial\Ubt}{\partial r} - (\Ubt\cdot\eb_r)\frac{\partial w_0}{\partial r}\eb_r - \frac{1}{\rho}\nabla\pi + \frac{1}{\rho_0}\frac{\partial\pi_0}{\partial r}\eb_r - \frac{(\rho-\rho_0)}{\rho}g\eb_r.\label{Perturbational Velocity Equation}
\end{equation}
We are going to use a 1D Taylor series extrapolation in space and time to compute time-centered normal components of $\Ubt^{\trans}$ at face-centers.  By 1D, we mean that we omit any spatial derivatives that are not in the direction of the extrapolation.  We first compute ``left/right'' states, and then use a Riemann solver to pick the final state.\\
\subsection{2D Cartesian Case}
For the rest of this paper, for 2D Cartesian problems, $\Ub = (u,w)$ and $\Ubt = (u,\wt)$.  We will use the shorthand $\ib = (i,r)$.\\

For the x-faces, the extrapolation is as follows:
\begin{eqnarray}
u_L &=& u_{\ib-\eb_x}^n + \left(\half - \frac{\dt}{h}u_{\ib-\eb_x}^n\right)\Delta_x u_{\ib-\eb_x}^n, \\
u_R &=& u_{\ib}^n - \left(\half + \frac{\dt}{h}u_{\ib}^n\right)\Delta_x u_{\ib}^n.
\end{eqnarray}
We pick the final edge states using a Riemann solver:
\begin{equation}
u^{\trans}_{\ib-\half\eb_x} =
\begin{cases}
0, & (u_L \le 0 ~ {\rm AND} ~ u_R \ge 0) ~ {\rm OR} ~ |u_L + u_R| < \epsilon, \\
u_L, & u_L + u_R > 0, \\
u_R, & u_L + u_R < 0, \\
\end{cases}
\end{equation}
For the radial faces, we must include the radial derivative terms proportional to $w_0$ and $(\Ubt\cdot\eb_r) = \wt$.  The term proportional to $w_0$ is included by using $w$ rather than $\wt$ as the advection speed.  The term proportional to $\wt$ is an additional term:
\begin{eqnarray}
\wt_L &=& \wt_{\ib-\eb_r}^n + \left(\half - \frac{\dt}{h}w_{\ib-\eb_r}^n\right)\Delta_r \wt_{\ib-\eb_r}^n + \frac{\dt}{2}\wt_{\ib-\eb_r}^n\left(\frac{\partial w_0}{\partial r}\right)_{\ib-\eb_r},\label{Radial utrans_L} \\
\wt_R &=& \wt_{\ib}^n + \left(\half - \frac{\dt}{h}w_{\ib}^n\right)\Delta_r \wt_{\ib}^n + \frac{\dt}{2}\wt_{\ib}^n\left(\frac{\partial w_0}{\partial r}\right)_{\ib}.\label{Radial utrans_R}
\end{eqnarray}
The total radial velocity is defined as:
\begin{equation}
w_{\ib}^n = \wt_{\ib}^n + \frac{w_{0,r+\half}+w_{0,r-\half}}{2}.\label{Total Radial Velocity}
\end{equation}
The spatial derivatives of $w_0$ are given by the two-point centered difference:
\begin{equation}
\left(\frac{\partial w_0}{\partial r}\right)_{\ib} = \frac{w_{0,r+\half}-w_{0,r-\half}}{h}.\label{Radial Derivative of w_0}
\end{equation}
We pick the final edge states using:
\begin{equation}
\wt^{\trans}_{\ib-\half\eb_r} =
\begin{cases}
0, & (w_L \le 0 ~ {\rm AND} ~ w_R \ge 0) ~ {\rm OR} ~ |w_L + w_R| < \epsilon, \\
\wt_L, & w_L + w_R > 0, \\
\wt_R, & w_L + w_R < 0, \\
\end{cases}
\end{equation}
where $w_L = \wt_L + w_{0,r-\half}$ and $w_R = \wt_R + w_{0,r-\half}$.  As mentioned before, we now call {\tt addw0} to convert $\wt^{\trans}$ to $w^{\trans}.$
\subsection{3D Cartesian Case}
For the rest of this paper, for 3D Cartesian problems, $\Ub = (u,v,w)$ and $\Ubt = (u,v,\wt)$.  We will use the shorthand $\ib = (i,j,r)$.\\

The only difference between the 2D and 3D Cartesian cases is that we must also compute $v^{\trans}$.  The procedure is analogous to computing $u^{\trans}$:
\begin{eqnarray}
v_L &=& v_{\ib-\eb_y}^n + \left(\half - \frac{\dt}{h}v_{\ib-\eb_y}^n\right)\Delta_y v_{\ib-\eb_y}^n, \\
v_R &=& v_{\ib}^n - \left(\half + \frac{\dt}{h}v_{\ib}^n\right)\Delta_y v_{\ib}^n,
\end{eqnarray}
\begin{equation}
v^{\trans}_{\ib-\half\eb_y} =
\begin{cases}
0, & (v_L \le 0 ~ {\rm AND} ~ v_R \ge 0) ~ {\rm OR} ~ |v_L + v_R| < \epsilon, \\
v_L, & v_L + v_R > 0, \\
v_R, & v_L + v_R < 0. \\
\end{cases}
\end{equation}

\subsection{3D Spherical Case}
For the rest of this paper, for 3D spherical problems, $\Ub = (u,v,w)$ and $\Ubt = (\ut,\vt,\wt)$.  We will use the shorthand $\ib = (i,j,k)$.\\

For spherical problems, we first construct $\Ub$ by mapping $w_0$ onto a Cartesian array and adding it to $\Ubt$.  This is a spherical generalization of equation (\ref{Total Radial Velocity}).  We also map $\partial w_0/\partial r$ onto a Cartesian array.  This is a spherical generalization of equation (\ref{Radial Derivative of w_0}).  Then we use an extrapolation analogous to equations equations (\ref{Radial utrans_L}) and (\ref{Radial utrans_R}).  For example, for x-faces:
\begin{eqnarray}
\ut_L &=& \ut_{\ib-\eb_x}^n + \left(\half - \frac{\dt}{h}u_{\ib-\eb_x}^n\right)\Delta_x\ut_{\ib-\eb_x}^n + \frac{\dt}{2}\left[(\Ubt^n\cdot\eb_r)\left(\frac{\partial w_0}{\partial r}\right)(\eb_r\cdot\eb_x)\right]_{\ib-\eb_x}, \\
\ut_R &=& \ut_{\ib}^n + \left(\half - \frac{\dt}{h}u_{\ib}^n\right)\Delta_x\ut_{\ib}^n + \frac{\dt}{2}\left[(\Ubt^n\cdot\eb_r)\left(\frac{\partial w_0}{\partial r}\right)(\eb_r\cdot\eb_x)\right]_{\ib}.
\end{eqnarray}
Then, we use the same Riemann solver as the Cartesian case to obtain $\ut^{\trans}$.  To compute $\vt^{\trans}$ or $\wt^{\trans}$, simply replace all instaces of $\ut$ with $\vt$ (or $\wt$) and all instaces of $x$ with $y$ (or $z$).  Finally, we call {\tt addw0} to convert $\Ubt^{\trans}$ to $\Ub^{\trans}$.
\section{Computing $\Ub^{\mac}$}
In {\tt advance\_premac.f90}, we call {\tt make\_edge\_state} to compute $\Ubt^{\mac}$.
\subsection{2D Cartesian Case}
Here are the steps to create $u^{\mac}$ for 2D Cartesian problems.  First, we need to compute a time-centered $u$ at transverse faces:
\begin{eqnarray}
u_L &=& u_{\ib-\eb_y}^n + \left(\half - \frac{\dt}{2h}v_{\ib-\eb_y}^n\right)\Delta_y u_{\ib-\eb_y}^n, \\
u_R &=& u_{\ib} - \left(\half + \frac{\dt}{2h}v_{\ib}^n\right)\Delta_y u_{\ib}^n.
\end{eqnarray}
Then we solve a Riemann problem:
\begin{equation}
u_{\ib-\half\eb_y} =
\begin{cases}
\half(u_L + u_R), & \left|v^{\trans}_{\ib-\half\eb_y}\right| < \epsilon \\
u_L, & v^{\trans}_{\ib-\half\eb_y} > 0, \\
u_R, & v^{\trans}_{\ib-\half\eb_y} < 0. \\
\end{cases}
\end{equation}
Then, we extrapolate the normal velocity to normal faces:
\begin{eqnarray}
u_L^{\edge} &=& u_{\ib-\eb_x}^n + \left(\half - \frac{\dt}{2h}u_{\ib-\eb_x}\right)\Delta_x u_{\ib-\eb_x}^n + \frac{\dt}{2}f_{\ib-\eb_x} \nonumber \\
&& - \frac{\dt}{4h}\left(v_{\ib-\eb_x+\half\eb_y}^{\trans}+v_{\ib-\eb_x-\half\eb_y}^{\trans}\right)\left(u_{\ib-\eb_x+\half\eb_y}+u_{\ib-\eb_x-\half\eb_y}\right), \\
u_R^{\edge} &=& u_{\ib}^n + \left(\half - \frac{\dt}{2h}u_{\ib}\right)\Delta_x u_{\ib}^n + \frac{\dt}{2}f_{\ib} \nonumber \\
&& - \frac{\dt}{4h}\left(v_{\ib+\half\eb_y}^{\trans}+v_{\ib-\half\eb_y}^{\trans}\right)\left(u_{\ib+\half\eb_y}+u_{\ib-\half\eb_y}\right). \\
\end{eqnarray}
Finally, we solve a Riemann problem:
\begin{equation}
u_{\ib-\half\eb_x}^{\mac} =
\begin{cases}
0, & \left(u_L^{\edge} \le 0 ~ {\rm AND} ~ u_R^{\edge} \ge 0\right) ~ {\rm OR} ~ \left|u_L^{\edge} + u_R^{\edge}\right| < \epsilon, \\
u_L^{\edge}, & u_L^{\edge} + u_R^{\edge} > 0, \\
u_R^{\edge}, & u_L^{\edge} + u_R^{\edge} < 0. 
\end{cases}
\end{equation}

To compute $\vt^{\mac}$, we use an analogous procedure for computing $u^{\mac}$, except that we need to incorporate the terms in equation (\ref{Perturbational Velocity Equation}) proportional to $\partial w_0/\partial r$.  First, we need to compute a time-centered $\vt$ at transverse faces:
\begin{eqnarray}
\vt_L &=& \vt_{\ib-\eb_x}^n + \left(\half - \frac{\dt}{2h}u_{\ib-\eb_x}^n\right)\Delta_x \vt_{\ib-\eb_x}^n, \\
\vt_R &=& \vt_{\ib} - \left(\half + \frac{\dt}{2h}u_{\ib}^n\right)\Delta_x \vt_{\ib}^n.
\end{eqnarray}
Then we solve a Riemann problem:
\begin{equation}
\vt_{\ib-\half\eb_x} =
\begin{cases}
\half(\vt_L + \vt_R), & \left|u^{\trans}_{\ib-\half\eb_x}\right| < \epsilon \\
\vt_L, & u^{\trans}_{\ib-\half\eb_x} > 0, \\
\vt_R, & u^{\trans}_{\ib-\half\eb_x} < 0. \\
\end{cases}
\end{equation}
Then, we extrapolate the normal velocity to normal faces, incorporating the extra term proportional to $\partial w_0/\partial r$\footnote{The underbraced terms could be replaced by the average of the neighboring $v^{\trans}$.  We still don't know what the best option is.}:
\begin{eqnarray}
\vt_L^{\edge} &=& \vt_{\ib-\eb_y}^n + \left(\half - \frac{\dt}{2h}v_{\ib-\eb_y}\right)\Delta_y \vt_{\ib-\eb_y}^n + \frac{\dt}{2}f_{\ib-\eb_y} \nonumber \\
&& - \frac{\dt}{4h}\left(u_{\ib+\half\eb_x-\eb_y}^{\trans}+u_{\ib-\half\eb_x-\eb_y}^{\trans}\right)\left(\vt_{\ib+\half\eb_x-\eb_y}+\vt_{\ib-\half\eb_x-\eb_y}\right) \nonumber \\
&& - \frac{\dt}{2h}\underbrace{\vt_{\ib-\eb_y}}_{?}(w_{0,j-\half} - w_{0,j-\frac{3}{2}}), \\
\vt_R^{\edge} &=& \vt_{\ib}^n + \left(\half - \frac{\dt}{2h}v_{\ib}\right)\Delta_y \vt_{\ib}^n + \frac{\dt}{2}f_{\ib} \nonumber \\
&& - \frac{\dt}{4h}\left(u_{\ib+\half\eb_x}^{\trans}+u_{\ib-\half\eb_x}^{\trans}\right)\left(\vt_{\ib+\half\eb_x}+\vt_{\ib-\half\eb_x}\right) \nonumber \\
&& - \frac{\dt}{2h}\underbrace{\vt_{\ib}}_{?}(w_{0,j+\half} - w_{0,j-\half}).
\end{eqnarray}
Finally, we solve a Riemann problem:
\begin{equation}
\vt_{\ib-\half\eb_y}^{\mac} =
\begin{cases}
0, & \left(v_L^{\edge} \le 0 ~ {\rm AND} ~ v_R^{\edge} \ge 0\right) ~ {\rm OR} ~ \left|v_L^{\edge} + v_R^{\edge}\right| < \epsilon, \\
\vt_L^{\edge}, & v_L^{\edge} + v_R^{\edge} > 0, \\
\vt_R^{\edge}, & v_L^{\edge} + v_R^{\edge} < 0. 
\end{cases}
\end{equation}


\section{Computing $s^{\edge}$}


\section{Computing $\Ub^{\edge}$}


\section{Forces}

\cleardoublepage
\begin{table*}
\begin{center}
\caption{Forcing term into make edge scal \newline}
\begin{tabular}{|c|c|c|}
\multicolumn{1}{c}{PRED X}   & \multicolumn{1}{c}{Force} \\
\hline
  &   \\
T $(X)$ & $0$ \\
  &   \\
T $(\rho^\prime)$ & $-\rho^\prime \; \nabla \cdot (\ut+w_0 \er) - \nabla \cdot (\ut \rho_0) $  \\
  &   \\
\hline
\end{tabular}
\end{center}
\begin{center}
\begin{tabular}{|c|c|}
\multicolumn{1}{c}{H PRED TYPE}   & \multicolumn{1}{c}{Force} \\
\hline
   & \\
 1 $((\rho h)^\prime)$ &  $(\tilde{w} \frac{\partial p_0}{\partial r}) - (\rho h)^\prime \; \nabla \cdot (\ut+w_0 \er) - \nabla \cdot (\ut (\rho h)_0) $  \\
   & \\
 2 $(h)$ & $\frac{1}{\rho} (\psi + \tilde{w} \frac{\partial p_0}{\partial r})$  \\
   & \\
 3 $(T)$ & $\frac{1}{\rho c_p} \left[ (1 - \rho h_p) (\psi + \tilde{w} \frac{\partial p_0}{\partial r}) \right]$ \\
   & \\
 4 $(T)$ & $\frac{1}{\rho c_p} \left[ (1 - \rho h_p) (\psi + \tilde{w} \frac{\partial p_0}{\partial r}) \right]$ \\
   & \\
\hline
\hline
\end{tabular}
\end{center}
\end{table*}


\begin{table*}
\begin{center}
\caption{Quantity that goes into and out of make edge scal \newline}
\begin{tabular}{|c|c|c|c|}
\multicolumn{2}{c}{PREDICT} & \multicolumn{1}{c}{} & \multicolumn{1}{c}{} \\
\multicolumn{1}{c}{X} & \multicolumn{1}{c}{h pred type} & \multicolumn{1}{c}{Species } & \multicolumn{1}{c}{h/T } \\
\hline
T & 1 & $X$ and $\rho^\prime$ & $(\rho h)^\prime$ \\
T & 2 & $X$ and $\rho^\prime$ & $h$  \\
T & 3 & $X$ and $\rho^\prime$ & $T$  \\
T & 4 & $X$ and $\rho^\prime$ & $T$  \\
\hline
\end{tabular}
\end{center}
\end{table*}

\begin{table*}
\begin{center}
\caption{When predicting temp ... \newline}
\begin{tabular}{|c|c|c|c|c|}
\multicolumn{1}{c}{PRED X} & \multicolumn{1}{c}{type} & \multicolumn{1}{c}{Inputs } & 
\multicolumn{1}{c}{EOS inputs} & \multicolumn{1}{c}{output} \\
\hline
T & 3 & $X$ and $\rho^\prime$ & $\rho = \rho_0 + \rho^\prime$ &  $(\rho h)^\prime = \rho h - (\rho h)_0$ \\
  &   &                   & $   X = $ given & \\
  &   &                   &                 & \\
T & 4 & $X$ and $\rho^\prime$ & $\rho = \rho_0 + \rho^\prime$ & $h$ \\
  &   &                   & $   X = $ given & \\
  &   &                   &                 & \\
\hline
\end{tabular}
\end{center}
\end{table*}

\begin{table*}
\begin{center}
\caption{Quantity that goes into mkflux on edges \newline}
\begin{tabular}{|c|c|c|c|c|}
\multicolumn{2}{c}{PREDICT} & \multicolumn{1}{c}{} & \multicolumn{1}{c}{} \\
\multicolumn{1}{c}{X} & \multicolumn{1}{c}{h pred type} & \multicolumn{1}{c}{Species } & \multicolumn{1}{c}{h/T } & \multicolumn{1}{c}{$\rho^\prime$} \\
\hline
T & 1 & $X$ and $\rho^\prime$ & $(\rho h)^\prime$ & \\
T & 2 & $X$ and $\rho^\prime$ & $h$ & \\
T & 3 & $X$ and $\rho^\prime$ & $(\rho h)^\prime$ & \\
T & 4 & $X$ and $\rho^\prime$ & $h$ & \\
\hline
\end{tabular}
\end{center}
\end{table*}

\begin{table*}
\begin{center}
\caption{Quantity that is created in mkflux and passed into update scal\newline}
\begin{tabular}{|c|c|c|c|c|c|c|c|}
\multicolumn{2}{c}{PREDICT}   & 
\multicolumn{1}{c}{} & \multicolumn{1}{c}{}  
& \multicolumn{2}{c}{Add db?} & \multicolumn{2}{c}{Forcing term }\\
\multicolumn{1}{c}{X}   & \multicolumn{1}{c}{type} &
\multicolumn{1}{c}{Species } & \multicolumn{1}{c}{$(\rho h)$} 
& \multicolumn{1}{c}{X} & \multicolumn{1}{c}{h} & \multicolumn{1}{c}{$(\rho X)$} & \multicolumn{1}{c}{$(\rho h)$} \\
\hline
T & 1 & $(\ut+w_0)(\rho_0+\rho^\prime) X          $ & $(\ut+w_0)(\rho h)^\prime + \ut (\rho h)_0$ & N & Y & 0 & $ (\tilde{w} \frac{\partial p_0}{\partial r}) $ \\
T & 2 & $(\ut+w_0)(\rho_0+\rho^\prime) X          $ & $(\ut+w_0) (\rho_0+\rho^\prime) h$ & N & N & 0 & $ (\psi + \tilde{w} \frac{\partial p_0}{\partial r}) $ \\
T & 3 & $(\ut+w_0)(\rho_0+\rho^\prime) X          $ & $(\ut+w_0)(\rho h)^\prime + \ut (\rho h)_0$ & N & Y & 0 & $ (\tilde{w} \frac{\partial p_0}{\partial r}) $ \\
T & 4 & $(\ut+w_0)(\rho_0+\rho^\prime) X          $ & $(\ut+w_0) (\rho_0+\rho^\prime) h$ & N & N & 0 & $ (\psi + \tilde{w} \frac{\partial p_0}{\partial r}) $ \\
\hline
\end{tabular}
\end{center}
\end{table*}

\end{document}
