\documentclass[11pt]{article} 

\tolerance=600

\usepackage{amsmath}
\usepackage{amssymb}
\usepackage{epsfig}
\usepackage{color}

% Margins
\usepackage[lmargin=1.0in,rmargin=1.0in,tmargin=0.75in,bmargin=0.75in]{geometry}

\newcommand{\sfrac}[2]{\mathchoice
  {\kern0em\raise.5ex\hbox{\the\scriptfont0 #1}\kern-.15em/
   \kern-.15em\lower.25ex\hbox{\the\scriptfont0 #2}}
  {\kern0em\raise.5ex\hbox{\the\scriptfont0 #1}\kern-.15em/
   \kern-.15em\lower.25ex\hbox{\the\scriptfont0 #2}}
  {\kern0em\raise.5ex\hbox{\the\scriptscriptfont0 #1}\kern-.2em/
   \kern-.15em\lower.25ex\hbox{\the\scriptscriptfont0 #2}}
  {#1\!/#2}}

\def\half  {\frac{1}{2}}
\def\dt    {\Delta t}

\def\edge  {\rm EDGE}
\def\mac   {\rm MAC}
\def\trans {\rm TRANS}

\def\eb    {{\bf e}}
\def\ib    {{\bf i}}
\def\ub    {{\bf u}}
\def\Ub    {{\bf U}}

\def\Ubt   {\widetilde{\bf U}}
\def\ut    {\tilde{u}}
\def\vt    {\tilde{v}}
\def\wt    {\tilde{w}}

\title{Notes on the Godunov Step in {\tt MAESTRO}}

\begin{document}
\maketitle

These are working notes for the Godunov step in {\tt MAESTRO}.

\section{Computing $\ub^{\trans}$}
In {\tt advance\_premac.f90}, we call the function {\tt mkutrans}, which computes the edge-centered transverse velocities, $\ub^{\trans}$.  These transverse velocities do not contain $w_0$, so immediately following the call to {\tt mkutrans}, we call {\tt addw0}.  We only compute the normal component of velocity at each face.\\

The evolution equation for the perturbational velocity is:
\begin{equation}
\frac{\partial\Ubt}{\partial t} = -\Ubt\cdot\nabla\Ubt - w_0\frac{\partial\Ubt}{\partial r} - (\Ubt\cdot\eb_r)\frac{\partial w_0}{\partial r}\eb_r - \frac{1}{\rho}\nabla\pi + \frac{1}{\rho_0}\frac{\partial\pi_0}{\partial r}\eb_r - \frac{(\rho-\rho_0)}{\rho}g\eb_r.
\end{equation}
We are going to use a 1D Taylor series extrapolation in space and time to compute the normal components of $\Ubt$ at face-centers.  (Note that for Cartesian problems, $\Ubt = (u,v,\wt)$ and for spherical problems, $\Ubt = (\ut,\vt,\wt)$.)  By 1D, we mean that we omit any spatial derivatives that are not in the direction of the extrapolation.  We first compute ``left'' and ``right'' states, and then use a Riemann solver to pick the final state.\\

For Cartesian problems, for the x-faces, the extrapolation is as follows:
\begin{eqnarray}
u_L &=& u_{\ib-\eb_x}^n + \left(\half - \frac{\dt}{h}u_{\ib-\eb_x}^n\right)\Delta_x u_{\ib-\eb_x}^n, \\
u_R &=& u_{\ib}^n - \left(\half + \frac{\dt}{h}u_{\ib}^n\right)\Delta_x u_{\ib}^n,
\end{eqnarray}
where $\ib = (i,r)$ in 2D and $\ib = (i,j,r)$ in 3D.  For 3D Cartesian problems, for the y-faces, the extrapolation follows analogously:
\begin{eqnarray}
v_L &=& u_{\ib-\eb_y}^n + \left(\half - \frac{\dt}{h}u_{\ib-\eb_y}^n\right)\Delta_y u_{\ib-\eb_y}^n, \\
v_R &=& u_{\ib}^n - \left(\half + \frac{\dt}{h}u_{\ib}^n\right)\Delta_y u_{\ib}^n.
\end{eqnarray}
We pick the final edge states using:
\begin{equation}
u^{\trans}_{\ib-\half\eb_x} =
\begin{cases}
0, & (u_L \le 0 ~ {\rm AND} ~ u_R \ge 0) ~ {\rm OR} ~ |u_L + u_R| < \epsilon, \\
u_L, & u_L + u_R > 0, \\
u_R, & u_L + u_R < 0, \\
\end{cases}
\end{equation}
\begin{equation}
v^{\trans}_{\ib-\half\eb_y} =
\begin{cases}
0, & (v_L \le 0 ~ {\rm AND} ~ v_R \ge 0) ~ {\rm OR} ~ |v_L + v_R| < \epsilon, \\
v_L, & v_L + v_R > 0, \\
v_R, & v_L + v_R < 0. \\
\end{cases}
\end{equation}
For Cartesian problems, for the radial faces, we must include the radial derivative terms proportional to $w_0$ and $(\Ubt\cdot\eb_r) = \wt$:
\begin{eqnarray}
\wt_L &=& \wt_{\ib-\eb_r}^n + \left(\half - \frac{\dt}{h}w_{\ib-\eb_r}^n\right)\Delta_r \wt_{\ib-\eb_r}^n + \frac{\dt}{2}\wt_{\ib-\eb_r}^n\left(\frac{\partial w_0}{\partial r}\right)_{\ib-\eb_r},\label{Radial utrans_L} \\
\wt_R &=& \wt_{\ib}^n + \left(\half - \frac{\dt}{h}w_{\ib}^n\right)\Delta_r \wt_{\ib}^n + \frac{\dt}{2}\wt_{\ib}^n\left(\frac{\partial w_0}{\partial r}\right)_{\ib},\label{Radial utrans_R}
\end{eqnarray}
where the total radial velocity is defined as:
\begin{equation}
w_{\ib}^n = \wt_{\ib}^n + \frac{w_{0,r+\half}+w_{0,r-\half}}{2}\label{Total Radial Velocity}
\end{equation}
The spatial derivatives of $w_0$ are two point differences:
\begin{equation}
\left(\frac{\partial w_0}{\partial r}\right)_{\ib} = \frac{w_{0,r+\half}-w_{0,r-\half}}{h}.\label{Radial Derivative of w_0}
\end{equation}
We pick the final edge states using:
\begin{equation}
\wt^{\trans}_{\ib-\half\eb_r} =
\begin{cases}
0, & (w_L \le 0 ~ {\rm AND} ~ w_R \ge 0) ~ {\rm OR} ~ |w_L + w_R| < \epsilon, \\
\wt_L, & w_L + w_R > 0, \\
\wt_R, & w_L + w_R < 0, \\
\end{cases}
\end{equation}
where $w_L = \wt_L + w_{0,r-\half}$ and $w_R = \wt_R + w_{0,r-\half}$.  As mentioned before, we now call {\tt addw0} to convert $\wt^{\trans}$ to $w^{\trans}.$\\

For spherical problems, we first construct $\Ub$ by mapping $w_0$ onto a Cartesian array and adding it to $\Ubt$.  This is a spherical generalization of equation (\ref{Total Radial Velocity}).  We also map $\partial w_0/\partial r$ onto a Cartesian array.  This is a spherical generalization of equation (\ref{Radial Derivative of w_0}).  Then we use an extrapolation analogous to equations equations (\ref{Radial utrans_L}) and (\ref{Radial utrans_R}).  For example, for x-faces:
\begin{eqnarray}
\ut_L &=& \ut_{\ib-\eb_x}^n + \left(\half - \frac{\dt}{h}u_{\ib-\eb_x}^n\right)\Delta_x\ut_{\ib-\eb_x}^n + \frac{\dt}{2}\left[(\Ubt^n\cdot\eb_r)\left(\frac{\partial w_0}{\partial r}\right)(\eb_r\cdot\eb_x)\right]_{\ib-\eb_x}, \\
\ut_R &=& \ut_{\ib}^n + \left(\half - \frac{\dt}{h}u_{\ib}^n\right)\Delta_x\ut_{\ib}^n + \frac{\dt}{2}\left[(\Ubt^n\cdot\eb_r)\left(\frac{\partial w_0}{\partial r}\right)(\eb_r\cdot\eb_x)\right]_{\ib}.
\end{eqnarray}
Finally, we call {\tt addw0} to convert $\Ubt^{\trans}$ to $\Ub^{\trans}$.
\section{Computing $\ub^{\mac}$}
In {\tt advance\_premac.f90}, we call {\tt make\_edge\_state} to compute $\Ubt^{\mac}$.
\section{Computing $s^{\edge}$}
\section{Computing $\ub^{\edge}$}
\end{document}
