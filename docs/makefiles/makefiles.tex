\label{ch:make}

\section{Build Process Overview}

The \maestro\ build system relies on the definitions setup by \boxlib.  There are
4 main files that define the macros used by the build:
\begin{itemize}
\item {\tt \$\{BOXLIB\_HOME\}/Tools/F\_mk/GMakedefs.mak}

\item {\tt \$\{BOXLIB\_HOME\}/Tools/F\_mk/GMakeMPI.mak}

\item {\tt \$\{BOXLIB\_HOME\}/Tools/F\_mk/GMakerules.mak}

\item {\tt MAESTRO/GMaestro.mak}
\end{itemize}

The \boxlib\ build system separates the compiler-specific information
from the machine-specific information---this allows for reuse of the
compiler information.

rely on VPATH  / GPackage.mak

dependency generation

include files

files created at build-time: {\tt probin.f90}, {\tt build\_info.f90}, and {\tt network.f90}


\section{\maestro\ Problem Options}

\subsection{Problem-specific Files}

\subsection{Defining Network, EOS, and Conductivity Routines}

\subsection{Core MAESTRO modules}

\subsection{Unit Tests}


\section{Special Targets}

\subsection{Debugging ({\tt print-*})}

\subsection{{\tt clean} and {\tt realclean}}



\section{Extending the Build System}

\subsection{Adding a Machine}

\subsection{Adding a Compiler}

\subsection{Parallel Builds}





