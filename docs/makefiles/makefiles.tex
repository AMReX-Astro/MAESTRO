\label{ch:make}

\section{Build Process Overview}

The \maestro\ build system uses features of GNU make, which is
typically the default on systems.  The \maestro\ executable is built
in the problem's directory (one of the directories under {\tt
SCIENCE/}, {TEST\_PROBLEMS}, or {UNIT\_TESTS}).  This directory will
contain a makefile, {\tt GNUmakefile}, that includes all the necessary
information to build the executable.  

The main macros that define the build process are split across several
files.  The 4 main files are:
\begin{itemize}
\item {\tt \$\{BOXLIB\_HOME\}/Tools/F\_mk/GMakedefs.mak}:

  This setups the basic macros, includes the options for the selected
  compiler, builds the list of object and source files, and defines
  the compile and link command lines.

\item {\tt \$\{BOXLIB\_HOME\}/Tools/F\_mk/GMakeMPI.mak}:

  This implements any changes to the compiler names and library
  locations necessary to build a parallel executable with MPI.

\item {\tt \$\{BOXLIB\_HOME\}/Tools/F\_mk/GMakerules.mak}:

  This creates the various build targets and specifies the rules for
  building the object files, the list of dependencies, and some other
  lesser-used targets (tags for editors, documentation, etc.)

\item {\tt MAESTRO/GMaestro.mak}:

  This is a \maestro-specific file that gathers all of the various
  modules that are used to build a typical \maestro\ application
  and integrates with the \boxlib\ build system.  Every \maestro\
  problem's {\tt GNUmakefile} will include this file.

\end{itemize}

\maestro\ gets the location of the \boxlib\ library through the 
{\tt BOXLIB\_HOME} variable.  This should be set as an environment
variable in your shell start-up files (e.g.\ {\tt .bashrc} or {\tt
.cshrc}).

The \boxlib\ build system separates the compiler-specific information
from the machine-specific information---this allows for reuse of the
compiler information.  The only machine-specific parts of the build system
are for the MPI library locations, contained in {\tt GMakeMPI.mak}.
The compiler flags for the various compilers are listed in the
files in {\tt \$\{BOXLIB\_HOME\}/Tools/F\_mk/comps/}.

\subsection{Finding Source Files}

The \boxlib\ build system relies on the vpath functionality of {\tt
make}.  In a makefile, the {\tt vpath} variable holds search path used
to locate source files.  When a particular file is needed, the
directories listed in {\tt vpath} are searched until the file is
found.  The first instance of a file is used.  We exploit this feature
by always putting the build directory first in {\tt vpath} (this is
done in {\tt GMakerules.mak}).  This means that if a source file is
placed in the build directory, that copy will override any other
version in the source path.

In \maestro, the {\tt vpath} variable is set using the macros defined
in {\tt GMaestro.mak}.  A user does not need to set this variable
explicitly.  Additional source locations are added in the manner
described below (see \S~\ref{sec:make:otherfiles}).

\subsection{Dependencies}

dependency generation

include files

\subsection{Files Created at Compile-time}

Several files are created at build-time:
\begin{itemize}
\item {\tt probin.f90}

\item {\tt build\_info.f90}

\item {\tt network.f90} (only for the {\tt general\_null} network)
\end{itemize}



\section{\maestro\ Problem Options}

\subsection{Problem-specific Files}
\label{sec:make:otherfiles}


\subsection{Defining EOS, Network, and Conductivity Routines}

Each \maestro\ problem needs to define an equation of state, a
reaction network, and a routine to compute the conductivities (for
thermal diffusion).  This is true even if the problem is not doing
reactions of thermal diffusion.  These definitions are specified
in the problem's {\tt GNUmakefile}.

\begin{itemize}
\item {\tt EOS\_DIR}:

  This variable points to the path (relative to {\tt fParallel/}) of
  the equation of state used for the build.  Choices that work
  with \maestro\ are:
  \begin{itemize}
  \item {\tt extern/EOS/helmeos}
  \item {\tt extern/EOS/gamma\_law\_general}
  \end{itemize}

\item {\tt CONDUCTIVITY\_DIR}:

  This variable points to the conductivity routine used for the build
  (relative to {\tt fParallel/}).  Choices that work with \maestro\
  are:
  \begin{itemize}
  \item {\tt extern/conductivity/constant}
  \item {\tt extern/conductivity/timmes\_stellar}
  \end{itemize}
  If diffusion is not being used for the problem, this should be set
  to {\tt extern/conductivity/constant}.

\item {\tt NETWORK\_DIR}:

  This variable points to the reaction network used for the build
  (relative to {\tt fParallel/}).  A lot of options are present in
  {\tt extern/networks/}.  A network is required even if you are not
  doing reactions, since the network defines the species that are
  advected and interpreted by the equation of state.  

  A special choice, {\tt extern/networks/general\_null} is a general
  network that simply defines the properties of one or more species.
  This requires an inputs file, specified by {\tt
  GENERAL\_NET\_INPUTS}.  This inputs file is read at compile-time and
  used to build the {\tt network.f90} file that is compiled into the
  source.

\end{itemize}


\subsection{Core MAESTRO modules}

\subsection{Unit Tests}


\section{Special Targets}

\subsection{Debugging ({\tt print-*})}

To see the contents of any variable in the build system, you can build
the special target {\tt print-{\em varname}}, where {\tt {\em
varname}} is the name of the variable.  For example, to see what the
Fortran compiler flags are, you would do:
\begin{verbatim}
make print-FFLAGS
\end{verbatim}
This would give (for {\tt gfortran}, for example):
\begin{verbatim}
FFLAGS is -Jt/Linux.gfortran/m -I t/Linux.gfortran/m -O2 -fno-range-check
\end{verbatim}
This functionality is useful for debugging the makefiles.

\subsection{{\tt clean} and {\tt realclean}}



\section{Extending the Build System}

\subsection{Adding a Machine}

\subsection{Adding a Compiler}

\subsection{Parallel Builds}





