\label{ch:make}

\section{Build Process Overview}

The \maestro\ build system uses features of GNU make, which is typically the 
default on systems.  The main macros that define the build process are 
split across several files.  The 4 main files are:
\begin{itemize}
\item {\tt \$\{BOXLIB\_HOME\}/Tools/F\_mk/GMakedefs.mak}:
  This setups the basic macros, includes the options for the selected
  compiler, builds the list of object and source files, and defines
  the compile and link command lines.

\item {\tt \$\{BOXLIB\_HOME\}/Tools/F\_mk/GMakeMPI.mak}
  This implements any changes to the compiler names and library
  locations necessary to build a parallel executable with MPI.

\item {\tt \$\{BOXLIB\_HOME\}/Tools/F\_mk/GMakerules.mak}
  This creates the various build targets and specifies the rules for
  building the object files, the list of dependencies, and some other
  lesser-used targets (tags for editors, documentation, etc.)

\item {\tt MAESTRO/GMaestro.mak}
  This is a \maestro-specific file that gathers all of the various
  modules that are used to build a typical \maestro\ application
  and integrates with the \boxlib\ build system.  Every \maestro\
  problem's makefile will include this file.

\end{itemize}


The \boxlib\ build system separates the compiler-specific information
from the machine-specific information---this allows for reuse of the
compiler information.  The only machine-specific parts of the build system
are for the MPI library locations, contained in {\tt GMakeMPI.mak}.

The \maestro\ build system relies on the vpath functionality of {\tt
make}.  In a makefile, the {\tt vpath} variable holds search path used
to locate source files.  When a particular file is needed, the
directories listed in {\tt vpath} are searched until the file is
found.  The first instance of a file is used.  We exploit this feature
by always putting the build directory first in {\tt vpath} (this is
done in {\tt GMakerules.mak}).  This means that if a source file is
placed in the build directory, that copy will override any other
version in the source path.


dependency generation

include files

files created at build-time: {\tt probin.f90}, {\tt build\_info.f90}, and {\tt network.f90}


\section{\maestro\ Problem Options}

\subsection{Problem-specific Files}

\subsection{Defining Network, EOS, and Conductivity Routines}

\subsection{Core MAESTRO modules}

\subsection{Unit Tests}


\section{Special Targets}

\subsection{Debugging ({\tt print-*})}

\subsection{{\tt clean} and {\tt realclean}}



\section{Extending the Build System}

\subsection{Adding a Machine}

\subsection{Adding a Compiler}

\subsection{Parallel Builds}





