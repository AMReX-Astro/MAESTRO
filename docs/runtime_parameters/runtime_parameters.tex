
\section{Runtime Parameters}

%%%%%%%%%%%%%%%%
% symbol table
%%%%%%%%%%%%%%%%

{\small

\renewcommand{\arraystretch}{1.5}
%
\begin{center}
\begin{longtable}{|l|p{3.25in}|l|}
\caption[runtime parameters]{runtime parameters.} \label{table:runtime} \\
%
\hline \multicolumn{1}{|c|}{\textbf{parameter}} & 
       \multicolumn{1}{ c|}{\textbf{description}} & 
       \multicolumn{1}{ c|}{\textbf{default value}} \\ \hline 
\endfirsthead

\multicolumn{3}{c}%
{{\tablename\ \thetable{}---continued}} \\
\hline \multicolumn{1}{|c|}{\textbf{parameter}} & 
       \multicolumn{1}{ c|}{\textbf{description}} & 
       \multicolumn{1}{ c|}{\textbf{default value}} \\ \hline 
\endhead

\multicolumn{3}{|r|}{{\em continued on next page}} \\ \hline
\endfoot

\hline 
\endlastfoot


\verb=  anelastic_cutoff  = &   The density below which we modify the constraint to look like the anelastic constraint, instead of the low Mach constraint.  This prevents velocities from getting out of hand at the edge of the star. Refer to Section \ref{Sec:Anelastic Cutoff}.  &  3.d6 \\
\verb=  base_cutoff_density  = &   The density below which we keep the initial model constant. Refer to Section \ref{Sec:Base Cutoff Density}  &  3.d6 \\
\verb=  bcx_hi  = &   $+x$ boundary condition (valid values are listed in {\tt  boxlib/bc.f90})  &  SLIP\_WALL \\
\verb=  bcx_lo  = &   $-x$ boundary condition  &  SLIP\_WALL \\
\verb=  bcy_hi  = &   $+y$ boundary condition  &  SLIP\_WALL \\
\verb=  bcy_lo  = &   $-y$ boundary condition  &  SLIP\_WALL \\
\verb=  bcz_hi  = &   $+z$ boundary condition  &  SLIP\_WALL \\
\verb=  bcz_lo  = &   $-z$ boundary condition  &  SLIP\_WALL \\
\verb=  beta_type  = &   what type of coefficient to use inside the velocity divergence constraint. \newline {\tt beta\_type} = 1 uses $\beta_0$; \newline {\tt beta\_type} = 2 uses $\rho_0$ (anelastic); \newline {\tt beta\_type} = 3 uses 1 (small-scale combustion).  &  1 \\
\verb=  buoyancy_cutoff_factor  = &   The multiplicative factor (over base\_cutoff\_density) below which we do zero out the buoyancy term in the momentum equation.  &  5.0 \\
\verb=  burner_threshold_cutoff  = &   Mass fraction cutoff for burner\_threshold\_species  used in burner threshold  &  1.d-10 \\
\verb=  burner_threshold_species  = &   Name of the species to be used in burner threshold  &  "" \\
\verb=  cflfac  = &   CFL factor to use in the computation of the advection timestep constraint  &  0.5d0 \\
\verb=  cg_verbose  = &   Verbosity of bottom solver  &  0 \\
\verb=  check_base_name  = &   prefix to use in checkpoint file names  &  "chk" \\
\verb=  chk_int  = &   Number of timesteps between writing a checkpoint file  &  0 \\
\verb=  co_latitude  = &   latitude, in radians, for problems with rotation where the domain is only a subset of a full star.  &  ZERO \\
\verb=  dm_in  = &   dimensionality (valid values are 2 or 3)  &  2 \\
\verb=  do_alltoallv  = &    &  .false. \\
\verb=  do_burning  = &   turn on (.true.) or off (.false.) burning  &  .true. \\
\verb=  do_eos_h_above_cutoff  = &   After the advective enthalpy update, recompute the enthalpy if we are below the base cutoff density.  &  .true. \\
\verb=  do_initial_projection  = &   Do the initial projection.  &  .true. \\
\verb=  do_smallscale  = &   force $\rho_0 = (\rho h)_0 = 0$, {\tt evolve\_base\_state = F} and {\tt beta\_type} = 3  &  .false. \\
\verb=  do_sponge  = &   Use sponging.  &  .false. \\
\verb=  dpdt_factor  = &   factor in front of the volume discrepancy term (0.0 = off)  &  0.d0 \\
\verb=  drdxfac  = &   ratio of radial base state zones to Cartesian full state zones for spherical geometry  &  1 \\
\verb=  enthalpy_pred_type  = &   predict\_rhohprime = 1; \newline predict\_h = 2; \newline predict\_T\_then\_rhohprime = 3; \newline predict\_T\_then\_h = 4; \newline predict\_hprime = 5; \newline predict\_Tprime\_then\_h = 6.  &  predict\_rhohprime \\
\verb=  evolve_base_state  = &   turn on (.true.) or off (.false.) basestate evolution  &  .true. \\
\verb=  fixed_dt  = &   Fix the time step.  If -1.0, then use the standard time step.  &  -1.0d0 \\
\verb=  grav_const  = &   the gravitational acceleration (cm~s$^{-2}$) for plane-parallel geometry  &  -1.5d10 \\
\verb=  hg_bottom_solver  = &   valid values are $\le$ 0  &  -1 \\
\verb=  hg_dense_stencil  = &   In hgproject, in 2D, use a 9 point Laplacian (.true.) or 5-point Laplacian (.false.).  In 3D, use a 27 point Laplacian (.true.) or 7-point Laplacian (.false.).  &  .true. \\
\verb=  init_divu_iter  = &   Number of initial divu iterations.  &  4 \\
\verb=  init_iter  = &   Number of initial pressure iterations.  &  4 \\
\verb=  init_shrink  = &   the multiplicative factor ($\le 1$) to reduce the initial timestep as computed by the various timestep estimators  &  1.d0 \\
\verb=  job_name  = &   job name printed in output  &  "" \\
\verb=  lUsingNFiles  = &   If .true., use nOutFiles processors to write checkpoint and plotfiles. Fortran has the unfortunate feature of each processor only being able to write out 1-2GB each without crashing.  &  .true. \\
\verb=  max_dt_growth  = &   The maximum scale factor that the time step is allowed to grow by per time step.  &  1.1d0 \\
\verb=  max_grid_size  = &   The largest grid size that will be created using make\_new\_grids.  &  64 \\
\verb=  max_grid_size_base  = &   The largest grid size that will be created using make\_new\_grids for the coarsest level.  Defaults to {\tt max\_grid\_size}.  &  -1 \\
\verb=  max_levs  = &   Total number of levels.  1 = single level.  &  1 \\
\verb=  max_mg_bottom_nlevels  = &   if mg\_bottom\_solver == 4, then how many mg levels can the bottom solver mgt object have  &  1000 \\
\verb=  max_step  = &   Maximum number of steps in the simulation.  &  1 \\
\verb=  mg_bottom_solver  = &   valid values are $\le$ 0  &  -1 \\
\verb=  mg_verbose  = &   Verbsoity of the multigrid solver, but not the bottom solver.  &  0 \\
\verb=  min_eff  = &   parameter for cluster algorithm for making new grids in adaptive problems  &  0.7d0 \\
\verb=  min_width  = &   The minimum size on a side for a grid created using make\_new\_grids.  &  16 \\
\verb=  model_file  = &   input model file  &  "model.hse" \\
\verb=  nOutFiles  = &   If lUsingNFiles = .true., use this many processors to write checkpoint and plotfiles.  Fortran has the unfortunate feature of each processor only being able to write out 1-2GB each without crashing.  &  64 \\
\verb=  n_cellx  = &   Number of cells for the base level in the x-direction  &  -1 \\
\verb=  n_celly  = &   Number of cells for the base level in the y-direction  &  -1 \\
\verb=  n_cellz  = &   Number of cells for the base level in the z-direction  &  -1 \\
\verb=  perturb_model  = &   Turn on a perturbation in the initial data.  Problem specific.  &  .false. \\
\verb=  plot_base  = &   plot w0\_x, w0\_y, w0\_z, divw0, rho0, rhoh0, h0, and p0 in plotfile  &  .false. \\
\verb=  plot_base_name  = &   prefix to use in plotfile file names  &  "plt" \\
\verb=  plot_deltat  = &   rather than use a plot interva, plot a file after the solution has advanced past plot\_deltat in time  &  0.d0 \\
\verb=  plot_int  = &   plot interval  &  0 \\
\verb=  plot_spec  = &   plot species and omegadot in plotfile  &  .true. \\
\verb=  plot_trac  = &   plot tracers in plotfile  &  .false. \\
\verb=  pmask_x  = &   indicates whether the problem domain periodic in the x-direction  &  .false. \\
\verb=  pmask_y  = &   indicates whether the problem domain periodic in the y-direction  &  .false. \\
\verb=  pmask_z  = &   indicates whether the problem domain periodic in the z-direction  &  .false. \\
\verb=  ppm_type  = &   0 = no ppm \newline 1 = 1985 ppm \newline 2 = 2009 ppm  &  0 \\
\verb=  prob_hi_x  = &   physical coordinates of hi-x corner of problem domain  &  1.d0 \\
\verb=  prob_hi_y  = &   physical coordinates of hi-y corner of problem domain  &  1.d0 \\
\verb=  prob_hi_z  = &   physical coordinates of hi-z corner of problem domain  &  1.d0 \\
\verb=  prob_lo_x  = &   physical coordinates of lo-x corner of problem domain  &  ZERO \\
\verb=  prob_lo_y  = &   physical coordinates of lo-y corner of problem domain  &  ZERO \\
\verb=  prob_lo_z  = &   physical coordinates of lo-z corner of problem domain  &  ZERO \\
\verb=  ref_ratio  = &   Refinement ratio for multilevel problems  &  2 \\
\verb=  regrid_int  = &   How often we regrid.  &  -1 \\
\verb=  restart  = &   which file to restart from.  -1 means do not restart  &  -1 \\
\verb=  rotation_radius  = &   radius used for computing centrifugal term in rotation problems  &  1.0d6 \\
\verb=  rotational_frequency  = &   rotational frequency used for computing centrifugal term in rotation problems.  &  ZERO \\
\verb=  s0_interp_type  = &   The interpolation for filling a cell-centered multifab from a 1D bin-centered array. \newline 1 = piecewise constant; \newline 2 = piecewise linear; \newline 3 = quadratic  &  1 \\
\verb=  s0mac_interp_type  = &   The interpolation for filling an edge based multifab from a 1D bin-centered array. \newline 1 = Interpolate s0 to cell centers (with s0\_interp\_type), then average to edges; \newline 2 = Interpolate s0 to edges directly using linear interpolation; \newline 3 = Interpolate s0 to edges directly using quadratic interpolation.  &  1 \\
\verb=  single_prec_plotfiles  = &    &  .false. \\
\verb=  slope_order  = &   order of slopes in Godunov algorithm.  Options are 0, 2, or 4.  &  4 \\
\verb=  small_dens  = &    &  1.d-5 \\
\verb=  small_temp  = &    &  5.d6 \\
\verb=  spherical_in  = &   set to 1 if you are doing a spherical problem  &  0 \\
\verb=  sponge_kappa  = &   Parameter for sponge.  Problem dependent.  &  10.d0 \\
\verb=  stop_time  = &   simulation stop time  &  -1.d0 \\
\verb=  temp_diffusion_formulation  = &   How to compute the explicit thermal diffusion term.  \newline 1 = in terms of $T$; \newline 2 = in terms of $\rho,p_0,X$.  &  2 \\
\verb=  test_set  = &   Fixed grid file.  &  '' \\
\verb=  the_copy_cache_max  = &    Number of boxassoc layouts we keep in memory to avoid having to recompute the boxassoc, which is computationally expensive.  &  50 \\
\verb=  the_knapsack_verbosity  = &   Verbosity of the knapscak processor-to-grid algorithm.  &  .false. \\
\verb=  the_layout_verbosity  = &    &  0 \\
\verb=  the_sfc_threshold  = &   When assigning processors for grids, this determines whether we use the sfc algorithm or knapsack algorithm.  If the total number of grids divided by the number of processors is greater than this number, use sfc.  &  5 \\
\verb=  thermal_diffusion_type  = &   In the thermal diffusion solver, 1 = Crank-Nicholson; 2 = Backward Euler.  &  1 \\
\verb=  use_delta_gamma1_term  = &   turns on second order correction to delta gamma1 term  &  .false. \\
\verb=  use_divu_firstdt  = &   Use the divu constraint when computing the first time step.  &  .false. \\
\verb=  use_eos_coulomb  = &    &  .true. \\
\verb=  use_etarho  = &   turn on the etarho term as described in flow chart  &  .true. \\
\verb=  use_soundspeed_firstdt  = &   Use the soundspeed constraint when computing the first time step.  &  .false. \\
\verb=  use_tfromp  = &   When updating temperature, use $T=T(\rho,p_0,X) $ rather than $T=T(\rho,h,X)$.  &  .false. \\
\verb=  use_thermal_diffusion  = &   Use thermal diffusion.  &  .false. \\
\verb=  verbose  = &   General verbosity  &  0 \\
\verb=  w0_interp_type  = &   The interpolation for filling a cell-centered multifab from a 1D edge-centered array. \newline 1 = piecewise constant; \newline 2 = piecewise linear; \newline 3 = quadratic  &  2 \\
\verb=  w0mac_interp_type  = &   The interpolation for putting w0 on edges.  We only compute the normal component. \newline 1 = Interpolate w0 to cell centers (with w0\_interp\_type), then average to edges; \newline 2 = Interpolate w0 to edges directly using linear interpolation; \newline 3 = Interpolate w0 to edges directly using quadratic interpolation; \newline 4 = Interpolate w0 to nodes using linear interpolation, then average to edges.  &  1 \\


\end{longtable}
\end{center}

} % ends \small

%


