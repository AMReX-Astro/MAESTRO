
\section{Runtime Parameters}

Table~\ref{table:runtime} lists the runtime parameters available to
all MAESTRO problems.  Problem-specific runtime parameters are not
shown here.

%%%%%%%%%%%%%%%%
% symbol table
%%%%%%%%%%%%%%%%

\begin{landscape}


{\small

\renewcommand{\arraystretch}{1.5}
%
\begin{center}
\begin{longtable}{|l|p{5.25in}|l|}
\caption[ EOS
 parameters.]{ EOS
 parameters.} \label{table:runtime} \\
%
\hline \multicolumn{1}{|c|}{\textbf{parameter}} & 
       \multicolumn{1}{ c|}{\textbf{description}} & 
       \multicolumn{1}{ c|}{\textbf{default value}} \\ \hline 
\endfirsthead

\multicolumn{3}{c}%
{{\tablename\ \thetable{}---continued}} \\
\hline \multicolumn{1}{|c|}{\textbf{parameter}} & 
       \multicolumn{1}{ c|}{\textbf{description}} & 
       \multicolumn{1}{ c|}{\textbf{default value}} \\ \hline 
\endhead

\multicolumn{3}{|r|}{{\em continued on next page}} \\ \hline
\endfoot

\hline 
\endlastfoot


\rowcolor{tableShade}
\verb=  small_dens  = &    &  1.d-5 \\
\verb=  small_temp  = &    &  5.d6 \\
\rowcolor{tableShade}
\verb=  use_eos_coulomb  = &   use the Coulomb corrections between ions (if supported by the EOS)  &  .true. \\


\end{longtable}
\end{center}

} % ends \small


{\small

\renewcommand{\arraystretch}{1.5}
%
\begin{center}
\begin{longtable}{|l|p{5.25in}|l|}
\caption[ SDC
 parameters.]{ SDC
 parameters.} \label{table:runtime} \\
%
\hline \multicolumn{1}{|c|}{\textbf{parameter}} & 
       \multicolumn{1}{ c|}{\textbf{description}} & 
       \multicolumn{1}{ c|}{\textbf{default value}} \\ \hline 
\endfirsthead

\multicolumn{3}{c}%
{{\tablename\ \thetable{}---continued}} \\
\hline \multicolumn{1}{|c|}{\textbf{parameter}} & 
       \multicolumn{1}{ c|}{\textbf{description}} & 
       \multicolumn{1}{ c|}{\textbf{default value}} \\ \hline 
\endhead

\multicolumn{3}{|r|}{{\em continued on next page}} \\ \hline
\endfoot

\hline 
\endlastfoot


\rowcolor{tableShade}
\verb=  sdc_iters  = &   how many SDC iteractions to take (requires the code be compiled with {\tt SDC := t}  &  4 \\


\end{longtable}
\end{center}

} % ends \small


{\small

\renewcommand{\arraystretch}{1.5}
%
\begin{center}
\begin{longtable}{|l|p{5.25in}|l|}
\caption[ algorithm initialization
 parameters.]{ algorithm initialization
 parameters.} \label{table:runtime} \\
%
\hline \multicolumn{1}{|c|}{\textbf{parameter}} & 
       \multicolumn{1}{ c|}{\textbf{description}} & 
       \multicolumn{1}{ c|}{\textbf{default value}} \\ \hline 
\endfirsthead

\multicolumn{3}{c}%
{{\tablename\ \thetable{}---continued}} \\
\hline \multicolumn{1}{|c|}{\textbf{parameter}} & 
       \multicolumn{1}{ c|}{\textbf{description}} & 
       \multicolumn{1}{ c|}{\textbf{default value}} \\ \hline 
\endhead

\multicolumn{3}{|r|}{{\em continued on next page}} \\ \hline
\endfoot

\hline 
\endlastfoot


\rowcolor{tableShade}
\verb=  do_initial_projection  = &   Do the initial projection.  &  .true. \\
\verb=  init_divu_iter  = &   Number of initial divu iterations.  &  4 \\
\rowcolor{tableShade}
\verb=  init_iter  = &   Number of initial pressure iterations.  &  4 \\
\verb=  restart  = &   which file to restart from.  -1 means do not restart  &  -1 \\
\rowcolor{tableShade}
\verb=  restart_into_finer  = &   restart and add a level of refinement  &  .false. \\


\end{longtable}
\end{center}

} % ends \small


{\small

\renewcommand{\arraystretch}{1.5}
%
\begin{center}
\begin{longtable}{|l|p{5.25in}|l|}
\caption[ base state mapping
 parameters.]{ base state mapping
 parameters.} \label{table:runtime} \\
%
\hline \multicolumn{1}{|c|}{\textbf{parameter}} & 
       \multicolumn{1}{ c|}{\textbf{description}} & 
       \multicolumn{1}{ c|}{\textbf{default value}} \\ \hline 
\endfirsthead

\multicolumn{3}{c}%
{{\tablename\ \thetable{}---continued}} \\
\hline \multicolumn{1}{|c|}{\textbf{parameter}} & 
       \multicolumn{1}{ c|}{\textbf{description}} & 
       \multicolumn{1}{ c|}{\textbf{default value}} \\ \hline 
\endhead

\multicolumn{3}{|r|}{{\em continued on next page}} \\ \hline
\endfoot

\hline 
\endlastfoot


\rowcolor{tableShade}
\verb=  s0_interp_type  = &   The interpolation for filling a cell-centered multifab from a 1D bin-centered array. \newline 1 = piecewise constant; \newline 2 = piecewise linear; \newline 3 = quadratic  &  3 \\
\verb=  s0mac_interp_type  = &   The interpolation for filling an edge based multifab from a 1D bin-centered array. \newline 1 = Interpolate s0 to cell centers (with s0\_interp\_type), then average to edges; \newline 2 = Interpolate s0 to edges directly using linear interpolation; \newline 3 = Interpolate s0 to edges directly using quadratic interpolation.  &  1 \\
\rowcolor{tableShade}
\verb=  w0_interp_type  = &   The interpolation for filling a cell-centered multifab from a 1D edge-centered array. \newline 1 = piecewise constant; \newline 2 = piecewise linear; \newline 3 = quadratic  &  2 \\
\verb=  w0mac_interp_type  = &   The interpolation for putting w0 on edges.  We only compute the normal component. \newline 1 = Interpolate w0 to cell centers (with w0\_interp\_type), then average to edges; \newline 2 = Interpolate w0 to edges directly using linear interpolation; \newline 3 = Interpolate w0 to edges directly using quadratic interpolation; \newline 4 = Interpolate w0 to nodes using linear interpolation, then average to edges.  &  1 \\


\end{longtable}
\end{center}

} % ends \small


{\small

\renewcommand{\arraystretch}{1.5}
%
\begin{center}
\begin{longtable}{|l|p{5.25in}|l|}
\caption[ burning
 parameters.]{ burning
 parameters.} \label{table:runtime} \\
%
\hline \multicolumn{1}{|c|}{\textbf{parameter}} & 
       \multicolumn{1}{ c|}{\textbf{description}} & 
       \multicolumn{1}{ c|}{\textbf{default value}} \\ \hline 
\endfirsthead

\multicolumn{3}{c}%
{{\tablename\ \thetable{}---continued}} \\
\hline \multicolumn{1}{|c|}{\textbf{parameter}} & 
       \multicolumn{1}{ c|}{\textbf{description}} & 
       \multicolumn{1}{ c|}{\textbf{default value}} \\ \hline 
\endhead

\multicolumn{3}{|r|}{{\em continued on next page}} \\ \hline
\endfoot

\hline 
\endlastfoot


\rowcolor{tableShade}
\verb=  burner_threshold_cutoff  = &   Mass fraction cutoff for burner\_threshold\_species  used in burner threshold  &  1.d-10 \\
\verb=  burner_threshold_species  = &   Name of the species to be used in burner threshold  &  "" \\
\rowcolor{tableShade}
\verb=  do_burning  = &   turn on (.true.) or off (.false.) burning  &  .true. \\


\end{longtable}
\end{center}

} % ends \small


{\small

\renewcommand{\arraystretch}{1.5}
%
\begin{center}
\begin{longtable}{|l|p{5.25in}|l|}
\caption[ general MAESTRO
 parameters.]{ general MAESTRO
 parameters.} \label{table:runtime} \\
%
\hline \multicolumn{1}{|c|}{\textbf{parameter}} & 
       \multicolumn{1}{ c|}{\textbf{description}} & 
       \multicolumn{1}{ c|}{\textbf{default value}} \\ \hline 
\endfirsthead

\multicolumn{3}{c}%
{{\tablename\ \thetable{}---continued}} \\
\hline \multicolumn{1}{|c|}{\textbf{parameter}} & 
       \multicolumn{1}{ c|}{\textbf{description}} & 
       \multicolumn{1}{ c|}{\textbf{default value}} \\ \hline 
\endhead

\multicolumn{3}{|r|}{{\em continued on next page}} \\ \hline
\endfoot

\hline 
\endlastfoot


\rowcolor{tableShade}
\verb=  barrier_timers  = &   Call a parallel\_barrier before each of the simple timers we have coded in advance.f90.  Might cause a very slight overall performance hit.  &  .false. \\
\verb=  do_alltoallv  = &   use the MPI alltoallv call  &  .false. \\
\rowcolor{tableShade}
\verb=  job_name  = &   job name printed in output  &  "" \\
\verb=  the_knapsack_verbosity  = &   Verbosity of the knapscak processor-to-grid algorithm.  &  .false. \\
\rowcolor{tableShade}
\verb=  verbose  = &   General verbosity  &  0 \\


\end{longtable}
\end{center}

} % ends \small


{\small

\renewcommand{\arraystretch}{1.5}
%
\begin{center}
\begin{longtable}{|l|p{5.25in}|l|}
\caption[ grid
 parameters.]{ grid
 parameters.} \label{table:runtime} \\
%
\hline \multicolumn{1}{|c|}{\textbf{parameter}} & 
       \multicolumn{1}{ c|}{\textbf{description}} & 
       \multicolumn{1}{ c|}{\textbf{default value}} \\ \hline 
\endfirsthead

\multicolumn{3}{c}%
{{\tablename\ \thetable{}---continued}} \\
\hline \multicolumn{1}{|c|}{\textbf{parameter}} & 
       \multicolumn{1}{ c|}{\textbf{description}} & 
       \multicolumn{1}{ c|}{\textbf{default value}} \\ \hline 
\endhead

\multicolumn{3}{|r|}{{\em continued on next page}} \\ \hline
\endfoot

\hline 
\endlastfoot


\rowcolor{tableShade}
\verb=  amr_buf_width  = &   the number of buffer zones surrounding a cell tagged for refinement. note that this needs to be >= regrid\_int  &  -1 \\
\verb=  bcx_hi  = &   $+x$ boundary condition (valid values are listed in {\tt  boxlib/bc.f90})  &  SLIP\_WALL \\
\rowcolor{tableShade}
\verb=  bcx_lo  = &   $-x$ boundary condition  &  SLIP\_WALL \\
\verb=  bcy_hi  = &   $+y$ boundary condition  &  SLIP\_WALL \\
\rowcolor{tableShade}
\verb=  bcy_lo  = &   $-y$ boundary condition  &  SLIP\_WALL \\
\verb=  bcz_hi  = &   $+z$ boundary condition  &  SLIP\_WALL \\
\rowcolor{tableShade}
\verb=  bcz_lo  = &   $-z$ boundary condition  &  SLIP\_WALL \\
\verb=  blocking_factor  = &   grids must be an integer multiple of this number, if possible  &  8 \\
\rowcolor{tableShade}
\verb=  change_max_grid_size_1  = &   Change the max grid size on the base level on restart  &  .false. \\
\verb=  dm_in  = &   dimensionality (valid values are 2 or 3)  &  2 \\
\rowcolor{tableShade}
\verb=  drdxfac  = &   ratio of radial base state zones to Cartesian full state zones for spherical geometry  &  1 \\
\verb=  dump_grid_file  = &   dump out a file, named grids.out, containing the number of grids a each level. A new line is added each time regrid is called.  &  .false. \\
\rowcolor{tableShade}
\verb=  max_grid_size  = &   The largest grid size that will be created using make\_new\_grids.  &  64 \\
\verb=  max_grid_size_1  = &   The largest grid size that will be created using make\_new\_grids for the coarsest level.  Defaults to {\tt max\_grid\_size}.  &  -1 \\
\rowcolor{tableShade}
\verb=  max_grid_size_2  = &   The largest grid size that will be created using make\_new\_grids for level 2 (the first refined level).  Defaults to {\tt max\_grid\_size}.  &  -1 \\
\verb=  max_grid_size_3  = &   The largest grid size that will be created using make\_new\_grids for level 3 (the second refined level) and beyond.  Defaults to {\tt max\_grid\_size}.  &  -1 \\
\rowcolor{tableShade}
\verb=  max_levs  = &   Total number of levels.  1 = single level.  &  1 \\
\verb=  min_eff  = &   parameter for cluster algorithm for making new grids in adaptive problems  &  0.9d0 \\
\rowcolor{tableShade}
\verb=  minwidth  = &   The minimum size on a side for a grid created using make\_new\_grids.  &  8 \\
\verb=  n_cellx  = &   Number of cells for the base level in the x-direction  &  -1 \\
\rowcolor{tableShade}
\verb=  n_celly  = &   Number of cells for the base level in the y-direction  &  -1 \\
\verb=  n_cellz  = &   Number of cells for the base level in the z-direction  &  -1 \\
\rowcolor{tableShade}
\verb=  octant  = &   set octant = T if you just want to model an octant of a sphere (note: only takes effect for spherical geometry)  &  .false. \\
\verb=  prob_hi_x  = &   physical coordinates of hi-x corner of problem domain  &  1.d0 \\
\rowcolor{tableShade}
\verb=  prob_hi_y  = &   physical coordinates of hi-y corner of problem domain  &  1.d0 \\
\verb=  prob_hi_z  = &   physical coordinates of hi-z corner of problem domain  &  1.d0 \\
\rowcolor{tableShade}
\verb=  prob_lo_x  = &   physical coordinates of lo-x corner of problem domain  &  ZERO \\
\verb=  prob_lo_y  = &   physical coordinates of lo-y corner of problem domain  &  ZERO \\
\rowcolor{tableShade}
\verb=  prob_lo_z  = &   physical coordinates of lo-z corner of problem domain  &  ZERO \\
\verb=  ref_ratio  = &   Refinement ratio for multilevel problems  &  2 \\
\rowcolor{tableShade}
\verb=  regrid_int  = &   How often we regrid.  &  -1 \\
\verb=  spherical_in  = &   set to 1 if you are doing a spherical problem  &  0 \\
\rowcolor{tableShade}
\verb=  test_set  = &   Fixed grid file.  &  '' \\
\verb=  the_copy_cache_max  = &    Number of boxassoc layouts we keep in memory to avoid having to recompute the boxassoc, which is computationally expensive.  &  25 \\
\rowcolor{tableShade}
\verb=  the_layout_verbosity  = &    &  0 \\
\verb=  the_sfc_threshold  = &   When assigning processors for grids, this determines whether we use the sfc algorithm or knapsack algorithm.  If the total number of grids divided by the number of processors is greater than this number, use sfc.  &  5 \\


\end{longtable}
\end{center}

} % ends \small


{\small

\renewcommand{\arraystretch}{1.5}
%
\begin{center}
\begin{longtable}{|l|p{5.25in}|l|}
\caption[ heating
 parameters.]{ heating
 parameters.} \label{table:runtime} \\
%
\hline \multicolumn{1}{|c|}{\textbf{parameter}} & 
       \multicolumn{1}{ c|}{\textbf{description}} & 
       \multicolumn{1}{ c|}{\textbf{default value}} \\ \hline 
\endfirsthead

\multicolumn{3}{c}%
{{\tablename\ \thetable{}---continued}} \\
\hline \multicolumn{1}{|c|}{\textbf{parameter}} & 
       \multicolumn{1}{ c|}{\textbf{description}} & 
       \multicolumn{1}{ c|}{\textbf{default value}} \\ \hline 
\endhead

\multicolumn{3}{|r|}{{\em continued on next page}} \\ \hline
\endfoot

\hline 
\endlastfoot


\rowcolor{tableShade}
\verb=  do_heating  = &   use analytic heating  &  .false. \\


\end{longtable}
\end{center}

} % ends \small


{\small

\renewcommand{\arraystretch}{1.5}
%
\begin{center}
\begin{longtable}{|l|p{5.25in}|l|}
\caption[ hydrodynamics
 parameters.]{ hydrodynamics
 parameters.} \label{table:runtime} \\
%
\hline \multicolumn{1}{|c|}{\textbf{parameter}} & 
       \multicolumn{1}{ c|}{\textbf{description}} & 
       \multicolumn{1}{ c|}{\textbf{default value}} \\ \hline 
\endfirsthead

\multicolumn{3}{c}%
{{\tablename\ \thetable{}---continued}} \\
\hline \multicolumn{1}{|c|}{\textbf{parameter}} & 
       \multicolumn{1}{ c|}{\textbf{description}} & 
       \multicolumn{1}{ c|}{\textbf{default value}} \\ \hline 
\endhead

\multicolumn{3}{|r|}{{\em continued on next page}} \\ \hline
\endfoot

\hline 
\endlastfoot


\rowcolor{tableShade}
\verb=  anelastic_cutoff  = &   The density below which we modify the constraint to look like the anelastic constraint, instead of the low Mach constraint.  This prevents velocities from getting out of hand at the edge of the star. Refer to Section \ref{Sec:Anelastic Cutoff}.  &  3.d6 \\
\verb=  base_cutoff_density  = &   The density below which we keep the initial model constant. Refer to Section \ref{Sec:Base Cutoff Density}  &  3.d6 \\
\rowcolor{tableShade}
\verb=  bds_type  = &   0 = use ppm instead for multi-d integrator \newline 1 = bilinear  &  0 \\
\verb=  beta_type  = &   what type of coefficient to use inside the velocity divergence constraint. \newline {\tt beta\_type} = 1 uses $\beta_0$; \newline {\tt beta\_type} = 2 uses $\rho_0$ (anelastic); \newline {\tt beta\_type} = 3 uses 1 (small-scale combustion).  &  1 \\
\rowcolor{tableShade}
\verb=  buoyancy_cutoff_factor  = &   The multiplicative factor (over base\_cutoff\_density) below which we do zero out the buoyancy term in the momentum equation.  &  5.0 \\
\verb=  co_latitude  = &   latitude, in radians, for problems with rotation where the domain is only a subset of a full star.  &  ZERO \\
\rowcolor{tableShade}
\verb=  do_eos_h_above_cutoff  = &   After the advective enthalpy update, recompute the enthalpy if we are below the base cutoff density.  &  .true. \\
\verb=  do_planar_invsq_grav  = &   are we doing 1/r$^2$ gravity for plane-parallel  &  .false. \\
\rowcolor{tableShade}
\verb=  do_smallscale  = &   force $\rho_0 = (\rho h)_0 = 0$, {\tt evolve\_base\_state = F} and {\tt beta\_type} = 3  &  .false. \\
\verb=  do_sponge  = &   Use sponging.  &  .false. \\
\rowcolor{tableShade}
\verb=  dpdt_factor  = &   factor in front of the volume discrepancy term (0.0 = off)  &  0.d0 \\
\verb=  drive_initial_convection  = &   freeze the temperature used in the reaction network to the initial value.  This is useful for developing an initial convective field to carry away the energy, while preventing the reactions from going nonlinear.  &  .false. \\
\rowcolor{tableShade}
\verb=  enthalpy_pred_type  = &   predict\_rhohprime = 1; \newline predict\_h = 2; \newline predict\_T\_then\_rhohprime = 3; \newline predict\_T\_then\_h = 4; \newline predict\_hprime = 5; \newline predict\_Tprime\_then\_h = 6.  &  predict\_rhohprime \\
\verb=  evolve_base_state  = &   turn on (.true.) or off (.false.) basestate evolution  &  .true. \\
\rowcolor{tableShade}
\verb=  fix_base_state  = &   if true, don't call average to reset the base state at all, even during initialization  &  .false. \\
\verb=  grav_const  = &   the gravitational acceleration (cm~s$^{-2}$) for plane-parallel geometry  &  -1.5d10 \\
\rowcolor{tableShade}
\verb=  mach_max_abort  = &   maximum mach number before the code aborts  &  -1.d0 \\
\verb=  planar_invsq_mass  = &   the point mass for planar 1/r$^2$ gravity  &  0.d0 \\
\rowcolor{tableShade}
\verb=  plot_sponge_fdamp  = &   plot fdamp rather than sponge assumes sponge has the form 1/(1+dt*{\tt sponge\_kappa}*fdamp)  &  .false. \\
\verb=  ppm_type  = &   0 = no ppm \newline 1 = 1985 ppm \newline 2 = Hybrid Sekora/Colella McCorquodale Colella 2009/2010 ppm  &  1 \\
\rowcolor{tableShade}
\verb=  restart_with_vel_field  = &   restart the simulation using a result from a {\tt drive\_initial\_convection} = T run note that this uses the restart variable to specify which file to restart from.  After reading in the velocity information from the restart file, the time and timestep number are zeroed.  &  .false. \\
\verb=  rotation_radius  = &   radius used for computing centrifugal term in rotation problems  &  1.0d6 \\
\rowcolor{tableShade}
\verb=  rotational_frequency  = &   rotational frequency used for computing centrifugal term in rotation problems.  &  ZERO \\
\verb=  slope_order  = &   order of slopes in Godunov algorithm.  Options are 0, 2, or 4.  &  4 \\
\rowcolor{tableShade}
\verb=  species_pred_type  = &   Which quantities do we predict to the edges for computing the ($\rho X$) edge states?  \newline {\tt species\_pred\_type} = 1 means predict $\rho^\prime$ and $X$ separately.  \newline {\tt species\_pred\_type} = 2 means predict the full ($\rho X$) itself. \newline {\tt species\_pred\_type} = 3 means predict $\rho$ and $X$ separately.  &  1 \\
\verb=  sponge_center_density  = &   Center of the inner sponge.  &  3.d6 \\
\rowcolor{tableShade}
\verb=  sponge_kappa  = &   Parameter for sponge.  Problem dependent.  &  10.d0 \\
\verb=  sponge_start_factor  = &   The sponge begins at sponge\_center\_density * sponge\_start\_factor.  &  3.333d0 \\
\rowcolor{tableShade}
\verb=  use_delta_gamma1_term  = &   turns on second order correction to delta gamma1 term  &  .false. \\
\verb=  use_etarho  = &   turn on the etarho term as described in flow chart  &  .true. \\
\rowcolor{tableShade}
\verb=  use_linear_grav_in_beta  = &   how to represent gravity in the beta0 integration: .true. = piecewise linear .false. = piecewise constant  &  .false. \\
\verb=  use_tfromp  = &   When updating temperature, use $T=T(\rho,p_0,X) $ rather than $T=T(\rho,h,X)$.  &  .false. \\


\end{longtable}
\end{center}

} % ends \small


{\small

\renewcommand{\arraystretch}{1.5}
%
\begin{center}
\begin{longtable}{|l|p{5.25in}|l|}
\caption[ linear solvers
 parameters.]{ linear solvers
 parameters.} \label{table:runtime} \\
%
\hline \multicolumn{1}{|c|}{\textbf{parameter}} & 
       \multicolumn{1}{ c|}{\textbf{description}} & 
       \multicolumn{1}{ c|}{\textbf{default value}} \\ \hline 
\endfirsthead

\multicolumn{3}{c}%
{{\tablename\ \thetable{}---continued}} \\
\hline \multicolumn{1}{|c|}{\textbf{parameter}} & 
       \multicolumn{1}{ c|}{\textbf{description}} & 
       \multicolumn{1}{ c|}{\textbf{default value}} \\ \hline 
\endhead

\multicolumn{3}{|r|}{{\em continued on next page}} \\ \hline
\endfoot

\hline 
\endlastfoot


\rowcolor{tableShade}
\verb=  cg_verbose  = &   Verbosity of bottom solver  &  0 \\
\verb=  hg_bottom_solver  = &   valid values are $\le$ 0  &  -1 \\
\rowcolor{tableShade}
\verb=  hg_dense_stencil  = &   In hgproject, in 2D, use a 9 point Laplacian (.true.) or 5-point Laplacian (.false.).  In 3D, use a 27 point Laplacian (.true.) or 7-point Laplacian (.false.).  &  .true. \\
\verb=  max_mg_bottom_nlevels  = &   if mg\_bottom\_solver == 4, then how many mg levels can the bottom solver mgt object have  &  1000 \\
\rowcolor{tableShade}
\verb=  mg_bottom_solver  = &   valid values are $\le$ 0  &  -1 \\
\verb=  mg_verbose  = &   Verbsoity of the multigrid solver, but not the bottom solver.  &  0 \\
\rowcolor{tableShade}
\verb=  use_hypre  = &   use the hypre library  &  .false. \\


\end{longtable}
\end{center}

} % ends \small


{\small

\renewcommand{\arraystretch}{1.5}
%
\begin{center}
\begin{longtable}{|l|p{5.25in}|l|}
\caption[ output
 parameters.]{ output
 parameters.} \label{table:runtime} \\
%
\hline \multicolumn{1}{|c|}{\textbf{parameter}} & 
       \multicolumn{1}{ c|}{\textbf{description}} & 
       \multicolumn{1}{ c|}{\textbf{default value}} \\ \hline 
\endfirsthead

\multicolumn{3}{c}%
{{\tablename\ \thetable{}---continued}} \\
\hline \multicolumn{1}{|c|}{\textbf{parameter}} & 
       \multicolumn{1}{ c|}{\textbf{description}} & 
       \multicolumn{1}{ c|}{\textbf{default value}} \\ \hline 
\endhead

\multicolumn{3}{|r|}{{\em continued on next page}} \\ \hline
\endfoot

\hline 
\endlastfoot


\rowcolor{tableShade}
\verb=  check_base_name  = &   prefix to use in checkpoint file names  &  "chk" \\
\verb=  chk_int  = &   Number of timesteps between writing a checkpoint file  &  0 \\
\rowcolor{tableShade}
\verb=  diag_buf_size  = &   number of timesteps to buffer diagnostic output information before writing (note: not implemented for all problems)  &  10 \\
\verb=  lUsingNFiles  = &   If .true., use nOutFiles processors to write checkpoint and plotfiles. Fortran has the unfortunate feature of each processor only being able to write out 1-2GB each without crashing.  &  .true. \\
\rowcolor{tableShade}
\verb=  nOutFiles  = &   If lUsingNFiles = .true., use this many processors to write checkpoint and plotfiles.  Fortran has the unfortunate feature of each processor only being able to write out 1-2GB each without crashing.  &  64 \\
\verb=  plot_Hext  = &   plot external heating (Hext) in plotfile  &  .false. \\
\rowcolor{tableShade}
\verb=  plot_Hnuc  = &   plot nuclear energy generation rate (Hnuc) in plotfile  &  .true. \\
\verb=  plot_ad_excess  = &   plot the adiabatic excess  &  .false. \\
\rowcolor{tableShade}
\verb=  plot_base  = &   plot w0\_x, w0\_y, w0\_z, divw0, rho0, rhoh0, h0, and p0 in plotfile  &  .false. \\
\verb=  plot_base_name  = &   prefix to use in plotfile file names  &  "plt" \\
\rowcolor{tableShade}
\verb=  plot_cs  = &   plot soundspeed  &  .false. \\
\verb=  plot_deltat  = &   rather than use a plot interval, plot a file after the solution has advanced past plot\_deltat in time  &  -1.d0 \\
\rowcolor{tableShade}
\verb=  plot_eta  = &   plot $\etarho$ in plotfile  &  .false. \\
\verb=  plot_gpi  = &   plot pi and grad(pi)  &  .true. \\
\rowcolor{tableShade}
\verb=  plot_h_with_use_tfromp  = &   Turn on storing of enthalpy-based quantities in the plotfile when we are running with {\tt use\_tfromp}  &  .true. \\
\verb=  plot_int  = &   plot interval  &  0 \\
\rowcolor{tableShade}
\verb=  plot_omegadot  = &   plot omegadot in plotfile  &  .true. \\
\verb=  plot_processors  = &   create a field in the plotfile storing the processor number for each zone  &  .false. \\
\rowcolor{tableShade}
\verb=  plot_spec  = &   plot species and omegadot in plotfile  &  .true. \\
\verb=  plot_trac  = &   plot tracers in plotfile  &  .false. \\
\rowcolor{tableShade}
\verb=  single_prec_plotfiles  = &   store the state data in single precision  &  .false. \\


\end{longtable}
\end{center}

} % ends \small


{\small

\renewcommand{\arraystretch}{1.5}
%
\begin{center}
\begin{longtable}{|l|p{5.25in}|l|}
\caption[ particles
 parameters.]{ particles
 parameters.} \label{table:runtime} \\
%
\hline \multicolumn{1}{|c|}{\textbf{parameter}} & 
       \multicolumn{1}{ c|}{\textbf{description}} & 
       \multicolumn{1}{ c|}{\textbf{default value}} \\ \hline 
\endfirsthead

\multicolumn{3}{c}%
{{\tablename\ \thetable{}---continued}} \\
\hline \multicolumn{1}{|c|}{\textbf{parameter}} & 
       \multicolumn{1}{ c|}{\textbf{description}} & 
       \multicolumn{1}{ c|}{\textbf{default value}} \\ \hline 
\endhead

\multicolumn{3}{|r|}{{\em continued on next page}} \\ \hline
\endfoot

\hline 
\endlastfoot


\rowcolor{tableShade}
\verb=  store_particle_vels  = &   store the velocity of the particle?  &  .false. \\
\verb=  use_particles  = &   call the particle initialization, advection, etc. routines.  &  .false. \\


\end{longtable}
\end{center}

} % ends \small


{\small

\renewcommand{\arraystretch}{1.5}
%
\begin{center}
\begin{longtable}{|l|p{5.25in}|l|}
\caption[ problem initialization
 parameters.]{ problem initialization
 parameters.} \label{table:runtime} \\
%
\hline \multicolumn{1}{|c|}{\textbf{parameter}} & 
       \multicolumn{1}{ c|}{\textbf{description}} & 
       \multicolumn{1}{ c|}{\textbf{default value}} \\ \hline 
\endfirsthead

\multicolumn{3}{c}%
{{\tablename\ \thetable{}---continued}} \\
\hline \multicolumn{1}{|c|}{\textbf{parameter}} & 
       \multicolumn{1}{ c|}{\textbf{description}} & 
       \multicolumn{1}{ c|}{\textbf{default value}} \\ \hline 
\endhead

\multicolumn{3}{|r|}{{\em continued on next page}} \\ \hline
\endfoot

\hline 
\endlastfoot


\rowcolor{tableShade}
\verb=  model_file  = &   input model file  &  "model.hse" \\
\verb=  perturb_model  = &   Turn on a perturbation in the initial data.  Problem specific.  &  .false. \\


\end{longtable}
\end{center}

} % ends \small


{\small

\renewcommand{\arraystretch}{1.5}
%
\begin{center}
\begin{longtable}{|l|p{5.25in}|l|}
\caption[ thermal diffusion
 parameters.]{ thermal diffusion
 parameters.} \label{table:runtime} \\
%
\hline \multicolumn{1}{|c|}{\textbf{parameter}} & 
       \multicolumn{1}{ c|}{\textbf{description}} & 
       \multicolumn{1}{ c|}{\textbf{default value}} \\ \hline 
\endfirsthead

\multicolumn{3}{c}%
{{\tablename\ \thetable{}---continued}} \\
\hline \multicolumn{1}{|c|}{\textbf{parameter}} & 
       \multicolumn{1}{ c|}{\textbf{description}} & 
       \multicolumn{1}{ c|}{\textbf{default value}} \\ \hline 
\endhead

\multicolumn{3}{|r|}{{\em continued on next page}} \\ \hline
\endfoot

\hline 
\endlastfoot


\rowcolor{tableShade}
\verb=  conductivity_constant  = &   the constant conductivity (only applies if the conductivity/constant module is used)  &  1.0d0 \\
\verb=  temp_diffusion_formulation  = &   How to compute the explicit thermal diffusion term.  \newline 1 = in terms of $T$; \newline 2 = in terms of $\rho,p_0,X$.  &  2 \\
\rowcolor{tableShade}
\verb=  thermal_diffusion_type  = &   In the thermal diffusion solver, 1 = Crank-Nicholson; 2 = Backward Euler.  &  1 \\
\verb=  use_thermal_diffusion  = &   Use thermal diffusion.  &  .false. \\


\end{longtable}
\end{center}

} % ends \small


{\small

\renewcommand{\arraystretch}{1.5}
%
\begin{center}
\begin{longtable}{|l|p{5.25in}|l|}
\caption[ timestepping 
 parameters.]{ timestepping 
 parameters.} \label{table:runtime} \\
%
\hline \multicolumn{1}{|c|}{\textbf{parameter}} & 
       \multicolumn{1}{ c|}{\textbf{description}} & 
       \multicolumn{1}{ c|}{\textbf{default value}} \\ \hline 
\endfirsthead

\multicolumn{3}{c}%
{{\tablename\ \thetable{}---continued}} \\
\hline \multicolumn{1}{|c|}{\textbf{parameter}} & 
       \multicolumn{1}{ c|}{\textbf{description}} & 
       \multicolumn{1}{ c|}{\textbf{default value}} \\ \hline 
\endhead

\multicolumn{3}{|r|}{{\em continued on next page}} \\ \hline
\endfoot

\hline 
\endlastfoot


\rowcolor{tableShade}
\verb=  cflfac  = &   CFL factor to use in the computation of the advection timestep constraint  &  0.5d0 \\
\verb=  fixed_dt  = &   Fix the time step.  If -1.0, then use the standard time step.  &  -1.0d0 \\
\rowcolor{tableShade}
\verb=  init_shrink  = &   the multiplicative factor ($\le 1$) to reduce the initial timestep as computed by the various timestep estimators  &  1.d0 \\
\verb=  max_dt  = &   This is the maximum dt that is allowed  &  1.0d33 \\
\rowcolor{tableShade}
\verb=  max_dt_growth  = &   The maximum scale factor that the time step is allowed to grow by per time step.  &  1.1d0 \\
\verb=  max_step  = &   Maximum number of steps in the simulation.  &  1 \\
\rowcolor{tableShade}
\verb=  nuclear_dt_fac  = &   If $T_{max}^n > T_{max}^{n-1}$ set the new dt =   min[dt, dt*{\tt nuclear\_dt\_fac}*( $T_{max}^{n-1}$ / $(T_{max}^n-T_{max}^{n-1})$ ) ] for example, {\tt nuclear\_dt\_fac} = 0.01 means don't let the max temp grow more than approximately 1 percent not checkpoint-compatible yet since it wouldn't be backwards compatible  &  -1.0d0 \\
\verb=  small_dt  = &   the minimum allowed timestep -- we abort if dt drops below this value  &  1.d-10 \\
\rowcolor{tableShade}
\verb=  stop_time  = &   simulation stop time  &  -1.d0 \\
\verb=  use_divu_firstdt  = &   Use the divu constraint when computing the first time step.  &  .false. \\
\rowcolor{tableShade}
\verb=  use_soundspeed_firstdt  = &   Use the soundspeed constraint when computing the first time step.  &  .false. \\


\end{longtable}
\end{center}

} % ends \small


\end{landscape}

%


