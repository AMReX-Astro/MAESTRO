\begin{enumerate}

\item {\em  How do we turn off all the initial projections to look at the
   initial velocity field as specified in initdata, instead of as
   modified by the velocity constraint?} 
%
\begin{verbatim}
    max_step  = 1
    init_iter = 0
    init_divu_iter = 0
    do_initial_projection = T
\end{verbatim}


\item {\em How can we dump out a variable to a plotfile from any point in the
   code?} 
%
\begin{verbatim}
    use fabio_module

    call fabio_ml_multifab_write_d(uold,mla%mba%rr(:,1),"a_uold")
    call fabio_ml_multifab_write_d(umac(:,1),mla%mba%rr(:,1),"a_umacx")
\end{verbatim}

\item {\em Does it matter if we use a Fortran 90 compiler instead of a 
   Fortran 95 compiler?}


   On some machines it does---Fortran 95 specifies that any local
   allocated arrays are automatically deallocated when a subroutine
   ends.  Fortran 90 does not do this.  On IBM machines, for instance,
   using {\tt xlf90} instead of {\tt xlf95} for MAESTRO will produce
   memory leaks.


\item {\em How can I tell from a plotfile what runtime parameters were
   used for its run? or when it was created?}

   In each plotfile directory, there is a file called {\tt job\_info}
   (e.g.\ {\tt plt00000/job\_info}) that lists the build directory and
   date, as well as the value of every runtime parameter for the run.

\end{enumerate}
