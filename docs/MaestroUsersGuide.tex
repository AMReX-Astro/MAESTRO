\documentclass[11pt]{book} 

\tolerance=600

\usepackage{longtable,epsfig,epsf,amsmath}
%\usepackage{hyperref}
%\usepackage[T1]{fontenc}

\usepackage[usenames]{color}
\usepackage{listings}

\definecolor{AntiqueWhite3}{rgb}{0.804,0.753,0.69}
\definecolor{DarkerAntiqueWhite3}{rgb}{0.659,0.635,0.612}
\definecolor{orange}{rgb}{1.0,0.65,0.0}

\lstset{%
  keywordstyle=\color{blue}\ttfamily,%
  commentstyle=\itshape\color[gray]{0.5},%
  mathescape=true,%
  basicstyle=\small\ttfamily,%
  frameround=fttt,%
  frame=shadowbox,%
  rulesepcolor=\color{DarkerAntiqueWhite3},%
  backgroundcolor=\color{AntiqueWhite3},%
  emph={load,add_slice,save}, emphstyle=\color{blue},%
  emph={[2]In}, emphstyle=[2]\color{yellow},%
  emph={[3]Out}, emphstyle=[3]\color{orange},%
  mathescape=false}


% Margins
\usepackage[top=1.0in,bottom=1.0in,inner=0.85in,outer=0.7in]{geometry}

\newcommand{\rhozero}{\rho_0}
\newcommand{\pizero}{\pi_0}
\newcommand{\pizeroone}{\pi_0^{(1)}}
\newcommand{\pizerotwo}{\pi_0^{(2)}}
\newcommand{\gammabar}{\overline{\Gamma}_1}

\newcommand{\figpath}{./initial_models}
\newcommand{\lodensfigpath}{./lo_density}
\newcommand{\flowfigpath}{./flowchart}
\newcommand{\radbasefigpath}{./radial_base}
\newcommand{\sphericalfigpath}{./spherical_basestate}
\newcommand{\archfigpath}{./architecture}
\newcommand{\rotfigpath}{./rotation}
\newcommand{\visfigpath}{./visualization}

\newcommand{\apj}{ApJ}
\newcommand{\apjs}{ApJS}

\def\ptl   {\partial}

\def\eb    {{\bf e}}
\def\fb    {{\bf f}}
\def\ib    {{\bf i}}
\def\Ub    {{\bf U}}
\def\Vb    {{\bf V}}
\def\xb    {{\bf x}}
\def\ut    {\tilde{u}}
\def\vt    {\tilde{v}}
\def\wt    {\tilde{w}}

\newcommand{\Ubt}{\widetilde{\Ub}}

\def\edge  {\rm EDGE}
\def\mac   {\rm MAC}
\def\trans {\rm TRANS}

\newcommand{\nablab}{\mathbf{\nabla}}
\newcommand{\cdotb}{\mathbf{\cdot}}

\newcommand{\sfrac}[2]{\mathchoice
  {\kern0em\raise.5ex\hbox{\the\scriptfont0 #1}\kern-.15em/
   \kern-.15em\lower.25ex\hbox{\the\scriptfont0 #2}}
  {\kern0em\raise.5ex\hbox{\the\scriptfont0 #1}\kern-.15em/
   \kern-.15em\lower.25ex\hbox{\the\scriptfont0 #2}}
  {\kern0em\raise.5ex\hbox{\the\scriptscriptfont0 #1}\kern-.2em/
   \kern-.15em\lower.25ex\hbox{\the\scriptscriptfont0 #2}}
  {#1\!/#2}}

\newcommand{\half}{\frac{1}{2}}
\newcommand{\myhalf}{\sfrac{1}{2}}
\newcommand{\nph}{{n + \myhalf}}
\newcommand{\nmh}{{n - \myhalf}}

\newcommand{\Hext}{{H_{\rm ext}}}
\newcommand{\Hnuc}{{H_{\rm nuc}}}
\newcommand{\kth}{k_{\rm th}}
\def\init  {\rm init}
\def\model {\rm model}
\def\raw   {\rm raw}

\newcommand{\pred}{{\rm pred}}
\newcommand{\Sbar}{\overline{S}}

\newcommand{\inp}{\mathrm{in}}
\newcommand{\initp}{\mathrm{init}}
\newcommand{\outp}{\mathrm{out}}

\newcommand{\uadv}{\widetilde{\mathbf{U}}^{\mathrm{ADV}}}
\newcommand{\uadvone}{\widetilde{\mathbf{U}}^{\mathrm{ADV},\star}}
\newcommand{\uadvonedag}{\Ubt^{\mathrm{ADV},\dagger,\star}}
\newcommand{\uadvtwo}{\widetilde{\mathbf{U}}^{\mathrm{ADV}}}
\newcommand{\uadvtwodag}{\Ubt^{\mathrm{ADV},\dagger}}

\newcommand{\dt}{\Delta t}
\newcommand{\dr}{\Delta r}

\newcommand{\etarho}{\eta_{\rho}}
\def\etarhoec   {\etarho^{\rm ec}}
\def\etarhocc   {\etarho^{\rm cc}}
\def\etarhoflux {\etarho^{\rm flux}}
\def\divetarho  {\nabla\cdot(\etarho\eb_r)}

\newcommand{\ow}{\overline{w}_0}
\newcommand{\dw}{\delta w_0}
\newcommand{\thalf}{\sfrac{3}{2}}

\newcommand{\rhop}{{\rho^{\prime}}}

\newcommand{\omegadot}{\dot{\omega}}
\newcommand{\er}{\mathbf{e}_r}

\def\Omegab{{\bf \Omega}}
\def\rb    {{\bf r}}


\newcommand{\boxtype}{{\tt box\/}}
\newcommand{\fab}{{\tt fab\/}}
\newcommand{\multifab}{{\tt multifab\/}}
\newcommand{\boxarray}{{\tt boxarray\/}}
\newcommand{\mlboxarray}{{\tt ml\_boxarray\/}}
\newcommand{\layout}{{\tt layout\/}}
\newcommand{\mllayout}{{\tt ml\_layout\/}}
\newcommand{\bctower}{{\tt bc\_tower\/}}


\newcommand{\amrvis}{{\tt AmrVis}\ }
\newcommand{\visit}{{\tt VisIt}\ }
\newcommand{\yt}{{\tt yt}\ }



\setlength{\marginparwidth}{0.8in}
\newcommand{\MarginPar}[1]{\marginpar{%
\vskip-\baselineskip %raise the marginpar a bit
\raggedright\tiny\sffamily
\hrule\smallskip{\color{red}#1}\par\smallskip\hrule}}



\title{\Huge \bf MAESTRO User's Guide}

\begin{document}

\frontmatter

\maketitle

\setcounter{tocdepth}{1}
\tableofcontents

\addcontentsline{toc}{part}{list of figures}
\listoffigures


\cleardoublepage

\chapter*{Preface}
\addcontentsline{toc}{part}{preface}
\input preface/preface

\mainmatter

%-----------------------------------------------------------------------------
\part{Overview of the MAESTRO Algorithm}

\chapter{Introduction}
\input introduction/introduction

\chapter{Algorithm Flowchart}
\input flowchart/flowchart

%-----------------------------------------------------------------------------
\part{Using MAESTRO}

\chapter{Architecture}
\input architecture/architecture

\chapter{Runtime Parameters}
\input runtime_parameters/param_intro
\input runtime_parameters/runtime_parameters

\chapter{Initial Models}
\input initial_models/initial_models

\chapter{Notes on the Low Density Parameters in {\tt MAESTRO}}
\input lo_density/lo_density

\chapter{Managing Jobs}
\input managing_jobs/managingjobs

\chapter{Visualization}
\input visualization/visualization

\chapter{Analysis Routines}
\input analysis/analysis

\chapter{FAQ}
\input FAQ/faq

%-----------------------------------------------------------------------------
\part{MAESTRO Technical Details}

\chapter{Notes on the Volume Discrepancy Term}
\input volume_discrepancy/volume_discrepancy

\chapter{EOS and Temperature Notes}
\input eos_notes/eos_notes

\chapter{Notes on $\etarho$}
\input eta_notes/eta

\chapter{Notes on Perturbational Quantities}
\input pert_notes/pert

\chapter{Notes on Advection}
\input Godunov_notes/Godunov

\chapter{Multigrid Tolerances}
\input mg/mg

\chapter{Rotation}
\input rotation/rotation

\chapter{Radial Base State}
\input radial_base/radial_base

\chapter{Spherical Expansion}
\input spherical_basestate/basestate

\chapter{Thermodynamics}
\input thermo_notes/thermo_notes

\backmatter

%-----------------------------------------------------------------------------
\part{References}

\renewcommand\bibname{References}
\addcontentsline{toc}{chapter}{References}
\bibliographystyle{plain}
\bibliography{maestro_doc}

\end{document}
