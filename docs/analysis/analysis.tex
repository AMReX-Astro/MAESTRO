\section{The {\tt data\_processing} Routines}
\label{sec:analysis}

The {\tt fParallel/data\_processing/} directory contains a large
number of Fortran-based analysis routines for BoxLib datasets.  Many
of these can be used with both MAESTRO and the compressible
astrophysics code, Castro.

To compile any of the individual routines, edit the {\tt GNUmakefile}
add uncomment the line beginning with `{\tt programs +=}' containing
the routine you want to build.

\subsection{General Analysis Routines}

The following routines are generally applicable for any BoxLib-based
plotfile.  Typing the executable names without any arguments will
provide usage examples.

\begin{itemize}

\item {\tt faverage.f90}

  Laterally average each of the variables in a plotfile (works for
  both 2-d and 3-d).  This is written with MAESTRO
  plane-parallel geometry plotfiles in mind, and the averaging is done
  over the coordinate direction(s) perpendicular to gravity.


\item {\tt fboxinfo.f90}

  Print out some basic information about the number of boxes on each
  refinement level and (optionally) the bounds of each of the boxes.

\item {\tt fcompare.f90}

  Compare two plotfiles, zone-by-zone to machine precision, and report
  the L2-norm of the error (both absolute and relative) for each
  variable.  This assumes that the grids are identical. \\[-3mm]

  This is used by in the regression test suite in {\tt
  Parallel/util/regtests/}.


\item {\tt fextract.f90}

  Extract a 1-d line through a dataset (1-, 2-, or 3-d).  This works
  with both uniformly-gridded or AMR datasets.  For multi-dimensional
  datasets, the coordinate direction to extract along can be specified.
  The line is always taken through the center of the domain.  Either
  a single variable or all variables, along with the coordinate 
  information, are output to a file.
  

\item {\tt fextrema.f90}

  Report the min and max of each variable (or only a single variable)
  in one or more plotfiles.


\item {\tt fIDLdump.f90}, {\tt fIDLdump3d.f90}

  Create a binary file containing a uniformly gridded, single variable
  from a plotfile.  This is can then be read into IDL using the routines
  in {\tt fParallel/scripts/idlbl/}.


\item {\tt fsnapshot2d.f90}, {\tt fsnapshot3d.f90}

  Create an image (PPM file) of a single variable in a plotfile.  For
  3-d, the slice plane through the center of the domain is specified.
  Separate routines exist for 2-d and 3-d datasets.
  

\item {\tt ftime.f90}
 
  For each plotfile, simply print the simulation time.


\item {\tt fvarnames.f90}

  Simply print out the list of variables stored in a plotfile.


\subsection{Data Processing Example}

A well-commented example processing routine is provided in {\tt
  data\_processing/tutorial} as {\tt fspeciesmass2d.f90}.  This
routine simply computes the total mass of a particular species on the
entire domain for a 2-d dataset.  It is written to understand a
multilevel (AMR) dataset, and only considers the finest-available data
at any physical location in the computational domain.  

{\tt fspeciesmass2d.f90} should provide a good starting point for
writing a new analysis routine for BoxLib data.

\subsection{Particle routines}

The {\tt parseparticles.py} routine in the {\tt python/} subdirectory
can read in MAESTRO particle files containing particle histories
(usually named {\tt timestamp\_??}.  The driver {\tt test.py} shows
shows how to use this module to plot particle histories.

\end{itemize}

