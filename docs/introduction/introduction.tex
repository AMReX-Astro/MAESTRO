\section{History of MAESTRO}

MAESTRO describes the evolution of low Mach number flows that are in
hydrostatic equilibrium.  The idea for MAESTRO grew out of our success
in applying low Mach number combustion methods to small-scale
astrophysical flows.  Our original small-scale astrophysical
combustion algorithm is detailed in
\begin{itemize}
\item {\em Adaptive Low Mach Number Simulations of Nuclear Flames,}
J. B. Bell, M. S. Day, C. A. Rendleman, S. E. Woosley, \& M. Zingale
2004, JCP, 195, 2, 677 (henceforth BDRWZ)
\end{itemize}

\noindent MAESTRO was developed initially for modeling the convecting
phase in a white dwarf preceding the ignition of a Type Ia supernovae.
As such, we needed to incorporate the compressibility effects due to
large-scale stratification in the star.  The method closest in spirit
to MAESTRO is the pseudo-incompressible method of
Durran~\cite{durran}, developed for terrestrial atmospheric flows
(assuming an ideal gas).  Part of the complexity of the equations in
MAESTRO stems from the need to descibe a general equation of state.
Additionally, since reactions can significantly alter the hydrostatic
structure of a star, we incorporated extensions that capture the
expansion of the background state~\cite{almgren:2000}.  The low Mach
number equations for stellar flows were developed in a series of
papers leading up to the first application to this problem:
\begin{itemize}
\item {\em Low Mach Number Modeling of Type Ia
  Supernovae. I. Hydrodynamics,} A. S. Almgren, J. B. Bell, 
  C. A. Rendleman, \& M. Zingale 2006, ApJ, 637, 922 (henceforth
  paper I)
\item {\em Low Mach Number Modeling of Type Ia Supernovae. II. Energy
  Evolution,} A. S. Almgren, J. B. Bell, C. A. Rendleman, \& M. Zingale
  2006, ApJ, 649, 927 (henceforth paper II)
\item {\em Low Mach Number Modeling of Type Ia Supernovae. III. Reactions,}
  A. S. Almgren, J. B. Bell, A. Nonaka, \& M. Zingale
  2008, ApJ, 684, 449 (henceforth paper III)
\item {\em Low Mach Number Modeling of Type Ia Supernovae. IV. White Dwarf Convection,}
  M. Zingale, A. S. Almgren, J. B. Bell, A. Nonaka, \& S. E. Woosley
  2009, ApJ, 704, 196 (henceforth paper IV)
\end{itemize}

\noindent The current version of the algorithm is described in our
multilevel paper:
\begin{itemize}
\item {\em MAESTRO: An Adaptive Low Mach Number Hydrodynamics Algorithm for Stellar
  Flows,} A. Nonaka, A. S. Almgren, J. B. Bell, M. J. Lijewski, C. M. Malone,
  \& M. Zingale 2010, ApJS, 188, 358 (henceforth ``the multilevel paper'')
\end{itemize}

\noindent We have two papers in preparation that describe applications of the
method:
\begin{itemize}
\item {\em The Convective Phase Preceding Type Ia Supernovae,}
  M. Zingale, A. S. Almgren, J. B. Bell, C. M. Malone, A. Nonaka, \& S. E. Woosley 2010,
  in preparation (henceforth paper V)
\item {\em Multidimensional Modeling of Type I X-ray Bursts,}
  C. M. Malone, A. Nonaka, A. S. Almgren, J. B. Bell, \& M. Zingale 2010,
  in preparation (henceforth ``the XRB paper'')
\end{itemize}


\section{Projection Methods 101}

Most astrophysical hydrodynamics codes
(e.g.\ CASTRO~\cite{castro} or FLASH~\cite{flash}) solve the
compressible Euler equations, which can be written in the form:
\begin{equation}
U_t + \nabla \cdot F(U) = 0
\end{equation}
where U is the vector of conserved quantities, $U = (\rho, \rho u,
\rho E)$, with $\rho$ the density, $u$ the velocity, and $E$ the total
energy per unit mass.  This system of equations can be expressed 
as a system of advection equations:
\begin{equation}
q_t + A(q) q_x = 0
\end{equation}
where $q$ are called the primitive variables, and $A$ is the
Jacobian, $A \equiv \partial F / \partial U$.  The eigenvalues of the
matrix $A$ are the characteristic speeds---the speeds at which
information propagates.  For the Euler equations, these are $u$ and $u
\pm c$, where $c$ is the sound speed.  Solution methods for the
compressible equations make use of this wave-nature to compute fluxes
at the interfaces of grid cells to update the state in time.  An
excellent introduction to these methods is provided by LeVeque's book
\cite{leveque}.  The timestep for these methods is limited by the time
it takes for maximum characteristic speed to traverse one grid cell.
For very subsonic flows, this means that the timestep is dominated by
the propagation of soundwaves, which may not be important to the
overall dynamics of the flow.


In contrast, solving low Mach number systems (including the equations of
incompressible hydrodynamics) typically involves solving one or more
advection-like equations (representing, e.g.\ conservation of mass and
momentum) coupled with a divergence constraint on the velocity field.
For example, the equations of constant-density incompressible flow
are:
\begin{eqnarray}
U_t &=& -U \cdot \nabla U + \nabla p \label{eq:incompressible_u} \\
\nabla \cdot U &=& 0
\end{eqnarray}
Here, $U$ represents the velocity vector%
%
\footnote{Here we see an unfortunate conflict
of notation between the compressible hydro community and the
incompressible community.  In papers on compressible hydrodynamics,
$U$ will usually mean the vector of conserved quantities.  In 
incompressible / low speed papers, $U$ will mean the velocity vector.}
%
and $p$ is the dynamical pressure.  The time-evolution equation for
the velocity (eq.~[\ref{eq:incompressible_u}]) can be solved using
techniques similar to those developed for compressible hydrodynamics,
updating the old velocity, $U^n$, to the new time-level, $U^\star$.
Here the `$^\star$' indicates that the updated velocity does not, in
general, satisfy the divergence constraint.  A projection method will
take this updated velocity and force it to obey the constraint
equation.  The basic idea follows from the idea that any vector
field can be expressed as the sum of a divergence-free quantity and
the gradient of a scalar.  For the velocity, we can write:
\begin{equation}
U^\star = U^d + \nabla \phi \label{eq:decomposition}
\end{equation}
where $U^d$ is the divergence free portion of the velocity vector,
$U^\star$, and $\phi$ is a scalar.  Taking the divergence of
eq~(\ref{eq:decomposition}), we have
\begin{equation}
\nabla^2 \phi = \nabla \cdot U^\star
\end{equation}
(where we used $\nabla \cdot U^d = 0$).
With appropriate boundary conditions, this Poisson equation can be
solved for $\phi$, and the final, divergence free velocity can 
be computed as
\begin{equation}
U^{n+1} = U^\star - \nabla \phi
\end{equation}

Extensions to variable-density incompressible
flows~\cite{bellMarcus:1992b} involve a slightly different
decomposition of the velocity field and, as a result, a slightly
different Poisson equation.  There is also a variety of different ways
to express what is being projected~\cite{almgren:bell:crutchfield},
and different discretizations of the divergence and gradient operators
lead to slightly different mathematical properties of the methods
(leading to ``approximate
projections''~\cite{almgrenBellSzymczak:1996}).  Finally, for
second-order methods, two projections are typically done per timestep.
The first (the `MAC' projection~\cite{bellColellaHowell:1991})
operates on the half-time, edge-centered advective velocities, making
sure that they satisfy the divergence constraint.  These advective
velocities are used to construct the fluxes through the interfaces to
advance the solution to the new time.  The second/final projection
operates on the cell-centered velocities at the new time, again
enforcing the divergence constraint.  The MAESTRO algorithm performs
both of these projections.

The MAESTRO algorithm builds upon these ideas, using a different
velocity constraint equation that captures the compressibility
due to local sources and large-scale stratification.


%-----------------------------------------------------------------------------
% Notation
%-----------------------------------------------------------------------------

\section{Notation}

Throughout the papers describing MAESTRO, we've largely kept our
notation consistent.  Table~\ref{table:sym} defines the
frequently-used quantities and provides their units.



%%%%%%%%%%%%%%%%
% symbol table
%%%%%%%%%%%%%%%%

\renewcommand{\arraystretch}{1.5}
%
\begin{center}
\begin{longtable}{|l|p{4.0in}|l|}
\caption[definition of symbols.]{definition of symbols.} \label{table:sym} \\
%
\hline \multicolumn{1}{|c|}{\textbf{symbol}} & 
       \multicolumn{1}{ c|}{\textbf{meaning}} & 
       \multicolumn{1}{ c|}{\textbf{units}} \\ \hline 
\endfirsthead

\multicolumn{3}{c}%
{{\tablename\ \thetable{}---continued}} \\
\hline \multicolumn{1}{|c|}{\textbf{symbol}} & 
       \multicolumn{1}{ c|}{\textbf{meaning}} & 
       \multicolumn{1}{ c|}{\textbf{units}} \\ \hline 
\endhead

\multicolumn{3}{|r|}{{\em continued on next page}} \\ \hline
\endfoot

\hline 
\endlastfoot

$c_p$   & specific heat at constant pressure
          ($c_p \equiv \left . \partial h / \partial T \right |_{p,X_k}$)
        & erg~g$^{-1}$~K$^{-1}$ \\
%\hline
$f$     & volume discrepancy factor ($0 \le f \le 1$) & -- \\
%\hline
$g$     & gravitational acceleration                 & cm~s$^{-2}$ \\
%\hline
$h$     & specific enthalpy                          & erg~g$^{-1}$ \\
%\hline
$\Hext$ & external heating energy generation rate    & erg~g$^{-1}$~s$^{-1}$ \\
%\hline
$\Hnuc$ & nuclear energy generation rate             & erg~g$^{-1}$~s$^{-1}$ \\
%\hline
$h_p$   & $h_p \equiv \left . \partial h / \partial p \right |_{T,X_k}$ & cm$^{3}$~g$^{-1}$ \\
%\hline
$\kth$  & thermal conductivity                       & erg~cm$^{-1}$~s$^{-1}$~K$^{-1}$ \\
%\hline
$p_0$   & base state pressure                        & erg~cm$^{-3}$ \\
%\hline
$p_T$   & $p_T \equiv \left . \partial p / \partial T \right |_{\rho,X_k}$ & erg~cm$^{-3}$~K$^{-1}$ \\
%\hline
$p_{X_k}$ & $p_{X_k} \equiv \left . \partial p / \partial X_k \right |_{p,T,X_{j,j\ne k}}$ & erg~cm$^{-3}$ \\
%\hline
$p_\rho$ & $p_\rho \equiv \left . \partial p / \partial \rho \right |_{T,X_k}$ & erg~g$^{-1}$ \\
%\hline
$q_k$   & specific nuclear binding energy            & erg~g$^{-1}$  \\
%\hline
$r$     & radial coordinate (direction of gravity)   & cm \\
%\hline
$s$     & specific entropy                           & erg~g$^{-1}$~K$^{-1}$ \\
%\hline
$S$     & source term to the divergence constraint   & s$^{-1}$ \\
%\hline
$t$     & time                                       & s \\
%\hline
$T$     & temperature                                & K \\
%\hline
$\Ub$     & total velocity ($\Ub = \Ubt + w_0 \eb_r$) & cm~s$^{-1}$ \\
%\hline
$\Ubt$   & local velocity                             & cm~s$^{-1}$ \\
%\hline
$\uadv$ & advective velocity (edge-centered)         & cm~s$^{-1}$ \\
%\hline
$w_0$   & base state expansion velocity              & cm~s$^{-1}$ \\
%\hline
$X_k$   & mass fraction of the species ($\sum_k X_k = 1$) & -- \\
%\hline
$\beta_0$ & coefficient to velocity
            in velocity constraint equation  & g~cm$^{-3}$ \\
%\hline
$\Gamma_1$ & first adiabatic exponent ($\Gamma_1 \equiv \left . d \log p/d \log \rho \right |_s$) & -- \\
%\hline
$\etarho$ & $\etarho \equiv \overline{(\rho' \Ub \cdot \eb_r)}$ & g~cm$^{-2}$~s$^{-1}$ \\
%\hline
$\xi_k$ & $\xi_k \equiv \left . \partial h / \partial X_k \right |_{p,T,X_{j,j\ne k}}$ & erg~g$^{-1}$ \\
%\hline 
$\pi$   & dynamic pressure & erg~cm$^{-3}$ \\
%\hline
$\pizero$ & base state dynamic pressure & erg~cm$^{-3}$ \\
%\hline
$\rho$  & mass density  & g~cm$^{-3}$ \\
%\hline
$\rho_0$  & base state mass density  & g~cm$^{-3}$ \\
%\hline
$\rho'$  & perturbational density ($\rho' = \rho - \rho_0$) & g~cm$^{-3}$ \\
%\hline
$(\rho h)_0$ & base state enthalpy density & erg~cm$^{-3}$  \\
%\hline
$(\rho h)'$ & perturbational enthalpy density 
              $ \left [(\rho h)' = \rho h - (\rho h)_0 \right ]$ & erg~cm$^{-3}$  \\
%\hline
$\sigma$ & $\sigma \equiv p_T/(\rho c_p p_\rho)$ & erg$^{-1}$~g \\
%\hline
$\psi$  & $\psi \equiv D_0 p_0/Dt = \partial p_0/\partial t + w_0\partial p_0/\partial r$ & erg~cm$^{-3}$~s$^{-1}$ \\
%\hline
$\omegadot_k$ & creation rate for species $k$ ($\omegadot_k \equiv DX_k/Dt$) & s$^{-1}$ \\
\end{longtable}
\end{center}

\renewcommand{\arraystretch}{1.0}

