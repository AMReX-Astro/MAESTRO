\section{History of MAESTRO}

MAESTRO describes the evolution of low Mach number flows that are in
hydrostatic equilibrium.  The idea for MAESTRO grew out of our
small-scale astrophysical combustion algorithm detailed in
\begin{itemize}
\item {\em Adaptive Low Mach Number Simulations of Nuclear Flames,}
J. B. Bell, M. S. Day, C. A. Rendleman, S. E. Woosley, \& M. Zingale
2004, JCP, 195, 2, 677 (henceforth BDRWZ)
\end{itemize}

\noindent MAESTRO was developed initially for modeling the convecting
phase in a white dwarf preceding the ignition of a Type Ia supernovae.
The low Mach number equations were developed in a series of papers
leading up to the first application to this problem:
\begin{itemize}
\item {\em Low Mach Number Modeling of Type Ia
  Supernovae. I. Hydrodynamics,} A. S. Almgren, J. B. Bell, 
  C. A. Rendleman, \& M. Zingale 2006, ApJ, 637, 922 (henceforth
  paper I)
\item {\em Low Mach Number Modeling of Type Ia Supernovae. II. Energy
  Evolution,} A. S. Almgren, J. B. Bell, C. A. Rendleman, \& M. Zingale
  2006, ApJ, 649, 927 (henceforth paper II)
\item {\em Low Mach Number Modeling of Type Ia Supernovae. III. Reactions,}
  A. S. Almgren, J. B. Bell, A. Nonaka, \& M. Zingale
  2008, ApJ, 684, 449 (henceforth paper III)
\item {\em Low Mach Number Modeling of Type Ia Supernovae. IV. White Dwarf Convection,}
  M. Zingale, A. S. Almgren, J. B. Bell, A. Nonaka, \& S. E. Woosley
  2009, ApJ, 704, 196 (henceforth paper IV)
\end{itemize}

\noindent The current version of the algorithm is described in our multilevel paper:
\begin{itemize}
\item {\em MAESTRO: An Adaptive Low Mach Number Hydrodynamics Algorithm for Stellar
  Flows,} A. Nonaka, A. S. Almgren, J. B. Bell, M. J. Lijewski, C. M. Malone,
  \& M. Zingale 2010, submitted to ApJ
  (henceforth ``the multilevel paper'')
\end{itemize}

\noindent We have two papers in preparation:
\begin{itemize}
\item {\em The Convective Phase Preceding Type Ia Supernovae,}
  M. Zingale, A. S. Almgren, J. B. Bell, C. M. Malone, A. Nonaka, \& S. E. Woosley 2010,
  in preparation (henceforth paper V)
\item {\em Multidimensional Modeling of Type I X-ray Bursts,}
  C. M. Malone, A. Nonaka, A. S. Almgren, J. B. Bell, \& M. Zingale 2010,
  in preparation (henceforth ``the XRB paper'')
\end{itemize}


%-----------------------------------------------------------------------------
% Notation
%-----------------------------------------------------------------------------

\section{Notation}

Throughout the papers describing MAESTRO, we've largely kept our
notation consistent.  Table~\ref{table:sym} defines the
frequently-used quantities and provides their units.



%%%%%%%%%%%%%%%%
% symbol table
%%%%%%%%%%%%%%%%

\renewcommand{\arraystretch}{1.5}
%
\begin{center}
\begin{longtable}{|l|p{4.0in}|l|}
\caption[definition of symbols.]{definition of symbols.} \label{table:sym} \\
%
\hline \multicolumn{1}{|c|}{\textbf{symbol}} & 
       \multicolumn{1}{ c|}{\textbf{meaning}} & 
       \multicolumn{1}{ c|}{\textbf{units}} \\ \hline 
\endfirsthead

\multicolumn{3}{c}%
{{\tablename\ \thetable{}---continued}} \\
\hline \multicolumn{1}{|c|}{\textbf{symbol}} & 
       \multicolumn{1}{ c|}{\textbf{meaning}} & 
       \multicolumn{1}{ c|}{\textbf{units}} \\ \hline 
\endhead

\multicolumn{3}{|r|}{{\em continued on next page}} \\ \hline
\endfoot

\hline 
\endlastfoot

$c_p$   & specific heat at constant pressure
          ($c_p \equiv \left . \partial h / \partial T \right |_{p,X_k}$)
        & erg~g$^{-1}$~K$^{-1}$ \\
\hline
$f$     & volume discrepancy factor ($0 \le f \le 1$) & -- \\
\hline
$g$     & gravitational acceleration                 & cm~s$^{-2}$ \\
\hline
$h$     & specific enthalpy                          & erg~g$^{-1}$ \\
\hline
$\Hext$ & external heating energy generation rate    & erg~g$^{-1}$~s$^{-1}$ \\
\hline
$\Hnuc$ & nuclear energy generation rate             & erg~g$^{-1}$~s$^{-1}$ \\
\hline
$h_p$   & $h_p \equiv \left . \partial h / \partial p \right |_{T,X_k}$ & cm$^{3}$~g$^{-1}$ \\
\hline
$\kth$  & thermal conductivity                       & erg~cm$^{-1}$~s$^{-1}$~K$^{-1}$ \\
\hline
$p_0$   & base state pressure                        & erg~cm$^{-3}$ \\
\hline
$p_T$   & $p_T \equiv \left . \partial p / \partial T \right |_{\rho,X_k}$ & erg~cm$^{-3}$~K$^{-1}$ \\
\hline
$p_{X_k}$ & $p_{X_k} \equiv \left . \partial p / \partial X_k \right |_{p,T,X_{j,j\ne k}}$ & erg~cm$^{-3}$ \\
\hline
$p_\rho$ & $p_\rho \equiv \left . \partial p / \partial \rho \right |_{T,X_k}$ & erg~g$^{-1}$ \\
\hline
$q_k$   & specific nuclear binding energy            & erg~g$^{-1}$  \\
\hline
$r$     & radial coordinate (direction of gravity)   & cm \\
\hline
$s$     & specific entropy                           & erg~g$^{-1}$~K$^{-1}$ \\
\hline
$S$     & source term to the divergence constraint   & s$^{-1}$ \\
\hline
$t$     & time                                       & s \\
\hline
$T$     & temperature                                & K \\
\hline
$\Ub$     & total velocity ($\Ub = \Ubt + w_0 \eb_r$) & cm~s$^{-1}$ \\
\hline
$\Ubt$   & local velocity                             & cm~s$^{-1}$ \\
\hline
$\uadv$ & advective velocity (edge-centered)         & cm~s$^{-1}$ \\
\hline
$w_0$   & base state expansion velocity              & cm~s$^{-1}$ \\
\hline
$X_k$   & mass fraction of the species ($\sum_k X_k = 1$) & -- \\
\hline
$\beta_0$ & coefficient to velocity
            in velocity constraint equation  & g~cm$^{-3}$ \\
\hline
$\Gamma_1$ & first adiabatic exponent ($\Gamma_1 \equiv \left . d \log p/d \log \rho \right |_s$) & -- \\
\hline
$\etarho$ & $\etarho \equiv \overline{(\rho' \Ub \cdot \eb_r)}$ & g~cm$^{-2}$~s$^{-1}$ \\
\hline
$\xi_k$ & $\xi_k \equiv \left . \partial h / \partial X_k \right |_{p,T,X_{j,j\ne k}}$ & erg~g$^{-1}$ \\
\hline 
$\pi$   & dynamic pressure & erg~cm$^{-3}$ \\
\hline
$\pizero$ & base state dynamic pressure & erg~cm$^{-3}$ \\
\hline
$\rho$  & mass density  & g~cm$^{-3}$ \\
\hline
$\rho_0$  & base state mass density  & g~cm$^{-3}$ \\
\hline
$\rho'$  & perturbational density ($\rho' = \rho - \rho_0$) & g~cm$^{-3}$ \\
\hline
$(\rho h)_0$ & base state enthalpy density & erg~cm$^{-3}$  \\
\hline
$(\rho h)'$ & perturbational enthalpy density 
              $ \left [(\rho h)' = \rho h - (\rho h)_0 \right ]$ & erg~cm$^{-3}$  \\
\hline
$\sigma$ & $\sigma \equiv p_T/(\rho c_p p_\rho)$ & erg$^{-1}$~g \\
\hline
$\psi$  & $\psi \equiv D_0 p_0/Dt = \partial p_0/\partial t + w_0\partial p_0/\partial r$ & erg~cm$^{-3}$~s$^{-1}$ \\
\hline
$\omegadot_k$ & creation rate for species $k$ ($\omegadot_k \equiv DX_k/Dt$) & s$^{-1}$ \\
\end{longtable}
\end{center}

\renewcommand{\arraystretch}{1.0}
There are some great charts in the appendix of the multilevel paper demonstrating what 
the superscript notation represents.

