\section{EOS Calls}
\subsection{{\tt advance\_timestep}}
\begin{description}
\item[Step 1.] {\em Define the average expansion at time $t^\nph$ and the new $w_0.$}\\

\item[Step 2.] {\em Construct the provisional edge-based advective velocity, $\uadvone.$}\\

\item[Step 3.] {\em React the full state and base state through the first time interval 
of $\dt/2.$}\\

\item[Step 4.] {\em Advect the base state, then the full state, through a time interval 
of $\dt.$}\\

\item[Step 5.] {\em React the full state through a second time interval of $\dt/2.$}\\

\item[Step 6.] {\em Define a new average expansion rate at time $t^\nph.$}\\

\item[Step 7.] {\em Construct the final edge-based advective velocity, $\uadvtwo.$}\\

\item[Step 8.] {\em Advect the base state, then the full state, through a time interval 
of $\dt.$}\\

\item[Step 9.] {\em React the full state and base state through a second time interval 
of $\dt/2.$}\\

\item[Step 10.] {\em Compute $S^{n+1}$ for the final projection.}\\

\item[Step 11.] {\em Update the velocity.}\\

\item[Step 12.] {\em Compute a new $\dt.$}\\

\end{description}
\subsection{Initialization}
\subsection{{\tt make\_plotfile}}
\section{Temperature Usage}
\subsection{{\tt advance\_timestep}}
\begin{description}
\item[Step 1.] {\em Define the average expansion at time $t^\nph$ and the new $w_0.$}
\item[Step 2.] {\em Construct the provisional edge-based advective velocity, $\uadvone.$}
\item[Step 3.] {\em React the full state and base state through the first time interval 
of $\dt/2.$}
\begin{itemize}
\item{\tt react\_state} uses $T^n$ in burner and then computes $T^1$.
\end{itemize}
\item[Step 4.] {\em Advect the base state, then the full state, through a time interval 
of $\dt.$}\\
If {\tt use\_thermal\_diffusion} = T:
\begin{itemize}
\item {\tt make\_explicit\_thermal} computes coefficients for 
$\nabla\cdot\kth\nabla h + \cdots$ using $(\rho,T,X)^1$.
\item {\tt thermal\_conduct\_full\_alg} computes coefficients with $(\rho,T,X)^1$, then 
computes $T^{2,*}$ from $(\rho,h,X)^{2,*}$ or $(\rho,p,X)^{2,*}$ if {\tt use\_tfromp} = T.
\end{itemize}
If {\tt use\_thermal\_diffusion} = F:
\begin{itemize}
\item {\tt enthalpy\_advance} computes $T^{2,*}$ from $(\rho,h,X)^{2,*}$ or $(\rho,p,X)^{2,*}$
if {\tt use\_tfromp} = T.
\end{itemize}
\item[Step 5.] {\em React the full state through a second time interval of $\dt/2.$}
\begin{itemize}
\item {\tt react\_state} uses $T^{2,*}$ in burner and then computes $T^{n+1,*}$.
\end{itemize}
\item[Step 6.] {\em Define a new average expansion rate at time $t^\nph.$}
\begin{itemize}
\item {make\_explicit\_thermal} computes coefficients for 
$\nabla\cdot\kth\nabla h + \cdots$ using $(\rho,T,X)^{n+1,*}$.
\item {make\_S} thermodynamic variables are functions of $(\rho,T,X)^{n+1,*}$.
\end{itemize}
\item[Step 7.] {\em Construct the final edge-based advective velocity, $\uadvtwo.$}
\item[Step 8.] {\em Advect the base state, then the full state, through a time interval 
of $\dt.$}\\
If {\tt use\_thermal\_diffusion} = T:
\begin{itemize}
\item {\tt make\_explicit\_thermal} computes coefficients for 
$\nabla\cdot\kth\nabla h + \cdots$ using $(\rho,T,X)^1$.
\item {\tt thermal\_conduct\_full\_alg} computes coefficients with $(\rho,T,X)^1$ and
$(\rho,T,X)^{2,*}$, then computes $T^{2}$ from $(\rho,h,X)^{2}$ or $(\rho,p,X)^{2}$ if 
{\tt use\_tfromp} = T.
\end{itemize}
If {\tt use\_thermal\_diffusion} = F:
\begin{itemize}
\item {\tt enthalpy\_advance} computes $T^2$ from $(\rho,h,X)^2$ or $(\rho,p,X)^2$
if {\tt use\_tfromp} = T.
\end{itemize}
\item[Step 9.] {\em React the full state and base state through a second time interval 
of $\dt/2.$}
\begin{itemize}
\item {\tt react\_state} uses $T^2$ in burner and computes $T^{n+1}$.
\end{itemize}
\item[Step 10.] {\em Compute $S^{n+1}$ for the final projection.}
\begin{itemize}
\item {\tt make\_explicit\_thermal} computes coefficients for 
$\nabla\cdot\kth\nabla h + \cdots$ using $(\rho,T,X)^{n+1}$.
\item {\tt make\_S} thermodynamic variables are functions of $(\rho,T,X)^{n+1}$.
\end{itemize}
\item[Step 11.] {\em Update the velocity.}
\item[Step 12.] {\em Compute a new $\dt.$}
\end{description}
