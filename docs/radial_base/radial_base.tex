Since the one-dimensional spherical base state is not aligned with the
3-d Cartesian grid we use for the full star, computing quantities that
involve both the base state and the full state becomes complicated.
Therefore, when we go between the base state and the full state in
various parts of the algorithm (such as the averaging operations), we
need to map between the two geometries.

As there is no alignment between the 3-d Cartesian grid and the 1-d
spherical base state grid, we are free to choose their grid spacings
independently.  Experimentation has shown that the best choice results
from picking the 3-d Cartesian grid spacing, $\Delta x$ to be coarser
than the 1-d spherical base state spacing, $\Delta r$.
A typical simulation may use $5 \Delta r = \Delta x$.
We call the procedure to go from 1-d to 3-d {\tt fill\_3d} and the
complementary procedure that takes us from 3-d to 1-d {\tt average}.
\section{\tt fill\_3d}
There are three different {\tt fill\_3d} modes:
\begin{enumerate}
\item The 1D array is bin-centered, and we map to Cartesian cell-centers.
\item The 1D array is edge-centered, and we map to Cartesian cell-centers.
\item The 1D array is edge-centered, and we map to Cartesian edges.  
\end{enumerate}
\subsection{1D Cell-Centered to Cartesian Cell-Centered}
This mapping is done in the subroutine {\tt put\_1d\_array\_on\_cart}.
Figure \ref{fig:mapping} shows the Cartesian grid with the spherical
base state overlayed (for simplicity, the figure is drawn in 2-d).  
We compute the radius of the Cartesian cell center from the center of the star,
\begin{equation}
r = \sqrt{(x_i - x_c)^2 + (y_j - y_c)^2 + (z_k - z_c)^2}
\end{equation}
where $(x_c, y_c, z_c)$ are the coordinates of the center of the star.
We use this radius to find the corresponding radial bin as $n = \mathtt{int}(r
/ \Delta r)$ (here, our convention is to use 0-based indexing for the
base state).  We can then initialize a Cartesian cell quantity $q$ from its
corresponding base state quantity, $q_0$ as $q_{i,j,k} = q_{0,n}$.  In essence,
we are doing a piecewise constant interpolation of the radial array onto the 
Cartesian grid.  If the output is a vector quantity, rather than a scalar, we
multiply the result by the unit vector in the normal direction.
\subsection{1D Edge-Centered to Cartesian Cell-Centered}
This mapping is done in the subroutine {\tt put\_1d\_array\_on\_cart}.
Again, for each Cartesian cell-center, we compute the corresponding radial
bin.  We linearly interpolate from the two nearest radial edge values 
to get the Cartesian cell-centered value.  In essence,
we are doing a piecewise linear interpolation of the radial array onto the 
Cartesian grid.  If the output is a vector quantity, rather than a scalar, we
multiply the result by the unit vector in the normal direction.
\subsection{1D Edge-Centered to Cartesian Edges}
This mapping is done in the subroutine {\tt put\_1d\_array\_on\_cart}.
This case is only used to create {\tt w0mac} from {\tt w0} in the 
subroutine, {\tt put\_w0\_on\_edges}.  There are several variations 
of the interpolation, controlled by {\tt w0mac\_interp\_type}.  We 
currently use the default, {\tt w0mac\_interp\_type} = 1.  We first 
map $w_0$ onto Cartesian cell-centers, and multiply the result by 
the unit vector in the normal direction.  Then, for each Cartesian 
face, we obtain the normal velocity by averaging the corresponding 
velocities from the Cartesian cell-centers adjacent to the face.
\section{Avg}\label{Sec:Avg}
For the {\bf Avg} process, we first create a coarse 1D radial array 
with $\Delta x = \Delta r$.  Then, we compute the radius of the Cartesian cell 
center, and compute the index, $n$, of the corresponding coarse radial 
bin.  $q_{0,n}$ is then the average of all the $q_{i,j,k}$ whose 
Cartesian cell center maps into coarse radial bin $n$.  Then, we 
interpolate the coarse radial array onto the finer radial array.

\clearpage

\begin{figure}[tpb]
\begin{center}
\epsffile{\radbasefigpath/mapping.eps}
\caption{\label{fig:mapping} The Cartesian grid and spherical base
state (shown here in 2-d for simplicity).  Here we represent the
spherical base state as concentric shells (blue lines).  Since the
base state is not aligned with the Cartesian grid we use on which
we discretize the star, we need to map between the two configurations.
The `$+$' symbols represent the Cartesian zone centers.  In our
mapping, the zone marked with the red symbol maps into the gray-shaded
radial bin. }
\label{fig:mapping}
\end{center}
\end{figure}
