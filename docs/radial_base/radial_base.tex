Since the one-dimensional spherical base state is not aligned with the
3-d Cartesian grid we use for the full star, computing quantities that
involve both the base state and the full state becomes complicated.
Therefore, when we go between the base state and the full state in
various parts of the algorithm (such as the averaging operations), we
need to map between the two geometries.

As there is no alignment between the 3-d Cartesian grid and the 1-d
spherical base state grid, we are free to choose their grid spacings
independently.  Experimentation has shown that the best choice results
from picking the 3-d Cartesian grid spacing, $\Delta x$ to be coarser
than the 1-d spherical base state spacing, $\Delta r$.  A typical simulation
may use $5 \Delta r = \Delta x$.
We call the procedure to go from 1-d to 3-d {\tt fill\_3d} and the
complementary procedure that takes us from 3-d to 1-d {\tt average}.
\section{\tt fill\_3d}
Figure XXX shows the Cartesian grid with the spherical
base state overlayed (for simplicity, the figure is drawn in 2-d).  
The {\tt fill\_3d} procedure computes the radius of the cell center
from the center of the star,
\begin{equation}
r = \sqrt{(x_i - x_c)^2 + (y_j - y_c)^2 + (z_k - z_c)^2}
\end{equation}
where $(x_c, y_c, z_c)$ are the coordinates of the center of the star.
We use this radius to find the corresponding radial bin as $n = \mathtt{int}(r
/ \Delta r)$ (here, our convention is to use 0-based indexing for the
base state).  We can then initialize a Cartesian cell quantity $q$ from its
corresponding base state quantity, $q_0$ as $q_{i,j,k} = q_{0,n}$.
\subsection{\tt put\_1d\_array\_on\_cart}
\subsection{\tt put\_w0\_on\_edges}
\section{\tt average}
For the {\tt average} process, we again compute the radius of the
Cartesian cell center, and compute the index, $n$, of the
corresponding base state bin as above.  $q_{0,n}$ is then the average
of all the $q_{i,j,k}$ whose Cartesian cell center maps into radial
bin $n$.  However, if $\Delta r < \Delta x$, there may be some base
state bins that are not filled.  In this case, we will fill them via
interpolation. \MarginPar{more detail will be needed here.}
