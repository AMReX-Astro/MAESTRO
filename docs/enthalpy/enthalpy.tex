\section{Enthalpy Evolution}

In the low Mach number formulation, the continuity and momentum
equations together with the constraint on the velocity field
completely determine the solution.
\begin{eqnarray}
\frac{\partial(\rho X_k)}{\partial t} &=& -\nabla\cdot(\rho X_k\Ub) + 
\rho\omegadot_k,\label{eq:species}\\
\frac{\partial\Ub}{\partial t} &=& -\Ub\cdot\nabla\Ub  - 
  \frac{1}{\rho}\nabla\pi - 
  \frac{\rho-\rho_0}{\rho} g\eb_r,\label{eq:momentum}
\end{eqnarray}

The addition of an energy equation over-specifies the problem, but its
solution can help us in providing a diagnostic capability for the
solution.  Additionally, work from combustion suggests that low Mach
number methods perform better when considering reactions if we use
an enthalpy equation to determine the temperature.

The velocity constraint equation was constructed by diffrentiating 
the pressure along particle paths, using the equation of state.  


The full enthalpy equation, with no approximations, appears as:
\begin{equation}
\frac{\partial(\rho h)}{\partial t} = -\nabla\cdot(\rho h\Ub) + 
  \frac{Dp}{Dt} + \rho\Hnuc \label{eq:enthalpy}
\end{equation}
Here, $h = e + p/\rho$ is the specific enthalpy, with $e$ the specific
internal energy.  In the low Mach number formulation, we replace $p$
with $p_0$ in the $Dp/Dt$ term, however, the definition of enthalpy
implicitly contains a pressure.  When calling the equation of state,
we take $h$ and $\rho$ as inputs.  The equation of state is expressed
in terms of $T$ and $\rho$, so it iterates until it finds the $h$ that
we desire.  This $h$ will be of the form $h = e + p_\mathrm{EOS}/rho$,
where $p_\mathrm{EOS}$ is the pressure returned from the EOS.  Note that
$p_\mathrm{EOS}$ may not be equal to $p_0$---this causes us to drive
off of the constraint. \MarginPar{not completely sure what to say here}

The mismatch between the pressure implicit in the definition of $h$
and $p_0$ can be seen by substituting $h = e + p/\rho$ into the
enthalpy equation, where we replace $p$ with $p_0$ in the $Dp/Dt$ term:
\begin{eqnarray}
\frac{\partial(\rho h)}{\partial t} &=& -\nabla\cdot(\rho h\Ub) + 
  \frac{Dp_0}{Dt} + \rho\Hnuc \nonumber \\
%
\frac{\partial(\rho e)}{\partial t} + \frac{\partial p}{\partial t} &=&
 -\nabla\cdot(\rho e\Ub) -\nabla\cdot(p\Ub) + \frac{Dp_0}{Dt} + \rho\Hnuc \nonumber \\
%
\frac{\partial(\rho e)}{\partial t} &=&
 -\nabla\cdot(\rho e\Ub) - p\nabla\cdot\Ub + \rho\Hnuc + 
  \left \{ \frac{Dp_0}{Dt} - \frac{Dp}{Dt} \right \} \nonumber 
\end{eqnarray}
We note that the last equation appears to be an internal energy
evolution equation, but with an additional forcing term: $D(p_0 - p)/Dt$.

\section{Outstanding Questions}

\begin{itemize}
\item Why do we want to start with enthalpy instead of internal energy?


\item We always say that to substituting $p$ by $p_0$ in the enthalpy equation
   is what filters the acoustics from the system---did we ever prove this?


\item In the EOS, we take $h$ and construct $e$ by subtracting off
  $p$---depending on the inputs to the EOS, we treat the '$p$' part of $h$
  differently.

   if we come into the EOS with $\rho$, $T$, then we get $p$ and $e$ from
   the EOS and compute $h$.

   if we come into the EOS with $\rho$, $h$, then we find a $T$ such that
   $e + p(\rho,T)/\rho = h$, where $p(\rho,T)$ is the pressure from the EOS.
   It is not necessarily equal to $p_0$.

   if we come into the EOS with $\rho$, $p_0$, then find a $T$ that matches
   $p_0$, and then $h = e + p_0/\rho$.  In this case, there is no mismatch
   in the enthalpy equation.


\item do we want to always insist that the '$p$' hidden in the definition of
   $h$ is $p_0$?

\end{itemize}






