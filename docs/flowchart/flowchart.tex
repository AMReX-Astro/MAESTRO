Here we outline the algorithm currently implemented in the code.  Our
starting point for this description are the series of papers describing
the development of the algorithm:
\begin{itemize}
\item {\em Low Mach Number Modeling of Type Ia
  Supernovae. I. Hydrodynamics,} A. S. Almgren, J. B. Bell, 
  C. A. Rendleman, \& M. Zingale 2006, ApJ, 637, 922 (henceforth
  paper I)
\item {\em Low Mach Number Modeling of Type Ia Supernovae. II. Energy
  Evolution,} A. S. Almgren, J. B. Bell, C. A. Rendleman, \& M. Zingale
  2006, ApJ, 649, 927 (henceforth paper II)
\item {\em Low Mach Number Modeling of Type Ia Supernovae. III. Reactions,}
  A. S. Almgren, J. B. Bell, A. Nonaka, \& M. Zingale
  2008, ApJ, 684, 449 (henceforth paper III)
\item {\em Low Mach Number Modeling of Type Ia Supernovae. IV. White Dwarf Convection,}
  M. Zingale, A. S. Almgren, J. B. Bell, A. Nonaka, \& S. E. Woosley
  2009, ApJ, 704, 196 (henceforth paper IV)
\item {\em MAESTRO: An Adaptive Low Mach Number Hydrodynamics Algorithm for Stellar
  Flows,} A. Nonaka, A. S. Almgren, J. B. Bell, M. J. Lijewski, C. M. Malone,
  \& M. Zingale 2009, submitted to SIAM J. Sci. Comput. 
  (henceforth ``the multilevel paper'')
\end{itemize}
We also carry over some ideas from our small-scale low Mach number algorithm
for astrophysical flows:
\begin{itemize}
\item {\em Adaptive Low Mach Number Simulations of Nuclear Flames,}
J. B. Bell, M. S. Day, C. A. Rendleman, S. E. Woosley, \& M. Zingale
2004, JCP, 195, 2, 677 (henceforth BDRWZ)
\end{itemize}

\section{Changes Between Paper 3 and Paper 4}
\begin{enumerate}
\item We defined the mapping of data between a 1D radial array and the 3D Cartesian
grid for spherical problems (which we improve upon in the multilevel paper).
\item We update $T$ after the call to {\bf React State}.
\item We have created a {\tt burning\_cutoff\_density}, where the burning does
not happen below this density.  It is presently set to {\tt base\_cutoff\_density}.
\item Use corner coupling in advection.
\item We use {\tt use\_tfromp} to update temperature using $T=T(\rho,X_k,p_0)$ rather 
than $T=T(\rho,h,X_k)$.
\item For spherical problems, we have changed the discretization of 
$\Ubt\cdot\nabla p_0$ in the enthalpy update to 
$\nabla\cdot(\Ubt p_0) - p_0\nabla\cdot\Ubt$.
\item In paper III we discretized the enthalpy evolution equation in
terms of $T$.  Since then we have discovered that 
discretizing the enthalpy evolution in perturbational form, $(\rho h)'$,
leads to better numerical properties.  We use {\tt enthalpy\_pred\_type = 1}.
This is more like paper II.
\item We have turned off the evolution of $h$ above the atmosphere and instead
compute $h$ with the EOS using {\tt do\_eos\_h\_above\_cutoff = TRUE}.
\end{enumerate}
\section{Changes Between Paper 4 and the Multilevel Paper}
See the multilevel paper for the latest.
\section{Changes Between Papers 4.5 and Paper 5}
\begin{enumerate}
\item Added rotation.
\end{enumerate}
\section{Changes Between Paper 5 and XRB Paper}
\begin{enumerate}
\item We have added thermal diffusion, controlled by {\tt use\_thermal\_diffusion},
{\tt temp\_diffusion\_formulation}, and {\tt thermal\_diffusion\_type}.
\item We added the volume discrepancy term to the velocity constraint equation,
controlled by the input parameter, {\tt dpdt\_factor}.
\end{enumerate}

%-----------------------------------------------------------------------------
% Future Considerations
%-----------------------------------------------------------------------------

\section{Future Considerations}

\begin{itemize}

\item Should we use a predictor-corrector for updating the full-state density?
Specifically, after calling {\bf Correct Base}, should we do a full-state density 
advance and {\bf Correct Base} using the more accurate estimate of $\rho_0^{n+1}$?

\end{itemize}

%-----------------------------------------------------------------------------
% Notation
%-----------------------------------------------------------------------------

\section{Notation}

%%%%%%%%%%%%%%%%
% symbol table
%%%%%%%%%%%%%%%%

\renewcommand{\arraystretch}{1.5}
%
\begin{center}
\begin{longtable}{|l|p{4.0in}|l|}
\caption[definition of symbols.]{definition of symbols.} \label{table:sym} \\
%
\hline \multicolumn{1}{|c|}{\textbf{symbol}} & 
       \multicolumn{1}{ c|}{\textbf{meaning}} & 
       \multicolumn{1}{ c|}{\textbf{units}} \\ \hline 
\endfirsthead

\multicolumn{3}{c}%
{{\tablename\ \thetable{}---continued}} \\
\hline \multicolumn{1}{|c|}{\textbf{symbol}} & 
       \multicolumn{1}{ c|}{\textbf{meaning}} & 
       \multicolumn{1}{ c|}{\textbf{units}} \\ \hline 
\endhead

\multicolumn{3}{|r|}{{\em continued on next page}} \\ \hline
\endfoot

\hline 
\endlastfoot

$c_p$   & specific heat at constant pressure
          ($c_p \equiv \left . \partial h / \partial T \right |_{p,X_k}$)
        & erg~g$^{-1}$~K$^{-1}$ \\
\hline
$f$     & volume discrepancy factor ($0 \le f \le 1$) & -- \\
\hline
$g$     & gravitational acceleration                 & cm~s$^{-2}$ \\
\hline
$h$     & specific enthalpy                          & erg~g$^{-1}$ \\
\hline
$\Hext$ & external heating energy generation rate    & erg~g$^{-1}$~s$^{-1}$ \\
\hline
$\Hnuc$ & nuclear energy generation rate             & erg~g$^{-1}$~s$^{-1}$ \\
\hline
$h_p$   & $h_p \equiv \left . \partial h / \partial p \right |_{T,X_k}$ & cm$^{3}$~g$^{-1}$ \\
\hline
$\kth$  & thermal conductivity                       & erg~cm$^{-1}$~s$^{-1}$~K$^{-1}$ \\
\hline
$p_0$   & base state pressure                        & erg~cm$^{-3}$ \\
\hline
$p_T$   & $p_T \equiv \left . \partial p / \partial T \right |_{\rho,X_k}$ & erg~cm$^{-3}$~K$^{-1}$ \\
\hline
$p_{X_k}$ & $p_{X_k} \equiv \left . \partial p / \partial X_k \right |_{p,T,X_{j,j\ne k}}$ & erg~cm$^{-3}$ \\
\hline
$p_\rho$ & $p_\rho \equiv \left . \partial p / \partial \rho \right |_{T,X_k}$ & erg~g$^{-1}$ \\
\hline
$q_k$   & specific nuclear binding energy            & erg~g$^{-1}$  \\
\hline
$r$     & radial coordinate (direction of gravity)   & cm \\
\hline
$s$     & specific entropy                           & erg~g$^{-1}$~K$^{-1}$ \\
\hline
$S$     & source term to the divergence constraint   & s$^{-1}$ \\
\hline
$t$     & time                                       & s \\
\hline
$T$     & temperature                                & K \\
\hline
$\Ub$     & total velocity ($\Ub = \Ubt + w_0 \eb_r$) & cm~s$^{-1}$ \\
\hline
$\Ubt$   & local velocity                             & cm~s$^{-1}$ \\
\hline
$\uadv$ & advective velocity (edge-centered)         & cm~s$^{-1}$ \\
\hline
$w_0$   & base state expansion velocity              & cm~s$^{-1}$ \\
\hline
$X_k$   & mass fraction of the species ($\sum_k X_k = 1$) & -- \\
\hline
$\beta_0$ & coefficient to velocity
            in velocity constraint equation  & g~cm$^{-3}$ \\
\hline
$\Gamma_1$ & first adiabatic exponent ($\Gamma_1 \equiv \left . d \log p/d \log \rho \right |_s$) & -- \\
\hline
$\etarho$ & $\etarho \equiv \overline{(\rho' \Ub \cdot \eb_r)}$ & g~cm$^{-2}$~s$^{-1}$ \\
\hline
$\xi_k$ & $\xi_k \equiv \left . \partial h / \partial X_k \right |_{p,T,X_{j,j\ne k}}$ & erg~g$^{-1}$ \\
\hline 
$\pi$   & dynamic pressure & erg~cm$^{-3}$ \\
\hline
$\pizero$ & base state dynamic pressure & erg~cm$^{-3}$ \\
\hline
$\rho$  & mass density  & g~cm$^{-3}$ \\
\hline
$\rho_0$  & base state mass density  & g~cm$^{-3}$ \\
\hline
$\rho'$  & perturbational density ($\rho' = \rho - \rho_0$) & g~cm$^{-3}$ \\
\hline
$(\rho h)_0$ & base state enthalpy density & erg~cm$^{-3}$  \\
\hline
$(\rho h)'$ & perturbational enthalpy density 
              $ \left [(\rho h)' = \rho h - (\rho h)_0 \right ]$ & erg~cm$^{-3}$  \\
\hline
$\sigma$ & $\sigma \equiv p_T/(\rho c_p p_\rho)$ & erg$^{-1}$~g \\
\hline
$\psi$  & $\psi \equiv D_0 p_0/Dt = \partial p_0/\partial t + w_0\partial p_0/\partial r$ & erg~cm$^{-3}$~s$^{-1}$ \\
\hline
$\omegadot_k$ & creation rate for species $k$ ($\omegadot_k \equiv DX_k/Dt$) & s$^{-1}$ \\
\end{longtable}
\end{center}

\renewcommand{\arraystretch}{1.0}
There are some great charts in the appendix of the multilevel paper demonstrating what 
the superscript notation represents.
\section{Shorthand Functions}
See the multilevel paper for the latest.

%-----------------------------------------------------------------------------
% Time Advancement Algorithm
%-----------------------------------------------------------------------------

\section{Time Advancement Algorithm}\label{Sec:Time Advancement Algorithm}

We now describe the full time advancement algorithm, making frequent
use of the shorthand developed above.  Here, we assume that the
problem is already properly initialized.  We describe the details of
the initialization in \S \ref{Sec:Initialization}.
The advance of the state through a single timestep appears as:

%--------------------------------------------------------------------------
% STEP 1
%--------------------------------------------------------------------------

\begin{description}

\item[Step 1.] {\em Define the provisional time-centered expansion, $S^{n+\myhalf,\star}$, 
provisional base state velocity, $w_0^{n+\myhalf,\star}$, and provisional base state velocity 
forcing.}

\begin{enumerate}
\renewcommand{\theenumi}{{\bf \alph{enumi}}}

\item At the beginning of each time step, we need an estimate for the time-centered
source term in the velocity divergence constraint, as given in paper III equation (19),
\begin{equation}
  S =  -\sigma  \sum_k  \xi_k \omegadot_k  + 
  \frac{1}{\rho p_\rho} \sum_k p_{X_k}  {\omegadot}_k  + \sigma \Hnuc + \sigma \Hext 
  + \frac{\sigma}{\rho}\nabla\cdot\kth\nabla T \enskip .
\label{eq:defineS} 
\end{equation}

If this is the first step of the calculation ($n=0$), we set
\begin{equation}
S^{\myhalf,\star} = \frac{S^0 + S^1}{2} \enskip ,
\end{equation}
where $S^1$ is found through the iterative process that initializes the calculation.
Otherwise, following the method used in our small-scale low Mach number algorithm
(BDRWZ), we extrapolate to the half-time using the source
terms at the previous and current time levels
\begin{equation}
S^{\nph,\star} = S^n + \frac{\Delta t^n}{2} \frac{S^n - S^{n-1}}{\Delta t^{n-1}} \enskip .
\end{equation}
\item Compute
\begin{equation}
\overline{S^{\nph,\star}} = {\mathrm{\bf Avg}} \left(S^{\nph,\star}\right) \enskip .
\end{equation}
\item Compute $w_0^{n+\myhalf,\star}$:
\begin{description}
\item For planar geometry,\\
{\bf Compute} {\boldmath $w_0$} {\bf Planar}$[\overline{S^{\nph,\star}},\gammabar^n,p_0^n,
\psi^{n-\myhalf}] \rightarrow [w_0^{n+\myhalf,\star}]$.

If we are using the volume discrepancy term, compute 
$\overline{p^{\inp}} = \overline{p(\rho,h,X_k)^n}$.
\item For spherical geometry,\\
{\bf Compute} {\boldmath $w_0$} {\bf Spherical}$[\overline{S^{\nph,\star}},\gammabar^n,
\rhozero^n,p_0^n,g^n,\etarho^{n-\myhalf}] \rightarrow [w_0^{n+\myhalf,\star}]$.
\end{description}

\item Define the base state velocity forcing using equation (38) from paper III,
\begin{equation}
-\frac{1}{\rho_0} \frac{\partial \pizero}{\partial r} 
= \frac{\partial w_0}{\partial t} + w_0 \frac{\partial w_0}{\partial r} 
\enskip, \label{eq:pizero}
\end{equation}
with the following discretization:
\begin{equation}
\left ( \frac{1}{\rho_0} \frac{\partial \pi_0}{\partial r} \right )^{n,\star} = 
-\frac{w_0^{\nph,\star} - w_0^\nmh}{(\dt^n+\dt^{n-1})/2} 
- w_0^{n,\star} \left(\frac{\partial w_0}{\partial r}\right)^{n,\star} \enskip ,
\end{equation} 
where $w_0^{n,\star}$ and $(\partial w_0 / \partial r)^{n,\star}$ are defined through 
paper III equation (53),
\begin{equation}
w_0^{n,\star} = \frac{\dt^{n} w_0^{\nmh} + \dt^{n-1} w_0^{\nph,\star}}{\dt^n+\dt^{n-1}} 
\enskip,
\end{equation}
\begin{equation}
\left(\frac{\partial w_0}{\partial r}\right)^{n,\star} = 
\frac{1}{\dt^n+\dt^{n-1} } 
\left [ \dt^{n} \left(\frac{\partial w_0 }{ \partial r}\right)^{\nmh}
+ \dt^{n-1} \left(\frac{\partial w_0 }{ \partial r}\right)^{\nph,\star} \right ] 
\enskip.
\end{equation}
If $n=0$, we use $\dt^{-1} = \dt^0$.

\end{enumerate}

%--------------------------------------------------------------------------
% STEP 2
%--------------------------------------------------------------------------
\item[Step 2.] {\em Construct the provisional time-centered advective velocity on 
edges, $\uadvone$.}

The local velocity field is described by paper III equation (37),
\begin{equation}
\frac{\partial\Ubt}{\partial t} = 
- \left(\Ubt+w_0\right) \cdot \nabla \Ubt
- \left(\Ubt \cdot \eb_r\right) \frac{\partial w_0}{\partial r} \eb_r
- \frac{1}{\rho} \nabla\pi
+ \frac{1}{\rho_0} \frac{\partial \pizero}{\partial r} \eb_r
- \frac{(\rho-\rhozero)}{\rho} \; g \; \eb_r  \label{eq:utildeupd}  \enskip .
\end{equation}

We construct time-centered velocities at edges, $\uadvone$, 
using a second-order unsplit Godunov procedure followed by an elliptic
solve, as described in Appendix B of paper III.  We note that 
$\uadvone$ satisfies the discrete versions of paper III equation (21),
\begin{equation}
\int_{\Omega_H} \Ubt \cdot \eb_r \; dA = 0 \label{eq:udoterzero} \enskip ,
\end{equation}
and paper III equation (39),
\begin{equation}
\nabla \cdot (\beta_0 \Ubt )  = \beta_0 \left(S - \Sbar \right )\enskip ,
\label{eq:tildeconstraint}
\end{equation}
with the discretization
\begin{equation}
\nabla \cdot \left(\beta_0^n \uadvone\right) = 
\beta_0^n \left(S^{\nph,\star} - \overline{S^{\nph,\star}}\right) \enskip .
\end{equation}
The discretization with the volume discrepancy correction is
\begin{equation}
\nabla \cdot \left(\beta_0^n \uadvone\right) = 
\beta_0^n \left\{ \left(S^{\nph,\star} - \overline{S^{\nph,\star}}\right)
+ \frac{f}{\gammabar^n p_0^n}
\left[\frac{p(\rho,h,X_k)^n - \overline{p(\rho,h,X_k)^n}}{\Delta t^n}\right]\right\} \enskip .
\end{equation}

%--------------------------------------------------------------------------
% STEP 3
%--------------------------------------------------------------------------
\item[Step 3.] {\em React the full state through the first time interval of $\dt / 2.$}

\begin{enumerate}
\renewcommand{\theenumi}{{\bf \alph{enumi}}}

\item {\bf React State}$[\rho^n, (\rho h)^n, X_k^n, T^n, (\rho\Hext)^n, p_0^n] \rightarrow
[ \rho^{(1)},(\rho h)^{(1)},X_k^{(1)},T^{(1)},(\rho \omegadot_k)^{(1)},(\rho \Hnuc)^{(1)} ]$.

\item Set $(\rho h)_0^n$ to be the lateral average of $(\rho h)^{(1)}$:
\begin{description}
\item For planar geometry, set $(\rho h)_0^n = $ {\bf Avg}$[(\rho h)^{(1)}]$
\item For spherical geometry, set $(\rho h)_0^n = (\rho h)_0^n +$ {\bf Avg}$[(\rho h)^{(1)} - (\rho h)_0^n]$
\end{description}

\end{enumerate}

%--------------------------------------------------------------------------
% STEP 4
%--------------------------------------------------------------------------
\item[Step 4.] {\em Advect the base state and full state, through a time interval of $\dt.$}

\begin{enumerate}
\renewcommand{\theenumi}{{\bf \alph{enumi}}}

\item {\bf Advect Base Density}$[\rhozero^{n},w_0^{\nph,\star}] \rightarrow$ 
$[\rho_0^{(2a),\star}, \rho_0^{\nph,\star,{\rm pred}}]$.

\item \label{step:species update} 
Update the species.  Here we consider only the advection terms, 
neglecting the reaction terms, and solve a discretized version of
\begin{equation}
\frac{\partial (\rho X_k)}{\partial t} + \nabla \cdot (\Ub \rho X_k) = 0 \enskip .
\label{eq:species}
\end{equation}
The update consists of two steps:

  \begin{enumerate}
  \renewcommand{\labelenumii}{{\bf \roman{enumii}}.}

  \item Compute the species edge states, $(\rho X_k)^{(1),\nph,\star}$,
  for the conservative update of $(\rho X_k)^{(1)}$. 
  Here we predict $\rho^{'(1)} = \rho^{(1)} - \rhozero^n$ and 
  $X_k^{(1)} = (\rho X_k)^{(1)} / \rho^{(1)}$ to time-centered edges to obtain 
  $\rho^{'(1),\nph,\star}$ and $X_k^{(1),\nph,\star}$ using a second-order
  unsplit Godunov procedure, as described in paper II,
  Appendix A, using $\Vb = \uadvone+w_0^{\nph,\star} \eb_r$.  Convert the 
  perturbational edge state to a full state using
\begin{equation}
\rho^{(1),\nph,\star} = 
\rho^{'(1),\nph,\star} + \frac{\rho_0^n + \rho_0^{(2a),\star}}{2} \enskip .
\label{eq:rhoXedgestate}
\end{equation}
  In planar geometry, we use 4th-order spatial averaging to compute $\rho_0$ on 
  radial faces.  In spherical geometry, we first map $\rho_0$ to Cartesian cell-centers 
  using the mapping in section \ref{Sec:1D Cell-Centered to Cartesian Cell-Centered}, 
  and then use 2nd-order spatial averaging to compute $\rho_0$ on all faces.

  \item Evolve $(\rho X_k)^{(1)} \rightarrow (\rho X_k)^{(2),\star}$
  without explicitly including the reaction terms,
\begin{equation}
(\rho X_k)^{(2),\star} = (\rho X_k)^{(1)} 
 - \dt \; \left\{ \nabla \cdot \left[ \left(\uadvone+w_0^{\nph,\star} \eb_r\right)  
  (\rho X_k)^{(1),\nph,\star} \right] \right\} \enskip ,
\end{equation}
\begin{equation}
\rho^{(2),\star} = \sum_k (\rho X_k)^{(2),\star} \enskip ,
\end{equation}
\begin{equation}
X_k^{(2),\star} = (\rho X_k)^{(2),\star} / \rho^{(2),\star}
\end{equation}

\end{enumerate}

\item Define an edge-centered $\etarho^{\nph,\star}$:

\begin{description}
\item For planar geometry,
\begin{equation}
 \etarho^{\nph,\star} =  {\rm {\bf Avg}} \sum_k \left[ \left(\uadvone \cdot \eb_r + w_0^{\nph,\star}\right) (\rho X_k)^{(1),\nph,\star} \right] - w_0^{\nph,\star} \rho_0^{\nph,\star,{\rm pred}} \enskip ,
\end{equation}
\item For spherical geometry, first construct 
$\eta_{\rho}^{{\rm cart},\nph,\star} = [\rho'(\Ubt\cdot\eb_r)]^{n+\myhalf,\star}$ using:
\begin{equation}
\eta_{\rho}^{{\rm cart},\nph,\star} = \left[\left(\frac{\rho^{(1)}+\rho^{(2),\star}}{2}\right)-\left(\frac{\rho_0^n+\rho_0^{(2a),\star}}{2}\right)\right] \sum_d \left(\uadvone \cdot \eb_d\right)n_d.
\end{equation}
Then, $\etarho^{\nph,\star}$ is the cell-centered average of $\eta_{\rho}^{{\rm cart},\nph,\star}$,
\begin{equation}
\etarho^{\nph,\star} = \overline{\eta_{\rho}^{{\rm cart},\nph,\star}}.
\end{equation}
This gives a cell-centered $\etarho^{\nph,\star}$.  To get $\etarho^{\nph,\star}$ at 
radial edges, average the two neighboring radial cell-centers.
\end{description}

\item Set $\rho_0^{n+1,\star}$ to be the lateral average of $\rho^{(2),\star}$:
\begin{description}
\item For planar geometry, set $\rho_0^{n+1,\star} =$ {\bf Avg}$(\rho^{(2),\star})$
\item For spherical geometry, set $\rho_0^{n+1,\star} = \rho_0^{(2a),\star} +$ {\bf Avg}$(\rho^{(2),\star} - \rho_0^{(2a),\star})$
\end{description}

\item {\bf Compute Gravity}$[\rho_0^{n+1,\star}] \rightarrow [g^{n+1,\star}]$.

%{\bf Advect Base Pressure Planar}$[p_0^n, w_0^{\nph,\star}, \etarho^{n+\myhalf,\star}]
%\rightarrow [p_0^{n+1,\star},\psi^{\nph,\star}]$.
%\item For spherical geometry,\\
%{\bf Advect Base Pressure Spherical}$[p_0^n,w_0^{\nph,\star},\gammabar^{(1)},
%\overline{S^{\nph,\star}}, \rho^{(2),\star}, X_k^{(2),\star}]$\\
%$~~~~~~~~~~~~~~~~~~~~~~~~~~~~~~~~~~~~~~~~~~~~~~~~~~~~~~~~~~~~~~~~~~~~~~~~~~~~~~~~~~~
%\rightarrow [p_0^{n+1,\star},\psi^{n,\star},\psi^{\nph,\star}]$.
\item {\bf Enforce HSE}$[p_0^n,\rho_0^{n+1,\star},g^{n+1,\star}] \rightarrow [p_0^{n+1,\star}]$.

\item Compute $\psi$.
\begin{description}
\item For planar geometry,
\begin{equation}
\psi_j^{n+\myhalf,\star} = \frac{1}{2} \left(\eta_{\rho,j-\myhalf}^{n+\myhalf,\star} 
+ \eta_{\rho,j+\myhalf}^{n+\myhalf,\star}\right) g \enskip .
\end{equation}
\item For spherical geometry, first compute:
\begin{equation}
\gammabar^{(1)} = {\rm{\bf Avg}} \left[ \Gamma_1\left(\rho^{(1)}, p_0^{n}, 
X_k^{(1)}\right) \right] \enskip ,
\end{equation}
\begin{equation}
\gammabar^{(2),\star} = {\rm{\bf Avg}} \left[ \Gamma_1\left(\rho^{(2),\star}, p_0^{n+1,\star}, 
X_k^{(2),\star}\right) \right] \enskip .
\end{equation}
Then, define a base time and time-centered $\psi$:
\begin{equation}
\psi_j^{n,\star} = 
\left(\gammabar^{(1)} p_0^n\right)_j
\left \{ \Sbar_j^{n+\myhalf,\star} - 
\frac{1}{r_j^2} \left [ (r^2 w_0^{n+\myhalf,\star})_{j+\myhalf} -
          (r^2 w_0^{n+\myhalf,\star})_{j-\myhalf} \right ] \right \} \enskip ,
\end{equation}
\begin{eqnarray}
\psi_j^{n+\myhalf,\star} &=& 
\left(\frac{\gammabar^{(1)}+\gammabar^{(2),\star}}{2}\right)_j
\left(\frac{p_0^n+p_0^{n+1,\star}}{2}\right)_j \nonumber \\
&& \left \{ \Sbar_j^{n+\myhalf,\star} - 
\frac{1}{r_j^2} \left [ (r^2 w_0^{n+\myhalf,\star})_{j+\myhalf} -
          (r^2 w_0^{n+\myhalf,\star})_{j-\myhalf} \right ] \right \} \enskip .
\end{eqnarray}
\end{description}

\item Update $(\rho h)_0$.
\begin{description}
\item For planar geometry,\\
{\bf Advect Base Enthalpy Planar}$[(\rho h)_0^{n}, w_0^{\nph,\star}, \psi^{\nph,\star}]
\rightarrow [(\rho h)_0^{n+1,\star}]$.
\item For spherical geometry,\\
{\bf Advect Base Enthalpy Spherical}$[(\rho h)_0^{n}, w_0^{\nph,\star}, 
\psi^{n,\star},\psi^{\nph,\star}] \rightarrow [(\rho h)_0^{n+1,\star}]$.
\end{description}

\item Update the enthalpy.  Here we consider only the advective terms, 
neglecting the reaction terms and thermal diffusion terms, and solve a discretized version of
  \begin{equation}
  \frac{\partial (\rho h)}{\partial t}  = - \nabla \cdot (\Ub \rho h)
+ \psi + (\Ubt \cdot \eb_r) \frac{\partial p_0}{\partial r} 
\enskip , \label{eq:rhohupdadv} 
  \end{equation}
For spherical geometry, experience has shown that solving the analytically equivalent form,
\begin{equation}
  \frac{\partial (\rho h)}{\partial t}  = - \nabla \cdot (\Ub \rho h)
+ \psi + \nabla \cdot (\Ubt p_0) - p_0 \nabla \cdot \Ubt  
\enskip , \label{eq:rhohupdadv2} 
\end{equation}
results in a better solution.

  \begin{enumerate}
  \renewcommand{\labelenumii}{{\bf \roman{enumii}}.}

  \item Compute the enthalpy edge state, $(\rho h)^{(1),\nph,\star},$
    for the conservative update of $(\rho h)^{(1)}.$  Here we predict 
    $(\rho h)' = (\rho h)^{(1)} - (\rho h)_0$ to time-centered edges to obtain 
    $(\rho h)^{'(1),\nph,\star}$, 
    using a second-order unsplit Godunov procedure, 
    as described in paper II, Appendix A, using $\Vb =
    \uadvone+w_0^{\nph,\star} \eb_r$.  We do not include the reaction
    terms in the enthalpy prediction, since we accounted for them
    already in {\bf React State}.  Convert the perturbational edge
    state to a full state using
    
\begin{equation}
(\rho h)^{(1),\nph,\star} = 
(\rho h)^{'(1),\nph,\star} + \frac{(\rho h)_0^n + (\rho h)_0^{n+1,\star}}{2}
\enskip .
\end{equation}
  For planar geometry, we use 4th-order spatial averaging to compute $(\rho h)_0$
  on radial faces.  In spherical geometry, we first map $h_0 \equiv (\rho h)_0/\rho_0$ 
  and $\rho_0$ to Cartesian cell-centers using the mapping in section 
  \ref{Sec:1D Cell-Centered to Cartesian Cell-Centered}, use 2nd-order spatial 
  averaging to compute $h_0$ and $\rho_0$ on all faces, and then multiply these 
  terms to get $(\rho h)_0$ on all faces.

  \item Evolve $(\rho h)^{(1)} \rightarrow (\rho h)^{(2),\star}$ without
  explicitly including the reaction terms.

\begin{description}
\item For planar geometry,

  \begin{eqnarray}
  (\rho h)^{(2),\star} &=& (\rho h)^{(1)} - \dt \; \left\{ \nabla
      \cdot \left[ \left(\uadvone+w_0^{\nph,\star} \eb_r\right) (\rho
      h)^{(1),\nph,\star} \right] \right\} \nonumber \\ && + \dt \;
    \left(\uadvone \cdot \eb_r\right) \left(\frac{\partial
      p_0}{\partial r} \right)^{n} + \dt \; \psi^{\nph,\star} \enskip ,
  \end{eqnarray}

\item For spherical geometry,

  \begin{eqnarray}
  (\rho h)^{(2),\star} &=& (\rho h)^{(1)} - \dt \; \left\{ \nabla
      \cdot \left[ \left(\uadvone+w_0^{\nph,\star} \eb_r\right) (\rho
      h)^{(1),\nph,\star} \right] \right\} \nonumber \\ 
    && + \dt \; \left \{ \nabla \cdot \left (\uadvone p_0^{n} \right ) 
       - p_0^{n} \nabla \cdot \uadvone \right \} \nonumber \\
    && + \dt \; \psi^{\nph,\star} \enskip ,
  \end{eqnarray}

\end{description}

\end{enumerate}

If {\tt do\_eos\_h\_above\_cutoff=T} (which is the default setting), then if
$\rho^{(2),\star} < {\tt base\_cutoff\_density}$, then we recompute enthalpy using
\begin{equation}
(\rho h)^{(2),\star} = \rho^{(2),\star}h\left(\rho^{(2),\star},p_0^{n+1,\star},X_k^{(2),\star}\right).
\end{equation}


%--------------------------------------------------------------------------
% STEP 4.1
%--------------------------------------------------------------------------

\item If we are using thermal diffusion, diffuse the enthalpy through a time interval of 
$\dt$.  First, define $(\rho h)^{(1a),\star} = (\rho h)^{(2),\star}$.  We recompute
$(\rho h)^{(2),\star}$ to account for thermal diffusion.  Here we begin
with the enthalpy equation, but consider only the 
diffusion terms,
\begin{equation}
  \frac{\partial (\rho h)}{\partial t}  = 
 \nabla\cdot\kth\nabla T 
\enskip . \label{eq:rhohupdthermal} 
  \end{equation}
We can recast this as an enthalpy-diffusion equation by applying the
chain-rule to $h(p_0,T,X_k)$,
\begin{equation}
\nabla h = h_p \nabla p_0 + c_p \nabla T + \sum_k \xi_k \nabla X_k \enskip ,
\end{equation}
giving
\begin{equation}
  \frac{\partial (\rho h)}{\partial t}  = 
 \nabla\cdot \frac{\kth}{c_p}\nabla h -  
 \sum_k \nabla\cdot \frac{\xi_k \kth}{c_p}\nabla X_k -
 \nabla\cdot \frac{h_p \kth}{c_p}\nabla p_0 
\enskip . \label{eq:rhohupdthermal2} 
  \end{equation}

Compute $\kth^{(1)}, c_p^{(1)}$, and $\xi_k^{(1)}$ from $\rho^{(1)}, T^{(1)}$, and $X_k^{(1)}$ as inputs to the equation of state.  The update is given by
\begin{eqnarray}
(\rho h)^{(2),\star} &=& (\rho h)^{(1a),\star} + \frac{\dt}{2}\nabla\cdot\left(\frac{\kth^{(1)}}{c_p^{(1)}}\nabla h^{(2),\star} + \frac{\kth^{(1)}}{c_p^{(1)}}\nabla h^{(1)}\right)\nonumber\\
&&- \frac{\dt}{2}\sum_k\nabla\cdot\left(\frac{\xi_k^{(1)}\kth^{(1)}}{c_p^{(1)}}\nabla X_k^{(2),\star} + \frac{\xi_k^{(1)}\kth^{(1)}}{c_p^{(1)}}\nabla X_k^{(1)}\right)\nonumber\\
&&- \frac{\dt}{2}\nabla\cdot\left(\frac{h_p^{(1)}\kth^{(1)}}{c_p^{(1)}}\nabla p_0^{n+1,\star} + \frac{h_p^{(1)}\kth^{(1)}}{c_p^{(1)}}\nabla p_0^{n}\right),
\end{eqnarray}
which is numerically implemented as a diffusion equation for $h^{(2),\star}$,
\begin{eqnarray}
\left(\rho^{(2),\star} - \frac{\dt}{2}\nabla\cdot\frac{\kth^{(1)}}{c_p^{(1)}}\nabla\right)h^{(2),\star} &=& (\rho h)^{(1a),\star} + \frac{\dt}{2}\nabla\cdot\frac{\kth^{(1)}}{c_p^{(1)}}\nabla h^{(1)}\nonumber\\
&&- \frac{\dt}{2}\sum_k\nabla\cdot\left(\frac{\xi_k^{(1)}\kth^{(1)}}{c_p^{(1)}}\nabla X_k^{(2),\star} + \frac{\xi_k^{(1)}\kth^{(1)}}{c_p^{(1)}}\nabla X_k^{(1)}\right)\nonumber\\
&&- \frac{\dt}{2}\nabla\cdot\left(\frac{h_p^{(1)}\kth^{(1)}}{c_p^{(1)}}\nabla p_0^{n+1,\star} + \frac{h_p^{(1)}\kth^{(1)}}{c_p^{(1)}}\nabla p_0^{n}\right),
\end{eqnarray}
\item Update the temperature using the equation of state.  For planar geometry,
we update using
\begin{equation}
T^{(2),\star} = T\left(\rho^{(2),\star}, h^{(2),\star}, X_k^{(2),\star}\right) \enskip ,
\end{equation}
but in spherical geometry, we set {\tt use\_tfromp = T}, and we use
\begin{equation}
T^{(2),\star} = T\left(\rho^{(2),\star}, p_0^{n+1,\star}, X_k^{(2),\star}\right) \enskip .
\end{equation}
The latter has the effect of completely decoupling the enthalpy equation from the 
evolution.

\end{enumerate}

%--------------------------------------------------------------------------
% STEP 5
%--------------------------------------------------------------------------
\item[Step 5.] {\em React the full state through a second time interval of $\dt / 2.$}

{\bf React State}$[ \rho^{(2),\star},(\rho h)^{(2),\star}, X_k^{(2),\star}, T^{(2),\star}, 
(\rho\Hext)^{(2),\star}, p_0^{n+1,\star}] $\\
$~~~~~~~~~~~~~~~~~~~~~~~~~~~~~~~~~~~~~~~~~~~~~
\rightarrow [ \rho^{n+1,\star},(\rho h)^{n+1,\star}, 
X_k^{n+1,\star}, T^{n+1,\star}, (\rho \omegadot_k)^{(2),\star}, (\rho \Hnuc)^{(2),\star} ].$

%--------------------------------------------------------------------------
% STEP 6
%--------------------------------------------------------------------------
\item[Step 6.] {\em Define the time-centered expansion, $S^{n+\myhalf}$, base state
velocity, $w_0^{n+\myhalf}$, and base state velocity forcing.}

\begin{enumerate}
\renewcommand{\theenumi}{{\bf \alph{enumi}}}
\item Define
\begin{equation}
  S^{n+1,\star} =  -\sigma  \sum_k  \xi_k  (\omegadot_k)^{(2),\star}  + 
  \sigma \Hnuc^{(2),\star} +
  \frac{1}{\rho p_\rho} \sum_k p_{X_k}  ({\omegadot}_k)^{(2),\star}  
  + \sigma \Hext + \frac{\sigma}{\rho}\nabla\cdot\kth\nabla T^{n+1,\star} \enskip ,
\end{equation} 
where $(\omegadot_k)^{(2),\star} = (\rho \omegadot_k)^{(2),\star} / \rho^{(2),\star}$
and the thermodynamic quantities are defined using $\rho^{n+1,\star}, X_k^{n+1,\star},$ 
and $T^{n+1,\star}$ as inputs to the equation of state.
Then define
\begin{equation}
 S^\nph = \frac{S^n + S^{n+1,\star}}{2} \enskip. 
\end{equation}

\item Compute
\begin{equation}
\overline{S^{\nph}} = {\mathrm{\bf Avg}} (S^{\nph}) \enskip.
\end{equation}

\item Define:
\begin{eqnarray}
\gammabar^{\nph} = \frac{\gammabar^{n} + \gammabar^{n+1,\star}}{2}\enskip, \qquad
\rhozero^{\nph} = \frac{\rhozero^{n} + \rhozero^{n+1,\star}}{2}\enskip, \nonumber \\
p_0^{\nph} = \frac{p_0^{n} + p_0^{n+1,\star}}{2}\enskip, \qquad
g^\nph = \frac{g^n + g^{n+1,\star}}{2}\enskip ,
\end{eqnarray}
with
\begin{equation}
 \gammabar^{n+1,\star} = {\rm{\bf Avg}} 
\left[ \Gamma_1\left(\rho^{n+1,\star}, p_0^{n+1,\star}, X_k^{n+1,\star}\right) \right] 
\enskip .
\end{equation}

\item Compute $w_0^{n+\myhalf}$:
\begin{description}
\item For planar geometry,\\
{\bf Compute} {\boldmath $w_0$} {\bf Planar}$[\overline{S^{\nph}},
\gammabar^{n+\myhalf},p_0^{n+\myhalf},g^{n+\myhalf},\psi^{n+\myhalf,\star}]\rightarrow [w_0^{n+\myhalf}]$.

If we are using the volume discrepancy term, compute
\begin{equation}
\overline{p^{\inp}} = \frac{\overline{p(\rho,h,X_k)^n} + \overline{p(\rho,h,X_k)^{n+1,\star}}}{2}.
\end{equation}
\item For spherical geometry,\\
{\bf Compute} {\boldmath $w_0$} {\bf Spherical}$[\overline{S^{\nph}},\gammabar^{n+\myhalf},
\rhozero^{n+\myhalf},p_0^{n+\myhalf},\etarho^{n+\myhalf}] \rightarrow [w_0^{n+\myhalf}]$.
\end{description}

\item Define the base state velocity forcing using equation (\ref{eq:pizero}) 
with the following discretization:
\begin{equation}
\left ( \frac{1}{\rho_0} \frac{\partial \pi_0}{\partial r} \right )^n = 
-\frac{w_0^{\nph} - w_0^\nmh}{\myhalf(\dt^n+\dt^{n-1})} 
- w_0^n \left(\frac{\partial w_0}{\partial r}\right)^n,
\end{equation}
where $w_0^{n}$ and $(\partial w_0 / \partial r)^{n}$ are defined as
\begin{equation}
w_0^n = \frac{\dt^{n} w_0^{\nmh} + \dt^{n-1} w_0^{\nph}}{\dt^n+\dt^{n-1}}\enskip,
\end{equation}
\begin{equation}
\left(\frac{\partial w_0}{\partial r}\right)^{n} = \frac{1}{\dt^n+\dt^{n-1} } \left [ \dt^{n} \left(\frac{\partial w_0 }{ \partial r}\right)^{\nmh}
+ \dt^{n-1} \left(\frac{\partial w_0 }{ \partial r}\right)^{\nph} \right ] \enskip.
\end{equation}
If $n=0$, we use $\dt^{-1} = \dt^0$.

\end{enumerate}

%--------------------------------------------------------------------------
% STEP 7
%--------------------------------------------------------------------------
\item[Step 7.] {\em Construct the time-centered advective velocity on edges, $\uadvtwo$.}

The procedure to construct $\uadvtwo$ is described in detail in Appendix B of paper III
and is analogous to the procedure used in step 2, but with updated values
for $w_0$ and $\pizero.$  We note that $\uadvtwo$ satisfies the discrete versions of 
equations (\ref{eq:udoterzero}) and (\ref{eq:tildeconstraint}), with the discretization
\begin{equation}
\nabla \cdot \left(\beta_0^{\nph} \uadvtwo\right) =
\beta_0^{\nph}\left(S^{\nph} - \overline{S^{\nph}}\right) \enskip ,
\end{equation}
\begin{equation}
\beta_0^{\nph} = \frac{ \beta_0^n +  \beta_0^{n+1,\star} }{2}; ~~~~~
 \beta_0^{n+1,\star} = \beta \left(\rho_0^{n+1,\star}, p_0^{n+1,\star}, \gammabar^{n+1,\star}, g^{n+1,\star}\right) \enskip ,
\end{equation}
where $\beta_0$ is computed as described in paper III, Appendix C.

The discretization with the volume discrepancy correction is
\begin{equation}
\nabla \cdot \left(\beta_0^{\nph} \uadvtwo\right) =
\beta_0^{\nph}\left\{\left(S^{\nph} - \overline{S^{\nph}}\right)
 + \frac{f}{\gammabar^{\nph} p_0^{\nph}}\cdot 
\left[\frac{p(\rho,h,X_k)^{\nph} - \overline{p(\rho,h,X_k)^{\nph}}}{\Delta t^n}\right]\right\} \enskip ,
\end{equation}

%--------------------------------------------------------------------------
% STEP 8
%--------------------------------------------------------------------------
\item[Step 8.] {\em Advect the base state and full state through a time interval of $\dt.$}

\begin{enumerate}
\renewcommand{\theenumi}{{\bf \alph{enumi}}}

\item {\bf Advect Base Density}$[\rho_0^{n},w_0^{\nph}] \rightarrow$ 
$[\rho_0^{(2a)}, \rho_0^{\nph,{\rm pred}}]$.

\item Update the species.  This step is identical to step 4b with the following changes:

\begin{enumerate}
\renewcommand{\labelenumii}{{\bf \roman{enumii}}.}

\item Compute the species edge states, $(\rho X_k)^{(1),n+\myhalf}$, 
using $\Vb = \uadvtwo+w_0^{\nph}\eb_r$.  Convert the perturbational edge 
state to a full state using 
\begin{equation}
\rho^{(1),\nph} = \rho^{'(1),\nph} + \frac{\rho_0^n + \rho_0^{(2a)}}{2}\enskip .
\end{equation}

\item Evolve $(\rho X_k)^{(1)} \rightarrow (\rho X_k)^{(2)}$ using
\begin{equation}
(\rho X_k)^{(2)} = (\rho X_k)^{(1)} 
- \dt \; \left\{ \nabla \cdot \left[\left(\uadvtwo+w_0^{\nph} \eb_r\right)  
(\rho X_k)^{(1),\nph} \right] \right\}  \enskip ,
\end{equation}
\begin{equation}
\rho^{(2)} = \sum_k (\rho X_k)^{(2)} \enskip ,
\end{equation}
\begin{equation}
X_k^{(2)} = (\rho X_k)^{(2)} / \rho^{(2)}
\end{equation}

\end{enumerate}

\item Define an edge-centered $\etarho^{\nph}$:

\begin{description}

\item For planar geometry,
\begin{equation}
 \etarho^{\nph} = {\rm {\bf Avg}} \sum_k \left [\left(\uadvtwo \cdot \eb_r + w_0^{\nph}\right) (\rho X_k)^{(1),\nph} \right] - w_0^{\nph} \rho_0^{\nph,{\rm pred}} \enskip ,
\end{equation}
\item For spherical geometry, first construct 
$\eta_{\rho}^{{\rm cart},\nph} = [\rho'(\Ubt\cdot\eb_r)]^{n+\myhalf}$ using:
\begin{equation}
\eta_{\rho}^{{\rm cart},\nph} = \left[\left(\frac{\rho^{(1)}+\rho^{(2)}}{2}\right)-\left(\frac{\rho_0^n+\rho_0^{(2a)}}{2}\right)\right] \sum_d \left(\uadvtwo \cdot \eb_d\right)n_d.
\end{equation}
Then, $\etarho^{\nph}$ is the cell-centered average of $\eta_{\rho}^{{\rm cart},\nph}$,
\begin{equation}
\etarho^{\nph} = \overline{\eta_{\rho}^{{\rm cart},\nph}}.
\end{equation}
This gives a cell-centered $\etarho^{\nph}$.  To get $\etarho^{\nph}$ at 
radial edges, average the two neighboring cell-centers.
\end{description}

\item Set $\rho_0^{n+1}$ to be the lateral average of $\rho^{(2)}$:
\begin{description}
\item For planar geometry, set $\rho_0^{n+1} =$ {\bf Avg}$(\rho^{(2)})$
\item For spherical geometry, set $\rho_0^{n+1} = \rho_0^{(2a)} +$ {\bf Avg}$(\rho^{(2)} - \rho_0^{(2a)})$
\end{description}

\item {\bf Compute Gravity}$[\rho_0^{n+1}] \rightarrow [g^{n+1}]$.

Then, set $g^{n+\myhalf} = (g^n + g^{n+1})/2$.

%\item Update $p_0$ and compute $\psi$.
%\begin{description}
%\item For planar geometry,\\
%{\bf Advect Base Pressure Planar}$[p_0^n, w_0^{\nph}, \etarho^{n+\myhalf}]$ 
%$\rightarrow [p_0^{n+1},\psi^{\nph}]$.
%\item For spherical geometry,\\
%{\bf Advect Base Pressure Spherical}$[p_0^n,w_0^{\nph},\gammabar^{(1)},\overline{S^{\nph}},
%\rho^{(2)},X_k^{(2)}]$\\
%$~~~~~~~~~~~~~~~~~~~~~~~~~~~~~~~~~~~~~~~~~~~~~~~~~~~~~~~~~~~~~~~~~~~~~~~~~~~~~~~~~~~
% \rightarrow [p_0^{n+1},\psi^n,\psi^{\nph}]$.
%\end{description}
\item {\bf Enforce HSE}$[p_0^n,\rho_0^{n+1},g^{n+1}] \rightarrow [p_0^{n+1}]$.

\item Compute $\psi$.
\begin{description}
\item For planar geometry, 
\begin{equation}
\psi_j^{n+\myhalf} = \frac{1}{2} \left(\eta_{\rho,j-\myhalf}^{n+\myhalf} 
+ \eta_{\rho,j+\myhalf}^{n+\myhalf}\right) g\enskip .
\end{equation}
\item For spherical geometry, first compute:
\begin{equation}
\gammabar^{(2)} = {\rm{\bf Avg}} \left[ \Gamma_1\left(\rho^{(2)}, p_0^{n+1}, 
X_k^{(2)}\right) \right] \enskip .
\end{equation}
Then, define a base time and time-centered $\psi$:
\begin{equation}
\psi_j^{n} = 
\left(\gammabar^{(1)} p_0^n\right)_j
\left \{ \Sbar_j^{n+\myhalf} - 
\frac{1}{r_j^2} \left [ (r^2 w_0^{n+\myhalf})_{j+\myhalf} -
          (r^2 w_0^{n+\myhalf})_{j-\myhalf} \right ] \right \} \enskip ,
\end{equation}
\begin{eqnarray}
\psi_j^{n+\myhalf} &=& 
\left(\frac{\gammabar^{(1)}+\gammabar^{(2)}}{2}\right)_j
\left(\frac{p_0^n+p_0^{n+1}}{2}\right)_j \nonumber \\
&& \left \{ \Sbar_j^{n+\myhalf} - 
\frac{1}{r_j^2} \left [ (r^2 w_0^{n+\myhalf})_{j+\myhalf} -
          (r^2 w_0^{n+\myhalf})_{j-\myhalf} \right ] \right \} \enskip .
\end{eqnarray}
\end{description}

\item Update $(\rho h)_0$.
\begin{description}
\item For planar geometry,\\
{\bf Advect Base Enthalpy Planar}$[(\rho h)_0^n, w_0^{\nph}, \psi^{\nph}]
\rightarrow [(\rho h)_0^{n+1}]$.
\item For spherical geometry,\\
{\bf Advect Base Enthalpy Spherical}$[(\rho h)_0^n, w_0^{\nph}, \psi^n, \psi^{\nph}]
\rightarrow [(\rho h)_0^{n+1}]$.
\end{description}

\item Update the enthalpy.  This step is identical to step 4i with the following changes:

\begin{enumerate}
\renewcommand{\labelenumii}{{\bf \roman{enumii}}.}

\item Compute the enthalpy edge state, $(\rho h)^{(1),\nph}$, using
$\Vb = \uadvtwo+w_0^{\nph} \eb_r$.  Convert the perturbational edge 
state to a full state using
\begin{equation}
(\rho h)^{(1),\nph} = 
(\rho h)^{'(1),\nph} + \frac{(\rho h)_0^n + (\rho h)_0^{n+1}}{2} \enskip .
\end{equation}

\item Evolve $(\rho X_k)^{(1)} \rightarrow (\rho X_k)^{(2)}$
without explicitly including the reaction terms,

\begin{description}
\item For planar geometry,
\begin{eqnarray}
(\rho h)^{(2)} &=& (\rho h)^{(1)} - \dt \; \left\{ \nabla \cdot \left[ \left(\uadvtwo+w_0^{\nph} \eb_r\right)  
(\rho h)^{(1),\nph} \right] \right\} \nonumber \\
&& + \frac{\dt}{2} \; \left(\uadvtwo \cdot \eb_r\right)
\left [ \left(\frac{\partial p_0}{\partial r} \right)^{n}
      + \left(\frac{\partial p_0}{\partial r} \right)^{n+1}  \right ] 
+ \dt \; \psi^{\nph} \enskip ,
\end{eqnarray}

\item For spherical geometry,
\begin{eqnarray}
(\rho h)^{(2)} &=& (\rho h)^{(1)} - \dt \; \left\{ \nabla \cdot \left[ \left(\uadvtwo+w_0^{\nph} \eb_r\right)  
(\rho h)^{(1),\nph} \right] \right\} \nonumber \\
    && + \dt \; \left \{ \nabla \cdot \left (\uadvtwo p_0^{\nph} \right ) 
       - p_0^{\nph} \nabla \cdot \uadvtwo \right \} \nonumber \\
    && + \dt \; \psi^{\nph} \enskip ,
\end{eqnarray}

\end{description}

\end{enumerate}

If {\tt do\_eos\_h\_above\_cutoff=T} (which is the default setting), then if
$\rho^{(2)} < {\tt base\_cutoff\_density}$, then we recompute enthalpy using
\begin{equation}
(\rho h)^{(2)} = \rho^{(2)}h\left(\rho^{(2)},p_0^{n+1},X_k^{(2)}\right).
\end{equation}

%--------------------------------------------------------------------------
% STEP 8.1
%--------------------------------------------------------------------------
\item If we are using thermal diffusion, diffuse the enthalpy through a time interval of 
$\dt$.  First, define $(\rho h)^{(1a)} = (\rho h)^{(2)}$.  We recompute $(\rho h)^{(2)}$ to 
account for thermal diffusion.  Compute $\kth^{(2),\star}, c_p^{(2),\star}$, and 
$\xi_k^{(2),\star}$, from $\rho^{(2),\star}, T^{(2),\star}$, and $X_k^{(2),\star}$ as inputs to 
the equation of state.  The update is given by
\begin{eqnarray}
(\rho h)^{(2)} &=& (\rho h)^{(1a)} + \frac{\dt}{2}\nabla\cdot\left(\frac{\kth^{(2),\star}}{c_p^{(2),\star}}\nabla h^{(2)} + \frac{\kth^{(1)}}{c_p^{(1)}}\nabla h^{(1)}\right)\nonumber\\
&&- \frac{\dt}{2}\sum_k\nabla\cdot\left(\frac{\xi_k^{(2),\star}\kth^{(2),\star}}{c_p^{(2),\star}}\nabla X_k^{(2)} + \frac{\xi_k^{(1)}\kth^{(1)}}{c_p^{(1)}}\nabla X_k^{(1)}\right)\nonumber\\
&&- \frac{\dt}{2}\nabla\cdot\left(\frac{h_p^{(2),\star}\kth^{(2),\star}}{c_p^{(2),\star}}\nabla p_0^{n+1} + \frac{h_p^{(1)}\kth^{(1)}}{c_p^{(1)}}\nabla p_0^{n}\right),
\end{eqnarray}
which is numerically implemented as a diffusion equation for $h^{(2)}$,
\begin{eqnarray}
\left(\rho^{(2)} - \frac{\dt}{2}\nabla\cdot\frac{\kth^{(2),\star}}{c_p^{(2),\star}}\nabla\right)h^{(2)} &=& (\rho h)^{(1a)} + \frac{\dt}{2}\nabla\cdot\frac{\kth^{(1)}}{c_p^{(1)}}\nabla h^{(1)}\nonumber\\
&&- \frac{\dt}{2}\sum_k\nabla\cdot\left(\frac{\xi_k^{(2),\star}\kth^{(2),\star}}{c_p^{(2),\star}}\nabla X_k^{(2)} + \frac{\xi_k^{(1)}\kth^{(1)}}{c_p^{(1)}}\nabla X_k^{(1)}\right)\nonumber\\
&&- \frac{\dt}{2}\nabla\cdot\left(\frac{h_p^{(2),\star}\kth^{(2),\star}}{c_p^{(2),\star}}\nabla p_0^{n+1} + \frac{h_p^{(1)}\kth^{(1)}}{c_p^{(1)}}\nabla p_0^{n}\right),
\end{eqnarray}
\item Update the temperature using the equation of state.  For planar geometry we
update using
\begin{equation}
T^{(2)} = T\left(\rho^{(2)}, h^{(2)}, X_k^{(2)}\right) \enskip ,
\end{equation}
but in spherical geometry, we set {\tt use\_tfromp = T}, and we use
\begin{equation}
T^{(2)} = T\left(\rho^{(2)}, p_0^{n+1}, X_k^{(2)}\right) \enskip .
\end{equation}
The latter has the effect of completely decoupling the enthalpy equation from the 
evolution.

\end{enumerate}

%--------------------------------------------------------------------------
% STEP 9
%--------------------------------------------------------------------------
\item[Step 9.] {\em React the full state through a second time interval of $\dt / 2.$}

{\bf React State}$[\rho^{(2)},(\rho h)^{(2)}, X_k^{(2)},T^{(2)}, (\rho\Hext)^{(2)}, p_0^{n+1}]$\\
$~~~~~~~~~~~~~~~~~~~~~~~~~~~~~~~~~~~~~~~~~~~~~~~~~~~~~~\rightarrow [\rho^{n+1}, 
(\rho h)^{n+1}, X_k^{n+1}, T^{n+1}, (\rho \omegadot_k)^{(2)}, (\rho \Hnuc)^{(2)} ].$  


%--------------------------------------------------------------------------
% STEP 10
%--------------------------------------------------------------------------
\item[Step 10.] {\em Define the new time expansion $S^{n+1}$, and $\gammabar^{n+1}$.}

\begin{enumerate}
\renewcommand{\theenumi}{{\bf \alph{enumi}}}
\item Define
\begin{equation}
  S^{n+1} =  -\sigma  \sum_k  \xi_k (\omegadot_k)^{(2)}  + \sigma \Hnuc^{(2)} +
  \frac{1}{\rho p_\rho} \sum_k p_{X_k}  ({\omegadot}_k)^{(2)}  
   + \sigma \Hext + \frac{\sigma}{\rho}\nabla\cdot\kth\nabla T^{n+1} \enskip ,
\end{equation}
where $(\omegadot_k)^{(2)} = (\rho \omegadot_k)^{(2)} / \rho^{(2)}$
and the thermodynamic quantities are defined using $\rho^{n+1}, X_k^{n+1},$
and $T^{n+1}$ as inputs to the equation of state.

\item Compute
\begin{equation}
\overline{S^{n+1}} = {\mathrm{\bf Avg}} (S^{n+1}) \enskip.
\end{equation}

\item Define
\begin{equation}
 \gammabar^{n+1} = {\rm{\bf Avg}}\left[\Gamma_1\left(\rho^{n+1}, p_0^{n+1}, 
X_k^{n+1}\right) \right] \enskip .
\end{equation}

\end{enumerate}


%--------------------------------------------------------------------------
% STEP 11
%--------------------------------------------------------------------------
\item[Step 11.] {\em Update the velocity}.  

The velocity update happens analogously to paper II, using $S^{n+1}$ from step 10.
We update the velocity field $\Ubt^n$ to $\Ubt^{n+1,\star}$ by discretizing 
equation (\ref{eq:utildeupd}) as
\begin{eqnarray}
\Ubt^{n+1,\star} &=& \Ubt^n - \dt \;
 \left[\left(\uadvtwo+ w_0^{\nph} \eb_r\right) \cdot \nabla \Ubt \right]^\nph
      - \dt \; \left(\uadvtwo \cdot \eb_r\right)  \left(\frac{\partial w_0^{\nph}}{\partial r} \right) \eb_r \nonumber \\
   &&   + \dt \left[ - \frac{1}{\rho^\nph} {\mathbf{G}} \pi^\nmh
        + \frac{1}{\rho_0} {\mathbf{G}} \pi_0^n
        - \frac{(\rho^\nph-\rhozero^\nph)}{\rho^\nph} g^{n+\myhalf} \eb_r \right] \enskip ,
\end{eqnarray}
where $\rho^{\nph}$ is defined as
\begin{equation}
\rho^\nph = \frac{\rho^n + \rho^{n+1}}{2}\enskip ,
\end{equation}
and $\mathbf{G}$ approximates a cell-centered gradient from nodal data.
The construction of $[(\uadvtwo+ w_0^{\nph} \eb_r \cdot \nabla) \Ubt ]^\nph$
is described in paper II, Appendix A, with $\Vb = \uadvtwo+w_0^{\nph} \eb_r$ and $s$ 
set to each component of $\Ubt^n$ individually.  Here, the $\star$ superscript 
does not refer to the predictor-corrector aspect of the algorithm, but rather 
than the updated velocity does not satisfy the divergence constraint.

Finally, we impose the divergence constraint from equation (\ref{eq:tildeconstraint}):
\begin{equation}
\nabla \cdot \left(\beta_0^{\nph} \Ubt^{n+1} \right) 
= \beta_0^{\nph} \left(S^{n+1} - \overline{S^{n+1}} \right) \enskip,
\end{equation}
where $\beta_0^{n+\myhalf}$ is now defined as
\begin{equation}
\beta_0^{n+\myhalf} = \frac{\beta_0^n + \beta_0^{n+1}}{2}; ~~~~~ 
\beta_0^{n+1} = \beta \left(\rho_0^{n+1}, p_0^{n+1}, \gammabar^{n+1}, g^{n+1}\right)\enskip .
\end{equation}

The discretization with the volume discrepancy term is:
\begin{equation}
\nabla \cdot \left(\beta_0^{\nph} \Ubt^{n+1} \right)  = \beta_0^{\nph}\left\{  \left(S^{n+1} - \overline{S^{n+1}} \right)
+ \frac{f}{\gammabar^{n+1} p_0^{n+1}}
\left[\frac{p(\rho,h,X_k)^{n+1} - \overline{p(\rho,h,X_k)^{n+1}}}{\Delta t^n}\right]\right\}
\enskip .
\end{equation}

To solve this equation, we define
${\bf V} = \Ubt^{n+1,\star} + ( \dt / \rho^\nph ) \; \mathbf{G} \pi^\nmh$ 
and solve
\begin{equation}
 L_\beta^\rho \phi = D \left ( \beta_0^{\nph} {\bf V} \right) - \beta_0^{\nph} 
\left(S^{n+1}-\overline{S^{n+1}} \right) \enskip, \label{Linear System for Multigrid}
\end{equation}
for nodal values of $\phi$, where $L_\beta^\rho$ is the standard bilinear
finite element approximation to $\nabla \cdot ({\beta_0^\nph}/{\rho^\nph}) \nabla.$
The discretization with the volume discrepancy term is:
\begin{equation}
 L_\beta^\rho \phi =
   D \left ( \beta_0^{\nph} {\bf V} \right) - \beta_0^{\nph}\left\{ \left(S^{n+1}-\overline{S^{n+1}} \right)
- \frac{f}{\gammabar^{n+1} p_0^{n+1}}
\left[\frac{p(\rho,h,X_k)^{n+1} - \overline{p(\rho,h,X_k)^{n+1}}}{\Delta t^n}\right]\right\} \enskip.
\end{equation}
In this step, $D$ is a discrete second-order operator that approximates the 
divergence at nodes from cell-centered data and satisfies
$\mathbf{G} = -D^T.$ 
(See (cite almgrenBellSzymczak:1996) for a detailed discussion of this
approximate projection; see (cite almgren:bell:crutchfield) for a discussion
of this particular form of the projection operand.)  
We solve the linear system of equation (\ref{Linear System for Multigrid})
using multigrid V-cycles with Gauss-Seidel relaxation.

We determine the new time velocity field from
\begin{equation}
\Ubt^{n+1} = {\bf V} - \frac{1}{\rho^\nph} \mathbf{G} \phi \enskip ,
\end{equation}
and the new time-centered perturbational pressure from
\begin{equation}
  \pi^\nph = \frac{1}{\dt} \phi \enskip .
\end{equation}

%--------------------------------------------------------------------------
% STEP 12
%--------------------------------------------------------------------------
\item[Step 12.] {\em Compute a new $\dt.$}

Compute $\dt$ for the next time step with the procedure described in \S 3.4 of paper III
using $w_0$ as computed in step 6 and
$\Ubt^{n+1}$ as computed in step 11.  We use this $\dt$ in the next time step. 


\end{description}

\noindent This completes the time advancement of the algorithm.



%-----------------------------------------------------------------------------
% Initialization
%-----------------------------------------------------------------------------

\section{Initialization}\label{Sec:Initialization}

We start each calculation with user-specified initial values for
$\rho$, $X_k$ and $T,$ as well as an initial background state.  In
order for the low Mach number assumption to hold, the initial data
must be thermodynamically consistent with the initial background
state.  In addition, the initial velocity field must satisfy an
initial approximation to the divergence constraint.  We use an iterative
procedure to compute both an initial right-hand-side for the
constraint equation and an initial velocity field that satisfies
the constraint.  The user specifies the number of iterations,
$N_{\rm iters}^{S},$ in this first step of the initialization procedure.

The initial perturbational pressure also needs to be determined for
use in steps 2, 7 and 11. 
This is done through a second iterative procedure which follows the
time advancement algorithm as described in steps 1-11 in 
\S \ref{Sec:Time Advancement Algorithm}.  
The user specifies the number of iterations, 
$N_{\rm iters}^{\pi},$ in this second step of the initialization procedure.
The details for both iterations are given below.\\

%--------------------------------------------------------------------------
% STEP 0
%--------------------------------------------------------------------------
\noindent {\bf Step 0.} {\em Initialization}

First, we need to construct approximations to $S^0, w_0^{-\myhalf}, \Delta t^0$, 
and $\Ub^0$.  Start with initial data $X_k^{\initp}, \rho^{\initp},$ $T^{\initp},$ an 
initial base state, and an initial guess for the velocity, $\Ub^{\initp}$.
Set $w_0^0 = 0$ as an initial approximation.  Use the equation of state to 
determine $(\rho h)^{\initp}$.  Compute $\beta_0^0$ as a function of 
the initial data.  Then, project $\Ub^{\initp}$ using $\beta_0^0$ and 
$S = \rho\Hext$, giving $\Ub^{0,0}$.  The next part of the initialization process 
proceeds as follows.

\begin{enumerate}
\renewcommand{\theenumi}{{\bf \alph{enumi}}}
\renewcommand{\labelenumii}{\roman{enumii}.}

\item {\bf Do} {$\nu = 1,...,N_{\rm iters}^{S}$.}
  \begin{enumerate}

  \item Estimate $\Delta t^\nu$ using $\Ub^{0,\nu-1}$ and $w_0^{\nu-1}.$

  \item {\bf React State}$[ \rho^{\initp},(\rho h)^{\initp}, X_k^{\initp}, T^{\initp}, 
(\rho^{\initp} \Hext), p_0^{\initp}] \rightarrow [\rho^{\outp}, (\rho h)^{\outp}, 
X_k^{\outp}, T^{\outp}, (\rho \omegadot_k)^{0,\nu} ].$

  \item Compute $S^{0,\nu}$ from equation (\ref{eq:defineS}) 
        using $(\rho \omegadot_k)^{0,\nu}$ and the initial data.

  \item Compute $\overline{S^{0,\nu}} = {\mathrm{\bf Avg}} (S^{0,\nu}).$

  \item Compute $w_0^{\nu}$ as in step 1c using $\overline{S^{0,\nu}}$ and $\psi=0$.
        
  \item Project $\Ub^{0,\nu-1}$ using $\beta_0^0$ and 
        $(S^{0,\nu} - \overline{S^{0,\nu}})$ as the source term.  
        This yields $\Ub^{0,\nu}.$

  \end{enumerate}

  {\bf End do.}

  Define $S^0 = S^{0,N_{\rm iters}^S}$, $w_0^{-\myhalf} = w_0^{N_{\rm iters}^S}$, 
$\dt^0 = \Delta t^{N_{\rm iters}^S},$ and $\Ub^0 = \Ub^{0,N_{\rm iters}^S}.$

\end{enumerate}

Next, we need to construct approximations to $\etarho^{-\myhalf}, \psi^{-\myhalf}, S^1$,
and $\pi^{-\myhalf}$.  As initial approximations, set 
$\etarho^{-\myhalf}=0, \psi^{-\myhalf}=0, S^{1,0}=S^0$, and $\pi^{-\myhalf}=0.$
\begin{enumerate}
  \renewcommand{\theenumi}{{\bf \alph{enumi}}}
  \renewcommand{\labelenumii}{\roman{enumii}.}
  \addtocounter{enumi}{1}
  
\item {\bf Do} {$\nu = 1,...,N_{\rm iters}^{\pi}$.}
  
  \begin{enumerate}
  \item Perform steps 1--11 as described above, using 
    $S^{\myhalf,\star} = (S^0 + S^{1,\nu-1})/2$ in step 1 as described.
    The only other difference in the time advancement is that in step 11
    we define ${\bf V} = (\Ubt^{1,\star} - \Ubt^0)$ and solve
    \begin{equation}  L_\beta^\rho \phi =
      D \left ( \beta_0^{\myhalf} {\bf V} \right) - \beta_0^{\myhalf} \left[ \left(S^{1}-\overline{S^{1}}\right) - \left(S^{0}-\overline{S^{0}}\right) \right] \enskip . 
    \end{equation}
    (The motivation for this form of the projection in the initial pressure iterations
    is discussed in (cite almgren:bell:crutchfield).)
      We discard the new velocity resulting from this, but keep the new  
      value for $\pi^{\myhalf} = \pi^{-\myhalf} + (1 / \dt) \; \phi.$  
      These steps also yield new scalar data at time $\dt,$ which
      we discard,  and new values for $\etarho^{\myhalf}$ (step 8), $\psi^{\myhalf}$ (step 8), 
      $S^{1,\nu}$ (step 10), and $\pi^{\myhalf}$ (step 11), which we keep.
    \item Set $\pi^{-\myhalf} = \pi^{\myhalf}$, $\etarho^{-\myhalf} = \etarho^{\myhalf}$,
      and $\psi^{-\myhalf} = \psi^{\myhalf}$. 
    \end{enumerate}
    
    {\bf End do.}
    
    Finally, we define $S^1 = S^{1,N_{\rm iters}^\pi}.$
    
  \end{enumerate}
