\documentclass[11pt]{article} 

\tolerance=600

\usepackage{amsmath,color}

% Margins
\usepackage[lmargin=0.5in,rmargin=1.5in,tmargin=1.0in,bmargin=1.0in]{geometry}

\newcommand{\rhozero}{\rho_0}
\newcommand{\pizero}{\pi_0}
\newcommand{\pizeroone}{\pi_0^{(1)}}
\newcommand{\pizerotwo}{\pi_0^{(2)}}
\newcommand{\gammabar}{\overline{\Gamma}_1}

\newcommand{\nablab}{\mathbf{\nabla}}
\newcommand{\cdotb}{\mathbf{\cdot}}

\newcommand{\sfrac}[2]{\mathchoice
  {\kern0em\raise.5ex\hbox{\the\scriptfont0 #1}\kern-.15em/
   \kern-.15em\lower.25ex\hbox{\the\scriptfont0 #2}}
  {\kern0em\raise.5ex\hbox{\the\scriptfont0 #1}\kern-.15em/
   \kern-.15em\lower.25ex\hbox{\the\scriptfont0 #2}}
  {\kern0em\raise.5ex\hbox{\the\scriptscriptfont0 #1}\kern-.2em/
   \kern-.15em\lower.25ex\hbox{\the\scriptscriptfont0 #2}}
  {#1\!/#2}}

\newcommand{\half}{\sfrac{1}{2}}
\newcommand{\myhalf}{\sfrac{1}{2}}
\newcommand{\nph}{{n + \myhalf}}
\newcommand{\nmh}{{n - \myhalf}}

\newcommand{\Hext}{{H_{\rm ext}}}
\newcommand{\Hnuc}{{H_{\rm nuc}}}

\newcommand{\Sbar}{\overline{S}}

\newcommand{\inp}{\mathrm{in}}
\newcommand{\initp}{\mathrm{init}}
\newcommand{\outp}{\mathrm{out}}

\newcommand{\uadvone}{\widetilde{\mathbf{U}}^{\mathrm{ADV},(1)}}
\newcommand{\uadvtwo}{\widetilde{\mathbf{U}}^{\mathrm{ADV},(2)}}
\newcommand{\V}{\mathbf{V}}
\newcommand{\dt}{\Delta t}
\newcommand{\dr}{\Delta r}

\newcommand{\etarho}{\eta_{\rho}}

\newcommand{\ubold}{\mathbf{U}}
\newcommand{\ut}{\widetilde{\ubold}}

\newcommand{\omegadot}{\dot{\omega}}
\newcommand{\er}{\mathbf{e}_r}


\setlength{\marginparwidth}{1.0in}
\newcommand{\MarginPar}[1]{\marginpar{%
\vskip-\baselineskip %raise the marginpar a bit
\raggedright\tiny\sffamily
\hrule\smallskip{\color{red}#1}\par\smallskip\hrule}}

\title{Flowchart of Current Algorithm}

\begin{document}

\maketitle
\tableofcontents
\cleardoublepage

\section{Equations}
\begin{eqnarray}
\frac{dX_k}{dt} &=& \omegadot_k(\rho,X_k,T)\enskip, \label{eq:VODE1C} \\
\frac{dT}{dt} &=&\frac{1}{c_p} \left ( -\sum_k \xi_k  \omegadot_k  + \Hnuc \right )\enskip. \label{eq:tempreactC}
\end{eqnarray}
\begin{equation}
\label{eq:tempenthalpy}
\frac{DT}{Dt} = \frac{1}{\rho c_p} \left[ \left(1 - \rho h_p\right) \frac{D p}{D t}
 - \sum_k \rho \xi_k {\omegadot}_k 
 + \rho \Hnuc + \rho \Hext \right] \enskip , 
\end{equation}
\begin{eqnarray}
\frac{\partial \rhozero}{\partial t} &=& - \nablab \cdotb \left( \rhozero w_0 \er \right)
- \nablab \cdotb \left( \etarho \er \right) \enskip ,
\label{eq:rho0upd_new}
\end{eqnarray}
\begin{equation}
\frac{\partial p_0}{\partial t} = w_0 \rho_0 g  + \etarho g
                                 = - w_0 \frac{\partial p_0}{\partial r} + \etarho g \enskip .
\label{eq:p0upd_new} 
\end{equation}
\begin{equation}
\label{eq:constraint}
\nablab \cdotb \ubold + \alpha \left( \frac{\partial p_0}{\partial t} + \ubold \cdotb \nablab p_0 \right) =
 -\sigma  \sum_k \xi_k \omegadot_k  + 
  \frac{1}{\rho p_\rho} \sum_k p_{X_k}  {\omegadot}_k + \sigma \Hnuc + \sigma \Hext \equiv S \enskip ,
\end{equation}
\begin{eqnarray}
\frac{\partial w_0}{\partial r} &=& \frac{1}{\beta_0} \nablab \cdotb (\beta_0 w_0 \er) 
                                    - w_0 \frac{1}{\beta_0} \frac{\partial \beta_0}{\partial r}  \nonumber \\
                                &=&   \left( \Sbar - \frac{1}{\gammabar p_0}
                                      \frac{\partial p_0}{\partial t} \right)
                                     - w_0 \frac{1}{\beta_0} \frac{\partial \beta_0}{\partial r}  \nonumber \\
                                &=&   \Sbar - \frac{1}{\gammabar p_0}
                                      \left( \frac{\partial p_0}{\partial t}
                                       + w_0 \frac{\partial p_0}{\partial r}  \right) \nonumber \\
                                &=&   \Sbar - \frac{1}{\gammabar p_0} \etarho g \enskip \label{eq:divw0} ,
\end{eqnarray}
\begin{equation}
- \frac{1}{\rho_0} \frac{\partial \pizero}{\partial r} = \frac{\partial w_0}{\partial t} + w_0 \frac{\partial w_0}{\partial r} \enskip, \label{eq:pizero}
\end{equation}
\begin{equation}
\int_{\Omega_H} \ut \cdotb \er \; dA = 0 \label{eq:udoterzero} \enskip ,
\end{equation}
\begin{equation}
\nablab \cdotb (\beta_0 \ut )  = \beta_0 \left(S - \Sbar \right )
 \enskip .  \label{eq:tildeconstraint}
\end{equation}
\begin{equation}
w_0^{n,\star} = \frac{\dt^{n} w_0^{\nmh} + \dt^{n-1} w_0^{\nph,\star}}{\dt^n+\dt^{n-1}}\enskip,\label{eq:w0nstar}
\end{equation}
\begin{equation}
\left(\frac{\partial w_0}{\partial r}\right)^{n,\star} = \frac{1}{\dt^n+\dt^{n-1} } \left [ \dt^{n} \left(\frac{\partial w_0 }{ \partial r}\right)^{\nmh}
+ \dt^{n-1} \left(\frac{\partial w_0 }{ \partial r}\right)^{\nph,\star} \right ] \enskip.\label{eq:dw0drnstar}
\end{equation}
\begin{eqnarray}
\frac{\partial w_0}{\partial t} &=& - w_0 \frac{\partial w_0}{\partial r} - 
\frac{1}{\rho_0} \frac{\partial \pizero}{\partial r} \label{eq:w0upd} \enskip , \\
\frac{\partial\ut}{\partial t} &=& - \ut \cdotb \nablab \ut - w_0 \frac{\partial \ut}{\partial r}
                                 - \left(\ut \cdotb \er\right) \frac{\partial w_0}{\partial r} \er
                                 - \frac{1}{\rho} \nablab\pi
                                 + \frac{1}{\rho_0} \frac{\partial \pizero}{\partial r} \er
                                 - \frac{(\rho-\rhozero)}{\rho} \; g \; \er  \label{eq:utildeupd}  \enskip ,
\end{eqnarray}
\begin{equation}
 L_\beta^\rho \phi =
   D \left ( \beta_0^{\nph} {\bf V} \right) - \beta_0^{\nph} \left(S^{n+1}-\overline{S}^{n+1} \right)\enskip,
\label{eq:nodal_solve}
\end{equation}
\begin{equation}
  S =  -\sigma  \sum_k  \xi_k \omegadot_k  + 
  \frac{1}{\rho p_\rho} \sum_k p_{X_k}  {\omegadot}_k  + \sigma \Hnuc + \sigma \Hext \enskip .
\label{eq:defineS} 
\end{equation}

\section{Current Algorithm}
\subsection{Notation}

We make use of the following shorthand notations in outlining the algorithm.
\begin{enumerate}

\item For any quantity, $\phi,$ we define $\overline{\phi} = ${\bf Avg}$(\phi),$ 
the average over $\Omega_H,$ as
\begin{equation}
\overline{\phi}(r) = \frac{1}{\mathrm{A}(\Omega_H)}\int_{\Omega_H} \phi(r,{\bf x}) \; dA \enskip .
\end{equation}

\item {\bf React State}$(\rho^{\inp},(\rho h)^{\inp},X_k^{\inp},T^{\inp}, (\rho\Hext)^{\inp})\rightarrow$
$(\rho^{\outp}, (\rho h)^{\outp}, X_k^{\outp}, T^{\outp}, (\rho \omegadot_k)^{\outp})$ \newline
is the process by which we evolve the species and enthalpy from $X_k^{\inp} \rightarrow X_k^{\outp}$ 
and $(\rho h)^{\inp} \rightarrow (\rho h)^{\outp}$ by solving the following system of equations 
over a time interval of  $\Delta t/2$,
\begin{eqnarray}
\frac{\partial X_k}{\partial t} &=& \omegadot_k\enskip,\\
\frac{\partial (\rho h)}{\partial t} &=& \rho \Hnuc + \rho \Hext\enskip .
\end{eqnarray}
  In particular, to evolve the species, we solve the system:
\begin{eqnarray}
\frac{dX_k}{dt} &=& \omegadot_k(\rho,X_k,T)\enskip, \\
\frac{dT}{dt} &=&\frac{1}{c_p} \left ( -\sum_k \xi_k  \omegadot_k  + \Hnuc \right )\enskip.
\end{eqnarray}
using the stiff ordinary differential equation
 integration methods provided by the {\tt VODE} package (cite VODE).
The absolute error tolerances are set to $10^{-12}$ for the species,
 and a relative tolerance of $10^{-5}$ is used for the temperature.  
The integration yields the new values of the mass fractions, $X_k^{\outp}$.  
Equation (\ref{eq:tempreactC}) is derived from equation (\ref{eq:tempenthalpy}) by assuming that the pressure is constant during the burn state.  
In evolving these equations, we need to evaluate $c_p$ and $\xi_k$.  
In theory, this means evaluating the equation of state for each right-hand side evaluation that {\tt VODE} requires.  
In practice, we freeze $c_p$ and $\xi_k$ at the start of the integration time step and compute them using 
$\rho^{\inp}, X_k^{\inp},$ and $T^{\inp}$ as inputs to the equation of state.  
We note that while temperature is evolved when solving these equations, we do not keep the final temperature, 
nor do we use it to compute the final change in enthalpy.  
Therefore, $T^{\outp} = T^{\inp}$.  Also, note that the density remains unchanged during 
the {\bf React State} step, i.e., $\rho^{\outp} = \rho^{\inp}$.

After the new mass fractions have been computed, the reaction rates are defined as:
\begin{equation}
(\rho\omegadot_k)^{\outp} = \frac{\rho^{\outp} ( X_k^{\outp} - X_k^{\inp})}{\Delta t /2} \enskip,
\end{equation}
and the nuclear energy generation rate is defined as
\begin{equation}
(\rho\Hnuc) = -\sum_k(\rho\omegadot_k)^{\outp} \; q_k \enskip.
\end{equation}
The enthalpy update incorporates the external heating, $(\rho\Hext)^{\inp}$, and is updated by
\begin{equation}
(\rho h)^{\outp} = (\rho h)^{\inp} + \frac{\dt}{2} (\rho\Hnuc) + \frac{\dt}{2} (\rho\Hext)^{\inp}\enskip.
\end{equation}

\item {\bf Advect Base}$(\rho_0^{\inp}, p_0^{\inp}, \beta_0^{\inp}, w_0^{\inp}, \psi^{\inp} ) \rightarrow 
(\rho_0^{\outp}, \rho_0^{\outp,n+\myhalf}, p_0^{\outp})$ 
\newline
is the process by which we update the base state through $\dt$ in time given the
radial velocity $w_0^{\inp}.$ Here we discuss
the algorithm for plane-parallel geometries.  
The base state arrays are all one-dimensional 
in the radial coordinate; we think of the base state quantities as defined
at cell centers, with $w_0$ and $\beta_0$ defined halfway in-between.\\ \\
a)  For the density update, we discretize equation (\ref{eq:rho0upd_new}), neglecting the $\etarho$ term,
to compute the new base state density,
\begin{eqnarray}
\rho_{0,j}^{\outp} &=& \rho_{0,j}^{\inp} - \frac{\dt}{\dr} 
\left [ \left( {{\rho}_0}^{\outp,\nph} {w_0}^{\inp}\right)_{j+\myhalf} -  
         \left( {{\rho}_0}^{\outp,\nph} {w_0}^{\inp}\right)_{j-\myhalf} \right ] \nonumber  \enskip,
\end{eqnarray}
where $j$ refers to the one-dimensional index in the radial direction.
The interface states, ${{\rho}_0}^{\outp,\nph}$, are found via the procedure described in
paper II, Appendix A, and are saved for use later in the overall algorithm.\\ \\
b)  For the pressure update, We discretize equation (\ref{eq:p0upd_new}), 
to compute the new base state pressure,
\begin{eqnarray}
p_{0,j}^\outp &=& p_{0,j}^\inp - \frac{\dt}{2\dr} \left (w_{0,j+\myhalf}^\inp
+ w_{0,j-\myhalf}^\inp \right)
\left (p_{0,j+\myhalf}^{n+\myhalf} -p_{0,j-\myhalf}^{n+\myhalf} \right ) + \dt \; \psi_j^\inp \enskip,
\end{eqnarray}
where the interface states are again found via the procedure described in
paper II, Appendix A.
\\

\item {\bf Correct Base}$(\rho_0^{\inp} , \etarho^{\inp}) \rightarrow (\rho_0^{\outp})$
is the process by which we adjust the base state density given $\etarho$ using
\begin{eqnarray}
\rho_{0,j}^{\outp} &=& \rho_{0,j}^{\inp} - \frac{\dt}{\dr} 
\left( \eta_{\rho,j+\myhalf}^{\inp} - \eta_{\rho,j-\myhalf}^{\inp}  \right)  \enskip . 
\end{eqnarray} 
\end{enumerate}

\subsection{Time Advancement Algorithm}

We now describe the full time advancement algorithm, making frequent
use of the shorthand developed above.  Here, we assume that the
problem is already properly initialized.  We describe the details of
the initialization in \S~ref{Sec:Initialization}.

In paper II we discretized the density and enthalpy evolution equations in 
perturbational form, arguing that this would be less
susceptible to grid effects.  In the present algorithm, the analog would
be to use a perturbational form of the species equations.
Numerical testing has shown that the form of the algorithm presented
here is in fact more robust than the perturbational form 
when the base state mixing terms ($\etarho$) are included.  
In the algorithm described below, density is still 
extrapolated to edges in perturbational form, but 
we treat $X_k$ and $h$ in non-perturbational form.

%--------------------------------------------------------------------------
% STEP 1
%--------------------------------------------------------------------------
\noindent {\bf Step 1.} {\em Define the average expansion at time $t^\nph$ and the new $w_0.$}

\begin{enumerate}
\renewcommand{\theenumi}{{\bf \alph{enumi}}}

\item At the beginning of each time step, we need an estimate for the time-centered
source term in the velocity divergence constraint (see eq. [\ref{eq:constraint}]).
If this is the first step of the calculation ($n=0$), we set
\[
S^{\half,\star} = \frac{S^0 + S^1}{2} \enskip ,
\]
where $S^1$ is found through the iterative process that initializes the calculation.
Otherwise, following the method used in our small-scale low Mach number algorithm
(Cite Bell04), we extrapolate to the half-time using the source
terms at the previous and current time levels
\begin{equation}
S^{\nph,\star} = S^n + \frac{\Delta t^n}{2} \frac{S^n - S^{n-1}}{\Delta t^{n-1}} \enskip .
\end{equation}

\item Define
\[
\overline{S}^{\nph,\star} = {\mathrm{\bf Avg}} \left(S^{\nph,\star}\right) \enskip.
\]

\item Construct $w_0^{\nph,\star}$ by integrating equation (\ref{eq:divw0}) using 
the lagged $\psi^{\nmh},$
\begin{equation}
\frac{\partial w_0^{\nph,\star}}{\partial r} =  \overline{S}^{\nph,\star} 
- \frac{1}{\gammabar^{n} p_0^{n}} \psi^{\nmh} \enskip . \nonumber
\end{equation}
For spherical geometries, this equation would be modified.

\item Using equation (\ref{eq:pizero}), define the scaled pressure gradient
\begin{equation}
-\left ( \frac{1}{\rho_0^n} \frac{\partial \pizeroone}{\partial r} \right ) = 
\frac{w_0^{\nph,\star} - w_0^\nmh}{(\dt^n+\dt^{n-1})/2} 
+  w_0^{n,\star} \left(\frac{\partial w_0}{\partial r}\right)^{n,\star} \enskip ,
\end{equation} 
where $w_0^{n,\star}$ and $(\partial w_0 / \partial r)^{n,\star}$ are
\begin{equation}
w_0^{n,\star} = \frac{\dt^{n} w_0^{\nmh} + \dt^{n-1} w_0^{\nph,\star}}{\dt^n+\dt^{n-1}}\enskip,\end{equation}
\begin{equation}
\left(\frac{\partial w_0}{\partial r}\right)^{n,\star} = \frac{1}{\dt^n+\dt^{n-1} } \left [ \dt^{n} \left(\frac{\partial w_0 }{ \partial r}\right)^{\nmh}
+ \dt^{n-1} \left(\frac{\partial w_0 }{ \partial r}\right)^{\nph,\star} \right ] \enskip.\end{equation}
If $n=0$, we use $\dt^{-1} = \dt^0$.

\end{enumerate}

%--------------------------------------------------------------------------
% STEP 2
%--------------------------------------------------------------------------
\noindent {\bf Step 2.} {\em Construct the provisional edge-based advective velocity}, $\uadvone$: 

The procedure to construct $\uadvone$ is described in detail in Appendix B of paper III.
We note that  $\uadvone$ satisfies the discrete versions of equations
(\ref{eq:udoterzero}) and (\ref{eq:tildeconstraint}), 
in particular,
\begin{equation}
\nablab \cdotb \left(\beta_0^n \uadvone\right) = \beta_0^n \left(S^{\nph,\star} - \overline{S}^{\nph,\star}\right)
\enskip ,
\end{equation}

%--------------------------------------------------------------------------
% STEP 3
%--------------------------------------------------------------------------
\noindent {\bf Step 3.} {\em React the full state through the first time interval of $\dt / 2.$}

\begin{enumerate}
\renewcommand{\theenumi}{{\bf \alph{enumi}}}

\item {\bf React State}$(\rho^n, (\rho h)^n, X_k^n, T^n, (\rho^n \Hext))
                   \rightarrow$ $( \rho^{(1)}, (\rho h)^{(1)}, X_k^{(1)}, T^{(1)},
                                  (\rho \omegadot_k)^{(1)} ).$  

\item
Define
\begin{eqnarray}
\gammabar^{(1)}        &=& {\rm{\bf Avg}} \left( \Gamma_1(\rho^{(1)}, p_0^{n}, X_k^{(1)}) \right) 
                                                                \enskip , \nonumber \\
 {\beta   }_0^{(1)}    &=& \beta   \left(\rho_0^{n}, p_0^{n}, \gammabar^{(1)}\right) \enskip .
\end{eqnarray}

\end{enumerate}

%--------------------------------------------------------------------------
% STEP 4
%--------------------------------------------------------------------------
\noindent {\bf Step 4.} {\em Advect the base state, then the full state, through a time interval of $\dt.$}

\begin{enumerate}
\renewcommand{\theenumi}{{\bf \alph{enumi}}}

\item {\bf Advect Base}$(\rhozero^{n}, p_0^{n}, \beta_0^{(1)}, w_0^{\nph,\star}, \psi^{\nmh}) \rightarrow $
$(\rho_0^{(2),\star}, \rho_0^{\nph,\star,{\rm pred}}, p_0^{n+1,\star}).$ 

\item Compute the edge states, $(\rho X_k)^{(1),\nph,\star}$ and $(\rho h)^{(1),\nph,\star},$
      for the conservative update of $(\rho X_k)$ and $(\rho h).$  Here we
      predict $\rho'$, $T$, and $X_k$ to the edges, using
      a second-order Taylor expansion in space and time, as described in
      paper II, Appendix A, using $\V = \uadvone+w_0^{\nph,\star} \er$.  
      We explicitly include the reaction terms in the temperature prediction, 
      since we 
      did not update the temperature in {\bf React State}. 
      We use the equation of state and base state density, $\rho_0$,
      to convert these to edge states for $(\rho X_k)$ and $(\rho h)$. 
      We define $\rho^{(1),\nph,\star} = \sum_k (\rho X_k)^{(1),\nph,\star}.$

\item Evolve $(\rho X_k)^{(1)} \rightarrow (\rho X_k)^{(2),\star}$ and
$(\rho h)^{(1)} \rightarrow (\rho h)^{(2),\star}$ without explicitly including the 
reaction terms,
\begin{eqnarray}
(\rho X_k)^{(2),\star} &=& (\rho X_k)^{(1)} 
 - \dt \; \left[ \nablab \cdotb \left( (\uadvone+w_0^{\nph,\star} \er)  
  (\rho X_k)^{(1),\nph,\star} \right) \right] \enskip , \\ \nonumber  \\
(\rho h)^{(2),\star} &=& (\rho h)^{(1)} - \dt \; \left[ \nablab \cdotb \left( (\uadvone+w_0^{\nph,\star} \er)  
  (\rho h)^{(1),\nph,\star} \right) \right] \nonumber \\
&& + \dt \; \left(\uadvone \cdotb \er\right) \left(\frac{\partial p_0}{\partial r} \right)^{n} 
+ \dt \; \psi^{\nmh} \enskip , \\
\rho^{(2),\star} &=& \sum_k (\rho X_k)^{(2),\star} \enskip , \\ 
X_k^{(2),\star} &=& (\rho X_k)^{(2),\star} / \rho^{(2),\star} \enskip , \\
T^{(2),\star} &=& T\left(\rho^{(2),\star}, (\rho h)^{(2),\star},  X_k^{(2),\star}\right) {\hbox{ \rm using the equation of state.}} 
\end{eqnarray}

\item {\bf Correct Base}$(\rho_0^{(2),\star}, \etarho^{\nmh}) \rightarrow (\rho_0^{n+1,\star})$.

\item Define an edge-centered $\etarho^{\nph,\star}$ and cell-centered $\psi_j^{\nph,\star}$:
\begin{eqnarray}
 \etarho^{\nph,\star}   &=&  {\rm {\bf Avg}} \left( (\uadvone \cdotb \er + w_0^{\nph,\star}) 
\rho^{(1),\nph,\star} \right) - w_0^{\nph,\star} \rho_0^{\nph,\star,{\rm pred}} 
 \nonumber \enskip , \\
 \psi_j^{\nph,\star}  &=&  \frac{1}{2} \left(\eta_{\rho,j-\myhalf}^{\nph,\star} + \eta_{\rho,j+\myhalf}^{\nph,\star}\right) g \enskip .
\end{eqnarray}
For spherical geometries, $\psi_j^{\nph,\star}$ would have a different form. 

\end{enumerate}

%--------------------------------------------------------------------------
% STEP 5
%--------------------------------------------------------------------------
\noindent {\bf Step 5.} {\em React the full state through a second time interval of $\dt / 2.$}

\begin{enumerate}
\renewcommand{\theenumi}{{\bf \alph{enumi}}}

\item {\bf React State}$( \rho^{(2),\star},(\rho h)^{(2),\star}, X_k^{(2),\star}, 
                             T^{(2),\star}, (\rho^{(2),\star} \Hext)) $\\
$\rightarrow ( \rho^{n+1,\star},(\rho h)^{n+1,\star}, X_k^{n+1,\star}, T^{n+1,\star}, 
              (\rho \omegadot_k)^{(2),\star} ).$  

\item Define
\begin{eqnarray}
 \gammabar^{n+1,\star}    &=& {\rm{\bf Avg}} \left( \Gamma_1(\rho^{n+1,\star}, p_0^{n+1,\star}, 
                                                      X_k^{n+1,\star}) \right) \enskip , \nonumber \\
 {\beta   }_0^{n+1,\star}    &=& \beta   \left(\rho_0^{n+1,\star}, p_0^{n+1,\star}, \gammabar^{n+1,\star}\right) \enskip .
\end{eqnarray}

\end{enumerate}

%--------------------------------------------------------------------------
% STEP 6
%--------------------------------------------------------------------------
\noindent {\bf Step 6.} {\em Define a new average expansion rate at time $t^\nph.$}

\begin{enumerate}
\renewcommand{\theenumi}{{\bf \alph{enumi}}}
\item Define:
\begin{equation}
  S^{n+1,\star} =  -\sigma  \sum_k  \xi_k  (\omegadot_k)^{(2),\star}  + 
  \sigma \Hnuc^{(2),\star} +
  \frac{1}{\rho p_\rho} \sum_k p_{X_k}  ({\omegadot}_k)^{(2),\star}  
  + \sigma \Hext \enskip .
\end{equation} 
where $(\omegadot_k)^{(2),\star} = (\rho \omegadot_k)^{(2),\star} / \rho^{(2),\star}$
and the remaining quantities are defined using $X_k^{n+1,\star},$ $\rho^{n+1,\star},$ 
and $T^{n+1,\star}$ from step 5.
Then define
\[
 S^\nph = \frac{S^n + S^{n+1,\star}}{2} \enskip. 
\]

\item Define:
\[
\overline{S}^{\nph} = {\mathrm{\bf Avg}} (S^{\nph}) \enskip.
\]

\item Construct $w_0^{\nph}$ by integrating equation (\ref{eq:divw0}),
\begin{equation}
\frac{\partial w_0^{\nph}}{\partial r} =  \overline{S}^{\nph,\star} - \frac{1}{\gammabar^{\nph,\star} p_0^{\nph,\star}} \psi^{\nph,\star} \enskip , \nonumber
\end{equation}
for plane-parallel geometries, where 
$\gammabar^{\nph,\star} = (\gammabar^{n} + \gammabar^{n+1,\star} ) / 2 $ and
$ p_0^{\nph,\star} = (p_0^{n} + p_0^{n+1,\star} ) / 2.$

\item Using equation (\ref{eq:pizero}), define
\begin{equation}
-\left ( \frac{1}{\rho_0^n} \frac{\partial \pizerotwo}{\partial r} \right ) = 
\frac{w_0^{\nph} - w_0^\nmh}{\myhalf(\dt^n+\dt^{n-1})} 
+ w_0^n \left(\frac{\partial w_0}{\partial r}\right)^n.
\end{equation}
where $w_0^n$ and $(\partial w_0 / \partial r)^{n}$ are defined analogously to 
equations (\ref{eq:w0nstar}) and (\ref{eq:dw0drnstar}).

\end{enumerate}

%--------------------------------------------------------------------------
% STEP 7
%--------------------------------------------------------------------------
\noindent {\bf Step 7.} {\em Construct the final edge-based advective velocity}, $\uadvtwo$: 

The procedure to construct $\uadvtwo$ is described in detail in Appendix B of paper III
and is analogous to the procedure used in step 2, but with updated values
for $w_0$ and $\pizero.$ We note that  $\uadvtwo$ satisfies the discrete versions of 
equations (\ref{eq:udoterzero}) and (\ref{eq:tildeconstraint}),
in particular
\begin{equation}
\nablab \cdotb \left(\beta_0^{\nph,\star} \uadvtwo\right) = \beta_0^{\nph,\star} 
\left(S^{\nph} - \overline{S}^{\nph}\right) \enskip .
\end{equation}
where $ \beta_0^{\nph,\star} = ( \beta_0^n +  \beta_0^{n+1,\star}) / 2.$

%--------------------------------------------------------------------------
% STEP 8
%--------------------------------------------------------------------------
\noindent {\bf Step 8.} {\em Advect the base state, then the full state, through a time interval of $\dt.$}

\begin{enumerate}
\renewcommand{\theenumi}{{\bf \alph{enumi}}}

\item {\bf Advect Base}$(\rho_0^{n}, p_0^{n}, \beta_0^{(1)}, w_0^{\nph}, \psi^{\nph,\star} ) \rightarrow $
$(\rho_0^{(2)}, \rho_0^{\nph,{\rm pred}}, p_0^{n+1}).$ 

\item Compute the edge states, $(\rho X_k)^{(1),\nph}$ and $(\rho h)^{(1),\nph},$
      for the conservative update of $(\rho X_k)$ and $(\rho h).$  Here we
      predict $\rho'$, $T$, and $X_k$ to the edges, using
      a second-order Taylor expansion in space and time, as described in
      paper II, Appendix A, using $\V = \uadvtwo+w_0^{\nph} \er$. 
      Again, we explicitly include the reaction terms in the temperature prediction, 
      since we did not update  
      the temperature in {\bf React State}.
      We use the equation of state and base state density, $\rho_0$,
      to convert these to edge states for $(\rho X_k)$ and $(\rho h)$. 
      We define $\rho^{(1),\nph} = \sum_k (\rho X_k)^{(1),\nph}.$

\item Evolve $(\rho X_k)^{(1)} \rightarrow (\rho X_k)^{(2)}$ and
$(\rho h)^{(1)} \rightarrow (\rho h)^{(2)},$ 
\begin{eqnarray}
(\rho X_k)^{(2)} &=& (\rho X_k)^{(1)} 
- \dt \; \left[ \nablab \cdotb \left( (\uadvtwo+w_0^{\nph} \er)  
(\rho X_k)^{(1),\nph} \right) \right]  \enskip , \\ \nonumber \\
(\rho h)^{(2)} &=& (\rho h)^{(1)} - \dt \; \left[ \nablab \cdotb \left( (\uadvtwo+w_0^{\nph} \er)  
(\rho h)^{(1),\nph} \right) \right] \nonumber \\
&& + \frac{\dt}{2} \; \left(\uadvtwo \cdotb \er\right)
\left [ \left(\frac{\partial p_0}{\partial r} \right)^{n}
      + \left(\frac{\partial p_0}{\partial r} \right)^{n+1}  \right ] 
+ \dt \; \psi^{\nph,\star} \enskip , \\ 
\rho^{(2)} &=& \sum_k (\rho X_k)^{(2)} \enskip , \\
X_k^{(2)} &=& (\rho X_k)^{(2)} / \rho^{(2)} \enskip , \\
T^{(2)} &=& T\left(\rho^{(2)}, (\rho h)^{(2)},  X_k^{(2)}\right) {\hbox{ \rm using the
equation of state.}} 
\end{eqnarray}

\item {\bf Correct Base}$(\rho_0^{(2)}, \etarho^{\nph, \star}) \rightarrow (\rho_0^{n+1})$

\item Define an edge-centered $ \etarho^{\nph}$ and a cell-centered $\psi_j^{\nph}$,
\begin{eqnarray}
 \etarho^{\nph}   &=&  {\rm {\bf Avg}} \left( (\uadvtwo \cdotb \er + w_0^{\nph}) 
\rho^{(1),\nph} \right) - w_0^{\nph} \rho_0^{\nph,{\rm pred}} 
 \nonumber \enskip , \\
 \psi_j^{\nph}  &=&  \frac{1}{2} \left(\eta_{rho,j-\myhalf}^{\nph} + \eta_{\rho,j+\myhalf}^{\nph}\right) g \enskip .
\end{eqnarray}
\end{enumerate}

%--------------------------------------------------------------------------
% STEP 9
%--------------------------------------------------------------------------
\noindent {\bf Step 9.} {\em React the full state through a second time interval of $\dt / 2.$}

\begin{enumerate}
\renewcommand{\theenumi}{{\bf \alph{enumi}}}

\item {\bf React State}$(\rho^{(2)},(\rho h)^{(2)}, X_k^{(2)},T^{(2)}, (\rho^{(2)} \Hext) )
                   \rightarrow (\rho^{n+1}, (\rho h)^{n+1}, X_k^{n+1}, T^{n+1}, 
                               (\rho \omegadot_k)^{(2)} ).$  

\item Define
\begin{eqnarray}
 \gammabar^{n+1}    &=& {\rm{\bf Avg}}\left(\Gamma_1(\rho^{n+1}, p_0^{n+1}, {X_k}^{n+1}) \right) \enskip , \nonumber \\ 
 {\beta   }_0^{n+1}    &=& \beta   \left(\rho_0^{n+1}, p_0^{n+1},   \gammabar^{n+1}\right) \enskip .
\end{eqnarray}

\end{enumerate}

%--------------------------------------------------------------------------
% STEP 10
%--------------------------------------------------------------------------
\noindent {\bf Step 10.} {\em Compute $S^{n+1}$ for the final projection.}

\begin{enumerate}
\renewcommand{\theenumi}{{\bf \alph{enumi}}}
\item Define:
\begin{equation}
  S^{n+1} =  -\sigma  \sum_k  \xi_k (\omegadot_k)^{(2)}  + \sigma \Hnuc^{(2)} +
  \frac{1}{\rho p_\rho} \sum_k p_{X_k}  ({\omegadot}_k)^{(2)}  
   + \sigma \Hext \enskip ,
\end{equation}
where $(\omegadot_k)^{(2)} = (\rho \omegadot_k)^{(2)} / \rho^{(2)}$
and the remaining quantities are defined using $X_k^{n+1},$ $\rho^{n+1},$
and $T^{n+1}$ from step 9.

\item Define
\[
\overline{S}^{n+1} = {\mathrm{\bf Avg}} (S^{n+1}) \enskip.
\]

\end{enumerate}


%--------------------------------------------------------------------------
% STEP 11
%--------------------------------------------------------------------------
\noindent {\bf Step 11.} {\em Update the velocity}.  

The velocity update happens analogously to paper II, using $S^{n+1}$ from step 10.
We update the velocity field $\ut^n$ to $\ut^{n+1,\star}$ by discretizing
equation (\ref{eq:utildeupd}), 
\begin{eqnarray}
\ut^{n+1,\star} &=& \ut^n - \dt \;
 \left[\left((\uadvtwo+ w_0^{\nph} \er) \cdotb \nablab\right) \ut \right]^\nph
      - \dt \; \left(\uadvtwo \cdotb \er\right)  \left(\frac{\partial w_0^{\nph}}{\partial r} \right) \er \nonumber \\
   &&   + \dt \left[ - \frac{1}{\rho^\nph} {\mathbf{G}} \pi^\nmh
        + \frac{1}{\rho_0} {\mathbf{G}} \pizerotwo
        - \frac{(\rho^\nph-\rhozero^\nph)}{\rho^\nph} g \er \right] \enskip ,
\end{eqnarray}
where $\rho^\nph = ( \rho^n + \rho^{n+1} )/2$
and $\mathbf{G}$ approximates a cell-centered gradient from nodal data. 
The construction of $\left[((\uadvtwo+ w_0^{\nph} \er) \cdotb \nablab) \ut \right]^\nph$
is described in paper II, Appendix A, with $\V = \uadvtwo+w_0^{\nph} \er$ and~$s$ set to each
component of $\ut^n$ individually.

Finally, we impose the divergence constraint from equation (\ref{eq:tildeconstraint}),
\[
\nablab \cdotb \left(\beta_0^{\nph} \ut^{n+1} \right)  = \beta_0^{\nph} \left(S^{n+1} - \overline{S}^{n+1} \right)\enskip,
\]
by defining ${\bf V} = \ut^{n+1,\star} + ( \dt / \rho^\nph ) \; \mathbf{G} \pi^\nmh$ and solving
\begin{equation}
 L_\beta^\rho \phi =
   D \left ( \beta_0^{\nph} {\bf V} \right) - \beta_0^{\nph} \left(S^{n+1}-\overline{S}^{n+1} \right)\enskip,
\end{equation}
for nodal values of $\phi$, where
$\beta_0^\nph = ( \beta_0^n + \beta_0^{n+1} )/2$ and
$L_\beta^\rho$ is the standard bilinear
finite element approximation to $\nablab \cdotb ({\beta_0^\nph}/{\rho^\nph}) \nablab.$
In this step, $D$ is a discrete second-order operator that approximates the 
divergence at nodes from cell-centered data and satisfies
$\mathbf{G} = -D^T.$ 
(See (cite almgrenBellSzymczak:1996) for a detailed discussion of this
approximate projection; see (cite almgren:bell:crutchfield) for a discussion
of this particular form of the projection operand.)  
We solve the linear system of equation (\ref{eq:nodal_solve}) using multigrid V-cycles with Gauss-Seidel relaxation.

We determine the new time velocity field from
\begin{equation}
\ut^{n+1} = {\bf V} - \frac{1}{\rho^\nph} \mathbf{G} \phi \enskip ,
\end{equation}
and the new time-centered perturbational pressure from:
\[
  \pi^\nph = \frac{1}{\dt} \phi \enskip .
\]

%--------------------------------------------------------------------------
% STEP 12
%--------------------------------------------------------------------------
\noindent {\bf Step 12.} {\em Compute a new $\dt.$}

Compute $\dt$ for the next time step with the procedure described in \S~ref{sec:timesteps},
using $w_0$ as computed in step 6 and
$\ut^{n+1}$ as computed in step 11.  We use this $\dt$ in the next time step. 

\noindent This completes the time advancement of the algorithm.

\subsection{Initialization}

We start each calculation with user-specified initial values for
$\rho$, $X_k$ and $T,$ as well as an initial background state.  In
order for the low Mach number assumption to hold, the initial data
must be thermodynamically consistent with the initial background
state.  In addition, the initial velocity field must satisfy an
initial approximation to the divergence constraint.  We use an iterative
procedure to compute both an initial right-hand-side for the
constraint equation and an initial velocity field that satisfies
the constraint.  The user specifies the number of iterations,
$N_{\rm iters}^{S},$ in this first step of the initialization procedure.

The initial perturbational pressure also needs to be determined for
use in steps 2, 7 and 11. 
This is done through a second iterative procedure which follows the
time advancement algorithm as described in steps 1-11 in \S~ref{Sec:Alg}.  
The user specifies the number of iterations, 
$N_{\rm iters}^{\pi},$ in this second step of the initialization procedure.
The details for both iterations are given below.

%--------------------------------------------------------------------------
% STEP 0
%--------------------------------------------------------------------------
\noindent {\bf Step 0} {\em Initialization}

Start with initial data $X_k^{\initp}, \rho^{\initp},$ $T^{\initp},$ an 
initial base state, and an initial guess for the velocity, $\ubold^{\initp}.$
Use the equation of state to determine $(\rho h)^{\initp}$.  Set
$w_0^1 = 0$ as an initial approximation.  Compute $\beta_0^{\initp}$ as a function of 
the initial data.  Then, project $\ubold^{\initp}$ using $\beta_0^{\initp}$ and 
$S = \rho\Hext$, giving $\ubold^{0,1}$.  The next part of the initialization process 
proceeds as follows.

\begin{enumerate}
\renewcommand{\theenumi}{{\bf \alph{enumi}}}
\renewcommand{\labelenumii}{\roman{enumii}.}

\item {\bf Do} {$\nu = 1,...,N_{\rm iters}^{S}$.}
  \begin{enumerate}

  \item Estimate $\Delta t^\nu$ using $\ubold^{0,\nu}$ and $w_0^\nu.$

  \item {\bf React State}$( \rho^{\initp},(\rho h)^{\initp}, X_k^{\initp}, T^{\initp}, (\rho^{\initp} \Hext) )$\\
      $\rightarrow (\rho^{\outp}, (\rho h)^{\outp}, X_k^{\outp}, T^{\outp}, (\rho \omegadot_k)^{0,\nu} ).$

  \item Compute $S^{0,\nu}$ from equation (\ref{eq:defineS}) 
        using $(\rho \omegadot_k)^{0,\nu}$ and the initial data.

  \item Compute $\overline{S}^{0,\nu} = {\mathrm{\bf Avg}} (S^{0,\nu}).$

  \item Compute $w_0^{\nu+1}$ as in step 1c using $\overline{S}^{0,\nu}$ and $\psi^{\nmh} = 0$.
        

  \item Project $\ubold^{0,\nu}$ using $\beta_0^{\initp}$ and 
        $(S^{0,\nu} - \overline{S}^{0,\nu})$ as the source term.  
        This yields $\ubold^{0,\nu+1}.$

  \end{enumerate}

  {\bf End do.}

  Define $S^0 = S^{0,N_{\rm iters}^S}$, $w_0^{-\myhalf} = w_0^{N_{\rm iters}^S+1}$, 
$\dt^0 = \Delta t^{N_{\rm iters}^S},$ and $\ubold^0 = \ubold^{0,N_{\rm iters}^S+1}.$

\end{enumerate}

Next, we need to construct an approximation to the time-centered perturbational pressure,
$\pi,$  and an approximation to the divergence constraint at the end of the
first time step.  As initial approximations, set $S^{1,0} = S^0$,
$\etarho^{-\half} = 0$, $\psi^{-\half} = 0$, and $\pi^{-\half} = 0.$
\begin{enumerate}
  \renewcommand{\theenumi}{{\bf \alph{enumi}}}
  \renewcommand{\labelenumii}{\roman{enumii}.}
  \addtocounter{enumi}{1}
  
\item {\bf Do} {$\nu = 0,...,N_{\rm iters}^{\pi}-1$.}
  
  \begin{enumerate}
  \item Perform steps 1--11 as described above, using 
    $S^{\half,\star} = (S^0 + S^{1,\nu})/2$ in step 1 as described.
    The only other difference in the time advancement is that in step 11
    we define ${\bf V} = (\ut^{1,\star} - \ut^0)$ and solve
    \begin{equation}  L_\beta^\rho \phi =
      D \left ( \beta_0^{\half} {\bf V} \right) - \beta_0^{\half} \left( (S^{1}-\overline{S}^{1}) - (S^{0}-\overline{S}^{0}) \right) \enskip . 
    \end{equation}
    (The motivation for this form of the projection in the initial pressure iterations
    is discussed in (cite almgren:bell:crutchfield).)
      We discard the new velocity resulting from this, but keep the new  
      value for $\pi^{\half} = \pi^{-\half} + (1 / \dt) \; \phi.$  
      These steps also yield new scalar data at time $\dt,$ which
      we discard,  and new values for $\etarho^{\half}$ (step 8), $\psi^{\half}$ (step 8), 
      $S^{1,\nu+1}$ (step 10), and $\pi^{\half}$ (step 11), which we keep.
    \item Set $\pi^{-\half} = \pi^{\half}$, $\etarho^{-\half} = \etarho^{\half}$,
      and $\psi^{-\half} = \psi^{\half}$. 
    \end{enumerate}
    
    {\bf End do.}
    
    Finally, we define $S^1 = S^{1,N_{\rm iters}^\pi}.$
    
  \end{enumerate}

\end{document}
