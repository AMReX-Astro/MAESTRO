\documentclass[11pt]{article} 

\tolerance=600

\usepackage{amsmath,color}

% Margins
\usepackage[lmargin=0.5in,rmargin=0.5in,tmargin=0.75in,bmargin=0.75in]{geometry}

\def\half  {\frac{1}{2}}
\def\dt    {\Delta t}
\def\kth   {k_{\rm th}}

\def\edge  {\rm EDGE}
\def\mac   {\rm MAC}
\def\trans {\rm TRANS}

\def\eb    {{\bf e}}
\def\fb    {{\bf f}}
\def\ib    {{\bf i}}
\def\Ub    {{\bf U}}

\def\Ubt   {\widetilde{\bf U}}
\def\ut    {\tilde{u}}
\def\vt    {\tilde{v}}
\def\wt    {\tilde{w}}

\title{Notes on the low density parameters in {\tt MAESTRO}}

\setlength{\marginparwidth}{0.75in}
\newcommand{\MarginPar}[1]{\marginpar{%
\vskip-\baselineskip %raise the marginpar a bit
\raggedright\tiny\sffamily
\hrule\smallskip{\color{red}#1}\par\smallskip\hrule}}

\begin{document}

\maketitle
\tableofcontents

These are working notes for the low density parameters in {\tt MAESTRO}.
\section{anelastic\_cutoff}
\begin{itemize}
\item In {\tt base\_state}, we abort the program if {\tt anelastic\_cutoff} $<$ {\tt base\_cutoff\_density}.
\item After we initilize the base state in {\tt varden} and at the beginning of {\tt advance}, We compute the coordinate associated with {\tt anelastic\_cutoff}, $r_{\rm anel}$.  This is done by setting $r_{\rm anel} = i$ as soon as $\rho_0(i) <$ {\tt anelastic\_cutoff}.
\item In {\tt correct\_base}, we only update $\rho_0$ for $r\in[0,r_{\rm anel}-1]$.
\item In {\tt make\_div\_coeff}, we change how we compute {\tt div\_coeff} at and above $r_{\rm anel}$.
\item In {\tt make\_psi}, we only compute $\psi$ for $r\in[0,r_{\rm anel}-1]$.
\item in {\tt make\_S}, we set {\tt delta\_gamma1\_term} and {\tt delta\_gamma1} to zero at and above $r_{\rm anel}$.
\end{itemize}
\section{base\_cutoff\_density}
\begin{itemize}
\item In {\tt base\_state}, we abort the program if {\tt anelastic\_cutoff} $<$ {\tt base\_cutoff\_density}.
\item In {\tt base\_state}, we compute a physical cutoff location, {\tt r\_cutoff\_loc}, which is defined as the physical location of the first cell-center at the coarsest level for which $\rho_0 <$ {\tt base\_cutoff\_density}.
\item In {\tt base\_state}, when we are generating the initial background/base state, if we are above {\tt r\_cutoff\_loc}, just use the values for $\rho,T$, and $p$ at {\tt r\_cutoff\_loc}.
\item In {\tt make\_hgrhs}, we only add the volume discrepancy if $\rho_0 >$ {\tt base\_cutoff\_density}.
\item In {\tt make\_macrhs}, we only add the volume discrepancy if $\rho_0 >$ {\tt base\_cutoff\_density}.
\item In {\tt mkscalforce} subroutine {\tt mkrhohforce}, once $\rho_0 \le$ {\tt base\_cutoff\_density}, we compute $\nabla p_0$ with a difference stencil instead of simply setting it to $\rho_0 g$.
\item In {\tt rhoh\_vs\_t} function {\tt makeRhoHfromP}, the base state pressure at edges is computed differently based on where we are in relation to the {\tt base\_cutoff\_density}.  If we are at a equal or higher density than {\tt base\_cutoff\_density}, we integrate $\rho g$ from the nearest cell-centered $p_0$ value.  If we are at a lower density than {\tt base\_cutoff\_density}, we just use the cell-centered value.
\item In {\tt update\_scal}, if the new $\rho$ is less than {\tt base\_cutoff\_density}, we call the EOS to compute $\rho h$.
\end{itemize}
\section{burning\_cutoff}
\begin{itemize}
\item In {\tt react\_state}, we only call the burner if $\rho >$ {\tt burning\_cutoff\_density}.
\end{itemize}
\end{document}
