These are working notes for the low density parameters in {\tt MAESTRO}.

\section{Computing the Cutoff Values}

At several points in the algorithm, we compute {\tt anelastic\_cutoff\_coord}, 
{\tt base\_cutoff\_density\_coord}, and {\tt burning\_cutoff\_density\_coord} by 
setting the corresponding coordinate equal to $r$ as soon as $\rho_0(r)$ is less than 
that particular cutoff value.  We set the coordinates at the following points in the
code:

\begin{itemize}

\item After we initialize the base state in {\tt varden}.

\item At the beginning of {\tt advance\_timestep}.

\item After each call to {\tt advect\_base\_dens}.

\item After each call to {\tt correct\_base} in spherical 
  (or call to {\tt average} in plane-parallel.

\end{itemize}

\section{anelastic\_cutoff}

\begin{itemize}

\item In {\tt init\_base\_state}, we abort the program if 
  {\tt anelastic\_cutoff} $<$ {\tt base\_cutoff\_density}.

\item In {\tt make\_div\_coeff}, we change how we compute {\tt div\_coeff} 
  for $r\in[{\tt anelastic\_cutoff\_coord},{\tt r\_end\_coord}]$.

\item In {\tt make\_psi} for plane-parallel, we only compute $\psi$ for 
  $r\in[{\tt r\_start\_coord},{\tt anelastic\_cutoff\_coord}-1]$.

\item in {\tt make\_S}, we set {\tt delta\_gamma1\_term} and {\tt delta\_gamma1} 
  to zero for $r\in[{\tt anelastic\_cutoff\_coord},{\tt r\_end\_coord}]$.

\item In {\tt sponge}, {\tt anelastic\_cutoff} is used in a problem
  dependent way.

\item In {\tt probin}, {\tt anelastic\_cutoff} is set to $3\times 10^6$
  by default.

\end{itemize}

\section{base\_cutoff\_density}

\begin{itemize}

\item In {\tt probin}, {\tt base\_cutoff\_density} is set to $3\times 10^6$
  by default.

\item In {\tt base\_state}, we abort the program if {\tt
  anelastic\_cutoff} $<$ {\tt base\_cutoff\_density}.

\item In {\tt base\_state}, we compute a physical cutoff location,
  {\tt base\_cutoff\_density\_loc}, which is defined as the physical
  location of the first cell-center at the coarsest level for which
  $\rho_0 <$ {\tt base\_cutoff\_density}.

\item In {\tt base\_state}, when we are generating the initial
  background/base state, if we are above {\tt
    base\_cutoff\_density\_loc}, just use the values for $\rho,T$, and
  $p$ at {\tt base\_cutoff\_density\_loc}.

\item In {\tt make\_hgrhs}, we only add the volume discrepancy for
  $r\in[{\tt r\_start\_coord},{\tt base\_cutoff\_density\_coord}-1]$.

\item In {\tt make\_macrhs}, we only add the volume discrepancy for
  $r\in[{\tt r\_start\_coord},{\tt base\_cutoff\_density\_coord}-1]$.

\item In {\tt make\_w0}, we only add the volume discrepancy for
  $r\in[{\tt r\_start\_coord},{\tt base\_cutoff\_density\_coord}-1]$.

\item In {\tt mkscalforce} subroutine {\tt mkrhohforce}, for
  $r\in[{\tt base\_cutoff\_density\_coord},{\tt r\_end\_coord}]$ we
  compute $\nabla p_0$ with a difference stencil instead of simply
  setting it to $\rho_0 g$.

\item In {\tt makeRhoHfromP}, the base state pressure at edges is
  computed differently based on where we are in relation to the {\tt
    base\_cutoff\_density}.  For $r\in[{\tt r\_start\_coord},{\tt
      base\_cutoff\_density\_coord}-1]$ we integrate $\rho g$ from the
  nearest cell-centered $p_0$ value.  Otherwise, we just use the
  cell-centered value above the edge.

\item In {\tt update\_scal}, if the new $\rho$ is less than {\tt
  base\_cutoff\_density}, we call the EOS to compute $\rho h$.

\item In {\tt make\_grav}, if we are spherical, we do not add to the
  enclosed mass, $M(r)$ if $\rho$ is less than {\tt base\_cutoff\_density}.

\item In {\tt enforce\_HSE}, we set $p_0(r) = p_0(r-1)$ for 
  $r\in[{\tt base\_cutoff\_density\_coord}-1,{\tt r\_end\_coord}]$

\end{itemize}

\section{burning\_cutoff}

\begin{itemize}

\item In {\tt react\_state}, we only call the burner if 
  $\rho >$ {\tt burning\_cutoff\_density}.

\end{itemize}

