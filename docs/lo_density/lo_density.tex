These are working notes for the low density parameters in {\tt MAESTRO}.

\section{Computing the Cutoff Values}

At several points in the algorithm, we compute {\tt anelastic\_cutoff\_coord}, 
{\tt base\_cutoff\_density\_coord}, and {\tt burning\_cutoff\_density\_coord} by 
setting the corresponding coordinate equal to $r$ as soon as $\rho_0(r)$ is less than 
or equal to that particular cutoff value.  We compute the cutoff coordinate at the
coarsest level, and then set the coordinate at the next finer level to be twice that
of the coarser level.  We set the coordinates at the following points in the code:

\begin{itemize}

\item After we call {\tt initialize} in {\tt varden}.

\item At the beginning of {\tt advance\_timestep}.

\item After each call to {\tt advect\_base\_dens}.

\item After we ``{\tt correct base}'' (by setting $\rho_0 = \overline{\rho}$ in
  plane-parallel and $\rho_0 = \rho_0 + {\bf{Avg}}(\rho - \rho_0)$ in spherical).

\item At the beginning of the second-half of the algorithm (Step 6), we reset
  the coordinates to the base-time values using $\rho_0^n$.

\end{itemize}

\section{anelastic\_cutoff}\label{Sec:Anelastic Cutoff}

\begin{itemize}

\item In {\tt probin}, {\tt anelastic\_cutoff} is set to $3\times 10^6$ by default.

\item In {\tt init\_base\_state}, we give a warning if {\tt anelastic\_cutoff}
  is smaller than the minimum model density.  Then, if ${\tt anelastic\_cutoff} 
  < {\tt base\_cutoff\_density}$, we abort the program.

\item In {\tt make\_div\_coeff}, for 
  $r \ge {\tt anelastic\_cutoff\_coord}$, we set
  ${\tt div\_coeff}(n,r) = {\tt div\_coeff}(n,r-1) * \rho_0(n,r)/\rho_0(n,r-1)$.
  \MarginPar{I think we want $r > {\tt anelastic\_cutoff\_coord}$.}

\item In {\tt make\_psi} for plane-parallel, we only compute $\psi$ for 
  $r < {\tt anelastic\_cutoff\_coord}$.
  \MarginPar{I think we want to change this to $r \le {\tt anelastic\_cutoff\_coord}$.}

\item in {\tt make\_S}, we set {\tt delta\_gamma1\_term} and {\tt delta\_gamma1} 
  to zero for $r \ge {\tt anelastic\_cutoff\_coord}$.
  \MarginPar{I think we want $r > {\tt anelastic\_cutoff\_coord}$.}

\item In {\tt sponge}, {\tt anelastic\_cutoff} is used in a problem
  dependent way.

\end{itemize}

\section{base\_cutoff\_density}\label{Sec:Base Cutoff Density}

\begin{itemize}

\item In {\tt probin}, {\tt base\_cutoff\_density} is set to $3\times 10^6$ by default.

\item In {\tt init\_base\_state}, we give a warning of {\tt base\_cutoff\_density}
  is smaller than the minimum model density.  Then, if {\tt anelastic\_cutoff} 
  $<$ {\tt base\_cutoff\_density} we abort the program.

\item In {\tt base\_state}, we compute a physical cutoff location,
  {\tt base\_cutoff\_density\_loc}, which is defined as the physical
  location of the first cell-center at the coarsest level for which
  $\rho_0 \le {\tt base\_cutoff\_density}$.  This is a trick used for making
  the data consistent for multiple level problems.  When we are generating the 
  initial background/base state, if we are above {\tt base\_cutoff\_density\_loc}, 
  just use the values for $\rho,T$, and $p$ at {\tt base\_cutoff\_density\_loc}.
  When we check whether we are in HSE, we use {\tt base\_cutoff\_density\_loc}.

\item In {\tt make\_hgrhs}, {\tt make\_macrhs}, and {\tt make\_w0}, 
  we only add the volume discrepancy for $r < {\tt base\_cutoff\_density\_coord}$
  (in plane parallel) and if $\rho_0 > {\tt base\_cutoff\_density}$ (in spherical).
  \MarginPar{Revisit this.}

\item In {\tt mkrhohforce} for plane-parallel, for
  $r \ge {\tt base\_cutoff\_density\_coord}$, we
  compute $\nabla p_0$ with a difference stencil instead of simply
  setting it to $\rho_0 g$.
  \MarginPar{I think we want $r > {\tt base\_cutoff\_density\_coord}$.}

\item In {\tt update\_scal}, if the new $\rho$ is less than 
  {\tt base\_cutoff\_density} and {\tt do\_eos\_h\_above\_cutoff},  
  we call the EOS to compute $h$.

\item In {\tt make\_grav} for spherical, we only add the enclosed mass if
  $\rho_0 > {\tt base\_cutoff\_density}$.

\item In {\tt enforce\_HSE}, we set $p_0(r) = p_0(r-1)$ for 
  $r > {\tt base\_cutoff\_density\_coord}$.

\end{itemize}

\section{burning\_cutoff}

\begin{itemize}

\item In {\tt probin}, {\tt burning\_cutoff\_density} is set to 
  {\tt base\_cutoff\_density}.  There is no option to set 
  {\tt burning\_cutoff\_density} using the inputs file.

\item In {\tt react\_state}, we only call the burner if 
  $\rho >$ {\tt burning\_cutoff\_density}.

\end{itemize}

