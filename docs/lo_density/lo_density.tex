\section{Notes on the low density parameters in {\tt MAESTRO}}

These are working notes for the low density parameters in {\tt MAESTRO}.

\subsection{anelastic\_cutoff}

\begin{itemize}

\item In {\tt init\_base\_state}, we abort the program if {\tt
  anelastic\_cutoff} $<$ {\tt base\_cutoff\_density}.

\item After we initilize the base state in {\tt varden} and at the
  beginning of {\tt advance\_timestep}, We compute the coordinate
  associated with {\tt anelastic\_cutoff}, {\tt
    anelastic\_cutoff\_coord}.  This is done by setting ${\tt
    anelastic\_cutoff\_coord}= r$ as soon as $\rho_0(r) <$ {\tt
    anelastic\_cutoff}.

\item In {\tt correct\_base}, we only update $\rho_0$ for $r\in[{\tt
    r\_start\_coord},{\tt anelastic\_cutoff\_coord}-1]$.

\item In {\tt make\_div\_coeff}, we change how we compute {\tt
  div\_coeff} for $r\in[{\tt anelastic\_cutoff\_coord},{\tt
    r\_end\_coord}]$.

\item In {\tt make\_psi}, we only compute $\psi$ for $r\in[{\tt
    r\_start\_coord},{\tt anelastic\_cutoff\_coord}-1]$.

\item in {\tt make\_S}, we set {\tt delta\_gamma1\_term} and {\tt
  delta\_gamma1} to zero for $r\in[{\tt anelastic\_cutoff\_coord},{\tt
    r\_end\_coord}]$.
\end{itemize}

\subsection{base\_cutoff\_density}

\begin{itemize}

\item In {\tt base\_state}, we abort the program if {\tt
  anelastic\_cutoff} $<$ {\tt base\_cutoff\_density}.

\item In {\tt base\_state}, we compute a physical cutoff location,
  {\tt base\_cutoff\_density\_loc}, which is defined as the physical
  location of the first cell-center at the coarsest level for which
  $\rho_0 <$ {\tt base\_cutoff\_density}.

\item In {\tt base\_state}, when we are generating the initial
  background/base state, if we are above {\tt
    base\_cutoff\_density\_loc}, just use the values for $\rho,T$, and
  $p$ at {\tt base\_cutoff\_density\_loc}.

\item After we initilize the base state in {\tt varden} and at the
  beginning of {\tt advance\_timestep}, We compute the coordinate
  associated with {\tt base\_cutoff\_density}, {\tt
    base\_cutoff\_density\_coord}.  This is done by setting ${\tt
    base\_cutoff\_density\_coord}= r$ as soon as $\rho_0(r) <$ {\tt
    base\_cutoff\_density}.

\item In {\tt make\_hgrhs}, we only add the volume discrepancy for
  $r\in[{\tt r\_start\_coord},{\tt anelastic\_cutoff\_coord}-1]$.

\item In {\tt make\_macrhs}, we only add the volume discrepancy for
  $r\in[{\tt r\_start\_coord},{\tt anelastic\_cutoff\_coord}-1]$.

\item In {\tt mkscalforce} subroutine {\tt mkrhohforce}, for
  $r\in[{\tt base\_cutoff\_density\_coord},{\tt r\_end\_coord}]$ we
  compute $\nabla p_0$ with a difference stencil instead of simply
  setting it to $\rho_0 g$.

\item In {\tt makeRhoHfromP}, the base state pressure at edges is
  computed differently based on where we are in relation to the {\tt
    base\_cutoff\_density}.  For $r\in[{\tt r\_start\_coord},{\tt
      base\_cutoff\_density\_coord}-1]$ we integrate $\rho g$ from the
  nearest cell-centered $p_0$ value.  Otherwise, we just use the
  cell-centered value above the edge.

\item In {\tt update\_scal}, if the new $\rho$ is less than {\tt
  base\_cutoff\_density}, we call the EOS to compute $\rho h$.

\item In {\tt make\_grav}, if we are spherical, we do not add to the
  enclosed mass, $M(r)$ if $\rho$ is less than {\tt base\_cutoff\_density}.
\end{itemize}

\subsection{burning\_cutoff}

\begin{itemize}
\item After we initilize the base state in {\tt varden} and at the
  beginning of {\tt advance\_timestep}, We compute the coordinate
  associated with {\tt burning\_cutoff\_density}, {\tt
    burning\_cutoff\_density\_coord}.  This is done by setting ${\tt
    burning\_cutoff\_density\_coord}= r$ as soon as $\rho_0(r) <$ {\tt
    burning\_cutoff\_density}.

\item In {\tt react\_state}, we only call the burner if $\rho >$ {\tt
  burning\_cutoff\_density}.
%\item In {\tt react\_state}, we only call the burner for $r\in[{\tt r\_start\_coord},{\tt burning\_cutoff\_density\_coord}-1]$.
\end{itemize}

