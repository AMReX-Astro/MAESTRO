\documentclass[11pt]{article} 

\tolerance=600

\usepackage{amsmath,color}

% Margins
\usepackage[lmargin=0.5in,rmargin=1.5in,tmargin=1.0in,bmargin=1.0in]{geometry}

\def\half  {\frac{1}{2}}

\def\eb    {{\bf e}}
\def\Ub    {{\bf U}}
\def\Ubt   {\widetilde{\bf U}}
\def\ut   {\widetilde{\bf U}}
\def\wt    {\widetilde{w}}
\def\er   {\bf e_r}
\title{Notes on Perturbational Quantities}

\setlength{\marginparwidth}{1.0in}
\newcommand{\MarginPar}[1]{\marginpar{%
\vskip-\baselineskip %raise the marginpar a bit
\raggedright\tiny\sffamily
\hrule\smallskip{\color{red}#1}\par\smallskip\hrule}}

\begin{document}

\maketitle
\tableofcontents
\cleardoublepage

\section{Density Evolution}
The full density evolution equation is
\begin{equation}
\frac{\partial\rho}{\partial t} = -\nabla\cdot(\rho\Ub) = -\Ub\cdot\nabla\rho 
- \rho\nabla\cdot\Ub. \label{rho equation}
\end{equation}
\subsection{Predicting $\rho'$ ({\tt predict\_rho = F} - default)}
The base state density evolution equation is
\begin{equation}
\frac{\partial\rho_0}{\partial t} = -\nabla\cdot(\rho_0 w_0 \eb_r) = 
-w_0\frac{\partial\rho_0}{\partial r} 
\underbrace{-\rho_0\frac{\partial w_0}{\partial r}}_{"\rho_0 ~ \text{force}"}.
\label{rho0 equation}
\end{equation}
Subtract (\ref{rho0 equation}) from (\ref{rho equation}) and rearrange terms, noting that 
$\Ub = \Ubt + w_o\eb_r$, to obtain the perturbational density equation,
\begin{equation}
\frac{\partial\rho'}{\partial t} = -\Ub\cdot\nabla\rho' \underbrace{- \rho'\nabla\cdot\Ub 
- \nabla\cdot(\rho_0\Ubt)}_{"\rho' ~ \text{force}"}. 
\label{rhoprime equation}
\end{equation}
\subsubsection{Predicting $\rho'$ at edges}
We define $\rho' = \rho^n - \rho_0^n$.  Then we predict $\rho'$ to edges using 
{\tt make\_edge\_scal} in {\tt scalar\_advance} and the underbraced term in (\ref{rhoprime equation}) 
as the forcing.
\subsubsection{Predicting $\rho_0$ at edges}\label{Predicting rho0 at edges}
There are two ways to predict $\rho_0$ at edges.
\begin{enumerate}
\item We call {\tt make\_edge\_state\_1d} using the underbraced term in 
(\ref{rho0 equation}) as the forcing.  This gives us $\rho_0^{n+\half,{\rm pred}}$.
This term is used to advect $\rho_0$ in {\tt advect\_base} and therefore must also be used 
to compute $\eta_\rho$, which will be used in {\tt correct\_base}.
\item We define $\rho_0^{n+\half,{\rm avg}} = (\rho_0^n + \rho_0^{(2)})/2$.  We 
compute $\rho_0^{(2)}$ from $\rho_0^n$ using {\tt advect\_base}, which advances equation 
(\ref{rho0 equation}) through $\Delta t$ in time.  The $(2)$ in the superscript indicates 
that we have not called {\tt correct\_base} yet, which computes  $\rho_0^{n+1}$ from $\rho_0^{(2)}$. 
We use $\rho_0^{(2)}$ rather than $\rho_0^{n+1}$ to construct $\rho_0^{n+\half,{\rm avg}}$ to be 
consistent with the fact that we do not include an $\eta_\rho$ contribution to compute 
$\rho'$ at edges.  $\rho_0^{n+\half,{\rm avg}}$ is used to construct $\rho$ at edges from 
$\rho'$ at edges, and this $\rho$ at edges is used to compute fluxes for $\rho X_k$.  
\end{enumerate}
\subsubsection{Computing $\rho$ at edges}\label{Computing rho at edges}
For the non-radial edges, we directly add $\rho_0^{n+\half,{\rm avg}}$ to $\rho'$ since 
$\rho_0^{n+\half,{\rm avg}}$ is a cell-centered quantity.  For the radial edges, we 
interpolate to obtain $\rho_0^{n+\half,{\rm avg}}$ at radial edges before adding it to $\rho'$.
\subsubsection{Advancing $\rho X_k$}\label{Advancing rhoX_k}
The evolution equation for $\rho X_k$, ignoring the reaction terms that were already 
accounted for in {\tt react\_state}, and the associated discretization is
\begin{equation}
\frac{\partial\rho X_k}{\partial t} = -\nabla\cdot(\rho X_k\Ub) = 
-\nabla\cdot\left\{\left[\left(\rho_0^{n+\half,{\rm avg}} 
+ \rho'\right)X_k\right](\Ubt+w_0\eb_r)\right\}.
\end{equation}
\section{Energy Evolution}
The full enthalpy equation is
\begin{eqnarray}
\frac{\partial(\rho h)}{\partial t} &=& -\nabla\cdot(\rho h \Ub) + \frac{Dp_0}{Dt} 
+ \nabla\cdot\kappa\nabla T + \rho H_{\rm nuc} + \rho H_{\rm ext} \nonumber \\
&=& \underbrace{-\Ub\cdot\nabla(\rho h) - \rho h\nabla\cdot\Ub}_{-\nabla\cdot(\rho h\Ub)} 
+ \underbrace{\psi + \wt\frac{\partial p_0}{\partial r}}_{\frac{Dp_0}{Dt}} 
+ \nabla\cdot\kappa\nabla T + \rho H_{\rm nuc} + \rho H_{\rm ext}.
\end{eqnarray}
The full temperature equation is
\begin{equation}
\frac{\partial T}{\partial t} = -\Ub\cdot\nabla T
+ \frac{1}{\rho c_p}\left\{(1-\rho h_p)\left[\psi
+ \wt\frac{\partial p_0}{\partial r}\right] - \sum_k\rho\xi_k\dot\omega_k 
+ \rho H_{\rm nuc} + \rho H_{\rm ext}\right\}.
\end{equation}
\subsection{Predicting $(\rho h)'$ ({\tt enthalpy\_pred\_type = 1} - default)}
Due to operator splitting, the call to {\tt react\_state} has already been made.  Hence, the goal 
is to compute an edge state enthalpy starting from $(\rho h)^{(1)}$ using an enthalpy
equation that does not include the $\rho H_{\rm nuc}$ and $\rho H_{\rm ext}$ terms, where were 
already accounted for in {\tt react\_state}
\begin{equation}
\frac{\partial(\rho h)}{\partial t} = -\Ub\cdot\nabla(\rho h) - \rho h\nabla\cdot\Ub 
+ \psi + \wt\frac{\partial p_0}{\partial r} + \nabla\cdot\kappa\nabla T. \label{rhoh equation}
\end{equation}
We arbitrarily define a base state enthalpy evolution equation
\MarginPar{Does $\psi$ need to be included in the force?  I have written it in the base state equation for now, even though neither the base state or perturbational edge state prediction in the code uses $\psi$.  How about evolving density first,  then computing a time-centered $\psi$ to be used as the energy forcing?}
\begin{eqnarray}
\frac{\partial(\rho h)_0}{\partial t} &=& -\nabla\cdot[(\rho h)_0 w_0\eb_r] 
+ \frac{D_0p_0}{Dt} \nonumber \\
&=& -w_0\frac{\partial(\rho h)_0}{\partial r} 
- \underbrace{(\rho h)_0\frac{\partial w_0}{\partial r}}_{"(\rho h)_0 ~ \text{force}"}
+ \psi.\label{rhoh0 equation}
\end{eqnarray}
Subtracting (\ref{rhoh0 equation}) from (\ref{rhoh equation}) and rearranging terms gives 
the perturbational enthalpy equation
\begin{eqnarray}
\frac{\partial(\rho h)'}{\partial t} &=& -\nabla\cdot[(\rho h)'\Ub] 
- \nabla\cdot[(\rho h)_0\Ubt] + \wt\frac{\partial p_0}{\partial r} 
+ \nabla\cdot\kappa\nabla T\nonumber \\
&=& -\Ub\cdot\nabla(\rho h)' \underbrace{- (\rho h)'\nabla\cdot\Ub 
- \nabla\cdot[(\rho h)_0\Ubt] + \wt\frac{\partial p_0}{\partial r}
+ \nabla\cdot\kappa\nabla T}_{"(\rho h)' ~ \text{force}"}. \label{rhohprime equation}
\end{eqnarray}
\subsubsection{Predicting $(\rho h)'$ at edges}\label{Predicting rhohprime at edges}
{\bf What the code did before 8/6/08:}
 Instead of defining $(\rho h)_0^{(1)} = \overline{(\rho h)^{(1)}}$, the code carries 
through the complete evolution of $(\rho h)_0$ using $(\rho h)_0^n \rightarrow$ {\tt react\_base}
$\rightarrow (\rho h)_0^{(1)} \rightarrow$ {\tt advect\_base}
$\rightarrow (\rho h)_0^{(2)} \rightarrow$ {\tt react\_base} $\rightarrow (\rho h)_0^{n+1}$.  
There is no {\tt correct\_base} function for $(\rho h)_0$ written yet, nor is any adjustment made 
to sync up $(\rho h)_0$ after the thermal diffusion step.\\ \\
{\bf What the code currently does:}  Define $(\rho h)_0^{(1)} = \overline{(\rho h)^{(1)}}$  
and $(\rho h)' = (\rho h)^{(1)}-(\rho h)_0^{(1)}$.  Then we predict $(\rho h)'$ to 
edges using {\tt make\_edge\_scal} in {\tt scalar\_advance} and the underbraced term in 
(\ref{rhohprime equation}) as the forcing.
\subsubsection{Predicting $(\rho h)_0$ at edges}
We use an analogous procedure described in Section \ref{Predicting rho0 at edges} to 
obtain $(\rho h)_0^{n+\half,\rm{pred}}$ and $(\rho h)_0^{n+\half,\rm{avg}}$.  Note that 
$(\rho h)_0^{n+\half,{\rm avg}} = [(\rho h)_0^{(1)} + (\rho h)_0^{(2)}]/2$.
\subsubsection{Computing $\rho h$ at edges}
We use an analogous procedure described in Section \ref{Computing rho at edges} to compute
$\rho h$ at edges.
\subsubsection{Advancing $\rho h$}
{\bf What the did before 8/6/08:}
\begin{equation}
\frac{\partial(\rho h)}{\partial t} = -\nabla\cdot\left[(\rho h)'\left(\Ubt 
+ w_0\eb_r\right)\right] - \nabla\cdot\left[(\rho h)_0^{n+\half,{\rm avg}}\Ubt\right] 
\underbrace{- \nabla\cdot\left[(\rho h)_0^{n+\half,{\rm pred}}w_0\eb_r\right] + \psi}_{\text{this term used to come in through} ~ (\rho h)_0^{n+1} - (\rho h)_0^n} + \wt\frac{\partial p_0}{\partial r} \enskip .
\end{equation}
{\bf What the code does now:} We would like to update the 
non-forcing part of the enthalpy equation analogously to Section \ref{Advancing rhoX_k}
\begin{equation}
\frac{\partial(\rho h)}{\partial t} = -\nabla\cdot(\rho h\Ub) = 
-\nabla\cdot\left\{\left[(\rho h)_0^{n+\half,{\rm avg}} 
+ (\rho h)'\right]\left(\Ubt + w_0\eb_r\right)\right\} + \wt\frac{\partial p_0}{\partial r} + \psi  \enskip .
\end{equation}


\subsection{Predicting $h$ ({\tt enthalpy\_pred\_type = 2})}

\subsection{Predicting $T$ then $(rho h)'$ ({\tt enthalpy\_pred\_type = 3})}

\subsection{Predicting $T$ then $h$ ({\tt enthalpy\_pred\_type = 4})}

\subsection{Summary of Forcing}

\begin{table*}[h]
\begin{center}
\caption{Forcing term into {\tt make\_edge\_scal} \newline}
\begin{tabular}{c|c}
\hline
\hline
{H PRED TYPE} &   {Force} \\
\hline \\[-3mm]
1  $((\rho h)^\prime)$ &  $(\tilde{w} \frac{\partial p_0}{\partial r}) - 
 (\rho h)^\prime \; \nabla \cdot (\ut+w_0 \er) - 
 \nabla \cdot (\ut (\rho h)_0)$  \\[2 mm]
2  $(h)$ & $\frac{1}{\rho} (\psi + \tilde{w} 
  \frac{\partial p_0}{\partial r})$ \\[2 mm]
3  $(T)$ & $\frac{1}{\rho c_p} \left[ (1 - \rho h_p) 
  (\psi + \tilde{w} \frac{\partial p_0}{\partial r}) \right]$ \\[2 mm]
4  $(T)$ & $\frac{1}{\rho c_p} \left[ (1 - \rho h_p) (\psi + \tilde{w} 
\frac{\partial p_0}{\partial r}) \right]$ \\[2 mm]
\hline
\end{tabular}
\end{center}
\end{table*}

\begin{table*}[h]
\begin{center}
\caption{Quantity that goes into and out of make edge scal \newline}
\begin{tabular}{c|c}
\hline
\hline
{h pred type} & {h/T } \\
\hline \\[-3mm]
1 & $(\rho h)^\prime$ \\
2 & $h$  \\
3 & $T$  \\
4 & $T$  \\
\hline
\end{tabular}
\end{center}
\end{table*}

\begin{table*}
\begin{center}
\caption{When predicting temp ... \newline}
\begin{tabular}{c|c|c|c}
\hline
\hline
{type} & {Inputs } & {EOS inputs} & {output} \\
\hline
3 & $X$ and $\rho^\prime$ & $\rho = \rho_0 + \rho^\prime$ &  $(\rho h)^\prime = \rho h - (\rho h)_0$ \\
   &                   & $   X = $ given & \\
   &                   &                 & \\
 4 & $X$ and $\rho^\prime$ & $\rho = \rho_0 + \rho^\prime$ & $h$ \\
   &                   & $   X = $ given & \\
   &                   &                 & \\
\hline
\end{tabular}
\end{center}
\end{table*}


\begin{table*}
\begin{center}
\caption{Quantity that goes into mkflux on edges \newline}
\begin{tabular}{c|c|c}
\hline
\hline
{h pred type} & {Species } & {h/T } \\
\hline
1 & $X$ and $\rho^\prime$ & $(\rho h)^\prime$ \\
2 & $X$ and $\rho^\prime$ & $h$ \\
3 & $X$ and $\rho^\prime$ & $(\rho h)^\prime$ \\
4 & $X$ and $\rho^\prime$ & $h$ \\
\hline
\end{tabular}
\end{center}
\end{table*}

\begin{table*}
\begin{center}
\caption{Quantity that is created in mkflux and passed into update scal\newline\
}
\begin{tabular}{c|c|c|c|c}
       &            &              &\multicolumn{2}{c}{Forcing Term} \\
{type} & {Species } & {$(\rho h)$} & {$(\rho X)$} & {$(\rho h)$} \\
\hline 
1 & $(\ut+w_0)(\rho_0+\rho^\prime) X          $ & $(\ut+w_0)(\rho h)^\prime
 + \ut (\rho h)_0$ & 0 & $ (\tilde{w} \frac{\partial p_0}{\partial r}) $ \\[2mm]
2 & $(\ut+w_0)(\rho_0+\rho^\prime) X          $ & $(\ut+w_0) (\rho_0+\rho^\prime) h$ & 0 & $ (\psi + \tilde{w} \frac{\partial p_0}{\partial r}) $ \\[2mm]
3 & $(\ut+w_0)(\rho_0+\rho^\prime) X          $ & $(\ut+w_0)(\rho h)^\prime + \ut (\rho h)_0$ & 0 & $ (\tilde{w} \frac{\partial p_0}{\partial r}) $ \\[2mm]
4 & $(\ut+w_0)(\rho_0+\rho^\prime) X          $ & $(\ut+w_0) (\rho_0+\rho^\prime) h$ & 0 & $ (\psi + \tilde{w} \frac{\partial p_0}{\partial r}) $ \\[2mm]
\hline
\end{tabular}
\end{center}
\end{table*}


\end{document}
