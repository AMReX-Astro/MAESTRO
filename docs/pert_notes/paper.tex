\documentclass[11pt]{article} 

\tolerance=600

\usepackage{amsmath,color}

% Margins
\usepackage[lmargin=0.5in,rmargin=1.5in,tmargin=1.0in,bmargin=1.0in]{geometry}

\def\half  {\frac{1}{2}}

\def\eb    {{\bf e}}
\def\Ub    {{\bf U}}
\def\Ubt   {\widetilde{\bf U}}
\def\wt    {\widetilde{w}}

\title{Notes on Perturbational Quantities}

\setlength{\marginparwidth}{1.0in}
\newcommand{\MarginPar}[1]{\marginpar{%
\vskip-\baselineskip %raise the marginpar a bit
\raggedright\tiny\sffamily
\hrule\smallskip{\color{red}#1}\par\smallskip\hrule}}

\begin{document}

\maketitle
\tableofcontents
\cleardoublepage

\section{Density Evolution}
The full density evolution equation is
\begin{equation}
\frac{\partial\rho}{\partial t} = -\nabla\cdot(\rho\Ub) = -\Ub\cdot\nabla\rho 
- \rho\nabla\cdot\Ub. \label{rho equation}
\end{equation}
\subsection{Predicting $\rho'$ (predict\_rho = F - default)}
The base state density evolution equation is
\begin{equation}
\frac{\partial\rho_0}{\partial t} = -\nabla\cdot(\rho_0 w_0 \eb_r) = 
-w_0\frac{\partial\rho_0}{\partial r} - \rho_0\frac{\partial w_0}{\partial r}.
\label{rho0 equation}
\end{equation}
Subtract (\ref{rho0 equation}) from (\ref{rho equation}) and rearrange, noting that 
$\Ub = \Ubt + w_o\eb_r$, to obtain the perturbational density equation,
\begin{equation}
\frac{\partial\rho'}{\partial t} = -\Ub\cdot\nabla\rho' \underbrace{- \rho'\nabla\cdot\Ub 
- \nabla\cdot(\rho_0\Ubt)}_{"\rho' ~ \text{force}"}. 
\label{rhoprime equation}
\end{equation}
\subsubsection{Predicting $\rho'$ at edges}
We predict $\rho'$ to edges using make\_edge\_scal in scalar\_advance.  The underbraced 
forcing term in (\ref{rhoprime equation}) is computed in modify\_scal\_force.
\subsubsection{Computing $\rho$ at edges}\label{Computing rho at edges}
We first compute $\rho_0^{(2)}$ from $\rho^n$ using advect\_base.  The $(2)$ in the 
superscript indicates that we have not called correct\_base yet.  We compute $\rho$ on edges 
by adding $\rho_0^{n+\half} = (\rho_0^n + \rho_0^{(2)}) / 2$ to $\rho'$ at edges.  We use 
$\rho^{(2)}$ rather than $\rho^{n+1}$ to be consistent with the fact that we omitted the 
$\eta_\rho$ contribution to $\rho'$ at edges.  For the non-radial faces, we can directly add 
$\rho_0^{n+\half}$ to $\rho'$ since it is a cell-centered quantity.  For the radial faces, we 
interpolate to obtain $\rho_0^{n+\half}$ at edges.
\subsection{Predicting $\rho$ (predict\_rho = T)}
\subsubsection{Predicting $\rho$ at edges}
We predict $\rho$ to edges using make\_edge\_scal in scalar\_advance.  The forcing term 
is set to zero, but then we set is\_conservative = T to use the conservative 
discretization in make\_edge\_scal.
\subsection{Predicting $\rho_0$}
To compute $\eta_\rho$, we need $\rho_0$ at edge-centered, time-centered locations.  
However, we do NOT use the same edge-centered, time-centered $\rho_0^{n+\half}$ that was 
referred to in Section \ref{Computing rho at edges}.  Instead, we use a $\rho_0$ computed 
with a Godunov method in make\_edge\_state\_1d.  We call make\_edge\_state\_1d using the 
term proportional to $\rho_0$ in  (\ref{rho0 equation}) as the forcing.  The $\eta_\rho$ 
term is not 
included.\MarginPar{I think there is a bug in the code and the $\eta_\rho$ term needs to 
be included here.}  This gives us the so-called $\rho_0^{n+\half,{\rm pred}}$.
\section{Energy Evolution}
Full enthalpy equation:
\begin{eqnarray}
\frac{\partial(\rho h)}{\partial t} &=& -\nabla\cdot(\rho h \Ub) + \frac{Dp_0}{Dt} 
+ \nabla\cdot\kappa\nabla T + \rho H_{\rm nuc} + \rho H_{\rm ext} \nonumber \\
&=& \underbrace{-\Ub\cdot\nabla(\rho h) - \rho h\nabla\cdot\Ub}_{-\nabla\cdot(\rho h\Ub)} 
+ \underbrace{\psi + \Ubt\cdot\nabla p_0}_{\frac{Dp_0}{Dt}} + \nabla\cdot\kappa\nabla T 
+ \rho H_{\rm nuc} + \rho H_{\rm ext}\label{rhoh equation}
\end{eqnarray}
Full temperature equation:
\begin{equation}
\frac{\partial T}{\partial t} = -\Ub\cdot\nabla T
+ \frac{1}{\rho c_p}\left\{(1-\rho h_p)\left[\psi
+ (\Ubt\cdot\eb_r)\frac{\partial p_0}{\partial r}\right] - \sum_k\rho\xi_k\dot\omega_k 
+ \rho H_{\rm nuc} + \rho H_{\rm ext}\right\}.
\end{equation}
\subsection{Predicting $(\rho h)'$ (enthalpy\_pred\_type = 1 - default)}
These are working notes for what I'd like to do for the energy evolution.
\subsubsection{Predicting $(\rho h)'$ at edges}\label{Predicting rhohprime at edges}
We can arbitrarily define a base state enthalpy evolution equation:
\begin{eqnarray}
\frac{\partial(\rho h)_0}{\partial t} &=& -\nabla\cdot[(\rho h)_0 w_0\eb_r] 
+ \frac{D_0p_0}{Dt} + \overline{\nabla\cdot\kappa\nabla T} + \overline{\rho H_{\rm nuc}} 
+ \overline{\rho H_{\rm ext}} \nonumber \\
&=& -w_0\frac{\partial(\rho h)_0}{\partial r} - (\rho h)_0\frac{\partial w_0}{\partial r} 
+ \psi + \overline{\nabla\cdot\kappa\nabla T} + \overline{\rho H_{\rm nuc}} 
+ \overline{\rho H_{\rm ext}}.\label{rhoh0 equation}
\end{eqnarray}
We will predict both $(\rho h)_0$ and $(\rho h)'$ to edges using a Godunov method.  Then 
we will add the two answers together to get $\rho h$ at edges.  This is different from 
the density, in which we averaged an ``old'' and ``new'' base state density to get 
$\rho_0$ at edges.  By computing $\rho h$ at edges in this way, we avoid having to 
compute the complete evolution of $(\rho h)_0$, and therefore avoid having to write a 
correct\_base and diffuse\_base function for $\rho h$.\\

To simplify matters, the we have already called react\_state, so for now we will ignore 
the $\rho H_{\rm nuc}$ and $\rho H_{\rm ext}$ terms in the remaining derivations.  
Subtracting (\ref{rhoh0 equation}) from (\ref{rhoh equation}) and ignoring the 
aforementioned terms gives the perturbational enthalpy equation:
\begin{eqnarray}
\frac{\partial(\rho h)'}{\partial t} &=& -\nabla\cdot[(\rho h)'\Ub] 
- \nabla\cdot[(\rho h)_0\Ubt] + \wt\frac{\partial p_0}{\partial r} 
+ (\nabla\cdot\kappa\nabla T - \overline{\nabla\cdot\kappa\nabla T}) \nonumber \\
&=& -\Ub\cdot\nabla(\rho h)' \underbrace{- (\rho h)'\nabla\cdot\Ub 
- \nabla\cdot[(\rho h)_0\Ubt]} + \wt\frac{\partial p_0}{\partial r} 
+ (\nabla\cdot\kappa\nabla T 
- \overline{\nabla\cdot\kappa\nabla T})\label{rhohprime equation}
\end{eqnarray}
We use make\_edge\_scal to get $(\rho h)'$ at edges.  The function modify\_scal\_force 
computes the underbraced term in (\ref{rhohprime equation}).  The function mkrhohforce 
computes the $\wt$ term.\MarginPar{I need to write a function to handle the thermal 
diffusion terms.}
\subsubsection{Predicting $(\rho h)_0$ at edges}\label{Predicting rhoh0 at edges}
After the call to react\_state, we set $(\rho h)_0^{(1)} = \overline{(\rho h)^{(1)}}$.  
We use the Godunov algorithm in make\_edge\_state\_1d to predict $(\rho h)_0$ at edges, 
omitting the terms added in react\_state, so we are essentially evolving an abbreviated 
version of (\ref{rhoh0 equation}):
\begin{eqnarray}
\frac{\partial(\rho h)_0}{\partial t} &=& -\nabla\cdot[(\rho h)_0 w_0\eb_r] + \psi 
+ \overline{\nabla\cdot\kappa\nabla T} \nonumber \\
&=& -w_0\frac{\partial(\rho h)_0}{\partial r} - (\rho h)_0\frac{\partial w_0}{\partial r} 
+ \psi + \overline{\nabla\cdot\kappa\nabla T}.\label{rhoh0 equation}
\end{eqnarray}
In advect\_base, the forcing term is set to the term proportional to 
$(\rho h)_0$.\MarginPar{We are currently omitting the $\psi$ and thermal conduction 
terms.  I think this is a bug.}
\subsubsection{Computing $\rho h$ at edges}
Simply add the results from Sections \ref{Predicting rhohprime at edges} and 
\ref{Predicting rhoh0 at edges}, using spatial interpolation to get $(\rho h)_0$ 
at non-radial faces.
\subsection{Predicting $h$ (enthalpy\_pred\_type = 2)}
\subsection{Predicting $T$ into $(\rho h)'$ (enthalpy\_pred\_type = 3)}
\subsection{Predicting $T$ into $h$ (enthalpy\_pred\_type = 4)}
\end{document}
