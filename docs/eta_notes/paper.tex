\documentclass[11pt]{article} 

\tolerance=600

\usepackage{amsmath,color}

% Margins
\usepackage[lmargin=0.5in,rmargin=1.5in,tmargin=0.5in,bmargin=0.5in]{geometry}

\def\half   {\frac{1}{2}}

\def\eb         {{\bf e}}
\def\ib         {{\bf i}}
\def\Ub         {{\bf U}}

\def\etarho     {\eta_\rho}
\def\etarhocc   {\etarho^{\rm cc}}
\def\divetarho  {\nabla\cdot(\etarho\eb_r)}



\title{Notes on $\etarho$}

\setlength{\marginparwidth}{1.0in}
\newcommand{\MarginPar}[1]{\marginpar{%
\vskip-\baselineskip %raise the marginpar a bit
\raggedright\tiny\sffamily
\hrule\smallskip{\color{red}#1}\par\smallskip\hrule}}

\begin{document}

\maketitle

We carry around three different quantities: $\etarho$ (edge-centered), $\etarhocc$ (cell-centered), and $\divetarho$ (edge-centered).  These are each 1D in the radial direction.  These notes discuss when each of these is used, and how they are computed, in both plane-parallel and spherical.

\section{Computing $\etarho, \etarhocc$, and $\divetarho$}
This is done in {\tt make\_eta.f90}.
\subsection{Plane-Parallel}
We first compute a radial edge-centered multifab, $\eta_{\rho}^{\rm flux}$, using
\begin{equation}
\eta_{\rho,\ib+\half\eb_r}^{\rm flux} = \Ub_{\ib+\half\eb_r}^{n+\half} \rho_{\ib+\half\eb_r}^{n+\half} - w_{0,r+\half}^{n+\half}\rho_{0,r+\half}^{n+\half, {\rm pred}}
\end{equation}
$\etarho$ is the edge-centered ``average'' value of $\eta_{\rho}^{\rm flux}$,
\begin{equation}
\eta_{\rho,r+\half} = \overline{\eta_{\rho,\ib+\half\eb_r}^{\rm flux}}
\end{equation}
$\etarhocc$ is a cell-centered average of $\etarho$,
\begin{equation}
\eta_{\rho,r}^{\rm cc} = \frac{\eta_{\rho,r+\half} + \eta_{\rho,r-\half}}{2}.
\end{equation}
$\divetarho$ is the cell-centered divergence of $\etarho$,
\begin{equation}
\left[\divetarho\right]_r = \frac{\eta_{\rho,r+\half} - \eta_{\rho,r-\half}}{\Delta r}.
\end{equation}
\subsection{Spherical}

\end{document}
