\documentclass[11pt]{article} 

\tolerance=600

\usepackage{amsmath,color}

% Margins
\usepackage[lmargin=0.5in,rmargin=1.5in,tmargin=0.5in,bmargin=0.5in]{geometry}

\def\eb        {\bf{e}}

\def\etarho    {\eta_\rho}
\def\etarhocc  {\etarho^{\rm cc}}
\def\divetarho {\nabla\cdot(\etarho\eb_r)}

\title{Notes on $\etarho$}

\setlength{\marginparwidth}{1.0in}
\newcommand{\MarginPar}[1]{\marginpar{%
\vskip-\baselineskip %raise the marginpar a bit
\raggedright\tiny\sffamily
\hrule\smallskip{\color{red}#1}\par\smallskip\hrule}}

\begin{document}

\maketitle

We carry around three different quantities: $\etarho$ (edge-centered), $\etarhocc$ (cell-centered), and $\divetarho$ (edge-centered).  These notes discuss when each of these is used, and how they are computed, in both plane-parallel and spherical.


\end{document}
