We carry around three different 1D radial quantities: $\etarhoec$
(edge-centered), $\etarhocc$ (cell-centered), and $\divetarho$
(cell-centered).  These notes discuss when each of these is used, and
how they are computed, in both plane-parallel and spherical.

\section{The Mixing Term, $\etarho$}

The base state evolves in response to heating and mixing in the star.
The density evolution is governed by
\begin{eqnarray}
\frac{\partial \rho_0}{\partial t} &=& - 
 \nablab \cdotb \left( \rho_0 w_0 \er \right)
- \nablab \cdotb \left( \etarho \er \right) \enskip ,
\label{eq:rho0upd_new}
\end{eqnarray}
with
\begin{equation}
\etarho(r) = \overline{\left(\rhop \Ubt \cdot \er \right)} = \frac{1}{A(\Omega_H)} 
 \int_{\Omega_H}  (\rhop \Ubt \cdot \er ) \; dA \enskip , \label{eq:eta}
\end{equation}
designed to keep the average value of the full density, $\rho$, over a
layer of constant radius in the star equal to $\rho_0$.  To complete
the update of the base state, we need evolution equations for the
pressure, $p_0$, and velocity, $w_0$.  For spherical geometry, the
derivation of $w_0$ constraint equation is shown in
Appendix~\ref{app:gravity}, resulting in the following system
\begin{eqnarray}
w_0 &=& \ow + \delta w_0 \\
\frac{1}{r^2} \frac{\partial}{\partial r} \left (r^2 \ow \right ) &=& \Sbar \\
\frac{\partial}{\partial r} \left[ \frac{\gammabar p_0}{r^2} \frac{\partial}{\partial r} (r^2 \dw) \right] &=& - \frac{g}{r^2} \frac{\partial (r^2 \etarho)}{\partial r} - \frac{4 (\ow + \dw) \rho_0 g}{r} 
- 4 \pi G \rho_0 \etarho \label{eq:dw0constraint}
\end{eqnarray}

In paper~III, we introduced a mixing term, $\etarho$, to the density
evolution equation (eq.~[\ref{eq:rho0upd_new}]), with the objective of
keeping the base state density equal to the average of the density
over a layer.  For a spherical base state, it is best to define the 
average in terms of spherical coordinates,
\begin{equation}
\overline{q} = \frac{1}{4\pi} \int_{\Omega_H} q(r,\theta,\phi) \; d\Omega
\end{equation}
where $\int_{\Omega_H} d\Omega = 4\pi$ represents the integral over
the spherical $\theta$ and $\phi$ angles at constant radius.

Recall that if initially $ \overline{\rhop} = 1 /
{4\pi} \int_{\Omega_H} \rhop \; d\Omega = 0,$ 
\MarginPar{this section is here temporarily, just to better understand
what's going on here.  These notes came from an early version of
paper~III} 
there is no guarantee that $\overline{\rhop} = 0$ will hold at later
time.  To see this we start with the equation for $\rho'$, 
\begin{equation}
\frac{\partial \rhop}{\partial t} = - \nablab \cdotb (\rho_0 \Ubt)
-\nablab \cdotb (\rhop w_0 \er) - \nablab \cdotb (\rhop \Ubt )
\label{eq:rhoprimeold} \enskip ,
\end{equation}
rewritten by moving one term to the left-hand-side:
\begin{equation}
\frac{\partial \rhop}{\partial t} + \nablab \cdotb (\rhop \Ubt) 
= - \nablab \cdotb (\rho_0 \Ubt)
  -\nablab \cdotb (\rhop w_0 \er) \enskip .
\end{equation}
We integrate this over a spherical shell of thickness $2h$ at radius $r_0$, 
$\Omega_H \times (r_0-h, r_0+h)$, and normalize by the integration
volume, which is $\sim 4\pi r_0^2 \, 2h$ for small $h$, to obtain:
\begin{eqnarray}
\frac{1}{4\pi r_0^2 \, 2h} \int_{r_0-h}^{r_0+h} r^2 \; dr \int_{\Omega_H} 
    \left(\frac{\partial \rhop}{\partial t} + \nabla \cdotb ( \rhop \Ubt ) \right) \; d\Omega 
&=& - \frac{1}{4\pi r_0^2 \, 2h} \int_{r_0-h}^{r_0+h} r^2 \; dr 
\int_{\Omega_H} \nabla \cdotb \left( \rho_0 \Ubt + \rhop w_0 \er \right) \; d\Omega \nonumber  \\
&=& \left .  - \frac{1}{4\pi r_0^2 \, 2h}  \int_{\Omega_H} 
\left [ \rho_0  (\Ubt \cdot \er)  + \rhop w_0 \right] r^2
  \; d\Omega  \; \right  |_{r_0-h}^{r_0+h} \nonumber  \\
&=& 0 \nonumber \enskip ,
\end{eqnarray} 
where we have used the divergence theorem in spherical coordinates to transform
the volume integrals into area integrals over $\Omega_H.$
We see that both of the terms on the right hand side disappear. The first,
$\int_{\Omega_H} \rho_0 (\Ubt \cdot \er) d\Omega = 0$ follows from the definition of $\Ubt,$
and the second, $\int_{\Omega_H} \rhop w_0 d\Omega = w_0 \int_{\Omega_H} \rhop d\Omega = 0$
follows from our initial assumption.  
%\begin{eqnarray}
%\frac{1}{4\pi r_0^2 \, 2h} \frac{\partial}{\partial t} 
%  \int_{r_0-h}^{r_0+h} \int_{\Omega_H}  \rhop \; r^2 dr \; d\Omega &=&  
%\frac{1}{4\pi r_0^2 \, 2h} \int_{r_0-h}^{r_0+h} \int_{\Omega_H} \nabla \cdotb \left( \rho_0 \Ubt + \rhop w_0 \er \right)
%\; r^2 dr \; d\Omega &=&
%\\
%&=& \left .  - \frac{1}{4\pi r_0^2 \, 2h} \int_{\Omega_H} 
% \rhop (\Ubt \cdot \er)  \; r^2 d\Omega  \right  |_{r_0-h}^{r_0+h}
%\enskip ,
%\end{eqnarray} 
Now expanding the remaining terms and taking the limit as $h \rightarrow 0,$ we can write 
\begin{eqnarray} 
0 &=& \lim_{h\rightarrow 0} \frac{1}{4\pi r_0^2 \, 2h} \int_{r_0-h}^{r_0+h} r^2 \; dr \int_{\Omega_H} 
    \left(\frac{\partial \rhop}{\partial t} + \nabla \cdotb ( \rhop \Ubt ) \right) \; d\Omega \nonumber \\
&=& \frac{\partial}{\partial t} \left( \lim_{h\rightarrow 0} \frac{1}{4\pi r_0^2} \frac{1}{2h} 
  \int_{r_0-h}^{r_0+h} r^2 \; dr \int_{\Omega_H}  \rhop \;d\Omega \right) +
\lim_{h\rightarrow 0} \left( \frac{1}{4\pi r_0^2 \, 2h}  
  \int_{r_0-h}^{r_0+h} r^2 \; dr \int_{\Omega_H}  \nabla \cdotb ( \rhop \Ubt ) \; d\Omega \right) \nonumber \\
&=& \frac{\partial}{\partial t} \left(\frac{1}{4\pi} \int_{\Omega_H}  \rhop \; d\Omega \right) +
\lim_{h\rightarrow 0} \left( \left .  \frac{1}{4\pi r_0^2 \, 2h}  
\left [ \int_{\Omega_H} \rhop (\Ubt \cdot \er) \; d\Omega  \right]  r^2\; \right |_{r_0-h}^{r_0+h} \right) \nonumber  \\
&=&  \frac{\partial}{\partial t} \overline{\rhop} + \lim_{h\rightarrow 0} \left( \left .  \frac{1}{r_0^2 \, 2h}
\overline{[\rhop (\Ubt \cdot \er)]} \; r^2 \; \right |_{r_0-h}^{r_0+h} \right) \nonumber  \\
&=&  \frac{\partial}{\partial t} \overline{\rhop} + 
\lim_{h\rightarrow 0} \frac{1}{r_0^2 \, 2h} \int_{r_0-h}^{r_0+h} \nabla \cdot \left[\overline{\left(\rhop\Ubt \cdot \er\right)} \er \right]\;  r^2 \; dr \nonumber \\
&=&  \frac{\partial}{\partial t} \overline{\rhop} + \nabla \cdot \left[ \overline{\left(\rhop \Ubt \cdot \er\right)} \er \right]
\end{eqnarray} 
again using the divergence theorem, extracting the time derivative from the spatial integral,
and switching the order of operations as appropriate.

In short, 
\begin{equation}
\frac{\partial}{\partial t} \overline{\rhop} = 
- \nabla \cdot \left [ \overline{\left(\rhop \Ubt \cdot \er\right)} \, \er \right ] = 
- \nabla \cdot ( \etarho \er )
 \enskip .\label{eq:rhopbar} 
\end{equation} 

We need both $\etarho$ alone and its divergence for the
various terms in the construction of $w_0$ and the correction to
$\rho_0$.  The quantity $\etarho$ is edge-centered on our grid, and
for Cartesian geometries, we constructed it by averaging the
appropriate fluxes through the grid boundaries.  For a spherical base
state, this does not work, since the spherical shells do not line up
with the Cartesian grid boundaries.

Therefore, we take a different approach.  We compute $\etarho$ by
constructing the quantity $\rhop \Ubt \cdot \er$ in each cell, and then
use our average routine to construct a 1-d, cell-centered
$\eta_{\rho,r}$ (this is essentially numerically solving the integral
in Eq.~[\ref{eq:eta}]).  The edge-centered values of $\etarho$,
$\eta_{\rho,r+1/2}$ are then constructed by simple
averaging: \MarginPar{should we be doing a volume-weighted average?}
\begin{equation}
\eta_{\rho,r+1/2} = \frac{\eta_{\rho,r} + \eta_{\rho,r+1}}{2} \enskip .
\end{equation}

Instead of differencing $\eta_{\rho,r+1/2}$ to construct the 
divergence, we instead use equation~(\ref{eq:rhopbar}) directly, by writing:
\begin{equation}
\left [ \nabla \cdot (\etarho \er ) \right ]^{n+1/2}
= - \frac{\overline{\rhop^{n+1}} - \overline{\rhop^n}}{\Delta t}
= - \frac{\overline{\rhop^{n+1}}}{\Delta t} \enskip ,
\end{equation}
where we have made use of the fact that $\overline{\rhop^n} = 0$ by construction.


%-----------------------------------------------------------------------------
% flow chart
%-----------------------------------------------------------------------------
\section{$\eta$ Flow Chart}
\begin{enumerate}
\item Enter {\tt advance\_timestep} with $[\etarhoec, \etarhocc]^{n-\myhalf}$.
\item Call {\tt make\_w0}.  The spherical version uses uses $\etarhoec$ and $\etarhocc$.
\item Call {\tt density\_advance}.  The plane-parallel version computes $\etarhoflux$.
\item Call {\tt make\_etarho} to compute $[\etarhoec, \etarhocc]$
\item Call {\tt make\_psi}.  The plane-parallel version uses $\etarhocc$.
\item Call {\tt make\_w0}.  The spherical version uses uses $\etarhoec$ and $\etarhocc$.
\item Call {\tt density\_advance}.  The plane-parallel version computes $\etarhoflux$.
\item Call {\tt make\_etarho} to compute $[\etarhoec, \etarhocc]^{n+\myhalf}$
\item Call {\tt make\_psi}.  The plane-parallel version uses $\etarhocc$.
\end{enumerate}

%-----------------------------------------------------------------------------
% computing
%-----------------------------------------------------------------------------
\section{Computing $\etarhoec$ and $\etarhocc$}
This is done in {\tt make\_eta.f90}.

\subsection{Plane-Parallel}
We first compute a radial edge-centered multifab, $\eta_{\rho}^{\rm flux}$, using
\begin{equation}
\eta_{\rho,\ib+\myhalf\eb_r}^{\rm flux} = \left[\left(\Ubt_{\ib+\myhalf\eb_r}^{n+\myhalf}\cdot\eb_r\right) + w_{0,r+\myhalf}^{n+\myhalf}\right] \rho_{\ib+\myhalf\eb_r}^{n+\myhalf} - w_{0,r+\myhalf}^{n+\myhalf}\rho_{0,r+\myhalf}^{n+\myhalf, {\rm pred}}
\end{equation}
$\etarhoec$ is the edge-centered ``average'' value of $\eta_{\rho}^{\rm flux}$,
\begin{equation}
\eta_{\rho,r+\myhalf}^{\rm ec} = \overline{\eta_{\rho,\ib+\myhalf\eb_r}^{\rm flux}}
\end{equation}
$\etarhocc$ is a cell-centered average of $\etarhoec$,
\begin{equation}
\eta_{\rho,r}^{\rm cc} = \frac{\eta_{\rho,r+\myhalf}^{\rm ec} + \eta_{\rho,r-\myhalf}^{\rm ec}}{2}.
\end{equation}

\subsection{Spherical}\label{Sec:eta Spherical}
First, construct $\eta_{\rho}^{\rm cart} =
[\rho'(\Ubt\cdot\eb_r)]^{n+\myhalf}$ using:
\begin{equation}
\left[\frac{\rho^n+\rho^{n+1}}{2}-\left(\frac{\rho_0^n+\rho_0^{n+1}}{2}\right)^{\rm cart}\right] \sum_d\left(\frac{\Ubt_{\ib+\myhalf\eb_d}^{n+\myhalf}\cdot\eb_d+\Ubt_{\ib-\myhalf\eb_d}^{n+\myhalf}\cdot\eb_d}{2}\right)n_d.
\end{equation}
Then, $\etarhocc$ is the cell-centered average of $\eta_{\rho}^{\rm cart}$,
\begin{equation}
\etarhocc = \overline{\eta_{\rho}^{\rm cart}}.
\end{equation}
On interior faces, $\etarhoec$ is the average of $\etarhocc$,
\begin{equation}
\eta_{\rho,r-\myhalf}^{\rm ec} = \frac{\eta_{\rho,r-1}^{\rm cc} + \eta_{\rho,r}^{\rm cc}}{2}.
\end{equation}
At the upper and lower boundaries, we use
\begin{eqnarray}
\eta_{\rho,-\myhalf}^{\rm ec} &=& 0, \\
\eta_{\rho,{\rm nr}-\myhalf}^{\rm ec} &=& \eta_{\rho,{\rm nr}-1}^{\rm cc}.
\end{eqnarray}

%-----------------------------------------------------------------------------
% etarho_ec
%-----------------------------------------------------------------------------
\section{Using $\etarhoec$}
\subsection{Plane-Parallel}
NOT USED.

\subsection{Spherical}
In {\tt make\_w0}, $\etarhoec$ is used in the construction of the RHS
for the $\delta w_0$ equation.

%-----------------------------------------------------------------------------
% etarho_cc
%-----------------------------------------------------------------------------
\section{Using $\etarhocc$}
\subsection{Plane-Parallel}
In {\tt make\_psi}, $\psi = \etarhocc g$.

\subsection{Spherical}
In {\tt make\_w0}, $\etarhocc$ is used in the construction of the RHS
for the $\delta w_0$ equation.

