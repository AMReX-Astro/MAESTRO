\section{Implementing an SDC-enabled reaction network (no diffusion)}

Maestro integrates the following system of equations:
\begin{eqnarray}
\frac{\partial(\rho X_k)}{\partial t} &=& 
    -\nabla\cdot(\rho X_k\Ub) + \rho\omegadot_k,\label{eq:species}\\
\frac{\partial\Ub}{\partial t} &=& 
    -\Ub\cdot\nabla\Ub  - \frac{1}{\rho}\nabla\pi 
    - \frac{\rho-\rho_0}{\rho} g\eb_r,\label{eq:momentum}\\
\frac{\partial(\rho h)}{\partial t} &=& 
    -\nabla\cdot(\rho h\Ub) + \frac{Dp_0}{Dt} 
    + \rho\Hnuc ,\label{eq:enthalpy}
\end{eqnarray}
together with a constraint equation.

In the Strang-split version of Maestro, the reaction network solves just
the reaction portion of the species evolution equations along with a
temperature evolution equation.  This is done using a standard ODE solver
(like {\tt VODE}) on the system:
\begin{eqnarray}
\frac{dX_k}{dt} &=& \omegadot_k(\rho,X_k,T) \\
\frac{dT}{dt}   &=& 
    \frac{1}{c_p} \left ( -\sum_k \xi_k  \omegadot_k  + \Hnuc \right ).
\end{eqnarray}
Furthermore, we simplify the problem ``freezing'' the
thermodynamics---i.e., $c_p$ is evaluating at the start of the
integration and held constant thereafter.

In the SDC version, the VODE integration at the end of an SDC
iteration is responsible for updating all the thermodynamic quantities
\{$\rho$, $(\rho h)$, and $(\rho X_k$)\}, including both the advective
fluxes (incorporated via a piecewise constant advective flux source
term) and the reactions.  This provides a much stronger coupling between
the physical processes.  In particular, our system now looks like:
\begin{eqnarray}
\frac{d(\rho X_k)}{dt} &=& \rho \omegadot_k(\rho,X_k,T) + A_{\rho X_k} \\
\frac{d(\rho h)}{dt}   &=& \rho \Hnuc + A_{\rho h}
\end{eqnarray}
Here, $A_{\rho X_k}$ and $A_{\rho h}$ are piecewise-constant (in time)
approximations to the change in ${\rho X_k}$ and ${\rho h}$ (respectively)
due to the advection.  These are constructed by calling {\tt density\_advance}
and {\tt enthalpy\_advance} in Maestro and passed into the network solver
during the reaction step.  These terms provide approximate the following:
\begin{eqnarray}
A_{\rho X_k} &=& - \left [ \nabla \cdot (\rho X_k \Ub) \right ]^{n+1/2} \\
A_{\rho h}   &=& - \left [ \nabla \cdot (\rho h U) + \frac{Dp_0}{Dt} \right ]^{n+1/2}
\end{eqnarray}

The RHS routine that the ODE solver operates on will first construct
the density as:
\begin{equation}
\rho = \sum_k (\rho X_k)
\end{equation}
It will then derive the temperature from the equation of state.  If we
are running with {\tt use\_tfromp = T}, then we do
\begin{equation}
T = T(\rho, p_0, X_k)
\end{equation}
otherwise, we do
\begin{equation}
T = T(\rho, h, X_k)
\end{equation}
Note that in constrast to the Strang-split version, here we call the EOS
every time we enter the RHS routine.

The major changes from the non-SDC-enabled burners is the addition of
the advective terms to the system of ODEs, the fact that we integrate
$(\rho X_k)$ instead of just $X_k$, and the need to derive the
temperature from the input state for each RHS evaluation by {\tt
  VODE}.

Note also that the SDC integration by {\tt VODE} does not operate on 
the velocities at all.  That update is handled in the same fashion 
as the Strang-split version of the code.
