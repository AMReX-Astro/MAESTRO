\section{Derivatives With Respect to Composition}
\label{Sec:Derivatives With Respect to Composition}
In the following we assume that the molar mass of species $i$ is given by its
atomic mass number, $A_i = N_i + Z_i$ where $N_i$ is the number of neutrons
and $Z_i$ is the number of protons in the isotope.  \MarginPar{Note that this
  is common practice and that our EOS also makes this approximation.}  This 
is a slight approximation that ignores the mass difference between protons 
and neutrons as well as some minor binding energy terms.

The number density [cm$^-{3}$] of isotope $i$ is can be formed from the mass 
density and the molar mass of that isotope as follows:
\begin{equation}\label{eq:number density}
  n_i = \frac{\rho_i N_\text{A}}{A_i},
\end{equation}
where $N_\text{A}$ is Avogadro's number [\# / mole].  The molar abundance, 
$Y_i$, is a measure of the number of moles of species $i$ per gram in the 
system:
\begin{equation}\label{eq:molar abundance}
  Y_i = \frac{n_i}{\rho N_\text{A}} = \frac{\rho_i}{\rho}\frac{1}{A_i} 
  \equiv \frac{X_i}{A_i}
\end{equation}
where we have defined the mass fraction, $X_i = \frac{\rho_i}{\rho}$.  Note
\begin{equation}\label{eq:mass fraction sums to 1}
  \sum_i X_i = 1.
\end{equation}

We write the average molar mass and average proton number as:
\begin{align}\label{eq:abar}
  \bar{A} &= \frac{\sum_i A_i n_i}{\sum_i n_i} = \left(\sum_i X_i\right)
  \left(\sum_i \frac{X_i}{A_i}\right)^{-1}\\
  \label{eq:zbar}
  \bar{Z} &= \frac{\sum_i Z_i n_i}{\sum_i n_i} = \left(\sum_i Z_i
  \frac{X_i}{A_i}\right)\left(\sum_i \frac{X_i}{A_i}\right)^{-1}.
\end{align}
Our algorithm requires terms involving the derivative thermodynamic variables
($p$ or $e$, e.g.) with respect to composition.  Our EOS does not return such
derivatives but instead returns derivatives of these variables with respect 
to $\bar{A}$ and $\bar{Z}$.  Using the chain rule, we have
\begin{equation}\label{eq:p_xk}
  \frac{\ptl p}{\ptl X_k} = p_{X_k} = 
  \frac{\ptl p}{\ptl \bar{A}}\frac{\ptl \bar{A}}{\ptl X_k} + 
  \frac{\ptl p}{\ptl \bar{Z}}\frac{\ptl \bar{Z}}{\ptl X_k}.
\end{equation}
From \eqref{eq:abar} and \eqref{eq:zbar} we have
\begin{align}\label{eq:abar_X_k}
  \frac{\ptl \bar{A}}{\ptl X_k} &= \left(\sum_i \frac{X_i}{A_i}\right)^{-1}
  - \frac{\bar{A}^2}{A_k} = -\frac{\bar{A}}{A_k}\left(\bar{A} - A_k\right)\\
  \label{eq:zbar_X_k}
  \frac{\ptl \bar{Z}}{\ptl X_k} &= 
  \left(\frac{Z_k}{A_k}\right)\left(\sum_i \frac{X_i}{A_i}\right)^{-1}
  - \frac{\bar{Z}}{A_k}\left(\sum_i \frac{X_i}{A_i}\right)^{-1} = 
  -\frac{\bar{A}}{A_k}\left(\bar{Z} - Z_k\right),
\end{align}
where after differentiation we have used \eqref{eq:mass fraction sums to 1}
to write
\[
\left(\sum_i \frac{X_i}{A_i}\right)^{-1} = \bar{A}.
\]
We therefore have
\begin{equation}\label{eq:p_Xk_full}
  p_{X_k} = -\frac{\bar{A}}{A_k}\left(\bar{A} - A_k\right)
  \frac{\ptl p}{\ptl\bar{A}} - \frac{\bar{A}}{A_k}\left(\bar{Z} - Z_k\right)
  \frac{\ptl p}{\ptl\bar{Z}}.
\end{equation}
Before it was brought to our attention by Frank Timmes, we were missing the
second term in \eqref{eq:abar_X_k}.  The only place where such terms 
appear in our algorithm is in a sum over all species, such as:
\begin{equation}\label{eq:sum over species}
  \sum_i p_{X_i}\dot{\omega}_i = 
  -\bar{A}^2\frac{\ptl p}{\ptl\bar{A}}\sum_i \frac{\dot{\omega}_i}{A_i}
  +\bar{A}\frac{\ptl p}{\ptl\bar{A}}\sum_i \dot{\omega}_i
  -\bar{A}\bar{Z}\frac{\ptl p}{\ptl\bar{Z}}\sum_i \frac{\dot{\omega}_i}{A_i}
  +\bar{A}\frac{\ptl p}{\ptl \bar{Z}}\sum_i\frac{Z_i}{A_i}\dot{\omega}_i.
\end{equation}
\MarginPar{Note that if we are dealing with symmetric nuclei 
  (as we have been up through paper III) where $Z_k = N_k = \half A_k$ for all
  $k$, then the last term in \eqref{eq:sum over species} is identically zero
  as well.}
The second term in \eqref{eq:sum over species} is identically zero because
\[
\sum_k \dot{\omega}_k \equiv 0.
\]
This second term arises from what was added to \eqref{eq:abar_X_k} by 
Frank's correction.  Therefore, although important for individual derivatives
with respect to composition, this correction term has no effect on our 
solution.

\section{Adiabatic Excess}
\label{Sec:Adiabatic Excess}
The adiabatic excess, $\Delta\nabla$, is a quantity used to determine
if a system is stable ($\Delta\nabla < 0$) or unstable ($\Delta\nabla
> 0$) to convection under the Schwarzschild criterion (i.e. neglecting
compositional gradients).  Cox and Giuli (see chapter 9) define three
different ``adiabatic exponents'' that we will use:
\begin{eqnarray*}
  \Gamma_1 &\equiv&   \left(\frac{d\ln p}{d\ln\rho}\right)_\text{ad} \\
  \frac{\Gamma_2}{\Gamma_2-1} &\equiv& 
  \left(\frac{d\ln p}{d\ln T}\right)_\text{ad} \\
  \Gamma_3 - 1 &\equiv& \left(\frac{d\ln T}{d\ln\rho}\right)_\text{ad},
\end{eqnarray*} 
where the subscript ``ad'' means along an adiabat.  We can combine the
exponents to get the following relation
\begin{equation}\label{eq:Gamma relations}
  \Gamma_1 = \left(\frac{\Gamma_2}{\Gamma_2-1}\right)\left(\Gamma_3-1\right).
\end{equation}

The adiabatic excess is defined as
\begin{equation}\label{eq:adiabatic excess}
  \Delta\nabla = \nabla_\text{actual} - \nabla_\text{ad}
\end{equation}
where
\begin{equation}\label{eq:thermal gradient}
  \nabla \equiv \frac{d\ln T}{d\ln P}
\end{equation}
is the thermal gradient.  It is important to note that these thermal
gradients are only along the radial direction.  The ``actual''
gradient can be found from finite differencing the data whereas the
adiabatic term, $\nabla_\text{ad} = \left(\Gamma_2-1\right) /
\Gamma_2$, will need to be calculated at each point using
thermodynamic relations.  Our EOS only returns $\Gamma_1$ so we need
find another relation to use with \eqref{eq:Gamma relations} to solve
for the adiabatic excess.  

The Schwarzschild criterion does not care about changes in composition
and we therefore write $p = p(\rho,T)$ and
\begin{equation}\label{eq:dp}
  d\ln p = \chi_\rho d\ln\rho + \chi_T d\ln T
\end{equation}
where
\[
\chi_\rho = \left(\frac{d\ln p}{d\ln\rho}\right)_T,\qquad
\chi_T = \left(\frac{d\ln p}{d\ln T}\right)_\rho.
\]
Dividing \eqref{eq:dp} by $d\ln\rho$ and taking this along an adiabat
we have
\begin{equation}\label{eq:dp2}
  \left(\frac{d\ln p}{d\ln\rho}\right)_\text{ad} = \chi_\rho + \chi_T
  \left(\frac{d\ln T}{d\ln\rho}\right)_\text{ad}.
\end{equation}
Using the $\Gamma$'s, we have
\begin{equation}\label{eq:Gamma1 relation with Gamma2}
  \Gamma_1 = \chi_\rho + \chi_T\left(\Gamma_3-1\right).
\end{equation}
Combining \eqref{eq:Gamma relations} and \eqref{eq:Gamma1 relation
  with Gamma2} to eliminate $\Gamma_3$, we have:
\begin{equation}\label{eq:nabla_ad}
  \nabla_\text{ad} = \frac{\Gamma_1 - \chi_\rho}{\chi_T\Gamma_1}
\end{equation}
which uses only terms which are easily returned from an EOS call.


