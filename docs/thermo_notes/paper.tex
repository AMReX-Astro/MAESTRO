\documentclass[11pt]{article} 

\tolerance=600

\usepackage{amsmath,color,epsfig,epsf}

% Margins
\usepackage[lmargin=0.5in,rmargin=1.5in,tmargin=0.5in,bmargin=0.5in]{geometry}

\newcommand{\sfrac}[2]{\mathchoice
  {\kern0em\raise.5ex\hbox{\the\scriptfont0 #1}\kern-.15em/
   \kern-.15em\lower.25ex\hbox{\the\scriptfont0 #2}}
  {\kern0em\raise.5ex\hbox{\the\scriptfont0 #1}\kern-.15em/
   \kern-.15em\lower.25ex\hbox{\the\scriptfont0 #2}}
  {\kern0em\raise.5ex\hbox{\the\scriptscriptfont0 #1}\kern-.2em/
   \kern-.15em\lower.25ex\hbox{\the\scriptscriptfont0 #2}}
  {#1\!/#2}}

\def\half  {\frac{1}{2}}
\newcommand{\myhalf}{\sfrac{1}{2}}

\def\ptl   {\partial}


\title{Notes on thermodnyamic relations in {\tt MAESTRO}}

\setlength{\marginparwidth}{1.0in}
\newcommand{\MarginPar}[1]{\marginpar{%
\vskip-\baselineskip %raise the marginpar a bit
\raggedright\tiny\sffamily
\hrule\smallskip{\color{red}#1}\par\smallskip\hrule}}

\begin{document}

  \maketitle
  \tableofcontents

  \clearpage

  \section{Derivatives With Respect to Composition}
  \label{Sec:Derivatives With Respect to Composition}
  In the following we assume that the molar mass of species $i$ is given by its
  atomic mass number, $A_i = N_i + Z_i$ where $N_i$ is the number of neutrons
  and $Z_i$ is the number of protons in the isotope.  \MarginPar{Note that this
    is common practice and that our EOS also makes this approximation.}  This 
  is a slight approximation that ignores the mass difference between protons 
  and neutrons as well as some minor binding energy terms.
  
  The number density [cm$^-{3}$] of isotope $i$ is can be formed from the mass 
  density and the molar mass of that isotope as follows:
  \begin{equation}\label{eq:number density}
    n_i = \frac{\rho_i N_\text{A}}{A_i},
  \end{equation}
  where $N_\text{A}$ is Avogadro's number [\# / mole].  The molar abundance, 
  $Y_i$, is a measure of the number of moles of species $i$ per gram in the 
  system:
  \begin{equation}\label{eq:molar abundance}
    Y_i = \frac{n_i}{\rho N_\text{A}} = \frac{\rho_i}{\rho}\frac{1}{A_i} 
    \equiv \frac{X_i}{A_i}
  \end{equation}
  where we have defined the mass fraction, $X_i = \frac{\rho_i}{\rho}$.  Note
  \begin{equation}\label{eq:mass fraction sums to 1}
    \sum_i X_i = 1.
  \end{equation}
  
  We write the average molar mass and average proton number as:
  \begin{align}\label{eq:abar}
    \bar{A} &= \frac{\sum_i A_i n_i}{\sum_i n_i} = \left(\sum_i X_i\right)
    \left(\sum_i \frac{X_i}{A_i}\right)^{-1}\\
    \label{eq:zbar}
    \bar{Z} &= \frac{\sum_i Z_i n_i}{\sum_i n_i} = \left(\sum_i Z_i
    \frac{X_i}{A_i}\right)\left(\sum_i \frac{X_i}{A_i}\right)^{-1}.
  \end{align}
  Our algorithm requires terms involving the derivative thermodynamic variables
  ($p$ or $e$, e.g.) with respect to composition.  Our EOS does not return such
  derivatives but instead returns derivatives of these variables with respect 
  to $\bar{A}$ and $\bar{Z}$.  Using the chain rule, we have
  \begin{equation}\label{eq:p_xk}
    \frac{\ptl p}{\ptl X_k} = p_{X_k} = 
    \frac{\ptl p}{\ptl \bar{A}}\frac{\ptl \bar{A}}{\ptl X_k} + 
    \frac{\ptl p}{\ptl \bar{Z}}\frac{\ptl \bar{Z}}{\ptl X_k}.
  \end{equation}
  From \eqref{eq:abar} and \eqref{eq:zbar} we have
  \begin{align}\label{eq:abar_X_k}
    \frac{\ptl \bar{A}}{\ptl X_k} &= \left(\sum_i \frac{X_i}{A_i}\right)^{-1}
    - \frac{\bar{A}^2}{A_k} = -\frac{\bar{A}}{A_k}\left(\bar{A} - A_k\right)\\
    \label{eq:zbar_X_k}
    \frac{\ptl \bar{Z}}{\ptl X_k} &= 
    \left(\frac{Z_k}{A_k}\right)\left(\sum_i \frac{X_i}{A_i}\right)^{-1}
    - \frac{\bar{Z}}{A_k}\left(\sum_i \frac{X_i}{A_i}\right)^{-1} = 
    -\frac{\bar{A}}{A_k}\left(\bar{Z} - Z_k\right),
  \end{align}
  where after differentiation we have used \eqref{eq:mass fraction sums to 1}
  to write
  \[
  \left(\sum_i \frac{X_i}{A_i}\right)^{-1} = \bar{A}.
  \]
  We therefore have
  \begin{equation}\label{eq:p_Xk_full}
    p_{X_k} = -\frac{\bar{A}}{A_k}\left(\bar{A} - A_k\right)
    \frac{\ptl p}{\ptl\bar{A}} - \frac{\bar{A}}{A_k}\left(\bar{Z} - Z_k\right)
    \frac{\ptl p}{\ptl\bar{Z}}.
  \end{equation}
  Before it was brought to our attention by Frank Timmes, we were missing the
  second term in \eqref{eq:abar_X_k}.  The only place where such terms 
  appear in our algorithm is in a sum over all species, such as:
  \begin{equation}\label{eq:sum over species}
    \sum_i p_{X_i}\dot{\omega}_i = 
    -\bar{A}^2\frac{\ptl p}{\ptl\bar{A}}\sum_i \frac{\dot{\omega}_i}{A_i}
    +\bar{A}\frac{\ptl p}{\ptl\bar{A}}\sum_i \dot{\omega}_i
    -\bar{A}\bar{Z}\frac{\ptl p}{\ptl\bar{Z}}\sum_i \frac{\dot{\omega}_i}{A_i}
    +\bar{A}\frac{\ptl p}{\ptl \bar{Z}}\sum_i\frac{Z_i}{A_i}\dot{\omega}_i.
  \end{equation}
  \MarginPar{Note that if we are dealing with symmetric nuclei 
  (as we have been up through paper III) where $Z_k = N_k = \half A_k$ for all
  $k$, then the last term in \eqref{eq:sum over species} is identically zero
  as well.}
  The second term in \eqref{eq:sum over species} is identically zero because
  \[
  \sum_k \dot{\omega}_k \equiv 0.
  \]
  This second term arises from what was added to \eqref{eq:abar_X_k} by 
  Frank's correction.  Therefore, although important for individual derivatives
  with respect to composition, this correction term has no effect on our 
  solution.

\end{document}
