
Once {\sf MESA} was installed and working properly on a local machine (e.g. 
Bender), I tried installing it on Hopper, which is a Cray XE6 machine. On 
Bender, which is an Intel(R) Xeon(R) X5650 machine, the compiler was 
{\tt gcc} version 4.7.0, on Hopper I tried compiling {\sf MESA} with the 
Cray Compiler using version 8.0.5. The module that was loaded to enable this 
was {\tt PrgEnv-cray}.

There are several steps needed to compile {\sf MAESTRO} with {\sf MESA} on 
Hopper:
\begin{enumerate}
\item Download the {\tt MESA} source code using the command described in 
Section \ref{sec:setupmesa}
\item Edit the {\tt makefile\_header} in the {\tt utils} directory
\item Edit the {\tt build\_and\_test} script in the {\tt utils} directory
\item Edit the {\tt chem\_lib.f} file in the {\tt chem} directory
\item Edit the {\tt mtx} makefile
\item Edit the makefiles in {\tt screen}, {\tt utils}, {\tt num} and 
{\tt interp\_1d}
\item Edit the {\tt install} script in the main {\sf MESA} directory
\item Edit the {\tt BoxLib/Tools/F\_mk/GMakedefs.mak} makefile
\item Edit the {\tt BoxLib/Tools/F\_mk/comps/Linux\_cray.mak} file
\item Install {\sf MESA}
\item Compile {\sf MAESTRO}
\end{enumerate}

%%%%%%%%%%%%%%%%%%%%%%%%%%%%%%%%%%%%%%%%%%%%%%%%%%%%%%%%%%%%%%%%%%%%%%%
\subsection{{\tt makefile\_header}}
In following the setup outlined in this document, I changed the compilers, 
used the source version of {\tt LAPACK} and {\tt BLAS}, turned off 
{\tt PGPLOT} and turned off the {\tt SE} formatting library. I also added the 
appropriate compiler flags. The results of what I changed are shown below:
\begin{lstlisting}[language=make,mathescape=false]
  # Cray C-compiler
  CC = cc

  # Cray Fortran-Compiler
  FC = ftn

  # if you need special flags for the compiler, define them here:
  SPECIAL_FC_FLAGS = -h mpi0 -target=linux -emf
  # -h mpi0 disables mpi optimization
  # -target=linux specifies the build machine as linux based
  # -emf: the m flag says create NAME.mod files, the f flag says
  #        convert those to name.mod files

  # must explicity define all compilation flags
  FCbasic = $(SPECIAL_FC_FLAGS)
  FCimpno = -eI
  FCchecks =
  FCwarn = -m 3
  FCfixed = -f fixed -N 132
  FCfixed72 = -f fixed -N 72
  FCfree = -f free
  FCopt = -O 1
  FCdebug = -g -O0 -R bps
  FCstatic =

  FC_fixed_preprocess = -eZ
  FC_free_preprocess = -eZ

  # Cray compilers have omp ON by default
  FCopenmp =
  # to disable omp:
  #FCopenmp = -h noomp 
\end{lstlisting}

%%%%%%%%%%%%%%%%%%%%%%%%%%%%%%%%%%%%%%%%%%%%%%%%%%%%%%%%%%%%%%%%%%%%%%%
\subsection{{\tt build\_and\_test} script}

The {\tt build\_and\_test} script in the {\tt utils} directory is 
responsible for compiling the main module directory as well as compiling 
and running the test directory. We want to turn off the compiling and 
running of the test directory to ensure that {\sf MESA} compiles successfully 
(it failed to install otherwise). A portion of the {\tt build\_and\_test} 
script looks like this:
\begin{lstlisting}[language=bash,mathescape=false]
  cd make
  make
  check_okay
  cd ../test
  ./mk
  check_okay
  if [ -x test_quietly ]
  then
          ./test_quietly
          check_okay
  fi
  ./ck >& diff.txt
\end{lstlisting}
we wish to change it to this:
\begin{lstlisting}[language=bash,mathescape=false]
  cd make
  make
  check_okay
  #cd ../test
  #./mk
  check_okay
  if [ -x test_quietly ]
  then
  #        ./test_quietly
          check_okay
  fi
  #./ck >& diff.txt
\end{lstlisting}
This disables the compiling and running of the test directory.


%%%%%%%%%%%%%%%%%%%%%%%%%%%%%%%%%%%%%%%%%%%%%%%%%%%%%%%%%%%%%%%%%%%%%%%
\subsection{{\tt chem\_lib.f} Source File}

Open the {\tt chem\_lib.f} file located in {\tt chem/public}. In the 
{\tt generate\_nuclide\_set} subroutine, the implied do loop must be removed 
and replaced with an actual do loop. For an unknown reason, the implied do 
loop was causing a Segmentation Fault on Hopper with the Cray compiler. The 
implied do loop:
\begin{lstlisting}[language=fortran,mathescape=false]
  set = [(nuclide_set(names(i), i), i=1, size(names))]
\end{lstlisting}
should be changed to an explicit do loop:
\begin{lstlisting}[language=fortran,mathescape=false]
  do i=1, size(names)
     set(i) = nuclide_set(names(i), i)
  enddo
\end{lstlisting}


%%%%%%%%%%%%%%%%%%%%%%%%%%%%%%%%%%%%%%%%%%%%%%%%%%%%%%%%%%%%%%%%%%%%%%%
\subsection{{\tt mtx} makefile}

In the {\tt mtx/make/makefile}, there are a few compiler flags that were hard 
coded in. There is a {\tt -w} flag in five places and a {\tt -fno-common} 
flag in one place. The {\tt -w} flag instances look like this:
\begin{lstlisting}[language=make,mathescape=false]
  %.o: $(MODULE_DIR)/lapack_src/%.f
  	$(COMPILE_XTRA) -w $<

  # must turn off optimization for dlamch or can get infinite loop!!!
  dlamch.o: $(MODULE_DIR)/blas_src/dlamch.f
  	$(COMPILE_XTRA_NO_OPT) -w $<
\end{lstlisting}
The {\tt -w} should be deleted. There are three more instances immediately 
after these lines. The {\tt -fno-common} flag appears here:
\begin{lstlisting}[language=make,mathescape=false]
  KLU_C = $(CC) -O3 -fno-common -fexceptions
  KLU_I = -I$(KLU_DIR)
\end{lstlisting}
All three of the manually added flags ({\tt -O3}, {\tt -fno-common} and 
{\tt -fexceptions}) should be deleted from the {\tt KLU\_C} line.

%%%%%%%%%%%%%%%%%%%%%%%%%%%%%%%%%%%%%%%%%%%%%%%%%%%%%%%%%%%%%%%%%%%%%%%
\subsection{makefiles in {\tt screen}, {\tt num}, {\tt utils} and 
{\tt interp\_1d}}

When creating the library for these four modules, there is a {\tt LIB\_DEFS} 
variable in the makefile that holds the {\tt *\_def.o} file, e.g. 
{\tt utils\_def.o} or {\tt num\_def.o}. This file is never loaded into the 
library as seen below for the {\tt utils} directory. Certain modules have a 
{\tt *\_def.f} file with both definitions and declarations of data that will 
be needed at runtime, so they must be included in the library. Other modules 
have a {\tt *\_def.f} file that contains only definitions and no data 
declarations and therefor only needs to be included at compile time. The Cray 
compiler on Hopper for whatever reason complained that there were undefined 
references to certain {\tt *\_def.o} files. The undefined references only 
occurred for modules whose {\tt *\_def.f} file did not need to be included in 
the library. The fix was to manually include the {\tt *\_def.o} file in the 
library for all of the necessary module directories. The {\tt utils} directory 
is shown here:
\begin{lstlisting}[language=make,mathescape=false]
  LIB = libutils.a
  LIB_DEFS = utils_def.o
  LIB_OBJS = $(UTILS_ISNAN).o utils_nan.o utils_dict.o utils_lib.o

  $(LIB) : $(LIB_DEFS) $(LIB_OBJS)
  	$(LIB_TOOL) $(LIB) $(LIB_OBJS)
\end{lstlisting}
To include the {\tt LIB\_DEFS} file, change the above code to the following:
\begin{lstlisting}[language=make,mathescape=false]
  LIB = libutils.a
  LIB_DEFS = utils_def.o
  LIB_OBJS = $(UTILS_ISNAN).o utils_nan.o utils_dict.o utils_lib.o

  $(LIB) : $(LIB_DEFS) $(LIB_OBJS)
  	$(LIB_TOOL) $(LIB) $(LIB_DEFS) $(LIB_OBJS)
\end{lstlisting}
This must be done for each makefile in the following directories: 
{\tt screen/make}, {\tt num/make}, {\tt utils/make} and {\tt interp\_1d/make}.

%%%%%%%%%%%%%%%%%%%%%%%%%%%%%%%%%%%%%%%%%%%%%%%%%%%%%%%%%%%%%%%%%%%%%%%
\subsection{{\tt install} script}

Certain modules that are unneccessary for use with {\sf MAESTRO} do not need 
to be installed (and also cause some sort of installation error). In the 
main {\sf MESA} directory, edit the {\tt install} script and comment out 
the following four lines: {\tt do\_one sample}, {\tt do\_one star}, 
{\tt do\_one adipls} and {\tt do\_one astero}.

%%%%%%%%%%%%%%%%%%%%%%%%%%%%%%%%%%%%%%%%%%%%%%%%%%%%%%%%%%%%%%%%%%%%%%%
\subsection{{\sf BoxLib Linux\_cray.mak} file}

Two flags used in the compilation of {\sf MESA} with the Cray compiler are 
important; the {\tt -e m} or {\tt -em} flag and the {\tt -e f} or {\tt -ef} 
flag. The {\tt -em} flag tells the compiler to create {\tt .mod} files of the 
form {\tt NAME.mod}. The {\tt -e f} flag, which must be used with the 
{\tt -em} flag, instead outputs {\tt .mod} files of the form {\tt name.mod}. 
This is important for {\sf MESA} because the {\tt export} scripts copy the 
various libraries and {\tt .mod} files to the correct directories. The 
{\tt export} script in the {\tt const} directory for example looks like this:
\begin{lstlisting}[language=bash,mathescape=false]
  cp make/const_lib.mod ../include
  cp make/const_def.mod ../include

  cp make/libconst.a ../lib
  cd ../lib
  ranlib libconst.a
\end{lstlisting}
If the {\tt -ef} flag is not enabled, the {\tt .mod} file will be 
{\tt CONST\_LIB.mod} and will not get copied to the {\tt include} directory 
(the library is unaffected). The default for {\sf MAESTRO} is to not use 
the {\tt -ef} flag, but it does use the {\tt -em} flag. To make {\sf MAESTRO} 
and {\sf MESA} compatible, I added the {\tt -ef} flag to the 
{\tt Linux\_cray.mak} file located in {\tt BoxLib/Tools/F\_mk/comps}. I also 
added a few more debugging flags:
\begin{lstlisting}[language=make,mathescape=false]
  FFLAGS   += -J $(mdir) -I $(mdir) -emf
  F90FLAGS += -J $(mdir) -I $(mdir) -emf

  ifdef NDEBUG
    FFLAGS   += -O 1
    F90FLAGS += -O 1
    CFLAGS   += -O 1
  else
    FFLAGS   += -g -O0 -R bps
    F90FLAGS += -g -O0 -R bps
    CFLAGS   += -g -O0 -h bounds
  endif 
\end{lstlisting}

%%%%%%%%%%%%%%%%%%%%%%%%%%%%%%%%%%%%%%%%%%%%%%%%%%%%%%%%%%%%%%%%%%%%%%%
\subsection{{\sf BoxLib GMakedefs.mak} makefile}

In {\sf MAESTRO}, {\tt .mod} files are located in the {\tt \$(tdir)/m} 
directory, where {\tt tdir = t/\$(suf)} and {\tt \$(suf)} was in this case 
{\tt Linux.Cray}. This means the {\sf MAESTRO} routines are expecting to find 
module files with the name {\tt t/Linux.Cray/m/network.mod}. The addition of 
the {\tt -ef} flag affects the entire name of the module so the Cray compiler 
was creating files with the name {\tt t/linux.cray/m/network.mod}. To solve 
this, I modified what {\tt tdir} is defined to be when using the Cray compiler. 
This was done in the {\tt BoxLib/Tools/F\_mk/GMakedefs.mak} makefile by using 
a simple if statement:
\begin{lstlisting}[language=make,mathescape=false]
  ifeq ($(COMP), Cray)
    tdir = t/$(shell echo $(suf) | tr A-Z a-z)
  else
    tdir = t/$(suf)
  endif
  odir = $(tdir)/o
  mdir = $(tdir)/m
\end{lstlisting}
The shell command {\tt echo \$(suf) | tr A-Z a-z} has the effect of converting 
{\tt \$(suf)} to lowercase.

%%%%%%%%%%%%%%%%%%%%%%%%%%%%%%%%%%%%%%%%%%%%%%%%%%%%%%%%%%%%%%%%%%%%%%%
\subsection{Install {\sf MESA} and Compile {\sf MAESTRO}}

At this point the {\tt install} script in the top {\sf MESA} directory can be 
executed to install {\sf MESA}. This will compile the source code, generate 
the libraries and copy the necessary files to the correct directories. Once, 
{\sf MESA} is installed, {\sf MAESTRO} can be compiled and linked such that 
it can use {\sf MESA}'s EOS and/or network. Make sure the makefile variable 
{\tt COMP} is set to {\tt Cray} in the {\sf MAESTRO} problem directory 
so the Cray compiler is used.


