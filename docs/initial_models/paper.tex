\documentclass[11pt]{article} 

\tolerance=600

\usepackage{amsmath,color}

% Margins
\usepackage[lmargin=0.5in,rmargin=1.5in,tmargin=0.5in,bmargin=0.5in]{geometry}

\def\half  {\frac{1}{2}}
\def\dt    {\Delta t}
\def\kth   {k_{\rm th}}
\def\init  {\rm init}

\def\edge  {\rm EDGE}
\def\mac   {\rm MAC}
\def\trans {\rm TRANS}

\def\eb    {{\bf e}}
\def\fb    {{\bf f}}
\def\ib    {{\bf i}}
\def\Ub    {{\bf U}}

\def\Ubt   {\widetilde{\bf U}}
\def\ut    {\tilde{u}}
\def\vt    {\tilde{v}}
\def\wt    {\tilde{w}}

\title{Notes on generating initial models for mapping into {\tt MAESTRO}}

\setlength{\marginparwidth}{1.0in}
\newcommand{\MarginPar}[1]{\marginpar{%
\vskip-\baselineskip %raise the marginpar a bit
\raggedright\tiny\sffamily
\hrule\smallskip{\color{red}#1}\par\smallskip\hrule}}

\begin{document}

\maketitle
\tableofcontents

\clearpage

\section{Creating the Initial Data}
We begin with data from a model file, which we shall refer to as the ``model'' data.  Here is a list of the model data for each test problem:
\begin{eqnarray}
{\tt test2} & \rightarrow & {\tt ???} \nonumber \\
{\tt test\_convect} & \rightarrow & {\tt ???} \nonumber \\
{\tt wdconvect} & \rightarrow & {\tt ???} \nonumber \\
{\tt spherical\_heat} & \rightarrow & {\tt ???} \nonumber \\
{\tt xrb} & \rightarrow & {\tt initial\_models/xrb/sorted\_xrb.raw} \nonumber
\end{eqnarray}
We use a fortran subroutine to modify the model data to create ``initial'' data, denoted with the subcript ``init''.  The model data needs to be modified because there is no guarantee that it was generated with the same equation of state (EOS) that {\tt MAESTRO} uses, nor is there any guarantee that the pressure and density are in hydrostatic equilibrium (HSE) according to our choice of discretization given in equation (\ref{HSE Discretization}).  Here is a list of the fortran subroutine for each test problem:
\begin{eqnarray}
{\tt test2} & \rightarrow & {\tt ???} \nonumber \\
{\tt test\_convect} & \rightarrow & {\tt ???} \nonumber \\
{\tt wdconvect} & \rightarrow & {\tt ???} \nonumber \\
{\tt spherical\_heat} & \rightarrow & {\tt ???} \nonumber \\
{\tt xrb} & \rightarrow & {\tt initial\_models/xrb/init\_1d.f90} \nonumber
\end{eqnarray}
The fortran subroutine returns $p^{\init}, \rho^{\init}, T^{\init},$ and $X_k^{\init}$.  The fortran subroutine uses an iterative procedure to return initial data that is both thermodynamically consistent with the {\tt MAESTRO} EOS, and in HSE,
\begin{equation}
\frac{p_r^{\init} - p_{r-1}^{\init}}{\Delta r} = \frac{\rho_r^{\init} + \rho_{r-1}^{\init}}{2}g,\label{HSE Discretization}
\end{equation}
to a user-defined tolerance.

The initial data needs be generated at the same resolution as the base state for single-level problems and at the same resolution of the finest resolution base state for multi-level plane-parallel problems.  For spherical problems, we force $\Delta r = \Delta x$.
\section{Creating the Base State and Full State}
Given $p^{\init}, \rho^{\init}, T^{\init},$ and $X^{\init}$, this section describes how the base state ($\rho_0$ and $p_0$) and full state ($\rho, h, X$, and $T$) are computed.  We require that $\rho_0 = \overline\rho$, the base state is in HSE according to equation (\ref{HSE Discretization}), and the full state is in thermodynamic equilibrium with $p_0$.  Here are the steps:
\begin{enumerate}
\item Call the EOS to obtain a 1D $h^{\init} = h^{\init}(\rho^{\init},T^{\init},X^{\init})$.  Note that in the code, $p^{\init}$ is also overwritten, but the value of $p^{\init}$ does not change since we called $p^{\init} = p^{\init}(\rho^{\init},T^{\init},X^{\init})$ at the end of the previous fortran subroutine.
\item For plane-parallel problems, fill $\rho^{\init}, h^{\init}, X^{\init}$, and $T^{\init}$ onto a multifab to obtain the full state $\rho, h, X$, and $T$.  For spherical problems, fill $\rho^{\init}, p^{\init}$, and $X^{\init}$ onto a multifab to obtain the full state $\rho, p$, and $X$, and then call the EOS to obtain the full state $h,T = h,T(\rho,p,X)$.
\item For spherical problems, skip the remaining steps and instead set $p_0 = p^{\init}$ and $\rho_0 = \rho^{\init}$.
\item If {\tt perturb\_model} = F, skip the remaining steps and instead set $p_0 = p^{\init}$ and $\rho_0 = \rho^{\init}$.
\item Perturb $T$, then compute $\rho,h = \rho,h(p^{\init},T,X)$.
\item Set $\rho_0 = \overline\rho$.
\item Compute $p_0$ using equation (\ref{HSE Discretization}).
\item Compute the full state $T,h = T,h(\rho,p_0,X)$.
\end{enumerate}
Now $\rho_0 = \overline\rho$, the base state is in HSE, and the full state is in thermodynamic equilibrium with $p_0$.
\end{document}
