\documentclass[11pt]{article} 

\tolerance=600

\usepackage{amsmath,color}

% Margins
\usepackage[lmargin=0.5in,rmargin=0.5in,tmargin=0.75in,bmargin=0.75in]{geometry}

\def\half  {\frac{1}{2}}
\def\dt    {\Delta t}
\def\kth   {k_{\rm th}}

\def\edge  {\rm EDGE}
\def\mac   {\rm MAC}
\def\trans {\rm TRANS}

\def\eb    {{\bf e}}
\def\fb    {{\bf f}}
\def\ib    {{\bf i}}
\def\Ub    {{\bf U}}

\def\Ubt   {\widetilde{\bf U}}
\def\ut    {\tilde{u}}
\def\vt    {\tilde{v}}
\def\wt    {\tilde{w}}

\title{Notes on generating initial models for mapping into {\tt MAESTRO}}

\setlength{\marginparwidth}{0.75in}
\newcommand{\MarginPar}[1]{\marginpar{%
\vskip-\baselineskip %raise the marginpar a bit
\raggedright\tiny\sffamily
\hrule\smallskip{\color{red}#1}\par\smallskip\hrule}}

\begin{document}

\maketitle
\tableofcontents

\section{Creating the initial model}
\begin{itemize}
\item The initial model should be but into HSE according to $(p_{i+1} - p_i)/\Delta r = half (\rho_{i+1} - \rho_i) g_{i+\half}$

\item The initial model should be created at the same resolution as the base state.  This means that
when you change resolution of your simulation, use a new initial model.

\item The initial model should be thermodynamically consistent with the EOS that {\tt MAESTRO}
uses.

\end{itemize}

\section{Spherical notes}
\begin{itemize}
\item Spherical runs use $p$, $(\rho X)$, and $\rho$ from the initial model.  These are then
mapped onto the Cartesian grid, and then $(rho h)$ is computed via the EOS.
\end{itemize}

\end{document}
