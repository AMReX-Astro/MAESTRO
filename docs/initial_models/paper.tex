\documentclass[11pt]{article} 

\tolerance=600

\usepackage{amsmath,color}

% Margins
\usepackage[lmargin=0.5in,rmargin=1.5in,tmargin=0.5in,bmargin=0.5in]{geometry}

\newcommand{\sfrac}[2]{\mathchoice
  {\kern0em\raise.5ex\hbox{\the\scriptfont0 #1}\kern-.15em/
   \kern-.15em\lower.25ex\hbox{\the\scriptfont0 #2}}
  {\kern0em\raise.5ex\hbox{\the\scriptfont0 #1}\kern-.15em/
   \kern-.15em\lower.25ex\hbox{\the\scriptfont0 #2}}
  {\kern0em\raise.5ex\hbox{\the\scriptscriptfont0 #1}\kern-.2em/
   \kern-.15em\lower.25ex\hbox{\the\scriptscriptfont0 #2}}
  {#1\!/#2}}

\def\half  {\frac{1}{2}}
\newcommand{\myhalf}{\sfrac{1}{2}}

\def\dt    {\Delta t}
\def\kth   {k_{\rm th}}
\def\init  {\rm init}
\def\model {\rm model}
\def\raw   {\rm raw}

\def\edge  {\rm EDGE}
\def\mac   {\rm MAC}
\def\trans {\rm TRANS}

\def\eb    {{\bf e}}
\def\fb    {{\bf f}}
\def\ib    {{\bf i}}
\def\Ub    {{\bf U}}

\def\Ubt   {\widetilde{\bf U}}
\def\ut    {\tilde{u}}
\def\vt    {\tilde{v}}
\def\wt    {\tilde{w}}

\title{Notes on generating initial models for mapping into {\tt MAESTRO}}

\setlength{\marginparwidth}{1.0in}
\newcommand{\MarginPar}[1]{\marginpar{%
\vskip-\baselineskip %raise the marginpar a bit
\raggedright\tiny\sffamily
\hrule\smallskip{\color{red}#1}\par\smallskip\hrule}}

\begin{document}

\maketitle
\tableofcontents

\clearpage

\section{Creating the Model Data from Raw Data}
We begin with raw data, $\rho^{\raw}, T^{\raw}, p^{\raw}$, and $X^{\raw}$.  Here is the raw data file for each test problem:
\begin{eqnarray}
{\tt test2} & \rightarrow & {\tt ???} \nonumber \\
{\tt test\_convect} & \rightarrow & {\tt ???} \nonumber \\
{\tt wdconvect} & \rightarrow & {\tt ???} \nonumber \\
{\tt spherical\_heat} & \rightarrow & {\tt ???} \nonumber \\
{\tt xrb} & \rightarrow & {\tt initial\_models/xrb/sorted\_xrb.raw} \nonumber
\end{eqnarray}
Using the raw data, we need to generate model data with the same resolution as the base state.  Note that for spherical problems we set $\Delta r = \Delta x$.  We use a fortran subroutine to interpolate the raw data, yielding the model data, $\rho^{\model}, T^{\model}, p^{\model}$, and $X^{\model}$.  The fortran subroutine then uses an iterative procedure to modify the model data so that it is thermodynamically consistent with the {\tt MAESTRO} equation of state (EOS), and also satisfies our chosen hydrostatic equilibrium (HSE) discretization,
\begin{equation}
\frac{p_r - p_{r-1}}{\Delta r} = \frac{\rho_r + \rho_{r-1}}{2}g,\label{HSE Discretization}
\end{equation}
to a user defined tolerance.  Here are the fortran subroutines for each test problem:
\begin{eqnarray}
{\tt test2} & \rightarrow & {\tt ???} \nonumber \\
{\tt test\_convect} & \rightarrow & {\tt ???} \nonumber \\
{\tt wdconvect} & \rightarrow & {\tt ???} \nonumber \\
{\tt spherical\_heat} & \rightarrow & {\tt ???} \nonumber \\
{\tt xrb} & \rightarrow & {\tt initial\_models/xrb/init\_1d.f90} \nonumber
\end{eqnarray}
The model data is not generated at run-time - it must be generated in advance of running any MAESTRO examples.  The inputs file should point to the file containing the model data.
\subsection{Multilevel Plane-Parallel}
We need to generate model data with the same resolution as the finest base state resolution.
\section{Creating the Initial Data from the Model Data}
In {\tt base\_state.f90}, using the subroutine {\tt init\_base\_state.f90} (which is actually a terrible name since we are setting the initial data, not the base state), we set the initial data $\rho^{\init}, T^{\init}, p^{\init}$ and $X^{\init}$ equal to the model data.  Then, we set $h^{\init} = h^{\init}(\rho^{\init},T^{\init},X^{\init})$.  Note that in the code, $p^{\init}$ is also overwritten, but the value of $p^{\init}$ does not change since we called $p^{\model} = p^{\model}(\rho^{\model},T^{\model},X^{\model})$ at the end of the previous fortran subroutine.
\subsection{Multilevel Plane-Parallel}
For the finest level, we can set the initial data equal to the model data.  For the non-finest levels, we use linear interpolation to compute the initial data.  For cells that are covered by finer cells, we average $\rho^{\init}, (\rho h)^{\init},$ and $(\rho X)^{\init}$ from the fine cell to get the coarse cell value.  Then, for every cell not at the finest level, we call the EOS to compute $p^{\init},T^{\init} = p^{\init},T^{\init}(\rho^{\init},h^{\init},X^{\init})$.\MarginPar{Now $p^{\init}$ and $\rho^{\init}$ are not in HSE.  Is this a problem?}
\section{Creating the Base State and Full State}
Given $p^{\init}, \rho^{\init}, T^{\init},$ and $X^{\init}$, this section describes how the base state ($\rho_0$ and $p_0$) and full state ($\rho, h, X$, and $T$) are computed.  We require that $\rho_0 = \overline\rho$, the base state is in HSE according to equation (\ref{HSE Discretization}), and the full state is in thermodynamic equilibrium with $p_0$.  Here are the steps:
\begin{enumerate}
\item For plane-parallel problems, fill $\rho^{\init}, h^{\init}, X^{\init}$, and $T^{\init}$ onto a multifab to obtain the full state $\rho, h, X$, and $T$.  For spherical problems, fill $\rho^{\init}, p^{\init}$, and $X^{\init}$ onto a multifab to obtain the full state $\rho, p$, and $X$, and then call the EOS to obtain the full state $h,T = h,T(\rho,p,X)$.  Note that the multifab version of $p$ is temporary, and can be deallocated after this step.
\item For spherical problems, skip the remaining steps and instead set $p_0 = p^{\init}$ and $\rho_0 = \rho^{\init}$.
\item If {\tt perturb\_model} = F, skip the remaining steps and instead set $p_0 = p^{\init}$ and $\rho_0 = \rho^{\init}$.
\item Perturb $T$, then compute $\rho = \rho(p^{\init},T,X)$.
\item Set $\rho_0 = \overline\rho$.
\item Compute $p_0$ using equation (\ref{HSE Discretization}), using $p_{0,r=0}$ as the initial condition.
\item Compute the full state $T,h = T,h(\rho,p_0,X)$.
\end{enumerate}
Now $\rho_0 = \overline\rho$, the base state is in HSE, and the full state is in thermodynamic equilibrium with $p_0$.
\subsection{Multilevel Plane-Parallel}
Integrating the HSE discretization upward, when we encounter a change from coarse to fine, here is an equation to integrate up to the coarse-fine interface
\begin{equation}
\frac{p_{r-\myhalf}^l - p_{\sfrac{r}{2}-1}^{l-1}}{\Delta r^{l-1}/2} = \frac{\rho_{r-\myhalf}^l + \rho_{\sfrac{r}{2}-1}^{l-1}}{2}g ~~~ \rightarrow ~~~ p_{r-\myhalf}^l = p_{\sfrac{r}{2}-1}^{l-1} + \frac{\Delta r^{l-1} g}{4}\left(\rho_{r-\myhalf}^l + \rho_{\sfrac{r}{2}-1}^{l-1}\right).
\end{equation}
Here is an equation to integrate from the coarse-fine interface up to the cell center
\begin{equation}
\frac{p_r^l - p_{r-\myhalf}^l}{\Delta r^l/2} = \frac{\rho_r^l + \rho_{r-\myhalf}^l}{2}g ~~~ \rightarrow ~~~ p_r^l = p_{r-\myhalf}^l + \frac{\Delta r^l g}{4}\left(\rho_r^l + \rho_{r-\myhalf}^l\right).
\end{equation}
Combining equations gives
\begin{equation}
p_r^l = p_{\sfrac{r}{2}-1}^{l-1} + \frac{\Delta r^{l-1} g}{4}\left(\rho_{r-\myhalf}^l + \rho_{\sfrac{r}{2}-1}^{l-1}\right) + \frac{\Delta r^l g}{4}\left(\rho_r^l + \rho_{r-\myhalf}^l\right).
\end{equation}
We can simplify using
\begin{equation}
\Delta r^{l-1} = 2\Delta r^l,
\end{equation}
\begin{equation}
\rho_{r-\myhalf}^l = \frac{2}{3}\rho_r^l + \frac{1}{3}\rho_{\sfrac{r}{2}-1}^{l-1}.
\end{equation}
Simplifying
\begin{eqnarray}
p_r^l &=& p_{\sfrac{r}{2}-1}^{l-1} + \frac{\Delta r^l g}{2}\left(\frac{2}{3}\rho_r^l + \frac{1}{3}\rho_{\sfrac{r}{2}-1}^{l-1} + \rho_{\sfrac{r}{2}-1}^{l-1}\right) + \frac{\Delta r^l g}{4}\left(\rho_r^l + \frac{2}{3}\rho_r^l + \frac{1}{3}\rho_{\sfrac{r}{2}-1}^{l-1}\right) \nonumber \\
&=& p_{\sfrac{r}{2}-1}^{l-1} + \frac{3\Delta r^l g}{4}\left(\rho_{\sfrac{r}{2}-1}^{l-1} + \rho_r^l\right).
\end{eqnarray}
When we encounter a change from fine to coarse, analogously
\begin{equation}
p_{r}^l = p_{2(r-1)+1}^{l+1} + \frac{3\Delta r^{l+1}g}{4}\left(\rho_{2(r-1)+1}^{l+1} + \rho_{r}^l\right).
\end{equation}



\end{document}
