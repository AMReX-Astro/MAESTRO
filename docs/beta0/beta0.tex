The goal of $\beta_0$ is to capture the expansion of a displaced fluid
element due to the stratification of the atmosphere.  MAESTRO computes
$\beta_0$ as:
\begin{equation}
\beta_0(r,t) = \rho_0 \exp\left (  \int_0^r  \frac{1}{\gammabar p_0} \frac{\partial p_0}{\partial r^\prime} dr^\prime \right )
\end{equation}

\section{Constant Composition}
Consider an isentropically stratified atmosphere, with a constant
composition as a function of $r$.  If you displace a parcel of fluid
upwards, it will expand adabatically and continue to rise until its
density matches the background density.  Even if $\gammabar$ is not
constant in $r$, following from the definition of $\beta_0$,
\begin{equation}
\frac{1}{\beta_0} \frac{d\beta_0}{dr} = \frac{1}{\gammabar p_0} \frac{dp_0}{dr}
\end{equation}
and the definition of $\Gamma_1$,
\begin{equation}
\Gamma_1 = \left . \frac{d \log p}{d \log \rho} \right |_s
\end{equation}
So, at constant entropy, from the definition of $\Gamma_1$, it must hold
that
\begin{equation}
\frac{1}{\rho} \frac{d \rho}{dr} = \frac{1}{\Gamma_1 p} \frac{d p}{dz} \enskip .
\end{equation}
Comparing to the defintion of $\beta_0$ then
\begin{equation}
\frac{1}{\beta_0} \frac{\beta_0}{dr} =\frac{1}{\gammabar p_0}\frac{dp_0}{dr} = \frac{1}{\rho_0} \frac{d\rho_0}{dr}  \enskip .
\end{equation}
Therefore, $\beta_0 = \rho_0$.  

This means that if we have a constant composition and an
isentropically stratified atmosphere, as we displace a fluid element,
it will always remain neutrally buoyant.



\section{Composition Gradient}

If there is a change in composition with $r$, the situation is more
complicated.  Consider again an isentropically stratified atmosphere,
now with a composition gradient.  If you displace a parcel of fluid
upwards, it will rise.  If there are no processes that change the
composition (e.g.\ reactions), then the composition in the fluid
element will remain fixed.  As it rises, it will the ambient medium
will have a different composition that it has.  In this case, what is
the path to equilibrium?

