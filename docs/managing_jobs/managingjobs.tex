\section{Automatic Restarting and Archiving of Data}

The submission script {\tt jaguar.run} and shell script
{\tt process.jaguar} in {\tt wdconvect/job\_scripts/}
are designed to allow you to run \maestro\ with minimal interaction,
while being assured that the data is archived to HPSS on the NCCS
machines.

To use the scripts, first create a directory in HPSS that has the same
name as the directory on lustre you are running in (just the directory
name, not the full path).  E.g.\ if you are running in a directory
call {\tt wdconvect\_run}, then do:
\begin{verbatim}
hsi
mkdir wdconvect_run
\end{verbatim}

The script {\tt process.jaguar} is called from {\tt jaguar.run} and
will run in the background and continually wait until checkpoint or
plotfiles are created (actually, it always leaves the most recent one
alone, since data may still be written to it, so it waits until there
are more than 1 in the directory).

Then the script will use {\tt htar} to archive the plotfiles and checkpoints
to HPSS.  If the {\tt htar} command was successful, then the plotfiles are
copied into a {\tt plotfile/} subdirectory.  This is actually important,
since you don't want to try archiving the data a second time and
overwriting the stored copy, especially if a purge took place.

Additionally, if you have the path to the {\tt ftime} executable set in the
script ({\tt ftime.f90} lives in {\tt fParallel/data\_processing/}), then the
script will create a file called {\tt ftime.out} that lists the name of the
plotfile and the corresponding simulation time.

Finally, right when the job is submitted, the script will tar up all
of the diagnostic files created by {\tt diag.f90} and archive them on HPSS.
The {\tt .tar} file is given a name that contains the date-string to allow
multiple archives to co-exist.

This really allows you to run the job and have all of the data stored
long term automatically.  This way you don't need to worry about
filesystem purges.  It seems to work well.

Also, the {\tt jaguar.run} submission script has code in it that will look
at the most recently generated checkpoint files, make sure that they
were written out correctly (it looks to see if there is a Header file,
since that is the last thing written), and automatically set the
{\tt --restart} flag on the run line to restart from the most recent
checkpoint file.

This allows you to job-chain a bunch of submission and have them wait
until the previous job finished and then automatically queue up:
\begin{verbatim}
qsub -W depend=afterany:<JOB-ID>  <QSUB SCRIPT>
\end{verbatim}
where {\tt <JOB-ID>} is the id number of the job that must complete
before the new submission runs and {\tt QSUB SCRIPT} is the submission
script (e.g.\ {\tt jaguar.run}).
This way you can queue up a bunch of runs and literally leave things
alone and it will restart from the right place automatically and store
the data as it is generated.

When {\tt process.jaguar} is running, it creates a lockfile (called
{\tt process.pid}) that ensures that only one instance of the script
is running at any one time.  Sometimes if the machine crashes, the
{\tt process.pid} file will be left behind, in which case, the script
aborts.  Just delete that if you know the script is not running.

Analogous scripts exist for running on Atlas, with the same general
procedure.  The command to chain a job on atlas is:
\begin{verbatim}
msub -l depend=<JOB-ID> <MSUB SCRIPT>
\end{verbatim}
where, again, {\tt <JOB-ID>} is the id number of the job that must
complete before the current submission runs and {\tt <MSUB SCRIPT>}
is the job submission script (e.g.\ {\tt atlas.run}).
